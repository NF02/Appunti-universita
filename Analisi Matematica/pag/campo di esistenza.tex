\section{Campo di esistenza}
\begin{defi}
	lo svolgimento di una funzione fratta è il seguente -- va imposto il
	denominatore diverso da zero perché una frazione con al denominatore 0
	tende ad $\infty$.
	\begin{esempio}
		prendendo la funzione $y=\frac{x+1}{x-2}$, ricordando che il
		denominatore\\
		\underline{non deve assolutamente essere uguale a zero}
		quindi $x-2\neq 0$, per questo motivo $x\neq 2$ cioè
		\begin{equation}
			C.E.=(x\in \mathds{R} | x \neq 2)
		\end{equation}
		Il campo di esistenza è l'insieme delle x appartenenti ad $\mathds{R}$
		tale che x è diverso da 2, oppure in altra notazione
		$C.E.=(-\infty,2)\bigcup (2,+\infty)$ il campo di esistenza è l'insieme
		di tutti i punti della retta reale escluso il punto 2.
	\end{esempio}
\end{defi}
\begin{defi}
	lo svolgimento di una funziona radicale pari è il seguente -- il termine
	sotto radice deve essere maggiore o uguale a zero
	\begin{esempio}
		prendendo la funzione $y=\sqrt{x-3}$, visto che il radicale non può
		essere negativo, visto che si tratta di un radicale pari, quindi, si
		può dedurre che la x deve essere maggiore o uguale a 3 quindi la
		deduzione che si può avere è che in primo luogo $x\geq 0$ e anche
		$x\geq 3$ quindi nel solito modo il campo di esistenza è
		\begin{equation}
			C.E.=(x\in \mathds{R} | x\geq 3)
		\end{equation}
		Il campo di esistenza è l'insieme delle x appartenenti ad $\mathds{R}$
		tali che x è maggiore/uguale a 3 oppure altrimenti espresso come
		$C.E.=[3, +\infty)$.
	\end{esempio}
\end{defi}
\begin{defi}
	l'argomento del logaritmo deve essere maggiore di zero, Una funzione
	logaritmica se la x compare nell'argomento del logaritmo. Consideriamo una
	funzione logaritmica $y=\log(x+4)$, Poiché il logaritmo è definito solo per
	valori positivi dell'argomento, il termine dentro alle parentesi dovrà
	essere maggiore di zero $(x+4)>0$ segue $x>-4$ quindi il campo di esistenza
	sarà:
	\begin{equation}
		C.E.=\{x\in R | x>-4\}
	\end{equation}
	Il campo di esistenza è l'insieme di tutti i punti della retta reale
	maggiori di -4 oppure in altro notazione $C.E.=[-4,+\infty)$. Altrimenti,
	al di fuori di questi casi, il campo di esistenza è tutto l'asse reale.
\end{defi}
\begin{defi}
	Quando una funzione è continua in un modulo per disegnarla basta disegnare
	la funzione senza modulo poi riportare sopra l'asse delle x la parte che
	si trova sotto l'asse.
\end{defi}


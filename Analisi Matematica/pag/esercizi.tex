\chapter{Esercizi}
Per una maggior comprensione è giusto mostrare degli esercizi svolto e visto che ci sono alcuni casi in cui serve fare ulteriori operazioni per risolvere lo studio\dots
\section {Infinitesimi e infiniti}
\subsection{Esercitazione 1}
\begin{equation}
	y=1-\sqrt{1+x}
\end{equation}
\begin{equation*}
	\lim_{x\to0}\left[\frac{1-\sqrt{1+x}}{\sqrt{x}}\right]=\lim_{x\to0}\frac{\frac{1}{2\sqrt{1+x}}}{\frac{1}{2\sqrt{x}}}=-\frac{1}{\not{2}\sqrt{1+x}}*\not{2}\sqrt{x}=\frac{0}{1}=0
\end{equation*}
da questo si deduce che il limite non è di ordine $\frac{1}{2}$ come ipotizzato inizialmente ma risulta di un ordine maggiore, allora il passo successivo è tentare con l'ordine 1:
\begin{equation*}
	\lim_{x\to0}\left[\frac{1-\sqrt{1+x}}{x}\right]=\lim_{x\to0}-\frac{1}{2\sqrt{1+x}}=-\frac{1}{2}
\end{equation*}
e appunto in questo caso $ord\left(1-\sqrt{1+x}\right)=1$
\subsection{Esercitazione 2}
\begin{equation}
	f(x)=(1+x)^2-1
\end{equation}
\begin{equation*}
	\lim_{x\to0}\frac{(1+x)^2-1}{x}=\lim_{x\to0}2(1+x)=2
\end{equation*}
Da questo si evince che $ord\left[(1+x)^2-1\right]=1$, quindi è un infinitesimo di ordine 1.
\subsection{Esercitazione 3}
\begin{equation}
	f(x)=\tan x
\end{equation}
\paragraph{Ipotesi:} $\lim\limits_{x\to0}\tan x$ è di ordine 1
\begin{equation}
	\lim_{x\to0}\frac{\tan x}{x}=\frac{
\end{equation}

\section{Studio di funzione}
\subsection{Esercitazione 1}
\begin{equation}
	f(x)=\ln\left(x-x^3\right)
\end{equation}
\begin{enumerate}
	\item Dominio
	\begin{equation*}
		x-x^3>0
	\end{equation*}
	
	Per procedere dobbiamo raggruppare, mettendo come valor comune la $x$
	\begin{equation*}
		x(1-x^3)>0
	\end{equation*}
	Quindi prendiamo i due casi singolarmente e il risultato sarà $x>0$ e $x<\pm1$, quindi il dominio sarà
	\begin{equation*}
		\forall x\in (-\infty,-1) \vee (0,1)
	\end{equation*}
	\item Intersezione con l'asse x
		\begin{equation*}
			\begin{matrix}
				\ln\left(x-x^3\right)=0\\
				x-x^3=1\\
				-x^3+x-1=0
			\end{matrix}
		\end{equation*}
	\item comportamento agli estremi del dominio
		\begin{equation*}
			\lim_{x\to -\infty}\ln\left(x-x^3\right)=-\infty+infty
		\end{equation*}
		in questo caso non può essere utilizzata la regola di De l'Hopital per risolvere la forma indeterminata, quindi il metodo più semplice è quello di mettere in evidenza la x e procedere.
		
	\item Derivata prima
	\begin{equation*}
		f^\prime(x)=\frac{1}{(x-x^3)}\left(1-3x^2\right)=\frac{1-3x^2}{x-x^3}
	\end{equation*}
	Andremo a studiare solo il numeratore perché il denominatore è già stato definito nel dominio, quindi è l'unica cosa che può identificare il segno e quindi l'andamento della funzione.
	\begin{equation}
		\begin{matrix}
			1-3x^2=0\\
			-3x^2=-1\\
			3x^2=1\\
			x^2=\frac{1}{3}\\
			x=\pm\sqrt{\frac{1}{3}}=\pm\frac{1}{\sqrt{3}}
		\end{matrix}
	\end{equation}
	In questo caso è presente un punto di massimo e quindi dobbiamo calcolarlo per poterlo tracciare nel disegno.
	\begin{equation*}
		f\left(\frac{1}{\sqrt{3}}\right)=\ln\left(\frac{1}{\sqrt{3}}-\frac{1}{3\sqrt{3}}\right)=\ln\left(\frac{3-1}{3\sqrt{3}}\right)=\ln\left(\frac{2}{3\sqrt{3}}\right)
	\end{equation*}
	Quindi il massimo si trova in $\max\left[\frac{1}{\sqrt{3}}; \ln\left(\frac{2}{3\sqrt{3}} \right) \right]$
	\item grafico
	\begin{figure}[!ht]
		\centering
		\begin{tikzpicture}
			\node[] (pic) at (0,0) {\includegraphics[height=8cm]{img/esercitazioni/esercitazione 1.pdf}};
		\end{tikzpicture}
		\caption{Grafico di Funzione $f(x)=\ln\left(x-x^3\right)$}
	\end{figure}
\end{enumerate}


\section{Esercizi}
Per una maggior comprensione è giusto mostrare degli esercizi svolto e visto che ci sono alcuni casi in cui serve fare ulteriori operazioni per risolvere lo studio\dots
\subsection{Esercitazione 1}
\begin{equation}
	f(x)=\sqrt{1-|x|}
\end{equation}
Essendoci un valore assoluto la funzione è pari, quindi abbiamo già un dato certo.
\subsubsection{Dominio}
\begin{tasks}(2)
	\task \begin{equation*}
	1-|x|\geq 0
\end{equation*}
	\task \begin{equation*}
	-|x|\geq -1
\end{equation*}
	\task \begin{equation*}
	-|x|\geq -1
\end{equation*}
	\task \begin{equation*}
	|x|\leq 1
\end{equation*}
\end{tasks}
\[
	x\in[-1;1]
\]
\subsubsection{Intersezioni con gli assi}
Per poter tracciare il grafico un dei punti da svolgere è proprio l'intersezione con gli assi, che consente di comprendere dove la finzione passi effettivamente.
\paragraph{Intersezioni con l'asse x}
\begin{eqnarray*}
	A(-1;0)&B(1;0)
\end{eqnarray*}
\paragraph{Intersezioni con l'asse y}
\begin{eqnarray*}
	f(0)=\sqrt{1}=1&C(0;1)
\end{eqnarray*}
\subsubsection{Derivata prima}
Essendoci dei valori assoluto bisogna separare i due casi:
\begin{equation*}
	f(x)=\begin{cases}
		\sqrt{1-x}&\text{se }0\leq x\leq 1\\
		\sqrt{1+x}&\text{se }-1\leq x\leq 0
	\end{cases}
\end{equation*}
Quindi andando a derivare il tutto

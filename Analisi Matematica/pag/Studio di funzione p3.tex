\section{Equazioni differenziali lineari di ordine n}
\begin{equation}
	y^{(n)}+a_1(x)y^{(n-1)}+\dots+a_{n-1}(x)y^\prime+a_n(x)y=b(x)
\end{equation}
\begin{equation*}
	\left.
	\begin{aligned}
		 a_i(x)=\text{coefficienti}\\
		b(x)=\text{termine noto}
       	\end{aligned}
 	\right\} \text{Definizione in } I\subseteq \mathfrak{R}
\end{equation*}
Se $b(x)=0$ l'equazioone si dice \textit{omogenea}, altrimenti \textit{non omogenea}
\subsection{Teorema}
Se $y_1(x),\dots,y_n(x),\text{ }x\in I \subseteq R$, sono soluzioni particolari dell'equazione differenziale lineare omogenea di ordine n allora $c_1y_1+\dots+c_ny_n$ è soluzione.\\
L'integrale generale dell'equazione differenziale lineare omogenea di ordine n è
\begin{equation*}
	y_0(x)=c_1y_1(x)+\dots+c_ny_n(x)
\end{equation*}
$y_1(x),\dots,y_n(x)$, sono n soluzioni linearmente indipendenti, $c_1,\dots, c_n$, sono n costanti arbitrarie.
\subsection{Definizione di funzione linearmente indipendente}
$y_1(x),\dots,y_n(x)$, sono funzioni \underline{linearmente indipendenti} se
\begin{equation*}
	c_1y_1+\dots+c_ny_n=0\Rightarrow c_1=c_2=\dots=c_n=0
\end{equation*}
\textit{Condizione necessaria e sufficiente affinché n soluzioni, di un'equazione differenziale di ordine n, siano linearmente indipendenti è che il determinante Wronskiano:}
\begin{equation*}
	\begin{vmatrix}
		y_1,&\cdots&y_n\\
		y_1^\prime&\cdots&y^\prime_n\\
		\vdots&&\vdots\\
		y_1^{(n-1)}&\cdots&y_n^{(n-1)}
	\end{vmatrix}\neq 0
\end{equation*}
Data l'equazione non omogenea
\begin{description}
	\item[(1) ] $y^{(n)}+a_1(x)y^{(n-1)}+\dots+a_{n-1}(x)y^\prime+a_n(x)y=b(x)$ 
\end{description}
e la sua omogenea associata:
\begin{description}
	\item[(2) ] $y^{(n)}+a_1(x)y^{(n-1)}+\dots+a_{n-1}(x)y^\prime+a_n(x)y=0$  
\end{description}
l'integrale generale di (1) è:
\begin{equation*}
	y(x)=y_0(x)+\overline{y}(x)
\end{equation*}
dove $y_0(x)$ è l'integrale generale di (2) e $\overline{y}(x)$ è un integrale particolare di (1).
\subsubsection{omogenee a coefficiente costanti}
\begin{equation}
	y^{(n)}+a_1(x)y^{(n-1)}+\dots+a_{n-1}(x)y^\prime+a_n(x)y=0
\end{equation}
$a_1,\dots,a_n\in \mathfrak{R}$\\
A tale equazione si associa l'equazione caratteristica:
\begin{equation*}
	\lambda^n+a_1\lambda^{n-1}+\dots+a_{n-1}\lambda+a_n=0
\end{equation*}
che, per il teorema fondamentale dell'algebra, ha in C n radici ciascuna ciascuna contata con la propria molteplicità.
$y=e^{\alpha x}$ è soluzione dell'equazione differenziale lineare omogenea se $\alpha$ è soluzione dell'equazione caratteristica
\begin{equation}
	\text{Infatti se } y=e^{\alpha x}, y^\prime=\alpha x, \dots, y^{(n)}=a^ne^{\alpha x}
\end{equation}
 sostituendo nell'equazione differenziale si ha
 \begin{equation}
	e^{\alpha x}(a^n+a_1\alpha^{n-1}+\dots+a_{n-1}\alpha+a_n)=0
\end{equation}
\begin{description}
\item[$\Rightarrow e^{\alpha x}$ ]  è soluzione dell'equazione omogenea se $\alpha$ è soluzione dell'equazione caratteristica
\end{description}
\begin{description}
	\item[1° Caso)] L'equazione caratteristica ammette n radici reali e distinte $\lambda, \dots,\lambda_n$, allora gli n integrali linearmente indipendenti (\textit{dell'equazione omogenea}) sono:
	\begin{equation*}
		y_1=\lambda1x,y_2=e^{\lambda2x},\dots, y_n=e^{\lambda_n x}
	\end{equation*}
 	e l'integrale generale è
	\begin{equation*}
		y_0=c_1e^{\lambda2x}+\dots+c_ne^{\lambda_nx}
	\end{equation*}
	Esempio $y^{\prime\prime}-5y^\prime+6y=0$
	\item [2°Caso)] L'equazione caratteristica ammette radici reali e multiple, per esempio se $\lambda_1$ è di moltiplicità m, allora m integrali particolari (\textit{dell'equazione omogenea}) sono:
	\begin{equation*}
		y=e^{\lambda1x},y_2=xe^{\lambda1x},\dots,y_m?x^{m_1}e^{\lambda 1x}
	\end{equation*}
	in generale per ogni radice $\lambda_k$ di moltiplicità $m_k$, gli n integrali linearmente indipemndenti sono
	\begin{eqnarray}
		e^{\lambda k^x},xe^{\lambda k^x},x^2e^{\lambda k^x}, \dots, x^{mk^{-1}}e^{\lambda k^x} & k=1,\dots,r,&m_1+m_2+\dots+m_r=n
	\end{eqnarray}
	Es. $y^{\prime\prime\prime}+y^{\prime\prime}=0$
	\item[3° Caso)] L'equazione caratteristica ammette radici complesse coniugate:
		\begin{eqnarray*}
			\lambda=\alpha+i\beta&\text{di molteplicità m} \\
			\overline{\lambda}=\alpha+i\beta&\text{di molteplicità m}
		\end{eqnarray*}
		allora:
		\begin{eqnarray*}
			e^{\alpha x}\cos \beta x,&xe^{\alpha x} \cos\beta x,\dots, x^{m-1}e^{\alpha x} \cos\beta x\\
			e^{\alpha x}\sin \beta x,&xe^{\alpha x} \sin\beta x,\dots, x^{m-1}e^{\alpha x} \sin\beta x
		\end{eqnarray*}
		sono soluzioni dell'equazione omogenea (2m). Si arriva a tali soluzioni considerando gli integrali.\\
		Si arriva a tali soluzioni considerando gli integrali
		\begin{eqnarray*}
			x^ke^{(\alpha+i\beta)x},&x^ke^{(\alpha+i\beta)x},&k=0,1,\dots m-1
		\end{eqnarray*}
		a cui vengono applicate le formule di Eulero\\
		Esempio $y^{(4)}+2y^{\prime\prime}+y=0$
	 
\end{description}


\documentclass{book}
\usepackage[a4paper,top=2.0cm,bottom=2.0cm,left=2.0cm,right=3.0cm]{geometry}

%\documentclass[pdftex,10pt,a4paper]{book}
%\usepackage[paperwidth=19cm,
%paperheight=26cm, outer=2cm, 
%top=1.5cm, bottom=1.5cm]{ geometry}

\usepackage[english,italian]{babel} %l'ultima lingua è quella che legge per i titoli
\usepackage[utf8]{inputenc}
\usepackage[T1]{fontenc,url}
\usepackage{titlesec}
\usepackage{easylist}
\usepackage{hanging}

\usepackage[pdftex,colorlinks]{hyperref}
\hypersetup{
	colorlinks=true,
	linkcolor=black,
	filecolor=magenta,
	urlcolor=cyan,
}
\usepackage{hypcap}
\usepackage{blindtext}
\usepackage{tipa}
\usepackage{epigraph}
\usepackage{enumerate}
\usepackage{longtable}
\usepackage{setspace}
\usepackage{verbatim}
\usepackage{graphicx}
\usepackage{amsmath}
\usepackage{pbox}
\usepackage{fancyhdr}
\usepackage{cancel}
\usepackage{tabularx}
\usepackage{booktabs}
\usepackage{multirow}
\usepackage{longtable}
\usepackage{tikz}
\usepackage{tikz-qtree}
\usepackage{subfig}
\usepackage{xcolor}
\usepackage{amssymb}
\usepackage{amsmath}
\usepackage{mathrsfs}
\usepackage{textcomp}
\usepackage{circuitikz}
\usepackage{pifont}
\usepackage{imakeidx}
\usepackage{verbatim}
\usepackage{dsfont}
\usepackage{listings}
\usepackage{color}
\usepackage{upgreek}
\usepackage{tasks}
\usepackage{exsheets}
\usepackage{pgfplots}
\usepackage{amsthm}
\usepackage{wasysym}


\usepackage{showframe}
\renewcommand\ShowFrameLinethickness{0.15pt}
%\renewcommand*\ShowFrameColor{\color{red}}

%\usepackage{showkeys} %serve per mostrare le etichette "tag" o target, va tolta per la versione definitiva;

\SetupExSheets[question]{type=exam}

\definecolor{mygreen}{rgb}{0,0.6,0}
\definecolor{mygray}{rgb}{0.5,0.5,0.5}
\definecolor{mymauve}{rgb}{0.58,0,0.82}

\lstset{ 
  backgroundcolor=\color{white},   % choose the background color; you must add \usepackage{color} or \usepackage{xcolor}; should come as last argument
  basicstyle=\footnotesize,        % the size of the fonts that are used for the code
  breakatwhitespace=false,         % sets if automatic breaks should only happen at whitespace
  breaklines=true,                 % sets automatic line breaking
  captionpos=b,                    % sets the caption-position to bottom
  commentstyle=\color{mygreen},    % comment style
  deletekeywords={...},            % if you want to delete keywords from the given language
  escapeinside={\%*}{*)},          % if you want to add LaTeX within your code
  extendedchars=true,              % lets you use non-ASCII characters; for 8-bits encodings only, does not work with UTF-8
  firstnumber=1000,                % start line enumeration with line 1000
  frame=single,	                   % adds a frame around the code
  keepspaces=true,                 % keeps spaces in text, useful for keeping indentation of code (possibly needs columns=flexible)
  keywordstyle=\color{blue},       % keyword style
  language=Octave,                 % the language of the code
  morekeywords={*,...},            % if you want to add more keywords to the set
  numbers=left,                    % where to put the line-numbers; possible values are (none, left, right)
  numbersep=5pt,                   % how far the line-numbers are from the code
  numberstyle=\tiny\color{mygray}, % the style that is used for the line-numbers
  rulecolor=\color{black},         % if not set, the frame-color may be changed on line-breaks within not-black text (e.g. comments (green here))
  showspaces=false,                % show spaces everywhere adding particular underscores; it overrides 'showstringspaces'
  showstringspaces=false,          % underline spaces within strings only
  showtabs=false,                  % show tabs within strings adding particular underscores
  stepnumber=2,                    % the step between two line-numbers. If it's 1, each line will be numbered
  stringstyle=\color{mymauve},     % string literal style
  tabsize=2,	                   % sets default tabsize to 2 spaces
  title=\lstname                   % show the filename of files included with \lstinputlisting; also try caption instead of title
}

\frenchspacing

\newcommand{\abs}[1]{\lvert#1\rvert}

\usepackage{floatflt,epsfig}

\usepackage{multicol}
\newcommand\yellowbigsqcup[1][\displaystyle]{%
  \fboxrule0pt
  \ifx#1\textstyle\fboxsep-0.6pt\else\fboxsep-1.25pt\fi
  \mathrel{\fcolorbox{white}{yellow}{$#1\bigsqcup$}}}

\newtheorem{es}[section]{Esercizio}
\title{Equazioni lineari: basi e proprietà}
\author{Nicola Ferru}
\begin{document}
\maketitle
In questo documento sono presenti, alcune nozioni ed esercizi necessari a comprendere
come svolgere le equazioni lineari in funzione di $x$ e anche di $a$.
\begin{es}
  \label{es:esercizio1}
  Svolgere la seguente equazione lineare, andando a ricavare il valore di $x$
  \begin{equation}
    \label{eq:esercizio1}
    ax-a+3=2(a-x)
  \end{equation}
  ora, analizzando questa equazione, la cosa più immediata da fare è partire da $a$,
  quindi, il primo passaggio da fare è quello di applicare la proprietà distributiva
  per moltiplicare 2 per $a-x$:
  \begin{eqnarray*}
    ax-a+3=2a-2x
  \end{eqnarray*}
  poi sarà possibile sottrarre $2a$ da entrami i lati:
  \begin{eqnarray*}
    ax-a+3-2a=-2x
  \end{eqnarray*}
  dopo aver fatto questo, sarà possibile combinare $a$ e $-2a$ per ottenere $-3a$.
  \begin{eqnarray*}
    ax-3a+3=-2x
  \end{eqnarray*}
  A questo punto, bisogna sottrarre 3 da entrambi i lati
  \begin{eqnarray*}
    ax-3a=-2x-3
  \end{eqnarray*}
  Bisogna combinare tutti i termini contenenti $a$.
  \begin{eqnarray*}
    (x-3)a=-2x-3
  \end{eqnarray*}
  dopo aver raggruppato per $a$, bisogna dividere entrambi i lati
  per $x-3$.
  \begin{eqnarray*}
    \frac{(x-3)a}{x-3}=\frac{-2x-3}{x-3}
  \end{eqnarray*}
  e in fine è possibile ricavare la formula per ricavare $a$, andando a dividere
  $-2x-3$ per $x-3$.
  \begin{eqnarray*}
    a=\frac{2x+3}{x-3}
  \end{eqnarray*}
  Dopo, aver ricavato questa parte (e aver capito che $x\neq 3$), sarà possibile andare a
  svolgere le operazioni per poter ricavare il teorico valore di $x$. Quindi ripartendo da
  l'equazione (\ref{eq:esercizio1}). bisognerà come nel primo caso, applicare la
  proprietà distributiva per moltiplicare 2 per $a-x$, ottenendo quindi, la
  medesima situazione di prima:
  \begin{eqnarray*}
    ax-a+3=2a-2x
  \end{eqnarray*}
  poi, si aggiunge $2x$ a entrambi i lati
  \begin{eqnarray*}
    ax-a+3+2x=2a
  \end{eqnarray*}
  bisogna, aggiungere $a$ da entrambi i lati
  \begin{eqnarray*}
    ax+3+2x=2a+a
  \end{eqnarray*}
  poi bisogna cambianare $2a$ e $a$ per ottenere $3a$.
  \begin{eqnarray*}
    ax+3+2x=3a
  \end{eqnarray*}
  a questo punto, è possibile sottrarre 3 da entrambi i lati:
  \begin{eqnarray*}
    ax+2x=3a-3
  \end{eqnarray*}
  Combinando tutti i termini contenenti $x$
  \begin{eqnarray*}
    (a+2)x=3a-3
  \end{eqnarray*}
  e quindi bisogna dividere da entrambi i lati per $a+2$
  \begin{eqnarray*}
    \frac{(a+2)x}{a+2}=\frac{3a-3}{a+2}
  \end{eqnarray*}
  Bisogna dividere per $a+2$ annullando la moltiplicazione per $a+2$.
  \begin{eqnarray*}
    x=\frac{3a-3}{a+2}
  \end{eqnarray*}
  Poi per concludere la situazione basta dividere $-3+3a$ per $a+2$
  \begin{eqnarray*}
    x=\frac{3(a-1)}{a+2}
  \end{eqnarray*}
  e quindi, $a\neq -2$. Da questi fattori si comprende che questa equazione
  sia impossibile.
\end{es}
\clearpage
\begin{es}
  \label{es:esercizio2}
  Svolgere la seguente equazione lineare, per ricavare, il valore di $a$ e di $x$
  \begin{equation}
    \label{eq:esercizio2}
    a^2(x-2)=3(ax-6)
  \end{equation}
  In questo caso la prima cosa da fare è quella di utilizzare la proprietà distributiva,
  per moltiplicare $a^2$ per $x-2$.
  \begin{eqnarray*}
    a^2x-2a^2=3(ax-6)
  \end{eqnarray*}
  e poi è necessario fare lo stesso dall'altro lato, utilizzando la medesima proprietà
  con 3 e $ax-6$, ottenendo un $3ax-18$, quindi
  \begin{eqnarray*}
    a^2x-2a^2=3ax-18
  \end{eqnarray*}
  fatto questo, basterà sottrarre $3ax$ da entrambi i lati per $a^2-3a$
  \begin{eqnarray*}
    \frac{(a^2-3a)x}{2a^2-18}=\frac{2a^2-18}{a^2-3a}
  \end{eqnarray*}
  uno dei passaggi successivi, è quello di dividere per $a^2-3a$ annullando la
  moltiplicazione per $a^2-3a$
  \begin{eqnarray*}
    x=\frac{2a^2-18}{a^2-3a}
  \end{eqnarray*}
  e poi alla fine $2a^2-18$ per $a^2-3a$.
  \begin{eqnarray*}
    x=2+\frac{6}{a}
  \end{eqnarray*}
  Ora, questa può anche essere scritta in quest'altra forma
  \begin{eqnarray*}
    x=\frac{2(a+3)}{a}
  \end{eqnarray*}
  da questa è possibile dedurre che $a=0$ e $a=3$, ora vedendo la natura dell'equazione
  si denota che sia indeterminata e anche impossibile, perché $a\neq 0 \wedge a\neq 3$.
\end{es}
\begin{es}
  \label{es:esercizio3}
  Svolgere la seguente operazione, per ricavare $b$
  \begin{equation}
    \label{eq:esercizio3}
    b[(b+x)-2]=b^2+bx
  \end{equation}
  basterà sostituire, quindi per $b$
  \begin{eqnarray*}
    b(x+b)
  \end{eqnarray*}
  quindi visto che in questo caso si ottiene:
  \begin{eqnarray*}
    \begin{cases}
      b=0\\
      x=b
    \end{cases}
  \end{eqnarray*}
  in questo caso si tratta di una equazione indeterminata e anche impossibile,
  perché $b\neq0$
\end{es}
\begin{es}
  Svolgere la seguente equazione, per ricavare $k$ e $x$
  \begin{equation}
    \label{eq:esercizio4}
    7(kx+x-1)=k(3x-1)-5-x
  \end{equation}
  Visto che comunque bisogna partire da qualche punto è il caso di partire da $k$,
  in questo caso bisogna, utilizzare la proprietà distributiva per moltiplicare
  7 per $kx+x-1$
  \begin{eqnarray*}
    7kx+7x-7=k(3x-1)-5-x
  \end{eqnarray*}
  bisogna fare la stessa cosa dall'altro lato, moltiplicando $k$ per $3x-1$
  \begin{eqnarray*}
    7kx+7x-7=3kx-k-5-x
  \end{eqnarray*}
  poi a questo punto, bisogna sottrarre per $3kx$ da entrambi i lati e poi bisogna
  aggiungere $k$
  \begin{eqnarray*}
    7kx+7x-7-3kx=-k-5-x
  \end{eqnarray*}
  poi dopo aver spostato alla sinistra del ugualianza, adesso è il caso
  di sottrarre $-3kx$ a $7kx$:
  \begin{eqnarray*}
    4kx+7x-7=-k-5-x
  \end{eqnarray*}
  ora, bisogna comunque spostare a sinistra la variabile $k$, per fare questo
  basterà sommare per $k$ da ambo i lati
  \begin{eqnarray*}
    4kx-7+k=-5-x-7x
  \end{eqnarray*}
  ora basterà spostare a destra $7x$, andando a sottrarlo da ambo i lati
  \begin{eqnarray*}
    4kx-7+k=-5-x-7x
  \end{eqnarray*}
  dopo aver spostato $7x$, lo si sottrae a $-x$ ottenendo un $-8x$
  \begin{eqnarray*}
    4kx+k-7=-5-8x
  \end{eqnarray*}
  ora per togliere di mezzo il $-5$ e il $-7$ in un solo passaggio, basterà
  sommare $7$ da ambo i lati
  \begin{eqnarray*}
    4kx+k = 2-8x
  \end{eqnarray*}
  a questo punto a sinistra si può raggruppare per $k$
  \begin{eqnarray*}
    (4x+1)k=2-8x
  \end{eqnarray*}
  poi si divide tutto per $4x+1$
  \begin{eqnarray*}
    \frac{(4x+1)k}{4x+1}=\frac{2-8x}{4x+1}
  \end{eqnarray*}
  e alla fine si divide per $2-8x$ per $4x+1$.
  \begin{eqnarray*}
    k=\frac{2(1-4x)}{4x+1}
  \end{eqnarray*}
  Adesso, è il caso di trovare la $x$, quindi partendo dalla consueta base (\ref{eq:esercizio4},
  è possibile andare ad applicare la proprietà distributiva per moltiplicare 7 per $kx+x-1$ e anche
  k per $3x-1$
  \begin{eqnarray*}
    7kx+7x-1=3kx-k-5-x
  \end{eqnarray*}
  poi è possibile sottrarre $3kx$ a $7kx$ e poi si può anche aggiungere $x$ da ambo i lati per poter
  spostare $x$ a sinistra dell'ugualianza
  \begin{eqnarray*}
    4kx+7x-7+x=-k-x
  \end{eqnarray*}
  dopo questo, basta semplicemente, andare a combinare $7x$ e $x$ per poter ottenere $8x$ e poi si
  aggiunge $7$ da ambo i lati per poterlo spostare a destra dell'ugualianza, per poi andare a sommarlo
  a $-5$
  \begin{eqnarray*}
    4kx+8x=-k+2
  \end{eqnarray*}
  poi è possibile mettere in evvidenza $x$
  \begin{eqnarray*}
    (4k+8)x=2-k
  \end{eqnarray*}
  poi si va a dividere tutto per $4k+8$ per annullare la moltiplicazione per $4k+8$
  \begin{eqnarray*}
    x=\frac{2-k}{4k+8}
  \end{eqnarray*}
  a questo punto è possibile mettere in evvidenza 4
  \begin{eqnarray*}
    x=\frac{2-k}{4(k+2)}
  \end{eqnarray*}
  quindi visto il contesto, è logico che questa equazione sia impossibile, avendo come casi di esistenza
  $k\neq 0$ e $x\neq -\frac{1}{4}$.
\end{es}

\end{document}

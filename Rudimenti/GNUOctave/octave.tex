\documentclass{book}

\usepackage[utf8]{inputenc}
\usepackage{titlesec}
\usepackage{easylist}
\usepackage{hanging}
\usepackage{hyperref}
\usepackage[a4paper,top=2.0cm,bottom=2.0cm,left=2.0cm,right=2.0cm]{geometry}
\usepackage{blindtext}
\usepackage{tipa}
\usepackage{epigraph}
\usepackage{enumerate}
\usepackage{longtable}
\usepackage{setspace}
\usepackage{verbatim}
\usepackage[T1]{fontenc}
\usepackage{graphicx}
\usepackage[italian]{babel}
\usepackage{amsmath}
\usepackage{pbox}
\usepackage{fancyhdr}
\usepackage{cancel}
\usepackage{tabularx}
\usepackage{booktabs}
\usepackage{multirow}
\usepackage{longtable}
\usepackage{tikz}
\usepackage{tikz-qtree}
\usepackage{subfig}
\usepackage{xcolor}
\usepackage{amssymb}
\usepackage{mathrsfs}
\usepackage{textcomp}

\usepackage{listings}
\usepackage{color}
\usepackage{matlab-prettifier}

\definecolor{mygreen}{rgb}{0,0.6,0}
\definecolor{mygray}{rgb}{0.5,0.5,0.5}
\definecolor{mymauve}{rgb}{0.58,0,0.82}

\lstset{ 
  backgroundcolor=\color{white},   % choose the background color; you must add \usepackage{color} or \usepackage{xcolor}; should come as last argument
  basicstyle=\footnotesize,        % the size of the fonts that are used for the code
  breakatwhitespace=false,         % sets if automatic breaks should only happen at whitespace
  breaklines=true,                 % sets automatic line breaking
  captionpos=b,                    % sets the caption-position to bottom
  commentstyle=\color{mygreen},    % comment style
  deletekeywords={...},            % if you want to delete keywords from the given language
  escapeinside={\%*}{*)},          % if you want to add LaTeX within your code
  extendedchars=true,              % lets you use non-ASCII characters; for 8-bits encodings only, does not work with UTF-8
  firstnumber=1000,                % start line enumeration with line 1000
  frame=single,	                   % adds a frame around the code
  keepspaces=true,                 % keeps spaces in text, useful for keeping indentation of code (possibly needs columns=flexible)
  keywordstyle=\color{blue},       % keyword style
  language=Octave,                 % the language of the code
  morekeywords={*,...},            % if you want to add more keywords to the set
  numbers=left,                    % where to put the line-numbers; possible values are (none, left, right)
  numbersep=5pt,                   % how far the line-numbers are from the code
  numberstyle=\tiny\color{mygray}, % the style that is used for the line-numbers
  rulecolor=\color{black},         % if not set, the frame-color may be changed on line-breaks within not-black text (e.g. comments (green here))
  showspaces=false,                % show spaces everywhere adding particular underscores; it overrides 'showstringspaces'
  showstringspaces=false,          % underline spaces within strings only
  showtabs=false,                  % show tabs within strings adding particular underscores
  stepnumber=2,                    % the step between two line-numbers. If it's 1, each line will be numbered
  stringstyle=\color{mymauve},     % string literal style
  tabsize=2,	                   % sets default tabsize to 2 spaces
  title=\lstname                   % show the filename of files included with \lstinputlisting; also try caption instead of title
}

\linespread{1.5} % l'interlinea

\newtheorem{defi}{Definizione}

\frenchspacing

\newcommand{\abs}[1]{\lvert#1\rvert}

\usepackage{floatflt,epsfig}

\usepackage{multicol}
\newcommand\yellowbigsqcup[1][\displaystyle]{%
  \fboxrule0pt
  \ifx#1\textstyle\fboxsep-0.6pt\else\fboxsep-1.25pt\fi
  \mathrel{\fcolorbox{white}{yellow}{$#1\bigsqcup$}}}

\lstset{language=Octave,keywordstyle={\bfseries \color{red}}}

\title{Manuale base GNU/Octave}
\author{Nicola Ferru}
\begin{document}
\maketitle

\chapter{Introduzione}
\label{chap:intro}
\begin{defi}
  GNU/Octave è un applicativo per il calcolo matriciale che consente di svilgere
  tutte le operazioni base e non solo a riguardo, dallo somma, divisione,
  moltiplicazioni e sottrazioni tra matrici, calcolo del determinante, del grado e
  tanto altro.
\end{defi}

\section{Pacchetti e impostazioni base}
\label{sec:packbase}

\subsection{Pacchetti}
\label{sec:pack}

\begin{table}[th]
  \centering
  \begin{tabular}{ll}
    {\bf Nome} & {\bf Descrizione}\\\hline
    \href{https://gnu-octave.github.io/packages/fuzzy-logic-toolkit/}{fuzzy-logic-toolkit} & Un toolkit di logica fuzzy per lo più
                                                                                               compatibile con MATLAB per Octave \\\hline
    \href{https://gnu-octave.github.io/packages/symbolic/}{symbolic} & Aggiunge funzionalità di calcolo simbolico a GNU
                        Octave \\\hline
    \href{https://gnu-octave.github.io/packages/ocs/}{Circuit Simulator (OCS)} & Risolvere equazioni di circuiti elettrici DC e transitori. \\\hline
    \href{https://gnu-octave.github.io/packages/control/}{Control} & Strumenti CACSD ({\it Computer-Aided Control System
                       Design}) per GNU Octave,\\ &basati sulla libreria SLICOT.\\\hline
    \href{https://gnu-octave.github.io/packages/instrument-control/}{instrument-control} & Funzioni I/O di basso livello per interfacce seriali, i2c, parallele, tcp, gpib, vxi11,\\
               &udp e usbtmc.\\\hline 
  \end{tabular}
  \caption{pacchetti utili}
  \label{tab:pachutil}
\end{table}

\subsection{Impostazioni e formati}
\label{sec:formImp}

\begin{table}[ht]
  \centering
  \begin{tabular}{ll}
    {\bf Nome} & {\bf Descrizione}\\\hline
    rat & aspetto rateo (invece dei numeri reali rende numeri frazionari)\\\hline
  \end{tabular}
  \caption{Impostazioni e formati}
  \label{tab:form}
\end{table}
\chapter{Funzioni base}
\label{chap:funbase}

\section{Addizioni e sottrazioni tra matrici}
\label{sec:addesottmtx}

\begin{equation}
  \label{eq:es1}
  A=
  \begin{vmatrix}
    2 & 0 \\
    3 & -1
  \end{vmatrix}, \text{ } B=
  \begin{vmatrix}
    4 & -1\\
    1 & 2
  \end{vmatrix} \in M_2(\mathbf{R})
\end{equation}
Calcolare $2A-3B$ e $3A-2B$, per svolgerlo non è complesso, infatti, il primo
step è moltiplicare le matrici per il valore presente esternamente e poi fare
la sottrazione tra matrici, il risultato è il seguente:
\begin{eqnarray*}
  \label{eq:es1_sv}
  2A-3B = 2 \begin{vmatrix}
    2 & 0 \\
    3 & -1
  \end{vmatrix} - 3\begin{vmatrix}
    4 & -1\\
    1 & 2
  \end{vmatrix}=\begin{vmatrix}
    {\color{red}2}\cdot2 & {\color{red}2}\cdot0 \\
    {\color{red}2}\cdot3 & {\color{red}2}\cdot-1
  \end{vmatrix} +\begin{vmatrix}
    {\color{red}-3}\cdot4 & {\color{red}-3}\cdot1\\
    {\color{red}-3}\cdot1 & {\color{red}-3}\cdot2
                 \end{vmatrix}\\
  =
  \begin{vmatrix}
    4 &0\\
    6 &-1
  \end{vmatrix} +
  \begin{vmatrix}
    -12 & 3\\
    -3 &-6
  \end{vmatrix}=
  \begin{vmatrix}
    -8 & 3\\
    3 & -8
  \end{vmatrix}
\end{eqnarray*}
stessa cosa ma con valori inversi 
\begin{eqnarray*}
  3A-2B=
  \begin{vmatrix}
    -2 & 2 \\
    7 & -7
  \end{vmatrix}
\end{eqnarray*}

\subsection{Soluzione per Octave o Mathlab}
\label{sec:solmatoctes1}
\lstset{language=matlab,caption={svolgimento di una sottrazione tra matrixi 2x2},label=solmtx2x2}
\lstinputlisting[language=matlab, style=Matlab-editor]{source/moltM2x3.m}
\end{document}
    
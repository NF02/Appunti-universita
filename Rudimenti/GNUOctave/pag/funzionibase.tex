\chapter{Funzioni base}
\label{chap:funbase}

\section{Addizioni e sottrazioni tra matrici}
\label{sec:addesottmtx}

\begin{equation}
  \label{eq:es1}
  A=
  \begin{vmatrix}
    2 & 0 \\
    3 & -1
  \end{vmatrix}, \text{ } B=
  \begin{vmatrix}
    4 & -1\\
    1 & 2
  \end{vmatrix} \in M_2(\mathds{R})
\end{equation}
Calcolare $2A-3B$ e $3A-2B$, per svolgerlo non è complesso, infatti, il primo
step è moltiplicare le matrici per il valore presente esternamente e poi fare
la sottrazione tra matrici, il risultato è il seguente:
\begin{eqnarray*}
  \label{eq:es1_sv}
  2A-3B = 2 \begin{vmatrix}
    2 & 0 \\
    3 & -1
  \end{vmatrix} - 3\begin{vmatrix}
    4 & -1\\
    1 & 2
  \end{vmatrix}=\begin{vmatrix}
    {\color{red}2}\cdot2 & {\color{red}2}\cdot0 \\
    {\color{red}2}\cdot3 & {\color{red}2}\cdot-1
  \end{vmatrix} +\begin{vmatrix}
    {\color{red}-3}\cdot4 & {\color{red}-3}\cdot1\\
    {\color{red}-3}\cdot1 & {\color{red}-3}\cdot2
                 \end{vmatrix}\\
  =
  \begin{vmatrix}
    4 &0\\
    6 &-1
  \end{vmatrix} +
  \begin{vmatrix}
    -12 & 3\\
    -3 &-6
  \end{vmatrix}=
  \begin{vmatrix}
    -8 & 3\\
    3 & -8
  \end{vmatrix}
\end{eqnarray*}
stessa cosa ma con valori inversi 
\begin{eqnarray*}
  3A-2B=
  \begin{vmatrix}
    -2 & 2 \\
    7 & -7
  \end{vmatrix}
\end{eqnarray*}

\subsection{Soluzione per Octave o Mathlab}
\label{sec:solmatoctes1}
\lstset{language=matlab,caption={svolgimento di una sottrazione tra matrici 2x2},label=solmtx2x2}
\lstinputlisting[language=matlab, style=Matlab-editor]{source/moltM2x2.m}
\paragraph{Stampa a schermo}
\begin{verbatim}
ris =

  -8   3
   3  -8

ris =

  -2   2
   7  -7
\end{verbatim}


\section{Determinante di una matrice}
\label{sec:mtxdet}
Un operazione molto utile è il determinante della matrice, fondamentale per
lavorare su questa categoria di strutture, per calcolarlo non è difficile, in
programmi come GNU/Octave e anche Matlab este la funzione \lstinline|det(M)|,
che fa il classico svolgimento, prendendo un esempio concreto:
\begin{equation}
  \label{eq:es2}
  \begin{vmatrix}
    3 & 5\\
    8 & 4
  \end{vmatrix}
\end{equation}
Partendo da questa base dobbiamo fare la sequente operazione
\begin{equation}
  \label{eq:es2_sv}
  \det(A) =\det \begin{vmatrix}
    3 & 5\\
    8 & 4
  \end{vmatrix}= 3\cdot 4 -5\cdot 8 = 12 - 40 = -28
\end{equation}
Quindi il determinante della matrice 2x2 $A$ è -28, questo è anche il metodo che
potete utilizzare su octave per fare la verifica del valore ottenuto con la
funzione già pronta.
\subsection{Soluzione per Octave o Mathlab}
\label{sec:solmatoctes2}


\lstset{language=matlab,caption={svolgimento del determinante di una matrice 2x2},label=solmtx2x2}
\lstinputlisting[language=matlab, style=Matlab-editor]{source/detMtx2x2.m}
\paragraph{Stampa a schermo}

\begin{verbatim}
A =

   3   5
   8   4

ris = -28
ver = -28
\end{verbatim}

\section{Matrice inversa}
\label{sec:mtxInv}
Un operazione fondamentale è proprio la matrice inversa che serve per diverse
formule presenti nel percorso di Ingegneria. quindi per calcolare l'inversa basta utilizzare il comando \lstinline|inv(M)|, uno dei problemi che si può
riscontrare in questo caso è il fatto che il risultato possa essere espresso in
numeri reali, cosa non molto pratica, quindi per sistemare questo problema
basta applicare il formato rateo, come speficicato sopra, infatti, esiste
una funziona di formato chiamata rat che può essere attivata con il semplice
comando \lstinline|format rat| e il problema verra risolto.
Ma il metodo migliore è quello di fare un esempio. Prendiamo una matrice 3x3
\begin{equation}
  \label{eq:es3}
  A=\begin{vmatrix}
    2  & 4 & 5\\
    3  & 6 & 10\\
    9  & 1 & 7
  \end{vmatrix}
\end{equation}
La sua inversa sarà $A^{-1}$ che sarà composta dei seguenti valori

\section{Diagonale di una matrice}
\label{sec:diag}
Un altra funzione che in Matlab e octave viene fatta in modo pratico e veloce
è la stampa della diagonale. Infatti, dentro l'ambiente viene utilizzato il
comando \lstinline{diag(M)}.
\begin{equation*}
  A=\begin{pmatrix}
      2 & 20 & 1 & 3\\
      4 & 9 & 12 & 0\\
      6 & 4 & 13 & 7\\
      10 & 39 & 37 & 5
  \end{pmatrix}
\end{equation*}
Questo comando va a creare un vettore composto da i numeri presenti nella
diagonale della matrice, in questo caso l'istruzione \lstinline{diag(A)},
selezionera i numeri scritti in rosso:
\begin{equation*}
  A=\begin{pmatrix}
      {\color{red}2} & 20 & 1 & 3\\
      4 & {\color{red}9} & 12 & 0\\
      6 & 4 & {\color{red}13} & 7\\
      10 & 39 & 37 & {\color{red}5}
  \end{pmatrix}
\end{equation*}
e quindi $ans =
\begin{pmatrix}
  2 & 9 & 13 & 5
\end{pmatrix}
$, ovviamente questo accade nel caso base, perché il comando \lstinline|diag|
accetta al suo interno più di un parametri, infatti, se noi andiamo ad accodare
al nominativo della matrice un numero possiamo ottenere le altre diagonali.
Ed esempio se facciamo \lstinline|diag(A,1)|, il risultato sarà il seguente:
\begin{equation*}
  A=\begin{pmatrix}
      2 & {\color{red}20} & 1 & 3\\
      4 & 9 & {\color{red}12} & 0\\
      6 & 4 & 13 & {\color{red}7}\\
      10 & 39 & 37 & 5
  \end{pmatrix}
\end{equation*}
Quindi il vettore risultante sarà composto nel seguente modo $ans =
\begin{pmatrix}
  20 & 12 & 7
\end{pmatrix}
$ da questo si denota che il parametro che andiamo a passare serve semplicemente a distanziarsi positivamente o negativamente dalla diagonale 0, quella che
divide la matrice in due perfettamente. Nel caso in cui venga passato un
parametro negativo, in questo caso il -1, \lstinline|diag(A,-1)| il risultato sarà il seguente:
\begin{equation*}
  A=\begin{pmatrix}
      2 & 20 & 1 & 3\\
      {\color{red}4} & 9 & 12 & 0\\
      6 & {\color{red}4} & 13 & 7\\
      10 & 39 & {\color{red}37} & 5
  \end{pmatrix}
\end{equation*}
\begin{notab}
  Il termine ans sta per Answer ed è il valore che viene salvato automaticamente
  dal calcolatore per rendelo, esso ha una funzione temporanea visto che verrà
  sovrascritto alla prossima operazione. 
\end{notab}

\subsection{Esempio in Matlab o Octave}
\label{sec:diag_ex}
\lstset{language=matlab,caption={Esempio di utilizzo della funzione \lstinline|diag()|},label=solmtx2x2}
\lstinputlisting[language=matlab, style=Matlab-editor]{source/diagMtx.m}
\clearpage

\subsubsection{Stampa a schermo}
\label{sec:stampScr}
\begin{verbatim}
A =

    2   20    1    3
    4    9   12    0
    6    4   13    7
   10   39   37    5

________________________
diagZ =

    2
    9
   13
    5

diagU =

   20
   12
    7

diagMU =

    4
    4
   37
\end{verbatim}

\section{Operazioni tra vettori e mattrici}
\label{sec:vettmat}

Un'altra operazione tipica è la somma tra un vettore ``matrice unidimensionale''
e una matrice NxM, quindi anche qui ci sono dei matodi grafici di svolgimento,
ma partiamo dalle basi, prendiamo un vettore A e una matrici v
\begin{eqnarray}
  \label{eq:vetmatex}
  A =
  \begin{pmatrix}
    2 & 3 & 4\\
    1 & 5 & 7\\
    9 & 8 & 6        
  \end{pmatrix}, & \vec{v}=
                  \begin{pmatrix}
                    3 & 4 & 5
                  \end{pmatrix}
\end{eqnarray}
\clearpage
\subsection{Addizioni e sottrazioni tra matrici}
\label{sec:addesottvandmtx}

\subsubsection{Addizioni}
\label{sec:addvectmtx}
non modo non molto dissimile a quello che avveniva con la somma tra mattrici,
anche nella somma tra un vettore e una Matrice si va assomare i membri dell'uno
per quelli dell'altra, in questo caso nello specifico la prima colonna della
matrice è stata moltiplicata per il primo elemento del vettore, la seconda
colonna per il secondo elemento e così via.
\begin{eqnarray*}
   A+\vec{v} =
  \begin{pmatrix}
    {\color{red}3+}2 & {\color{red}4+}3 & {\color{red}5+}4\\
    {\color{red}3+}1 & {\color{red}4+}5 & {\color{red}5+}7\\
    {\color{red}3+}9 & {\color{red}4+}8 & {\color{red}5+}6        
  \end{pmatrix}=
  \begin{pmatrix}
    5 & 7 & 9\\
    4 & 9 & 12\\
    12 & 12 & 11
  \end{pmatrix}
\end{eqnarray*}

\subsubsection{Sottrazioni}
\label{sec:sottvectmtx}
\begin{eqnarray*}
   A-\vec{v} =
  \begin{pmatrix}
    {\color{red}3-}2 & {\color{red}4-}3 & {\color{red}5-}4\\
    {\color{red}3-}1 & {\color{red}4-}5 & {\color{red}5-}7\\
    {\color{red}3-}9 & {\color{red}4-}8 & {\color{red}5-}6        
  \end{pmatrix}=
  \begin{pmatrix}
    -1 & -1 & -1\\
    -2 & 1 & 2 \\
    6 & 4 & 1
  \end{pmatrix}
\end{eqnarray*}

\subsection{Moltiplicazioni e divisioni}
\label{sec:moltedivvetmtx}
\begin{eqnarray*}
   A\cdot\vec{v} =
  \begin{pmatrix}
    {\color{red}3\cdot}2 & {\color{red}4\cdot}3 & {\color{red}5\cdot}4\\
    {\color{red}3\cdot}1 & {\color{red}4\cdot}5 & {\color{red}5\cdot}7\\
    {\color{red}3\cdot}9 & {\color{red}4\cdot}8 & {\color{red}5\cdot}6        
  \end{pmatrix}=
  \begin{pmatrix}
    6 & 12 & 20\\
    3 & 20 & 35\\
    27 & 32 & 30
  \end{pmatrix}
\end{eqnarray*}
In questo caso l'ambiguità non sta nell'operazione in se e per se ma nella sintassi di
matlab e Octave che presentano due diverse funzioni per la moltiplicazione e la
divisione, la prima è \lstinline|A*M| che serve a fare una moltiplicazioni tra matrici
della stessa dimensione e poi c'è quella per che forza la condizione \lstinline|A.*M|
che funziona con matrici di dimensione anche differente.
\begin{oss}
  Per evitare che avvenga una moltiplicazione con i valori in verticale come
  visto qui sopra si può adottare questa soluzione, invece di settare il vettore
  orizzontalmente $\vec{v}=
                  \begin{pmatrix}
                    3 & 4 & 5
                  \end{pmatrix}$ lo si può impostare il vettore verticalente
                  $\vec{v}=
                  \begin{pmatrix}
                    3 \\ 4 \\ 5
                  \end{pmatrix}$, in modo che la situazione sia la seguente:
  \begin{equation*}
    A-\vec{v} =
    \begin{pmatrix}
        {\color{red}3-}2 & {\color{red}3-}3 & {\color{red}3-}4\\
        {\color{red}4-}1 & {\color{red}4-}5 & {\color{red}4-}7\\
        {\color{red}5-}9 & {\color{red}5-}8 & {\color{red}5-}6        
    \end{pmatrix}=
    \begin{pmatrix}
      -1 &  0 &  1\\
      -3 &  1 &  3\\
       4 &  3 &  1
    \end{pmatrix}
  \end{equation*}
  Ecco qui risolto uno dei problemi che può avvenire su matlab e octave.
\end{oss}
\section{Rank}
\label{sec:rank}

\begin{equation*}
  A=\begin{pmatrix}
      1 & 3 & 9\\
      9 & 4 & 12\\
      5 & 90 & 3
  \end{pmatrix}
\end{equation*}
In questo caso il rango è 3, su octave o matlab, basta utilizzare il comando
\lstinline|rank(A)|.\\

\section{Matrici Trasposte}
\label{sec:matt}
\begin{equation*}
  A = \begin{pmatrix}
    2 & 3 & 4\\
    1 & 5 & 7\\
    9 & 8 & 6        
  \end{pmatrix}
\end{equation*}
Per fare la matrice trasposta, basta scambiare i valore di N e M quindi il risultato è
\begin{equation*}
  A^\prime=
  \begin{pmatrix}
    2  & 1 &  9\\
    3  & 5 &  8\\
    4  & 7 &  6
  \end{pmatrix}
\end{equation*}
Praticamente i valori sono stati scambiati tagliando per la diagonale,
infatti, 2, 1 e 6 restano nelle stesse posizioni. In matlab e Octave esiste una
funzione dedicata \lstinline|A.'|, che fa automaticamente l'operazione,
ovviamente per sfruttare al massimo la matrice va salvata in una variabile.
\begin{verbatim}
octave:1> A = [2, 3, 4; 1, 5, 7; 9, 8,6]
A =

   2   3   4
   1   5   7
   9   8   6

octave:2> A'
ans =

   2   1   9
   3   5   8
   4   7   6
\end{verbatim}

\section{Autovettori e autovalori}
\label{sec:autovettorieautovalori}

Autovalori e autovettori costituiscono un aspetto fondamentale dello studio della
diagonalizzabilità e della triangolarizzabilità di una matrice e sono alla base
della costruzione della forma coninica di Jordan.
\begin{defi}
  Si dice forma canonica di Jordan di una matrice quadrata A una particolare matrice a
  blocchi triangolari superiore e simile ad $A$. Viene solitamente indicata con $J_A$ ed
  è caratterizzata dall'avere gli autovalori di A sulla diagonale principale
  (\lstinline|diag(A)| in Octave), degli 0 o degli 1 sulla diagonale soprastante e tutti
  0 altrove. 
\end{defi}

\section{Rouché-Capelli}
\label{sec:rouche-capelli}
La teoria dei sistemi (teorema di \textit{Rouché-Capelli}) ci ha insegnato che ammette
soluzioni non nulle se e solo se
\begin{equation}
  \label{eq:rc}
  \abs{A-\lambda I}=0
\end{equation}
Posto $P(A-\lambda I)$, si può osservare che $P(\lambda)$ è un polinomio di grado $n$ nella variabile $\lambda \in \mathds{R}$; $P(\lambda)$ 
è detto polinomio caratteristico A. Le soluzioni in $\mathds{R}$ dell'equazione (\ref{eq:rc}), cioè $P(\lambda)=0$, sono dette \textit{autovalore}
di $A$,se $\lambda$ è un autovalore di $A$, si può definire
\begin{equation*}
  V_\lambda =\{X\in\mathds{R}^n:[A-\lambda I]\cdot X=\vec{0}\}
\end{equation*}
Sappiamo che $\lambda$ è un autovalore di $\mathds{R}^n$: scriveremo
\begin{equation*}
  m_g(\lambda)=\dim V_\lambda
\end{equation*}
Il numero naturale $M_g(\lambda)$ è chiamato molteplicità geometrica dell'autovalore $\lambda$.
\begin{equation*}
  m_g(\lambda)=n-p(A-\lambda I)
\end{equation*}
In sostanza, $m_g(\lambda)$ coincide col numero di incognite libere del sistema omogeneo, come prescritto dal teorema di Rouché-Capelli. Il sottospazio
$V_\lambda$ è detto \textit{autospazio} associato all'autovalore $\lambda$; i suoi elementi non nulli che ogni autovalore $\lambda$, essendo una radice
di $P(\lambda)$, cioè di un polinomio di grado $n$, ha una sua moltiplicità algebrica, denotata $m_a(\lambda)$. Questo è il momento di eseguire alcuni
esercizi per prendere confidenza con questi concetti.

\subsection{Esercizio}
\label{sec:esAutovettori}

\begin{equation}
  \label{eq:esAuto}
  A=
  \begin{pmatrix}
    1 & 1\\
    0 & 1 
  \end{pmatrix} \in M_2 (\mathds{R})
\end{equation}
\begin{enumerate}
\item Determinare gli autovalori di A;
\item Per ogni $\lambda$, indicare $m_a(\lambda)$ e $m_g(\lambda)$;
\item Determinare una base degli eventuali autospazi.
\end{enumerate}

\paragraph{Soluzione}
\begin{enumerate}
\item Il polinomio caratteristico è
  \begin{equation*}
    P(\lambda)= [A-\lambda I] =
    \begin{vmatrix}
      \begin{pmatrix}
        1 & 1\\
        0 & 1
      \end{pmatrix}-
      \begin{pmatrix}
        \lambda & 0\\
        0 & \lambda
      \end{pmatrix}
    \end{vmatrix}=
    \begin{vmatrix}
      (1-\lambda) & 1 \\
      1 & (1-\lambda)
    \end{vmatrix}= (1-\lambda)^2
  \end{equation*}
Per risolvere il problema tocca trovare $\lambda$ svolgendo il quadrato di binomio, $\lambda^2-2\lambda+1$, poi dobbiamo dobbiamo ricavare il $\Delta$ e poi in fine calcolare $\lambda_{1,2}$.
\begin{eqnarray*}
  \Delta = b^2-4ac = 4 - 4 = 0\\
  \lambda_{1,2} = \frac{2\pm\sqrt{0}}{2}=1
\end{eqnarray*}
Ne segue che $A$ possiede un unico autovalore $\lambda_1=1$
\item Abbiamo
  \begin{eqnarray*}
    m_a(\lambda_1)=2, & m_g(\lambda_1)=n-p(A-\lambda_1I)=\\
    & = 2-p
      \begin{vmatrix}
        \begin{pmatrix}
          0 & 1 \\
          0 & 0
        \end{pmatrix}
      \end{vmatrix}= 2 - 1 = 1 
  \end{eqnarray*}
\item $V_{\lambda_1}$ è definito come l'insieme delle soluzioni di
  \begin{equation*}
    [A-\lambda_1I]\cdot\begin{vmatrix}
                         x_1\\
                         x_2
    \end{vmatrix}=
    \begin{vmatrix}
      0\\0
    \end{vmatrix}
  \end{equation*}
  cioè
  \begin{equation*}
    \begin{cases}
      x_2=0\\
      0=0
    \end{cases}
  \end{equation*}
  Ora, $\dim V_{\lambda_1}=1$, e una sua base è $\{^t[1,0]\}$.
\end{enumerate}
\lstset{language=matlab,caption={Svolgimento dell'autovalore},label=autovalore}
\lstinputlisting[language=matlab, style=Matlab-editor]{source/autoveteautoval.m}

\subsubsection{Stampa a schermo}
\label{sec:stauto}

\begin{verbatim}
A =

   1   1
   0   1

ris =

   1
   1

ris2 = 1
\end{verbatim}

\section{Criterio di diagonalizzabilità}
\label{sec:diagonalizzabilità}
\begin{equation}
  \label{eq:critDiag1}
  P^{-1}\cdot A \cdot P=
  \begin{bmatrix}
    \lambda_1 & 0 & \dots & 0\\
    0 & \lambda_2 && \vdots\\
    \vdots && \ddots & 0\\
    0 & \dots & 0 & \lambda_n
  \end{bmatrix}
\end{equation}
Attraverso la (\ref{eq:critDiag1}) abbiamo definito il concetto di matrice diagonalizzabile. Il seguente criterio
consente di stabilire sotto quali condizioni una matrice è diagonalizzabile se e solo se soddisfatte le due
proprietà seguenti:
\begin{enumerate}
\item \textit{Criterio di diagonalizzabilità:} Sia $A=[a_{ij}]\in M_n(\mathds{R})$. Allora $A$ è diagonalizzabile
  se e solo se sono soddisfatte le due proprietà seguenti:
  \begin{enumerate}
  \item Tutte le radici di $P(\lambda)$ sono reali;
  \item per ognuna di esse (autovalore), si ha $m_g(\lambda)=m_g(\lambda)$.
  \end{enumerate}
\end{enumerate}
\begin{oss}
  Se $P(\lambda)$ ammette $n$ radici reali distinte fra loro, allora $A$ è diagonalizzabile. Infatti, in questo
  caso è ovvio che ogni autovalore abbia molteplicità algebrica 1, e quindi anche la sua molteplicità geometrica
  deve valere 1, per la (\ref{eq:critDiag2})
\end{oss}
\begin{equation}
  \label{eq:critDiag2}
  1\leq m_g(\lambda)\leq m_a(\lambda)
\end{equation}
\begin{oss}
  Se $A$ è diagonalizzabile, in numeri reali $\lambda_1,\dots,\lambda_n$ che compaiono nella matrice a destra in
  (\ref{eq:critDiag1}) sono precisamente gli autovalori di $A$, ognuno contato un numero di volte pari alla
  propria moltiplicità. Della costruzione della matrice diagonalizzante $P$.  
\end{oss}
\begin{esempio}
  Sia
  \begin{equation}
    \label{eq:critDiag3}
    A=
    \begin{bmatrix}
      2 & 0 & 0\\
      6 & 1 & 6\\
      0 & 0 & 2
    \end{bmatrix} \in M_3(\mathds{R})
  \end{equation}
\paragraph{Soluzione}
Il polinomio caratteristico è
\begin{equation}
  \label{eq:critDiag4}
  P(\lambda)=\abs{A-\lambda I}=
  \begin{vmatrix}
    (2 - \lambda) & 0 & 0\\
    6 & (1-\lambda) & 6\\
    0 & 0 & (2-\lambda)
  \end{vmatrix}=(2-\lambda)^2(1-\lambda)
\end{equation}
Quindi il punto (a) è soddisfatta, con 2 autovlori:
\begin{equation}
  \label{eq:critDiag5}
  \begin{matrix}
    \lambda_{1}=2, & m_a(\lambda_1)=2 \text{ e } \lambda_2=1, & m_a(\lambda_2)=1 
  \end{matrix}
\end{equation} 
\clearpage
Per stabilire se $A$ è diagonalizzabile, bisogna calcolare $m_g(\lambda_1)$\footnote{Il calcolo di $m_g(\lambda_2)$ non
è necessario, per la (\ref{eq:critDiag2})}
\begin{equation}
  \label{eq:critDiag6}
  m_g (\lambda_1)=n-p(A-\lambda_1I)=3-\rho
  \begin{pmatrix}
    \begin{bmatrix}
      0 & 0 & 0\\
      6 & -1 & 6\\
      0 & 0 & 0
    \end{bmatrix}
  \end{pmatrix}
  =3-1=2
\end{equation}
Quindi soddisfa il punto (b) e concludiamo che a è diagonalizzabile.
\end{esempio}
\begin{esempio}
  Sia
  \begin{equation}
    \label{eq:critDiag7}
    A=
    \begin{bmatrix}
      2 & 0 & 0 \\
      0 & 1 & 2 \\
      0 & 2 & 1
    \end{bmatrix} \in M_3(\mathds{R})
  \end{equation}
  Stabire se A è duaginalizzabile.
  
\paragraph{Soluzione}

Il polinomio ccaratteristico è
\begin{eqnarray*}
  P(\lambda)-\abs{A-\lambda I}=
  \begin{vmatrix}
    (2-\lambda) & 0 & 0\\
    0 & (1-\lambda) & 2\\
    0 & 2 & (1-\lambda)
  \end{vmatrix} = (2-\lambda)[(1-\lambda)-4]
\end{eqnarray*}
per ottenere il valore riportato qui sopra abbiamo preso il primo elemento
a sinitra della matrice ``$(2-\lambda)$'' e poi abbiamo moltiplicato a croce
con la sotto matrice $
\begin{vmatrix}
   (1-\lambda) & 2\\
   2 & (1-\lambda)
\end{vmatrix}
$ ed ecco spiegato il $[(1-\lambda)-4]$, tutto questo è possibile per il semplice fatto che la prima riga e la prima colonna sono composte da zeri tranne il primo elemento.\\
Calcolando i discriminante $\lambda$ ricaviamo:
\begin{eqnarray*}
  \lambda_1=2, & \lambda_2=3 \text{ e } \lambda_3=-1
\end{eqnarray*}
Ora, $A$ è diagonalizzabile per l'osservazione 1 sopra.
\end{esempio}

\documentclass{report}

\usepackage[utf8]{inputenc}
\usepackage{titlesec}
\usepackage{easylist}
\usepackage{hanging}
\usepackage{hyperref}
\usepackage[a4paper,top=2.5cm,bottom=1.8cm,left=2.5cm,right=2.5cm]{geometry}
\usepackage{blindtext}
\usepackage{tipa}
\usepackage{epigraph}
\usepackage{enumerate}
\usepackage{longtable}
\usepackage{setspace}
\usepackage{verbatim}
\usepackage[T1]{fontenc}
\usepackage{graphicx}
\usepackage[italian]{babel}
\usepackage{amsmath}
\usepackage{pbox}
\usepackage{fancyhdr}
\usepackage{cancel}
\usepackage{tabularx}
\usepackage{booktabs}
\usepackage{multirow}
\usepackage{longtable}
\usepackage{tikz}
\usepackage{tikz-qtree}
\usepackage{subfig}
\usepackage{xcolor}
\usepackage{amssymb}
\usepackage{mathrsfs}
\usepackage{textcomp}

\usepackage{listings}
\usepackage{color}

\definecolor{mygreen}{rgb}{0,0.6,0}
\definecolor{mygray}{rgb}{0.5,0.5,0.5}
\definecolor{mymauve}{rgb}{0.58,0,0.82}

\lstset{ 
  backgroundcolor=\color{white},   % choose the background color; you must add \usepackage{color} or \usepackage{xcolor}; should come as last argument
  basicstyle=\footnotesize,        % the size of the fonts that are used for the code
  breakatwhitespace=false,         % sets if automatic breaks should only happen at whitespace
  breaklines=true,                 % sets automatic line breaking
  captionpos=b,                    % sets the caption-position to bottom
  commentstyle=\color{mygreen},    % comment style
  deletekeywords={...},            % if you want to delete keywords from the given language
  escapeinside={\%*}{*)},          % if you want to add LaTeX within your code
  extendedchars=true,              % lets you use non-ASCII characters; for 8-bits encodings only, does not work with UTF-8
  firstnumber=1000,                % start line enumeration with line 1000
  frame=single,	                   % adds a frame around the code
  keepspaces=true,                 % keeps spaces in text, useful for keeping indentation of code (possibly needs columns=flexible)
  keywordstyle=\color{blue},       % keyword style
  language=Octave,                 % the language of the code
  morekeywords={*,...},            % if you want to add more keywords to the set
  numbers=left,                    % where to put the line-numbers; possible values are (none, left, right)
  numbersep=5pt,                   % how far the line-numbers are from the code
  numberstyle=\tiny\color{mygray}, % the style that is used for the line-numbers
  rulecolor=\color{black},         % if not set, the frame-color may be changed on line-breaks within not-black text (e.g. comments (green here))
  showspaces=false,                % show spaces everywhere adding particular underscores; it overrides 'showstringspaces'
  showstringspaces=false,          % underline spaces within strings only
  showtabs=false,                  % show tabs within strings adding particular underscores
  stepnumber=2,                    % the step between two line-numbers. If it's 1, each line will be numbered
  stringstyle=\color{mymauve},     % string literal style
  tabsize=2,	                   % sets default tabsize to 2 spaces
  title=\lstname                   % show the filename of files included with \lstinputlisting; also try caption instead of title
}

\linespread{1.2} % l'interlinea

\frenchspacing

\newcommand{\abs}[1]{\lvert#1\rvert}

\usepackage{floatflt,epsfig}

\usepackage{multicol}
\newcommand\yellowbigsqcup[1][\displaystyle]{%
  \fboxrule0pt
  \ifx#1\textstyle\fboxsep-0.6pt\else\fboxsep-1.25pt\fi
  \mathrel{\fcolorbox{white}{yellow}{$#1\bigsqcup$}}}

\newtheorem{nota}{Nota}
\newtheorem{descrizione}{Descrizione}
\newtheorem{notab}{Nota bene}

\title{Laboratory of image processing for computer vision}
\author{Nicola Ferru}
\begin{document}
\maketitle
\tableofcontents

\chapter{Introduzione}
\label{chap:Intro}

\section{Argomenti}
\label{sec:argomento}
Prendendo un immagine come una funzione $f(x,y)$, quindi sul fronte di
sviluppo sarà possibile costruire funzioni e soluzioni per poter
lavorare sulle stesse.

\section{Cosa bisogna sapere sulla rapresentazione raster delle immagini}
\label{sec:raster}

Le immagini raster o a rappresentazioni di pixel, sono la tipoligia di
immagine più utilizzate per la rappresentazione e l'elaborazione grafica,
visto che comunque nel contesto comune si rappresentano delle situazioni
più o mano complesse tipicamente di applicazione reale. Per rendere
efficiente la matrice si usano due sistemi:
\begin{itemize}
\item \textbf{RGB}: Red, Green, Blue (\textit{{\color{red}Rosso}, {\color{green}Verde} e {\color{blue}blue}}), con questo
  sistema è possibile costruire qualunque gradiente colore in modo
  estremamente preciso (valori compreso tra 0 e 255 per tre valori\footnote{Un sistema di rappresentazione dei colori ad 8bit});
\item \textbf{Graylevel}: a Gradienti di grigio (valore compreso tra 0 a
  255 da nero a bianco).
\end{itemize}
\subsubsection{Colori RGB}
\label{sec:rgbcolor}
\begin{table}[ht!]
  \centering
  \begin{tabular}{lccc}
    \textbf{Colore}& \textbf{R} & \textbf{G} & \textbf{B}\\\hline
    Nero & 0 & 0 & 0\\
    Bianco & 255 & 255 & 255\\
    {\color{red}Rosso} & 255 & 0 & 0\\
    {\color{yellow}Giallo} & 255 & 255 & 0\\
    {\color{gray}Grigio} & 127 & 127 & 127\\\hline
  \end{tabular}
  \caption{Colori RGB}
  \label{tab:rgb}
\end{table}
E visto che si tratta di un sistema di rappresentazione a 8 bit per 3
colonne, quindi:
\begin{equation*}
  2^{3\cdot 8} =2^{24} = 16.777.216 \text{ colori}
\end{equation*}
Una gamma cromatica abbastanza sostanziosa da poter rappresentare in modo
sufficientemente fedele qualunque oggetto reale.

\subsubsection{Altri per rappresentare il colore}
\label{sec:altcolorr}

Oltre al sistema di rappresentazione del colore RGB esistoo altri sistemi
di codifica, tra i quali:
\begin{itemize}
\item \textbf{HSL}: Hue, Saturazione, Luce;
\item \textbf{HSV}: Hue, Saturazione, Valore;
\item \textbf{CMYK}: Ciano, Magenta, Giallo, Nero (\textit{Sistema di
    stampa}).
\end{itemize}

\subsection{I pixel}
\label{sec:pixel}

I \textit{pixel} sono il sistema di rappresentazione dell'immagine,
rappresentano l'informazione unitaria, infatti, all'interno del singolo
pixel può esser contenuto solo un valore (\texttt{un colore}), essendo
quadrati per poter rappresentare un immagine che presenta delle forme
stondate con una buona qualità sarà necessaria una matrice di una
dimensione abbondante.
\subsection{La scalabilità di una immagine}
\label{sec:scala}
Uno dei punti che bisogna sempre considarare è proprio la scala e
la scalabilità di una immagine, infatti, non esiste una
dimensione\footnote{Dimensione della matrice di pixel} corratta per ogni
situazione, infatti, dipende tutto dal caso di utilizzo, non a caso è
necessaria fare una valutazione e considerare diversi fattori:
\begin{itemize}
\item \textbf{Dimensione}: il numero di pixel necessari al comporre
  l'immagine, ad esempio un icona \textit{64x64px} oppure
  \textit{128x128px} è perfetta per l'utilizzo desktop, ma magari non è
  idonea per una esposizione fotografica o per un analisi
  biometrica\footnote{un lettore di impronte digitale sta sul
    500x250px, ovviamente dipende dalla tipologia di sensore}.
\item \textbf{Peso}: Il peso dell'immagine è significativa pensando
  all'ambito, il fatto stesso che l'immagine abbia un canale alpha o
  meno cambia il peso nella codifica, assieme al numero di livelli che
  la compongono. 
\item \textbf{Gamma cromatica}: La gamma cromatica è la fedeltà nella
  trasposizione dei colori che un formato riesce ad avere rispetto al
  caso reale e dipende anche dalla paletta cromatica che la codicica di
  suddetto formato possieda. Questo parametro è utile soprattutto nel
  settore fotografico e anche per tutti quei casi in cui è necessario
  un alto livello di dettaglio.
\end{itemize}
\begin{nota}
  La questione della gamma cromatica dipende tanto dal formato, infatti,
  a lato informatico esistono diversi formati per la rappresentazione
  delle immagini, non compressi (\textbf{\color{red}raw}) a quelli
  compressi (\textbf{\color{blue}quelli Lossy e quelli Lossless}).
  \begin{description}
  \item[Lossy] Il contesto di utilizzo in cui la fedeltà e qualità della
    rappresentazione non è il punto saliente. Tipicamente cerca di
    rimuovere con criterio alcune sfumature cromatiche che l'occhio umano
    in primo achitto non nota. (Utile per la condivisione online)
  \item[Lossless] Anche se conpresso cerca di mantenere il più possibile
    la qualità e fedeltà all'immagine non compressa. (Utile anche in
    contesti di editing fotografico)
  \end{description}
  Sapendo questo sarà già possibile muoversi al meglio in questo mondo.
\end{nota}

\section{Pareidolia e illusioni visive}
\label{sec:pareidolia}
Per pareidolia o illusione subconscia, intendiamo il modo in cui il
nostro cervello identifichi all'interno di un oggetto, un immagine o
altro, un volto o una forma familiare, questo perché il cervello umano
è pensato per identificare quello che è famigliare per una questione
pratica ed evolutiva.

\section{Ambiti in cui può essere utilizzata l'elaborazione digitale
  delle immagini?}
\label{sec:ambito}
Quello dell'elaborazione digitale delle immagini può essere utilizzato in
tantissimi settori diversi, da quello biomedico a quello areonautico e
areospaziale, all'automotive, etc. Per questi motivi è sempre più
centrale nel mondo ingegneristico e militare l'adozione di tali sistemi.
Ma facendo un esempio concreto, un sensore di frenata che riconoscra il
pedone e consenta una frenata più efficace, in anticipo rispetto ai
comuni riflessi umani.

\chapter{Proprietà delle immagini digitali}
\label{chap:propdelimmdig}

L'immagine digitale è un metodo per rappresentare di un qualcosa di
reale, e va a rappresentarlo con una matrice di pixel.

\section{Creazione di una immagine digitale}
\label{sec:creadiunimgdig}

\begin{description}
\item[Sampling] il processo che consiste nella downscaling dell'immagine
  o nella rimozione di pixel (\textit{pratica che fa pardere
    delle info});
  \begin{itemize}
  \item \textit{Lineare}: va a scalare tutta l'area dell'immagine in
    modo lineare e uniforme.
  \item \textit{Adattivo}: va a fare uno sampling in base all'importanza
    delle aree.
  \end{itemize}
\item[Quantizzazione] il processo che che va a motificare la profondità
  del colore
  \begin{itemize}
  \item Viene utilizzato per l'analisi dell'immagine e anche per le
    immagini personali.
  \item Problemi da risolvere: Memorizzazione e trasferimento.
  \end{itemize}
\end{description}
E su questi principi sono la base dell'elaborazione dell'immagine
digitale e del campionamento.

\subsection{Tipo di immagine}
\label{sec:tipodiimm}
\begin{itemize}
\item Numero di livello di intensità;
\item Numero di piani dell'immagine:
  \begin{itemize}
  \item Immagine binaria: è un immagine composta unicamente da bianco e
    nero;
  \item A livelli di grigio: è un'immagine che funziona a sfumature di
    grigio;
  \item A colori: sono composte da sfumature RGB che vanno da 0 a 255
    per creare una paletta cromatica accettabile (\ref{sec:raster}).
  \end{itemize}
\end{itemize}
Per generare le tipoligie di immagini si utilizzano dei canali:
\begin{table}[ht!]
  \centering
  \begin{tabular}{lrc}
    \textbf{Tipologie}&\textbf{Numero di canali}& \textbf{Dimensione}
    \\\hline
    bianco e nero & 1 &(1bit)\\
    Immagini a livelli di grigio & 1 &(8bit)\\
    Immagini a colori & 3 &(8bit)\\\hline
  \end{tabular}
  
  \caption{Numero di canali per le tipoligie di immagine}
  \label{tab:numcanimg}
\end{table}

\subsection{Rappresentazione di immagine}
\label{sec:raprimg}
\begin{itemize}
\item \textbf{Iconic Images}: Immagini che contengono i dati originali
  con l'intensità dei singoli pixel;
\item \textbf{Segmented images}: Immagini in cui i pixel sono divisi in
  gruppi in base all'appartenenza o meno agli oggetti;
\item \textbf{Geometric representations}: conoscenza attuale delle forme;
\item \textbf{Relational models}: presentare la conoscenza sugli oggetti
  e sulle relazioni con altri oggetti nell'immagine.
\end{itemize}

\section{Strutture dati}
\label{sec:strutdati}
Le strutture dati, sono le entità matematiche che consentono di
ragruppare più di un valore all'interno dello stesso insieme, per poter
più facilmente gestire, ordinare, ed elaborare il contenuto. Cosa
non possibile con variabili primitive\footnote{È una variabile che può
  contenere solo una tipologia singolarmente, da esse si possono
  creare le così dette \lstinline|struct| e anche delle strutture dati.}.
\subsection{Matrice}
\label{sec:matrice}
\begin{itemize}
\item La struttura dati più comune per rappresentare le immagini;
\item Gli elementi dell'array sono numeri interi;
\item Le caratteristiche spaziali sono implicitamente disponibili.
\end{itemize}
\begin{equation}
  \label{eq:matrice}
  \begin{array}{c|ccccccc}
      & 1 & 2 & 3 & 4 & 5 & 6 & 7 \\\hline
    1 & 1 & 2 & 1 & 3 & 1 & 1 & 1 \\
    2 & 2 & 1 & 1 & 1 & 2 & 3 & 4 \\
    3 & 1 & 2 & 1 & 1 & 1 & 1 & 4 \\
    4 & 1 & 2 & 2 & 2 & 1 & 3 & 1 \\
    5 & 1 & 2 & 2 & 2 & 1 & 1 & 4 \\
    6 & 2 & 2 & 2 & 1 & 1 & 1 & 1 \\
    7 & 1 & 4 & 1 & 1 & 1 & 1 & 1
  \end{array}
\end{equation}

\subsection{Vettore}
\label{sec:vettore}

\begin{itemize}
\item Molto comune per estrarre e archiviare informazioni dall'immagine;
\item Gli esempi di array sono in genere numeri interi;
\item Anche in questo caso il suo contenuto è accessibile attraverso
  informazioni implicite;
\item Gli array non sono altro che array bidimensionali.
\end{itemize}
\begin{equation}
  \label{eq:vettore}
  \begin{array}{ccccc}
    0&1&2&3&4\\\hline
    2&2&1&1&4\\\hline
  \end{array}
\end{equation}
\begin{nota}
  A lato logico all'interno di quasi tutti i linguaggi di
  programmazione, sia i vettori che le matrici fanno parte dello stesso
  tipo di struttura dati, i così detti Array, che possono avere
  \texttt{N} dimensioni. Tipicamente per scorrere suddette strutture si
  utilizzano delle variabili definite ``contatori'', indicate con le
  lettere: \texttt{i,j,k,h} assegnate per comodità visto che stanno in
  un area della tastiera vicina al posizionamento delle mani.
\end{nota}

\subsection{Topologie}
\label{sec:topologie}
Sono una categoria di strutture dati che possono servire ad indicare
logicamente le relazioni. Esse vengono utilizzate anche nel ambito
del processing delle immagini per distrivere le singole relazioni,
ne esistono di diversa tipologie:
\begin{itemize}
\item Grafo: tipologia ad albero utilizzato in tutto quello che prevede
  stati o connessioni;
\item Insiemi: permettono di stabilire appertenenze a ordini gerarchici.
\end{itemize}
\begin{figure}[ht!]
  \centering
  \def\b{.3}
\def\cr{.53}
\begin{tikzpicture} 
%\draw (1.5,1.5) circle [radius=\cr];
\node at (1.5,4.5) {$N= ?$ $R=?$};
% centro a maglia
\draw (1.5,1) circle [radius=\b];
\draw (1,1.9) circle [radius=\b];
\draw (2,1.9) circle [radius=\b];
\draw (2,1.9) -- (1,1.9) -- (1.5,1) -- (2,1.9);
% end centro 

\draw (1.5,1) -- (0,0);
\draw (0,0) circle [radius=\b];

\draw (1.5,1) -- (1.5,0);
\draw (1.5,0) circle [radius=\b];

\draw (3,0) -- (1.5,1);
\draw (3,0) circle [radius=\b];

\draw (1,1.9) -- (0,3);
\draw (0,3) circle [radius=\b];

\draw (2,1.9) -- (3,3);
\draw (3,3) circle [radius=\b];

\draw (2,1.9) -- (2.5,3.7);
\draw (2.5,3.7) circle [radius=\b];
\end{tikzpicture}
  \caption{Grafo}
  \label{fig:grafo}
\end{figure}
Mentre, nel caso dell'insieme la situazione è la seguente:
\begin{figure}[ht!]
  \centering
  \resizebox{8cm}{!}{
  \newcommand{\boundellipse}[3]% center, xdim, ydim
{(#1) ellipse (#2 and #3)
}

\begin{tikzpicture}
	\draw [draw=black] (-6,-5) rectangle  (6,4);
	\node at (-5,3.4){1};
	\draw \boundellipse{2,0}{1}{2};
	\draw \boundellipse{0,0}{5}{3};
	\draw (-4,-1.8) arc (-35:37.7:3);
	\node at (-4,1) {3};
	\node at (0.4,2){4};
	\node at (2,1)  {5};
	\node at (2.5,4.5)  {0};
	\draw (4,4) arc (-180:-90.7:2);
	\node at (5.5,3.5){2};
\end{tikzpicture}
}
  \caption{Insieme}
  \label{fig:insieme}
\end{figure}
\begin{nota}
  In entrambi i casi, la situazione è molto leggibile e si capisce
  anche la relazione in base al livello oppure al sotto insieme a cui
  il valore appartiene.
\end{nota}

\subsection{Gerarchico}
\label{sec:hierarchical}
\begin{itemize}
\item Sono nati per alleggerire il calcolo di operazioni molto complesse,
  ad esempio suddividendo i calcoli tra più computer;
\item Dividendo l'immagine in blocchi
\item Spesso non è così semplice dividere i compiti tra più computer
\item Le informazioni presenti in un blocco possono essere utili per
  elaborare quello adiacente
\item Non tutti i blocchi immagine devono essere elaborati allo stesso
  modo
\end{itemize}

\subsection{Piramidale (pyramids)}
\label{sec:pyramid}

\subsubsection{M-pyramids}
\label{sec:mpyram}
\begin{itemize}
\item Sono una sequenza di immagini con risoluzioni diverse;
\item Utilizzato quando è necessario lavorare con risoluzioni diverse
  contemporaneamente
\item Ovviamente un array di grado inferiore contiene 4 volte meno dati
  e può essere elaborato circa 4 volte più velocemente
\end{itemize}
Ed esempio, è possibile vedere come nella matrice $I_3$ sia una matrice
8x8, mentre, la matrice $I_2$ è una matrice 4x4, nella matrice $I_1$ è
un 2x2 e $I_0$ è una cella singola. 
\begin{eqnarray}
  \label{eq:mpyram}
  I_3=
  \begin{pmatrix}
    2 & 4 & 8 & 0 & 0 & 2 & 6 & 3\\
    3 & 0 & 0 & 1 & 1 & 3 & 6 & 4\\
    2 & 2 & 0 & 1 & 1 & 4 & 6 & 3\\
    3 & 2 & 1 & 1 & 1 & 2 & 6 & 3\\
    4 & 3 & 1 & 1 & 2 & 3 & 6 & 3\\
    4 & 4 & 2 & 1 & 2 & 4 & 7 & 5\\
    5 & 4 & 2 & 1 & 1 & 4 & 7 & 5\\
    5 & 5 & 1 & 0 & 0 & 2 & 7 & 5
  \end{pmatrix} &
  I_2=\begin{pmatrix}
    2 & 2 & 2 & 5\\
    2 & 1 & 2 & 5\\
    4 & 1 & 3 & 5\\
    5 & 1 & 2 & 6
  \end{pmatrix} & I_1=
  \begin{pmatrix}
    2 & 4\\
    3 & 4
  \end{pmatrix} I_0=
  \begin{pmatrix}
    3
  \end{pmatrix}
\end{eqnarray}
\begin{nota}
  Il rapporto tra un immagine e l'altra è un rapporto ad esponenziale di
  con base 2, infatti $I_3$ è un $2^3$, mentre, $I_0$ è un $2^0$. 
\end{nota}
\subsubsection{T-Pyramid}
\label{sec:treepy}
\begin{figure}[ht!]
  \centering
  \resizebox{15cm}{!}{
  \usetikzlibrary{graphs}

% Node styles
\tikzset{
% Two node styles for game trees: solid and hollow
solid node/.style={circle,draw,inner sep=1.5,fill=black},
hollow node/.style={circle,draw,inner sep=1.5}
}

\begin{tikzpicture} [scale=1,font=\footnotesize]
\tikzstyle{level 1}=[sibling distance=50mm] 
\tikzstyle{level 2}=[sibling distance=25mm] 
\tikzstyle{level 3}=[sibling distance=6mm] 
%radice
\node(0)[solid node]{}
% braccio destro
child{ node(1)[solid node][left]{}
child{ node(2)[solid node][left]{}
child{node(17)[solid node][left]{}}
child{node(18)[solid node][left]{}}
child{ node(3)[solid node][right]{}
edge from parent node[left]{livello 2}}
child{ node(12)[solid node][left]{}}
edge from parent node[left]{Livello 1}}
child{ node(4)[solid node][right]{}
child{node(10)[solid node][left]{}}
child{node(11)[solid node][left]{}}
child{node(15)[solid node][right]{}}
child{node(16)[solid node][right]{}}}
edge from parent node[left]{Livello 0}}
% braccio sinistro 2
child{ node(34)[solid node][right]{} 
child {node(35)[solid node][left]{}
child{node(36)[solid node][left]{}}
child{node(37)[solid node][left]{}}
child{node(38)[solid node][right]{}}
child{node(39)[solid node][right]{}}}
child{node(40)[solid node][left]{}
child{node(41)[solid node][left]{}}
child{node(42)[solid node][left]{}}
child{node(43)[solid node][right]{}}
child{node(44)[solid node][right]{}}}}
% braccio destro
child{ node(5)[solid node][right]{} 
child {node(6)[solid node][left]{}
child{node(7)[solid node][left]{}}
child{node(8)[solid node][left]{}}
child{node(19)[solid node][right]{}}
child{node(20)[solid node][right]{}}}
child{node(9)[solid node][left]{}
child{node(13)[solid node][left]{}}
child{node(14)[solid node][left]{}}
child{node(21)[solid node][right]{}}
child{node(22)[solid node][right]{}}}}
% braccio destro 2
child{ node(23)[solid node][right]{} 
child {node(24)[solid node][left]{}
child{node(25)[solid node][left]{}}
child{node(26)[solid node][left]{}}
child{node(27)[solid node][right]{}}
child{node(28)[solid node][right]{}}}
child{node(29)[solid node][left]{}
child{node(30)[solid node][left]{}}
child{node(31)[solid node][left]{}}
child{node(32)[solid node][right]{}}
child{node(33)[solid node][right]{}}}};

\end{tikzpicture}}
  \caption{Esempio di albero piramidale}
  \label{fig:alberopir}
\end{figure}
Da questo grafico è possibile dedurre che:
\begin{itemize}
\item Viene utilizzato per dividere gerarchicamente l'immagine in regioni
  adiacenti;
\item Queste regioni possono essere rappresentate da un albero;
\item Ogni nodo di questo albero ha 4 figli.
\end{itemize}
\clearpage
\subsubsection{Quad-tree}
\label{sec:quad-tree}
\begin{figure}[ht!]
  \centering
  \usetikzlibrary{graphs}

% Node styles
\tikzset{
% Two node styles for game trees: solid and hollow
solid node/.style={circle,draw,inner sep=2,fill=black},
hollow node/.style={circle,draw,inner sep=2},
rect hollow node/.style={rectangle,draw,inner sep=2},
rect solid node/.style={rectangle,draw, inner sep=2, fill=gray}
}

\begin{tikzpicture} [scale=1,font=\footnotesize]
\draw (0,2) -- (4,2);
\draw (2,0) -- (2,4);
\draw (2,1) -- (4,1);
\draw (3,0) -- (3,4);
\draw (0,3) -- (4,3);
\draw (1,2) -- (1,4);
\draw (1,2) -- (1,4);
\draw (1.5,3) -- (1.5,4);
\draw (2,3.5) -- (1,3.5);

\draw[brown, very thick] (0,0) rectangle (4,4);
\draw node[rect solid node] at (1.75,3.74) {E};
\draw node[rect solid node] at (1.25,3.25) {C};
\draw node[rect solid node] at (2.5,2.5) {A};
\draw node[rect solid node] at (3.5,1.5) {B};
\draw node[rect solid node] at (2.5,0.5) {D};
\end{tikzpicture}
  \caption{Quad-tree}
  \label{fig:qtree1}
\end{figure}
In questo modello partiamo da una rappresentazione inscritta in un
quadrato, che contiene al suo interno determinati valori, più il valore
è contenuto all'interno di un sotto quadrato più sarà in un livello
superiore\footnote{Gli alberi si leggono dall'alto verso il basso, quindi
  il livello sarà più alto più sarà in basso.}.
\begin{figure}[ht!]
  \centering
  \usetikzlibrary{graphs}

% Node styles
\tikzset{
% Two node styles for game trees: solid and hollow
solid node/.style={circle,draw,inner sep=2,fill=black},
hollow node/.style={circle,draw,inner sep=2},
rect hollow node/.style={rectangle,draw,inner sep=2},
rect solid node/.style={rectangle,draw, inner sep=2, fill=gray}
}

\begin{tikzpicture} [scale=1,font=\footnotesize]
\tikzstyle{level 1}=[sibling distance=25mm] 
\tikzstyle{level 2}=[sibling distance=7mm] 
\tikzstyle{level 3}=[sibling distance=6mm] 
%radice
\node(0)[hollow node]{}
child{node(1)[hollow node]{}
child{node(9)[rect hollow node]{}}
child{node(10)[hollow node]{}
child{node(13)[rect hollow node]{}}
child{node(14)[rect solid node]{E}}
child{node(15)[rect solid node]{C}}
child{node(16)[rect hollow node]{}}}
child{node(17)[rect hollow node]{}}
child{node(12)[rect hollow node]{}}
}
child{node(2)[hollow node]{}
child{node(5)[rect hollow node]{}}
child{node(6)[rect hollow node]{}}
child{node(7)[rect solid node]{A}}
child{node(8)[rect hollow node]{}}}
child{node(3)[rect hollow node]{}}
child{node(4)[hollow node]{}
child{node(14)[rect hollow node]{}}
child{node(15)[rect solid node]{B}}
child{node(16)[rect hollow node]{}}
child{node(17)[rect solid node]{D}}};

\end{tikzpicture}
  \caption{Quad-tree}
  \label{fig:qtree2}
\end{figure}
\begin{itemize}
\item Deriva da una modifica della T-pyramid
\item Non crea figli per ogni nodo ma solo quando i figli sono diversi
\end{itemize}

\subsection{Histogram (Istogramma)}
\label{sec:histogram}
L'istogramma nel processo di elaborazione dell'immagine, viene utilizzato
per verificare la presenta delle corrispettive tonalità di colore e
anche la sua frequenza o di scale di grigio in base alla situazione.
Un altro caso di utilizzo è la verifica del contrasto\footnote{Il
  contrasto è la differenza tra una porzioni d'immagine e il suo
sfondo, più esso è altro maggiore la parte in questione sarà risaltata}
di una immagine.

\end{document}

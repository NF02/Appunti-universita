\chapter{Introduzione}
\label{chap:Intro}

\section{Argomenti}
\label{sec:argomento}
Prendendo un immagine come una funzione $f(x,y)$, quindi sul fronte di
sviluppo sarà possibile costruire funzioni e soluzioni per poter
lavorare sulle stesse.

\section{Cosa bisogna sapere sulla rapresentazione raster delle immagini}
\label{sec:raster}

Le immagini raster o a rappresentazioni di pixel, sono la tipoligia di
immagine più utilizzate per la rappresentazione e l'elaborazione grafica,
visto che comunque nel contesto comune si rappresentano delle situazioni
più o mano complesse tipicamente di applicazione reale. Per rendere
efficiente la matrice si usano due sistemi:
\begin{itemize}
\item \textbf{RGB}: Red, Green, Blue (\textit{{\color{red}Rosso}, {\color{green}Verde} e {\color{blue}blue}}), con questo
  sistema è possibile costruire qualunque gradiente colore in modo
  estremamente preciso (valori compreso tra 0 e 255 per tre valori\footnote{Un sistema di rappresentazione dei colori ad 8bit});
\item \textbf{Graylevel}: a Gradienti di grigio (valore compreso tra 0 a
  255 da nero a bianco).
\end{itemize}
\subsubsection{Colori RGB}
\label{sec:rgbcolor}
\begin{table}[ht!]
  \centering
  \begin{tabular}{lccc}
    \textbf{Colore}& \textbf{R} & \textbf{G} & \textbf{B}\\\hline
    Nero & 0 & 0 & 0\\
    Bianco & 255 & 255 & 255\\
    {\color{red}Rosso} & 255 & 0 & 0\\
    {\color{yellow}Giallo} & 255 & 255 & 0\\
    {\color{gray}Grigio} & 127 & 127 & 127\\\hline
  \end{tabular}
  \caption{Colori RGB}
  \label{tab:rgb}
\end{table}
E visto che si tratta di un sistema di rappresentazione a 8 bit per 3
colonne, quindi:
\begin{equation*}
  2^{3\cdot 8} =2^{24} = 16.777.216 \text{ colori}
\end{equation*}
Una gamma cromatica abbastanza sostanziosa da poter rappresentare in modo
sufficientemente fedele qualunque oggetto reale.
\begin{figure}[ht!]
  \centering
  \resizebox{8cm}{!}{
  % RGB color mixing
% Author: Henrik Skov Midtiby <http://midtiby.blogspot.com/>

\begin{tikzpicture}
% Create the background in the circle, by drawing several slices
% each with a constant color given by the angle (which is converted
% to a color usin the hue, saturation and brightness color space).
\foreach \x in {0,0.0111,...,1} {
	\definecolor{currentcolor}{hsb}{\x, 1, 1}
	\draw[draw=none, fill=currentcolor]
		(-360*\x+88:2) -- (-360*\x+88:3.8)
		-- (-360*\x+92:3.8) -- (-360*\x+92:2) -- cycle;
}

% On top of the background draw three spotlights of the primary colors
% red, green and blue (they are primary in an additive colorspace where
% light are mixed)
\draw [draw=none, fill=red] (90:1.5) circle (2cm);
\draw [draw=none, fill=green] (-30:1.5) circle (2cm);
\draw [draw=none, fill=blue] (210:1.5) circle (2cm);

% Draw areas where two of the three primary colors are overlapping.
% These areas are the secondary colors yellow, cyan and magenta.
\begin{scope} % red + green = yellow
	\clip (90:1.5) circle(2cm);
	\draw [draw=none, fill=yellow] (-30:1.5) circle (2cm);
\end{scope} % blue + red = magenta
\begin{scope}
	\clip (210:1.5) circle(2cm);
	\draw [draw=none, fill=magenta] (90:1.5) circle (2cm);
\end{scope}
\begin{scope} % green + blue = cyan
	\clip (-30:1.5) circle(2cm);
	\draw [draw=none, fill=cyan] (210:1.5) circle (2cm);
\end{scope}

% Draw the center area which consists of all the primary colors.
\begin{scope} % red + green + blue = white
	\clip (90:1.5) circle(2cm);
	\clip (210:1.5) circle(2cm);
	\draw [draw=none, fill=white] (-30:1.5) circle (2cm);	
\end{scope}

% Draw a circle with markings along the perimeter, indicating which angles
% the hue function connects to certain colors.
\draw (0, 0) circle (3.9cm);
\foreach \x  in {0, 30, ..., 330}
	\draw (-\x+90:3.8) -- (-\x+90:4.0) (-\x+90:4.4) node {$\x^\circ$};

% Add labels with names of the primary and secondary colors.
\foreach \x/\text in {0/red, 60/yellow, 120/green, 180/cyan, 240/blue, 300/magenta}
	\draw (-\x+90:5.5) node {\text};
\end{tikzpicture}}
  \caption{Scala di colori RGB}
  \label{fig:rgb}
\end{figure}
\subsubsection{Altri per rappresentare il colore}
\label{sec:altcolorr}

Oltre al sistema di rappresentazione del colore RGB esistoo altri sistemi
di codifica, tra i quali:
\begin{itemize}
\item \textbf{HSL}: Hue, Saturazione, Luce;
\item \textbf{HSV}: Hue, Saturazione, Valore;
\item \textbf{CMYK}: Ciano, Magenta, Giallo, Nero (\textit{Sistema di
    stampa}).
\end{itemize}
\begin{nota}
  Nel caso del sistema CMYK quando tutti i colori vengono sommati si
  ottiene il nero, questo fu fatto per un ottica produttiva delle
  stampanti, la scelta di colori ciano, magenta e giallo insieme
  fanno il nero, motivo per la quale molti fotocopiatori riescono
  a lavorare anche con il toner nero quasi esaurito.
\end{nota}
\subsection{I pixel}
\label{sec:pixel}

I \textit{pixel} sono il sistema di rappresentazione dell'immagine,
rappresentano l'informazione unitaria, infatti, all'interno del singolo
pixel può esser contenuto solo un valore (\texttt{un colore}), essendo
quadrati per poter rappresentare un immagine che presenta delle forme
stondate con una buona qualità sarà necessaria una matrice di una
dimensione abbondante.
\subsection{La scalabilità di una immagine}
\label{sec:scala}
Uno dei punti che bisogna sempre considarare è proprio la scala e
la scalabilità di una immagine, infatti, non esiste una
dimensione\footnote{Dimensione della matrice di pixel} corratta per ogni
situazione, infatti, dipende tutto dal caso di utilizzo, non a caso è
necessaria fare una valutazione e considerare diversi fattori:
\begin{itemize}
\item \textbf{Dimensione}: il numero di pixel necessari al comporre
  l'immagine, ad esempio un icona \textit{64x64px} oppure
  \textit{128x128px} è perfetta per l'utilizzo desktop, ma magari non è
  idonea per una esposizione fotografica o per un analisi
  biometrica\footnote{un lettore di impronte digitale sta sul
    500x250px, ovviamente dipende dalla tipologia di sensore}.
\item \textbf{Peso}: Il peso dell'immagine è significativa pensando
  all'ambito, il fatto stesso che l'immagine abbia un canale alpha o
  meno cambia il peso nella codifica, assieme al numero di livelli che
  la compongono. 
\item \textbf{Gamma cromatica}: La gamma cromatica è la fedeltà nella
  trasposizione dei colori che un formato riesce ad avere rispetto al
  caso reale e dipende anche dalla paletta cromatica che la codicica di
  suddetto formato possieda. Questo parametro è utile soprattutto nel
  settore fotografico e anche per tutti quei casi in cui è necessario
  un alto livello di dettaglio.
\end{itemize}
\begin{nota}
  La questione della gamma cromatica dipende tanto dal formato, infatti,
  a lato informatico esistono diversi formati per la rappresentazione
  delle immagini, non compressi (\textbf{\color{red}raw}) a quelli
  compressi (\textbf{\color{blue}quelli Lossy e quelli Lossless}).
  \begin{description}
  \item[Lossy] Il contesto di utilizzo in cui la fedeltà e qualità della
    rappresentazione non è il punto saliente. Tipicamente cerca di
    rimuovere con criterio alcune sfumature cromatiche che l'occhio umano
    in primo achitto non nota. (Utile per la condivisione online)
  \item[Lossless] Anche se conpresso cerca di mantenere il più possibile
    la qualità e fedeltà all'immagine non compressa. (Utile anche in
    contesti di editing fotografico)
  \end{description}
  Sapendo questo sarà già possibile muoversi al meglio in questo mondo.
\end{nota}

\section{Pareidolia e illusioni visive}
\label{sec:pareidolia}
Per pareidolia o illusione subconscia, intendiamo il modo in cui il
nostro cervello identifichi all'interno di un oggetto, un immagine o
altro, un volto o una forma familiare, questo perché il cervello umano
è pensato per identificare quello che è famigliare per una questione
pratica ed evolutiva.

\section{Ambiti in cui può essere utilizzata l'elaborazione digitale
  delle immagini?}
\label{sec:ambito}
Quello dell'elaborazione digitale delle immagini può essere utilizzato in
tantissimi settori diversi, da quello biomedico a quello areonautico e
areospaziale, all'automotive, etc. Per questi motivi è sempre più
centrale nel mondo ingegneristico e militare l'adozione di tali sistemi.
Ma facendo un esempio concreto, un sensore di frenata che riconoscra il
pedone e consenta una frenata più efficace, in anticipo rispetto ai
comuni riflessi umani.

\documentclass{book}
\usepackage[a4paper,top=2.0cm,bottom=2.0cm,left=2.0cm,right=3.0cm]{geometry}

%\documentclass[pdftex,10pt,a4paper]{book}
%\usepackage[paperwidth=19cm,
%paperheight=26cm, outer=2cm, 
%top=1.5cm, bottom=1.5cm]{ geometry}

\usepackage[english,italian]{babel} %l'ultima lingua è quella che legge per i titoli
\usepackage[utf8]{inputenc}
\usepackage[T1]{fontenc,url}
\usepackage{titlesec}
\usepackage{easylist}
\usepackage{hanging}

\usepackage[pdftex,colorlinks]{hyperref}
\hypersetup{
	colorlinks=true,
	linkcolor=black,
	filecolor=magenta,
	urlcolor=cyan,
}
\usepackage{hypcap}
\usepackage{blindtext}
\usepackage{tipa}
\usepackage{epigraph}
\usepackage{enumerate}
\usepackage{longtable}
\usepackage{setspace}
\usepackage{verbatim}
\usepackage{graphicx}
\usepackage{amsmath}
\usepackage{pbox}
\usepackage{fancyhdr}
\usepackage{cancel}
\usepackage{tabularx}
\usepackage{booktabs}
\usepackage{multirow}
\usepackage{longtable}
\usepackage{tikz}
\usepackage{tikz-qtree}
\usepackage{subfig}
\usepackage{xcolor}
\usepackage{amssymb}
\usepackage{amsmath}
\usepackage{mathrsfs}
\usepackage{textcomp}
\usepackage{circuitikz}
\usepackage{pifont}
\usepackage{imakeidx}
\usepackage{verbatim}
\usepackage{dsfont}
\usepackage{listings}
\usepackage{color}
\usepackage{upgreek}
\usepackage{tasks}
\usepackage{exsheets}
\usepackage{pgfplots}
\usepackage{amsthm}


\usepackage{showframe}
\renewcommand\ShowFrameLinethickness{0.15pt}
%\renewcommand*\ShowFrameColor{\color{red}}

%\usepackage{showkeys} %serve per mostrare le etichette "tag" o target, va tolta per la versione definitiva;

\SetupExSheets[question]{type=exam}

\definecolor{mygreen}{rgb}{0,0.6,0}
\definecolor{mygray}{rgb}{0.5,0.5,0.5}
\definecolor{mymauve}{rgb}{0.58,0,0.82}

\lstset{ 
  backgroundcolor=\color{white},   % choose the background color; you must add \usepackage{color} or \usepackage{xcolor}; should come as last argument
  basicstyle=\footnotesize,        % the size of the fonts that are used for the code
  breakatwhitespace=false,         % sets if automatic breaks should only happen at whitespace
  breaklines=true,                 % sets automatic line breaking
  captionpos=b,                    % sets the caption-position to bottom
  commentstyle=\color{mygreen},    % comment style
  deletekeywords={...},            % if you want to delete keywords from the given language
  escapeinside={\%*}{*)},          % if you want to add LaTeX within your code
  extendedchars=true,              % lets you use non-ASCII characters; for 8-bits encodings only, does not work with UTF-8
  firstnumber=1000,                % start line enumeration with line 1000
  frame=single,	                   % adds a frame around the code
  keepspaces=true,                 % keeps spaces in text, useful for keeping indentation of code (possibly needs columns=flexible)
  keywordstyle=\color{blue},       % keyword style
  language=Octave,                 % the language of the code
  morekeywords={*,...},            % if you want to add more keywords to the set
  numbers=left,                    % where to put the line-numbers; possible values are (none, left, right)
  numbersep=5pt,                   % how far the line-numbers are from the code
  numberstyle=\tiny\color{mygray}, % the style that is used for the line-numbers
  rulecolor=\color{black},         % if not set, the frame-color may be changed on line-breaks within not-black text (e.g. comments (green here))
  showspaces=false,                % show spaces everywhere adding particular underscores; it overrides 'showstringspaces'
  showstringspaces=false,          % underline spaces within strings only
  showtabs=false,                  % show tabs within strings adding particular underscores
  stepnumber=2,                    % the step between two line-numbers. If it's 1, each line will be numbered
  stringstyle=\color{mymauve},     % string literal style
  tabsize=2,	                   % sets default tabsize to 2 spaces
  title=\lstname                   % show the filename of files included with \lstinputlisting; also try caption instead of title
}

% definizioni
\newtheorem{esercizio}{Esercizio}
\newtheorem{teorema}{Teorema}
\newtheorem{nota}{Nota}
\newtheorem{defi}{Definizione}
\newtheorem{esempio}{Esempio}
\newtheorem{svol}{Svolgimento}

\linespread{1.2} % l'interlinea

\frenchspacing

\newcommand{\abs}[1]{\lvert#1\rvert}

\usepackage{floatflt,epsfig}

\usepackage{multicol}
\newcommand\yellowbigsqcup[1][\displaystyle]{%
  \fboxrule0pt
  \ifx#1\textstyle\fboxsep-0.6pt\else\fboxsep-1.25pt\fi
  \mathrel{\fcolorbox{white}{yellow}{$#1\bigsqcup$}}}

\title{Appunti di Matematica}
\author{Nicola Ferru}
\date{}
\makeindex[columns=3, title=Alphabetical Index, intoc]

\begin{document}
%\maketitle
\begin{titlepage}
%	\tikz[remember picture,overlay] \node[opacity=0.50,inner sep=0pt, scale=0.8] at (current page.center){\includegraphics[width=\paperwidth,height=\paperheight]{./backround/backround.png}};
	\begin{center}
		
		% Upper part of the page
		\includegraphics[scale=.45]{./title/logo_CA.jpg}
		\\
		\vspace{1cm}
		\textsc{\large Università degli Studi di Cagliari}
		\vspace{1.2cm}
		
		\textsc{\Large DIEE}
		\vspace{0.5cm}
		
		\textsc{\large Dipartimento di Ingegneria e Architettura}
		\vspace{1.0cm}
		
		\textsc{\large Corso di Laurea {Triennale} in Ingegneria Elettrica industriale}\hspace{0.8cm}
		
		% Title
		\vspace{0.8cm}
		
		\Huge \doublespacing \bfseries \begin{spacing}{1}{ANALISI MATEMATICA 2}\end{spacing}
		\hfill
		\normalsize \itshape \begin{spacing}{1}{}\end{spacing}
		\hfill
		\normalsize\itshape \begin{spacing}{1}{edited by}\end{spacing}
		\hfill
		\Large\itshape  \begin{spacing}{1}{NICOLA FERRU}\end{spacing}
		\vspace{0.5cm}
		%		% Author and supervisor
		%		\begin{flushleft} \large
			%			\emph{Relatore:} \\
			%			Ch.mo Prof. Nome \textsc{Cognome}
			%		\end{flushleft}
		%		\vfill
		%		\begin{flushright} \large
			%			\emph{Laureando:}\\
			%			Nome \textsc{Cognome}
			%			
			%			\textsc{Matricola N. 1234567}
			%		\end{flushright}
		
		\hfill
		\vfill
		\Large \bfseries \begin{spacing}{1}{Unofficial Version}\end{spacing}
		\vspace{0.5cm}
		% Bottom of the page
		%{\small 4 Ottobre 2018 -  \today{}}
		{\small 2022 -  2023}
	\end{center}
	\clearpage
	\thispagestyle{empty}
	\vspace*{\fill}%
	{\centering [This page is intentionally left blank]\par}%
	\vspace{\fill}
\end{titlepage}


\tableofcontents
\listoftables
\listoffigures
\chapter{Introduzione all'Automazione}
\section{Sommario}
\begin{itemize}
    \item Automatica e Sistemi
    \item Problemi affrontati dall'Automatica
    \item Classificazione dei Sistemi
\end{itemize}
\section{Automatica e definizione di sistema}
\begin{defi}
  l'Automatica si occupa di studiare i sistemi e il loro controllo
  \begin{itemize}
    \item $\alpha\upsilon\tau o \mu\alpha\upsilon\upzeta$ in greco: ``che si muove da solo''
    \item \textit{automaton} in latino: ``macchina che opera da sola'';
  \end{itemize}
\end{defi}
\begin{nota}
  Il sistema per definizione del manuale dell'IEEE è un insieme di elementi che cooperano per
  svolgere una funzione altrimenti impossibile per ciascuno dei singoli componenti.
\end{nota}
Gli esempi più classici di sistema sono:
\begin{itemize}
    \item automibile, impianto termico, circuito elettrico;
    \item il corpo umano e l'ecosistema (per esempio Molentargius)
    \item un sistema economico (per esempio il mercato azionario)
    \item un programma di calcolatore
\end{itemize}
L'automatica ricerca leggi generali (dunque astratte) che possono essere usate in svariati dominii applicativi
\subsection{Problemi principali}
\begin{itemize}
    \item Modellazione;
    \item identificazione;
    \item Analisi;
    \item Controllo;
    \item Ottimizzazione;
    \item Diagnosi di guasto.
\end{itemize}
\section{Sistemi ad avanzamento temporale}
Nei sistemi ad avanzamento temporale ({\bf SAT}) il comportamento del sistema è descritto
da segnali ossia funzioni reali della variabile indipendente di tempo $t$.\\
Se la variabile tempo varia continuità si parla di {\bf SAT} a {\color{red} tempo continuo},
mentre, se essa prende valori in un insieme discreto si parla di {\bf SAT} {\color{red} a
  tempo discreto}.\\
Nel caso particolare dei sistemi a tempo discreto, è possibile identificare la sotto-classe
dei sistemi in cui anche i segnali in gioco, e non solo la variabile tempo, assumono valori
discreti.  
\begin{esempio}
  prendiamo l'esempio di un serbatoio che possiede due sensori di controllo, un per il troppo
  pieno e un per controllare il minimo. L'equazione che governa questo sistema è
  \begin{equation}
    \frac{d}{dt}V(t)=q_1(t)-q_2(t)
  \end{equation}
  Se le misure di volume e di portata sono disponibili solo ogni {\it T} unità di misura
  del tempo, si considerano le variabili a tempo discreto
  \begin{equation*}
    \begin{matrix}
      V(k)=V(kT), & q_1(k)=q_1(kT), & q_2(k)=q_2(kT), & k=0,1,\dots
    \end{matrix}
  \end{equation*}
  Posto $\Delta t=T$, approssimando la derivata con il rapporto incrementale
  \begin{equation*}
    \frac{d}{dt}V(t)\approx\frac{\Delta V}{\Delta t}=\frac{(k+1)-V(k)}{T}
  \end{equation*}
  e moltiplicando ambo i membri per $T$ la precedente equazione differenziale diventa una
  equazione alle differenze:
  \begin{equation*}
    V(k+1)-V(k)=Tq_1(k)-Tq_2(k)
  \end{equation*}
  il modello prevede 3 stati che variano in base al livello del liquido contenuto all'interno
  del serbatorio, infatti, il grafo che ne uscirà fuori è:
  \begin{figure}[!h]
    \begin{tikzpicture}[main/.style = {draw, circle}]
      \node[main] (1) {alto};
      \node[main] (2) [right of=1] {basso};
      \draw[->] (1) to [out=145,in=50,looseness=2.0] (2); 
    \end{tikzpicture}
  \end{figure} 
\end{esempio}
\chapter {Modelli matematici dei sistemi}
\section{Modello matematico}
Un \textbf{modello matematico} descrive una relazione che lega fra loro le grandezze che
descrivono il comportamento di un sistema.
\begin{defi}
  Il modello Ingresso-Uscita (IU) descrive il legame fra le uscita $y(t)$ (e le sue derivate) e
  l'ingresso $u(t)$ (e le derivate) sotto forma di una equazione differenziale. Alle volte viene
  utilizzato con la sua definizione inglese (Input/Output model) o modello I/O.
\end{defi}
\begin{defi}
  Il modello in Variabili di stato (VS) descrive come
  \begin{enumerate}
  \item L'evoluzione dello stato $x(t)\in \mathds{R}^n$ dipende dallo stato $x(t)\in \mathds{R}^n$ e dall'ingresso $u(t)$ ({\bf \color{red} equazione di stato}).
    \item l'uscita $y(t)$ dipende dallo stato di $x(t)$ e dall'ingresso u(t)
      (trasformazione di uscita)
    \end{enumerate}
  \end{defi}
  \section{Modello IU per un sistema SISO}
  \begin{defi}
    per sistema SISO si intende ``Single-input single-output system'', cioè un sistma con un solo
    ingresso e una sola uscita ed è meno complesso del sistema MIMO ``Multiple-input multiple-output system'' che invece possiende più interfaccie di input/output.
  \end{defi}
  
  \begin{equation*}
    h\left(\underbrace{y(t),\frac{d}{dt},\dots,\frac{d^n}{dt^n}y(t)}_{\text{\shortstack{uscita}}}, \underbrace{u(t),\frac{d}{d5}(t),\frac{d}{dt}u(t), t}_{\text{\shortstack{ingresso}}}\right)=0
  \end{equation*}
  n: grado massimo derivazione uscita = ordine del sistema\\
  m: grado massimo derivazione ingreso\\
  ovvero 
  \begin{equation*}
    h\left(\underbrace{y(t),\dot{y},\dots,y^{(n)}(t)}_{\text{\shortstack{uscita}}}, \underbrace{u(t),\dot{u(t)},\dots,u(t), t}_{\text{\shortstack{ingresso}}}\right)=0
  \end{equation*}

\printindex
\end{document}

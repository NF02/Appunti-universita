\documentclass{book}

\usepackage[utf8]{inputenc}
\usepackage{titlesec}
\usepackage{easylist}
\usepackage{hanging}
\usepackage{hyperref}
\usepackage[a4paper,top=2.0cm,bottom=2.0cm,left=2.0cm,right=2.0cm]{geometry}
\usepackage{blindtext}
\usepackage{tipa}
\usepackage{epigraph}
\usepackage{enumerate}
\usepackage{longtable}
\usepackage{setspace}
\usepackage{verbatim}
\usepackage[T1]{fontenc}
\usepackage{graphicx}
\usepackage[italian]{babel}
\usepackage{amsmath}
\usepackage{pbox}
\usepackage{fancyhdr}
\usepackage{cancel}
\usepackage{tabularx}
\usepackage{booktabs}
\usepackage{multirow}
\usepackage{longtable}
\usepackage{tikz}
\usepackage{qtree}
\usepackage{tikz-qtree}
\usepackage{subfig}
\usepackage{xcolor}
\usepackage{amssymb}
\usepackage{mathrsfs}
\usepackage{textcomp}
\usepackage{tasks}

\usepackage{amsthm}

\newtheorem{defi}{Definizione}
\linespread{1.2} % l'interlinea

\frenchspacing

\newcommand{\abs}[1]{\lvert#1\rvert}

\usepackage{floatflt,epsfig}

\usepackage{multicol}
\newcommand\yellowbigsqcup[1][\displaystyle]{%
  \fboxrule0pt
  \ifx#1\textstyle\fboxsep-0.6pt\else\fboxsep-1.25pt\fi
  \mathrel{\fcolorbox{white}{yellow}{$#1\bigsqcup$}}}

\title{Appunti fisica tecnica}
\author{Nicola Ferru}
\begin{document}
\maketitle
\chapter{Introduzione \label{intro}}
Lo scopo del corso è quello di introdurre e applicare i concetti necessari per arrivare a formulare
correttanente l'inquadramento termodinamico di un problema fisico in cui sono coinvolti scambi di calore, di
radiazione, di lavoro, di massa e materia. Tale formulazione deve avvenire in coerenza le approssimazioni e
relative stime, legate alla eventuali semplificazioni del caso in studio, e il fine è quello di stabilire
i rapporti quantitativi di causa ed effetto tra le entità di scambio, calore radiazione termica, lavoro
({\it in tutte le sue forme}) massa e materia e le coordinate termodinamiche ({\it che esprimono lo statto del
  sistema}). La conoscenza di tale legame, di causa ed effetto, permetterà di controllare o prevedere un
determinato processo o sarà utile semplicemente per conoscere le azioni necessarie affinché un sistema si
porti da uno stato iniziale (i) ad uno stato finale (f). 
\section{Il problema dell'inquadramento termico\label{pro.inq.term}}
La formulazione corretta dell'inquadramento termodinamico di un problema fisico è fontamentalmente basato su 4
concetti elementari e sul loro reciproco coordinamento. definiti come segue:
\begin{enumerate}
\item Sistema;
\item Ambiente circostante;
\item Entità di scambio;
\item coordinate termiche.
\end{enumerate}
\begin{defi}
  Il sistem è una porzione di materia su cui ricade il nostro interesse dal punto di vista termodinamico.
  Il sistema si trova in una regione di spazio che limita una qualsivoglia porzione di materia e che noi possiamo
  materialmente o concetualmente separare da tutto ciò che la circonda (un gas contenuto in un cilindro,
  oppure una particolare fluida, nella sua stessa individualità, circondata dal restante fluido a cui appartiene
  e in cui si trova immersa, un collettore solare che separa il fluido termovettore dell'ambiente esterno\dots)
  insomma qualsiasi regione di spazio che confina una porzione di materiale, nel momenbto in cui è oggetto di un
  interesse dal punto di vista termodinamico, individua SISTEMA termodinamico. Quindi il sistema è una
  qualsivoglia porzione di materiale che in un cetro istante $t_0$ occupa una certa regione di spazio e che può
  essere una separazione fisica oppure solo dentro la notra mente.  
\end{defi}
\begin{defi}
  L'Ambiente circostante è tutto ciò che non è sistema e che è esterno ad esso. Ma non è solo questo, orrorre
  prencisare che tutto ciò che non è sistema, per poter avere il suolo di {\it Ambiente Circostante} nei
  confronti del Sistema, deve avere anche un altro requisito. Deve essere in grado, in qualche modo e misura,
  di influinzzare/modificare direttamente lo stato termodinamico e il comportamento dello stesso. Questo
  significa che l'ambiente circostante è operativamente in grado di scambiare con il sistema deteminate grandezze
  che in seno all'inquadramento termodinamico vengono denominate ``Entità di scambio''. Le entità di cambio sono
  grandezze che possono essere colte e si manifestano all'atto dello scambio tra sistemi e ambiente circostante.
\end{defi}
\subsection{Entità di scambio}
\begin{tasks}
  \task Calore
  \task Radiazione terminca;
  \task Lavoro di tipo meccanico ed elettrico
  \task Massa e Materia.
\end{tasks}
Quindi una volta elencate queste grandezze quel'è l'ambiente circostante?\\
L'insieme di tutti i campi dell'universo, che scambiano con il sistema le entità
di scambio prima elencate, rappresenta {\it l'Ambiente Circostante.} L'insieme degli
altri corpi che non scambiano con gli altri sistemi nessuna entità di scambio allora
rapresenta, per il sistema in considerazione, il {\it resto dell'universo}.


\end{document}

\documentclass{article}

\usepackage[utf8]{inputenc}
\usepackage{titlesec}
\usepackage{easylist}
\usepackage{hanging}
\usepackage{hyperref}
\usepackage[a4paper,top=2.0cm,bottom=2.0cm,left=2.0cm,right=3.0cm]{geometry}
\usepackage{blindtext}
\usepackage{tipa}
\usepackage{epigraph}
\usepackage{enumerate}
\usepackage{longtable}
\usepackage{setspace}
\usepackage{verbatim}
\usepackage[T1]{fontenc}
\usepackage{graphicx}
\usepackage[italian]{babel}
\usepackage{amsmath}
\usepackage{pbox}
\usepackage{fancyhdr}
\usepackage{cancel}
\usepackage{tabularx}
\usepackage{booktabs}
\usepackage{multirow}
\usepackage{longtable}
\usepackage{tikz}
\usepackage{tikz-qtree}
\usepackage{subfig}
\usepackage{xcolor}
\usepackage{amssymb}
\usepackage{mathrsfs}
\usepackage{textcomp}
\usepackage{circuitikz}
\usepackage{pifont}

\usepackage{listings}
\usepackage{color}

\definecolor{mygreen}{rgb}{0,0.6,0}
\definecolor{mygray}{rgb}{0.5,0.5,0.5}
\definecolor{mymauve}{rgb}{0.58,0,0.82}

\lstset{ 
  backgroundcolor=\color{white},   % choose the background color; you must add \usepackage{color} or \usepackage{xcolor}; should come as last argument
  basicstyle=\footnotesize,        % the size of the fonts that are used for the code
  breakatwhitespace=false,         % sets if automatic breaks should only happen at whitespace
  breaklines=true,                 % sets automatic line breaking
  captionpos=b,                    % sets the caption-position to bottom
  commentstyle=\color{mygreen},    % comment style
  deletekeywords={...},            % if you want to delete keywords from the given language
  escapeinside={\%*}{*)},          % if you want to add LaTeX within your code
  extendedchars=true,              % lets you use non-ASCII characters; for 8-bits encodings only, does not work with UTF-8
  firstnumber=1000,                % start line enumeration with line 1000
  frame=single,	                   % adds a frame around the code
  keepspaces=true,                 % keeps spaces in text, useful for keeping indentation of code (possibly needs columns=flexible)
  keywordstyle=\color{blue},       % keyword style
  language=Octave,                 % the language of the code
  morekeywords={*,...},            % if you want to add more keywords to the set
  numbers=left,                    % where to put the line-numbers; possible values are (none, left, right)
  numbersep=5pt,                   % how far the line-numbers are from the code
  numberstyle=\tiny\color{mygray}, % the style that is used for the line-numbers
  rulecolor=\color{black},         % if not set, the frame-color may be changed on line-breaks within not-black text (e.g. comments (green here))
  showspaces=false,                % show spaces everywhere adding particular underscores; it overrides 'showstringspaces'
  showstringspaces=false,          % underline spaces within strings only
  showtabs=false,                  % show tabs within strings adding particular underscores
  stepnumber=2,                    % the step between two line-numbers. If it's 1, each line will be numbered
  stringstyle=\color{mymauve},     % string literal style
  tabsize=2,	                   % sets default tabsize to 2 spaces
  title=\lstname                   % show the filename of files included with \lstinputlisting; also try caption instead of title
}

\linespread{1.5} % l'interlinea

\frenchspacing

\newcommand{\abs}[1]{\lvert#1\rvert}

\usepackage{floatflt,epsfig}

\usepackage{multicol}
\newcommand\yellowbigsqcup[1][\displaystyle]{%
  \fboxrule0pt
  \ifx#1\textstyle\fboxsep-0.6pt\else\fboxsep-1.25pt\fi
  \mathrel{\fcolorbox{white}{yellow}{$#1\bigsqcup$}}}

\title{Appunti Fisica 1}
\author{Nicola Ferru}
\date{}
\begin{document}
\maketitle
\section{moto rettilineo uniformemente accelerato}
Moto rettilineo uniformemente accelerato. La definizione di moto rettilineo uniformemente accelerato è: il moto di un corpo con accelerazione costante lungo una traiettoria retta sempre nella stessa direzione e identico verso.
\begin{multicols}{3} 
$V_{S_0}=30,0m/s$\\
$V_F=5,00m/s$\\
$A_s=-2.00m/s$\\
$X_{F_0}=I_{SF}=155,5m$\\
$X_s(t)=X_{S_0}+X_{S_0}t+\frac{1}{2}A_st^2$\\
$X_s(t)=V_{S_0}+\frac{1}{2}A_st^2$\\
$X_F(t)=X_{F_0}+V_{F_0}t$
\end{multicols}
\begin{multicols}{2} 
 \begin{tikzpicture}[domain=-2:6] 
    \draw[very thin,color=gray] (-2.1,-1.1) grid (5.9,3.9);
    \draw[->] (-2.2,0) -- (6.2,0) node[right] {$t$}; 
    \draw[->] (0,-1.2) -- (0,4.2) node[above] {$x$};
  \end{tikzpicture}\\
  $(x_f(t)-x_{f_0})=X_{f}(t_0)$\\
  $X_s(t_c)=X_f(t_0)$\\
  $V_st_e+\frac{1}{2}A_st^2c=X_{F_0}+V_{F_0}+V_{F_0}+V_{F_0}tc$
\end{multicols}
$\alpha{x^2}+\beta{x}+\gamma=0$\\
$x=\frac{-\beta\pm\sqrt{\beta-\gamma}}{2\alpha}$ $\Delta\geq 0$\\
$\tilde{x^2}+\tilde{2\beta x}+\gamma=0$\\
$x=\sqrt{\tilde{\beta}}$\\
$\frac{1}{2}(V_{s_0}-V_{F_0})T_c-X_{F0}=0$\\
\section{I vettori}
\subsection{proiezione dei vettori prodotto scalare}
 \begin{tikzpicture}[domain=-2:6] 
    \draw[very thin,color=gray] (-2.1,-1.1) grid (5.9,3.9);
    \draw[->] (-2.2,0) -- (6.2,0) node[right] {$t$}; 
    \draw[->] (0,-1.2) -- (0,4.2) node[above] {$x$};
  \end{tikzpicture}\\
\begin{multicols}{3} 
$L*L=1$\\
$J*J=1$\\
$\overrightarrow{a}*\overrightarrow{i}=a_x$\\
$\overrightarrow{a}*$\\
$\overrightarrow{a}=\overrightarrow{a}_x\overrightarrow{I}+a_y\overrightarrow{J}$\\
$ax=\overrightarrow{a}*\overrightarrow{J}=||a||*||\overrightarrow{J}||\cos{\phi}=||\overrightarrow{a}||*\cos{\phi}$\\
$\overrightarrow{a}=a_x\overrightarrow{L}+a_y\overrightarrow{J}$\\
$\overrightarrow{b}=b_x\overrightarrow{L}+b_y\overrightarrow{J}$\\
$\overrightarrow{a}*\overrightarrow{b}=(a_x\overrightarrow{J}+a_y\overrightarrow{J})*(b_x\overrightarrow{J}+b_y\overrightarrow{J})$\\
$\overrightarrow{a}*\overrightarrow{b}=a_x*b_x+a_yb_y$\\
$||\overrightarrow{a}||=a_{x^*2}+a_{y^2}=\overrightarrow{a}*\overrightarrow{a}$\\
$\overrightarrow{r(t)}=\overrightarrow{r_0}+V_0t+\frac{1}{2}\overrightarrow{y}t^2$\\
$\overrightarrow{r}*\overrightarrow{J}=y=\overrightarrow{r}*\overrightarrow{J}+\overrightarrow{V_0}*\overrightarrow{J}$\\
$\cos{\frac{\pi}{2}*\phi}=\sin{\phi}$\\
$x=x_0+V_xt$\\
$y=y_0+V_0t-\frac{1}{2}gt^2$\\
\end{multicols} 
\subsubsection{moto balistico}
$x=x_0+V_{0x}t$\\
$y=y_0+V_{0y}t-\frac{1}{2}gt^2$\\
$x=0$\\
$y=h$\\
$V_{0y}=\overrightarrow{V_0}*\overrightarrow{J}=||\overrightarrow{V}||*||\overrightarrow{J}||$\\
$h=\frac{1}{2}gt^2$\\
$\tag$

\end{document}
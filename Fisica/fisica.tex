\documentclass{book}

\usepackage[utf8]{inputenc}
\usepackage{titlesec}
\usepackage{easylist}
\usepackage{hanging}
\usepackage{hyperref}
\usepackage[a4paper,top=2.0cm,bottom=2.0cm,left=2.0cm,right=3.0cm]{geometry}
\usepackage{blindtext}
\usepackage{tipa}
\usepackage{epigraph}
\usepackage{enumerate}
\usepackage{longtable}
\usepackage{setspace}
\usepackage{verbatim}
\usepackage[T1]{fontenc}
\usepackage{graphicx}
\usepackage[italian]{babel}
\usepackage{amsmath}
\usepackage{pbox}
\usepackage{fancyhdr}
\usepackage{cancel}
\usepackage{tabularx}
\usepackage{booktabs}
\usepackage{multirow}
\usepackage{longtable}
\usepackage{tikz}
\usepackage{tikz-qtree}
\usepackage{subfig}
\usepackage{xcolor}
\usepackage{amssymb}
\usepackage{mathrsfs}
\usepackage{textcomp}
\usepackage{circuitikz}
\usepackage{pifont}
\usepackage{mathtools}
\usepackage{amsmath}
\usepackage{listings}
\usepackage{color}

\definecolor{mygreen}{rgb}{0,0.6,0}
\definecolor{mygray}{rgb}{0.5,0.5,0.5}
\definecolor{mymauve}{rgb}{0.58,0,0.82}

\lstset{ 
  backgroundcolor=\color{white},   % choose the background color; you must add \usepackage{color} or \usepackage{xcolor}; should come as last argument
  basicstyle=\footnotesize,        % the size of the fonts that are used for the code
  breakatwhitespace=false,         % sets if automatic breaks should only happen at whitespace
  breaklines=true,                 % sets automatic line breaking
  captionpos=b,                    % sets the caption-position to bottom
  commentstyle=\color{mygreen},    % comment style
  deletekeywords={...},            % if you want to delete keywords from the given language
  escapeinside={\%*}{*)},          % if you want to add LaTeX within your code
  extendedchars=true,              % lets you use non-ASCII characters; for 8-bits encodings only, does not work with UTF-8
  firstnumber=1000,                % start line enumeration with line 1000
  frame=single,	                   % adds a frame around the code
  keepspaces=true,                 % keeps spaces in text, useful for keeping indentation of code (possibly needs columns=flexible)
  keywordstyle=\color{blue},       % keyword style
  language=Octave,                 % the language of the code
  morekeywords={*,...},            % if you want to add more keywords to the set
  numbers=left,                    % where to put the line-numbers; possible values are (none, left, right)
  numbersep=5pt,                   % how far the line-numbers are from the code
  numberstyle=\tiny\color{mygray}, % the style that is used for the line-numbers
  rulecolor=\color{black},         % if not set, the frame-color may be changed on line-breaks within not-black text (e.g. comments (green here))
  showspaces=false,                % show spaces everywhere adding particular underscores; it overrides 'showstringspaces'
  showstringspaces=false,          % underline spaces within strings only
  showtabs=false,                  % show tabs within strings adding particular underscores
  stepnumber=2,                    % the step between two line-numbers. If it's 1, each line will be numbered
  stringstyle=\color{mymauve},     % string literal style
  tabsize=2,	                   % sets default tabsize to 2 spaces
  title=\lstname                   % show the filename of files included with \lstinputlisting; also try caption instead of title
}

\linespread{1.5} % l'interlinea

\frenchspacing

\newcommand{\abs}[1]{\lvert#1\rvert}

\usepackage{floatflt,epsfig}

\usepackage{multicol}
\newcommand\yellowbigsqcup[1][\displaystyle]{%
  \fboxrule0pt
  \ifx#1\textstyle\fboxsep-0.6pt\else\fboxsep-1.25pt\fi
  \mathrel{\fcolorbox{white}{yellow}{$#1\bigsqcup$}}}

\title{Appunti Fisica}
\author{Nicola Ferru}
\date{}
\begin{document}
\maketitle
\tableofcontents
\listoftables
\listoffigures
% pagine dedicate
\section{Premesse\dots}

In questo repository, inoltre,  sono disponibili le dimostrazioni grafiche realizzate
con \textit{Geogebra}; consiglio a tutte le persone che usufruiranno di questo lavoro, di dare un occhiata alle dimostrazioni grafiche e stare attenti,  in quanto nel tempo potranno  essere presenti delle modifiche, cosi da apportare miglioramenti al contenuto degli stessi appunti.  Solitamente il lavoro di revisione viene fatto tre/quattro volte alla settimana perché sono in piena fase di sviluppo.  Ricordo a tutti che essendo un progetto volontario ci potrebbero essere dei rallentamenti per cause di ordine superiore e quindi
potrebbero esserci meno modifiche del solito oppure essere presenti degli errori.  Chiedo pertanto  la cortesia a voi lettori di contattarmi per apportare eventuali correzioni .  Tengo a precisare che tutto il progetto è puramente open source, pertanto vengono resi disponibili i sorgenti dei file LaTex  insieme ai PDF compilati.

\begin{center}
	Cordiali saluti
\end{center}
\newpage

\section{Simboli}
\begin{multicols}{3}
	$\in$ Appartiene\\
	$\notin$ Non appartiene\\
	$\exists$ Esiste\\
	$\exists !$ Esiste unico\\
	$\subset$ Contenuto strettamente\\
	$\subseteq$ Contenuto\\
	$\supset$ Contenuto strettamente\\
	$\supseteq$ Contiene\\
	$\Rightarrow$ Implica\\
	$\Longleftrightarrow$ Se e solo se\\
	$\neq$ Diverso\\
	$\forall$ Per ogni\\
	$\ni :$ Tale che\\
	$\leq$ Minore o uguale\\
	$\geq$ Maggiore o uguale\\
	$\alpha$ alfa\\
	$\beta$ beta\\
	$\gamma$ gamma\\
	$\Gamma$ Gamma\\
	$\delta,\Delta$ delta\\
	$\epsilon$ epsilon\\
	$\sigma,\Sigma$ sigma\\
	$\rho$ rho
\end{multicols}

\part{fisica 1}
\input{pag/unità di misura.tex}
\chapter{I moti}
\section{moto rettilineo uniformemente accelerato}
Moto rettilineo uniformemente accelerato. La definizione di moto rettilineo uniformemente accelerato è: il moto di un corpo con accelerazione costante lungo una traiettoria retta sempre nella stessa direzione e identico verso.
\subsection{Esercitazione 1}
Si lascia cadere un sasso in un pozzo. Se il tonfo nell'acqua viene percepito con un ritardo di 2,40s a quele distanza dell'imboccatura del pozzo si trova la superficie del l'acqua? la velocità del suono nell'aria è 336m/s. E se non teniamo conto del tempo cui il suono impiaga ad arrivare fino a noi, che errore percentuale commettiamo? Nel calcolare la profondità a cui si trova acqua?
\begin{equation*}
	V_{s}=336m/s\text{ } \Delta t_{tot}=2,40s \text{ legge oraria del sasso che cade}
\end{equation*}
\begin{multicols}{2}
	\begin{equation*}
		y(t)=y_0+V_0t+\frac{1}{2}at^2
	\end{equation*}
	\begin{equation*}
		y(t)=h-\frac{1}{2}gt^2
	\end{equation*}
	\begin{equation*}
		y_0=h
	\end{equation*}
	\begin{equation*}
		V_0=0
	\end{equation*}
	\begin{equation*}
		a=-g=9,81m/s^2
	\end{equation*}
\end{multicols}
\begin{equation*}
	\Delta t=t_{caduta}-t_{suono}
\end{equation*}
\begin{multicols}{2}
	\begin{equation*}
		h=V_0t_{suono}
	\end{equation*}
	\begin{equation*}
		t_{suono}=\frac{h}{V_s}
	\end{equation*}
	\begin{equation*}
		\begin{cases}
			y(t_{(caduta)}=0)=0 \\
			h-\frac{1}{2}gtc^2=0
		\end{cases}
	\end{equation*}
	\begin{equation*}
		tc=\sqrt{\frac{2h}{g}}
	\end{equation*}
\end{multicols}
\begin{equation*}
	\Delta t_{tot}=\sqrt{2h}{g}+\frac{h}{V_0}\to \frac{\sqrt{2h}}{g}=\Delta t_{tot} - \frac{h}{V_s}\to \frac{2h}{g}\to \frac{2h}{g}=\bigg(\Delta t-\frac{h}{V_s}\bigg)^2
\end{equation*}
\begin{equation*}
	\Rightarrow (\Delta t)^2+\frac{h^2}{V_{s^2}} -\frac{2h}{V_s}\Delta t=\frac{2h}{g}\to (\Delta t)^2-2 \bigg(\frac{\Delta t}{V_s}+\frac{1}{g}\bigg) h+\frac{h^2}{V_{s^2}} =0 
\end{equation*}
\paragraph{Forma ridotta\label{forma ridotta es.1.1}}
\begin{equation*}
	h^2-V_{s^2}\bigg(\frac{\Delta t+st}{V_s}+\frac{1}{g}\bigg)h+V_{s^2}\Delta t_{tot}^2=0
\end{equation*}
\begin{equation*}
	h=V_{s^2}\bigg(\frac{\Delta t_{tot}}{V_0}+\frac{1}{g}\bigg)\pm \sqrt{V_{s^2}\bigg(\frac{\Delta}{V_s}+\frac{1}{g}\bigg)^2-V^2_s\Delta t^2_{tot}}
\end{equation*}
\begin{equation*}
	h=V_{s^2}\bigg(\frac{\Delta t_{tot}}{V_s}+\frac{1}{g}\bigg)-\sqrt{V_{s^2}\bigg(\frac{\Delta t}{V_s}+\frac{1}{g}\bigg)^2-V_{s^2}\Delta t^2}\Rightarrow \Delta t_{tot}-\frac{h}{V_s}>0
\end{equation*}
\subsection{Esercitazione 2}
Due nuotatori, Alan e Beth, partono insieme dallo stesso punto dalla riva di un fiume molto largo, che scorre con velocità V. Entrambi si muovono alla stessa velocità C (C>V) relativa all'acqua. Alan nuota per uno distanza L seguendo la corrente e poi torna indietro coprendo la stessa distanza. Beth, invece, nuota in modo da avanzare in direzione perpendicolare alla sponda del fiume: anche lei copre una distanza L e poi torna indietro. Tutti e due tornano quindi al punto di partenza. chi arriva per primo?
È possibile intuire la riposta.
\begin{itemize}
	\item  C=velocità rispetto all'acqua
\end{itemize}
\begin{description}
	\item[ ] Alan $\to$ percorre L in direzione parallela alla velocità dell'acqua;
	\item[ ] Beth $\to$ percorre L perpendicolare, deve tener conto della forza dell'acqua.
\end{description}
\subsubsection{Caso Alan andata}
\begin{multicols}{2}
	\begin{equation*}
		\overrightarrow{V}=\text{velocità dell'acqua}
	\end{equation*}
	\begin{equation*}
		c=\text{velocità Alan rispetto all'acqua stessa di r}
	\end{equation*}
	Alan si muove seguendo l'acqua\\
	Andata:
	\begin{equation*}
		V=\overrightarrow{c}+\overrightarrow{V}
	\end{equation*}
	\begin{equation*}
		\overrightarrow{c}*\overrightarrow{i}=cx=||\overrightarrow{c}||||\overrightarrow{c}||\cos o=||\overrightarrow{c}||
	\end{equation*}
	\begin{equation*}
		\overrightarrow{V}*\overrightarrow{i}=||\overrightarrow{V}||=V
	\end{equation*}
\end{multicols}
La velocità dell'acqua nel riferimento mobile è nulla, visto la terra, Alan si muove di velocità $\overrightarrow{V}^{(t)}_{Alan}=\overrightarrow{c}+\overrightarrow{V}$
\subsubsection{Caso Alan ritorno}
in questo caso con 

\chapter{Modelli atomici}
\section{Modello atomico di Bohr-Sommerfeld}
Il modello atomico proposto da Niels Bohr nel 1913, successivamente ampliato da Arnold Sommerfeld nel 1916, è la più famosa applicazione della quantizzazione dell'energia che, insieme alle spiegazioni teoriche sulla radiazione del corpo nero, sull'effetto fotoelettrico e sullo scattering Compton, e all'equazione di Schrödinger, costituiscono la base della meccanica quantistica.\\
Il modello, proposto inizialmente per l'atomo di idrogeno, riusciva anche a spiegare, entro il margine di errore statistico, l'esistenza dello spettro sperimentale. Bohr presenta così un modello dell'atomo, facendo intuire che gli elettroni si muovono su degli orbitali. \textit{Questo modello viene ancora utilizzato nello studio dei Semiconduttori.} 
\begin{center}
	By \href{https://it.wikipedia.org/wiki/Modello_atomico_di_Bohr}{Wikipedia}
\end{center}

\end{document}

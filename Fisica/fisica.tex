\documentclass{book}

\usepackage[utf8]{inputenc}
\usepackage{titlesec}
\usepackage{easylist}
\usepackage{hanging}
\usepackage{hyperref}
\usepackage[a4paper,top=2.0cm,bottom=2.0cm,left=2.0cm,right=3.0cm]{geometry}
\usepackage{blindtext}
\usepackage{tipa}
\usepackage{epigraph}
\usepackage{enumerate}
\usepackage{longtable}
\usepackage{setspace}
\usepackage{verbatim}
\usepackage[T1]{fontenc}
\usepackage{graphicx}
\usepackage[italian]{babel}
\usepackage{amsmath}
\usepackage{pbox}
\usepackage{fancyhdr}
\usepackage{cancel}
\usepackage{tabularx}
\usepackage{booktabs}
\usepackage{multirow}
\usepackage{longtable}
\usepackage{tikz}
\usepackage{tikz-qtree}
\usepackage{subfig}
\usepackage{xcolor}
\usepackage{amssymb}
\usepackage{mathrsfs}
\usepackage{textcomp}
\usepackage{circuitikz}
\usepackage{pifont}
\usepackage{mathtools}
\usepackage{amsmath}
\usepackage{listings}
\usepackage{color}
\usepackage{tasks}

\definecolor{mygreen}{rgb}{0,0.6,0}
\definecolor{mygray}{rgb}{0.5,0.5,0.5}
\definecolor{mymauve}{rgb}{0.58,0,0.82}

\lstset{ 
  backgroundcolor=\color{white},   % choose the background color; you must add \usepackage{color} or \usepackage{xcolor}; should come as last argument
  basicstyle=\footnotesize,        % the size of the fonts that are used for the code
  breakatwhitespace=false,         % sets if automatic breaks should only happen at whitespace
  breaklines=true,                 % sets automatic line breaking
  captionpos=b,                    % sets the caption-position to bottom
  commentstyle=\color{mygreen},    % comment style
  deletekeywords={...},            % if you want to delete keywords from the given language
  escapeinside={\%*}{*)},          % if you want to add LaTeX within your code
  extendedchars=true,              % lets you use non-ASCII characters; for 8-bits encodings only, does not work with UTF-8
  firstnumber=1000,                % start line enumeration with line 1000
  frame=single,	                   % adds a frame around the code
  keepspaces=true,                 % keeps spaces in text, useful for keeping indentation of code (possibly needs columns=flexible)
  keywordstyle=\color{blue},       % keyword style
  language=Octave,                 % the language of the code
  morekeywords={*,...},            % if you want to add more keywords to the set
  numbers=left,                    % where to put the line-numbers; possible values are (none, left, right)
  numbersep=5pt,                   % how far the line-numbers are from the code
  numberstyle=\tiny\color{mygray}, % the style that is used for the line-numbers
  rulecolor=\color{black},         % if not set, the frame-color may be changed on line-breaks within not-black text (e.g. comments (green here))
  showspaces=false,                % show spaces everywhere adding particular underscores; it overrides 'showstringspaces'
  showstringspaces=false,          % underline spaces within strings only
  showtabs=false,                  % show tabs within strings adding particular underscores
  stepnumber=2,                    % the step between two line-numbers. If it's 1, each line will be numbered
  stringstyle=\color{mymauve},     % string literal style
  tabsize=2,	                   % sets default tabsize to 2 spaces
  title=\lstname                   % show the filename of files included with \lstinputlisting; also try caption instead of title
}

\linespread{1.5} % l'interlinea

\frenchspacing

\newcommand{\abs}[1]{\lvert#1\rvert}

\usepackage{floatflt,epsfig}

\usepackage{multicol}
\newcommand\yellowbigsqcup[1][\displaystyle]{%
  \fboxrule0pt
  \ifx#1\textstyle\fboxsep-0.6pt\else\fboxsep-1.25pt\fi
  \mathrel{\fcolorbox{white}{yellow}{$#1\bigsqcup$}}}

\title{Appunti Fisica}
\author{Nicola Ferru}
\date{}
\begin{document}
\maketitle
\tableofcontents
\listoftables
\listoffigures
% pagine dedicate
\section{Premesse\dots}

In questo repository, inoltre,  sono disponibili le dimostrazioni grafiche realizzate
con \textit{Geogebra}; consiglio a tutte le persone che usufruiranno di questo lavoro, di dare un occhiata alle dimostrazioni grafiche e stare attenti,  in quanto nel tempo potranno  essere presenti delle modifiche, cosi da apportare miglioramenti al contenuto degli stessi appunti.  Solitamente il lavoro di revisione viene fatto tre/quattro volte alla settimana perché sono in piena fase di sviluppo.  Ricordo a tutti che essendo un progetto volontario ci potrebbero essere dei rallentamenti per cause di ordine superiore e quindi
potrebbero esserci meno modifiche del solito oppure essere presenti degli errori.  Chiedo pertanto  la cortesia a voi lettori di contattarmi per apportare eventuali correzioni .  Tengo a precisare che tutto il progetto è puramente open source, pertanto vengono resi disponibili i sorgenti dei file LaTex  insieme ai PDF compilati.

\begin{center}
	Cordiali saluti
\end{center}
\newpage

\section{Simboli}
\begin{multicols}{3}
	$\in$ Appartiene\\
	$\notin$ Non appartiene\\
	$\exists$ Esiste\\
	$\exists !$ Esiste unico\\
	$\subset$ Contenuto strettamente\\
	$\subseteq$ Contenuto\\
	$\supset$ Contenuto strettamente\\
	$\supseteq$ Contiene\\
	$\Rightarrow$ Implica\\
	$\Longleftrightarrow$ Se e solo se\\
	$\neq$ Diverso\\
	$\forall$ Per ogni\\
	$\ni :$ Tale che\\
	$\leq$ Minore o uguale\\
	$\geq$ Maggiore o uguale\\
	$\alpha$ alfa\\
	$\beta$ beta\\
	$\gamma$ gamma\\
	$\Gamma$ Gamma\\
	$\delta,\Delta$ delta\\
	$\epsilon$ epsilon\\
	$\sigma,\Sigma$ sigma\\
	$\rho$ rho
\end{multicols}

\part{fisica 1}
\input{pag/unita di misura.tex}
\chapter{I moti}
\section{moto rettilineo uniformemente accelerato}
Moto rettilineo uniformemente accelerato. La definizione di moto rettilineo uniformemente accelerato è: il moto di un corpo con accelerazione costante lungo una traiettoria retta sempre nella stessa direzione e identico verso.
\subsection{Esercitazione 1}
Si lascia cadere un sasso in un pozzo. Se il tonfo nell'acqua viene percepito con un ritardo di 2,40s a quele distanza dell'imboccatura del pozzo si trova la superficie del l'acqua? la velocità del suono nell'aria è 336m/s. E se non teniamo conto del tempo cui il suono impiaga ad arrivare fino a noi, che errore percentuale commettiamo? Nel calcolare la profondità a cui si trova acqua?
\begin{equation*}
	V_{s}=336m/s\text{ } \Delta t_{tot}=2,40s \text{ legge oraria del sasso che cade}
\end{equation*}
\begin{multicols}{2}
	\begin{equation*}
		y(t)=y_0+V_0t+\frac{1}{2}at^2
	\end{equation*}
	\begin{equation*}
		y(t)=h-\frac{1}{2}gt^2
	\end{equation*}
	\begin{equation*}
		y_0=h
	\end{equation*}
	\begin{equation*}
		V_0=0
	\end{equation*}
	\begin{equation*}
		a=-g=9,81m/s^2
	\end{equation*}
\end{multicols}
\begin{equation*}
	\Delta t=t_{caduta}-t_{suono}
\end{equation*}
\begin{multicols}{2}
	\begin{equation*}
		h=V_0t_{suono}
	\end{equation*}
	\begin{equation*}
		t_{suono}=\frac{h}{V_s}
	\end{equation*}
	\begin{equation*}
		\begin{cases}
			y(t_{(caduta)}=0)=0 \\
			h-\frac{1}{2}gtc^2=0
		\end{cases}
	\end{equation*}
	\begin{equation*}
		tc=\sqrt{\frac{2h}{g}}
	\end{equation*}
\end{multicols}
\begin{equation*}
	\Delta t_{tot}=\sqrt{2h}{g}+\frac{h}{V_0}\to \frac{\sqrt{2h}}{g}=\Delta t_{tot} - \frac{h}{V_s}\to \frac{2h}{g}\to \frac{2h}{g}=\bigg(\Delta t-\frac{h}{V_s}\bigg)^2
\end{equation*}
\begin{equation*}
	\Rightarrow (\Delta t)^2+\frac{h^2}{V_{s^2}} -\frac{2h}{V_s}\Delta t=\frac{2h}{g}\to (\Delta t)^2-2 \bigg(\frac{\Delta t}{V_s}+\frac{1}{g}\bigg) h+\frac{h^2}{V_{s^2}} =0 
\end{equation*}
\paragraph{Forma ridotta\label{forma ridotta es.1.1}}
\begin{equation*}
	h^2-V_{s^2}\bigg(\frac{\Delta t+st}{V_s}+\frac{1}{g}\bigg)h+V_{s^2}\Delta t_{tot}^2=0
\end{equation*}
\begin{equation*}
	h=V_{s^2}\bigg(\frac{\Delta t_{tot}}{V_0}+\frac{1}{g}\bigg)\pm \sqrt{V_{s^2}\bigg(\frac{\Delta}{V_s}+\frac{1}{g}\bigg)^2-V^2_s\Delta t^2_{tot}}
\end{equation*}
\begin{equation*}
	h=V_{s^2}\bigg(\frac{\Delta t_{tot}}{V_s}+\frac{1}{g}\bigg)-\sqrt{V_{s^2}\bigg(\frac{\Delta t}{V_s}+\frac{1}{g}\bigg)^2-V_{s^2}\Delta t^2}\Rightarrow \Delta t_{tot}-\frac{h}{V_s}>0
\end{equation*}
\subsection{Esercitazione 2}
In un particolare gioco per bambini una pallina di massa 50.0 grammi viene lanciata su una pista orizzontale   che   in   un   certo   punto   inizia   a   piegarsi   per   formare   un   anello   verticale   completo  e circolare di raggio $R= 51.0 cm$.  Per lanciare la pallina si usa una molla di costante elastica $kel=100N/m$. Di quanto deve essere compressa la molla per poter fornire alla pallina la velocità  minima chele permette di non cadere nel punto più alto (\textit{si trascurino le forze di attrito; PRECISAZIONE: LA MASSA SCIVOLA SENZA ATTRITO}). 
\subsubsection{Soluzione}
Si può applicare il teorema di conservazione dell’energia meccanica considerando, per l’istante $t_1$, l’energia potenziale elastica associata alla massa ferma sulla molla compressa e per l’istante  $t_2$, l’energia  meccanica  della  massa  nel  punto  più  alto  ($2=h_R$)  della  sua  traiettoria.  Precisamente, possiamo scrivere:
\begin{equation*}
	K_1+U_1=K_2+U_2
\end{equation*}
Dove  $K_1$  e  $K_2$  sono le energie cinetiche negli istanti  $t_1$  e  $t_2$, rispettivamente, e  $U_1$  e  $U_2$  sono le energie potenziali negli istanti $t_1$ e $t_2$, rispettivamente. Sulla base delle indicazioni fornite dal testo del problema, possiamo scrivere
\begin{equation*}
	\begin{matrix}
		K_1=0;&K_2=\frac{1}{2}mv_2^2;&U_2=\frac{1}{2}k\Delta x_m^2;&U_2=mgh=2mgR
	\end{matrix}
\end{equation*}
dove $k$ è la costante elastica della molla, $\Delta x_m$ è la deformazione in compressione della molla, $v_2$ è la velocità della massa nel punto più alto della traiettoria. Al riguardo, la forza vincolare, ossia quella che costringe la massa a seguire la traiettoria circolare, può considerarsi nulla nel momento in cui si studia il problema nella condizione limite di ``distacco'' dalla pista. Ne segue che la sola forza che agisce sulla massa e, in modulo, $mg$. Dunque, essendo g perpendicolare alla velocità, risulta essere, all’istante $t_2$ anche l’accelerazione normale, ossia
\begin{equation*}
	a=g \to \frac{v_2^2}{R}=g\to v_2^2=gR\to K_2=\frac{1}{2}mbR
\end{equation*}
Pertanto, facendo le opportune sostituzioni, si ottiene
\begin{equation*}
	\frac{1}{2}k\Delta x_n^2=\frac{1}{2}mv_2^2\to \frac{1}{2}\Delta x_m^2=\frac{5}{2}mgR\to|\Delta x_mg|=\sqrt{\frac{5mgR}{k}}
\end{equation*}
\subsection{Esercitazione 3}
Una turbina idraulica è azionata da una corrente d'acqua ad alta velocità che urta contro le pale e rimbalza.   In   condizioni   ideali,   la   velocità   delle   particelle   d'acqua   dopo   l'urto   contro   la   pala   è esattamente nulla così che tutta l'energia dell'acqua si è trasferita alla turbina. Se la velocità delle particelle dell'acqua è 27.0 m/s, quanto vale la velocità ideale della pala della turbina? (\textit{Si consideri l'urto di una particella d'acqua contro la pala come un urto unidimensionale elastico})
\subsubsection{Soluzione}
La massa della singola molecola d’acqua è estremamente piccola rispetto a quella della pala, cosìche si può trattare il problema come quello dell’urto elastico unidimensionale di una massa m suuna parete (massa virtualmente infinita). Sappiamo che nelle suddette condizioni, nel sistema diriferimento in cui la parete e ferma, il modulo della velocità della massa rimane la stessa prima edopo l’urto. Precisamente, posto $v^\prime_1>0$  la proiezione sull’asse x (direzione dell’urto) del vettore velocità all’istante  $t_1$  (poco prima dell’urto), nel sistema di riferimento in cui la parete è ferma, e $v^\prime_2>0$  la proiezione sull’asse x del vettore velocità all’istante $t_1$ (poco dopo l'urto), nello stesso sistema di riferimento, risulta
\begin{equation*}
	v^\prime_2=v^\prime_1
\end{equation*}
Il testo del problema ci fornisce i dati delle velocità ($v_1=27.0m/2$ $v_2=0$) nel sistema di riferimento di terra, quello in cui la pala (parete) si muove con velocità incognita V (la pala si muove a regime costante e non cambia la sua velocità). Usando le relazioni di trasformazione delle velocità  tra sistemi di riferimento in moto relativo con velocità V possiamo scrivere
\begin{eqnarray*}
	v_1=V+v^\prime_1\\
	v_2=V+v^\prime_2
\end{eqnarray*}
sommando membro a membro e tenendo conto che $v_2=0$ si ottiene 
\begin{equation*}
	v_1=2V\to V=v_1/2
\end{equation*}
\subsection{Esercitazione 4}
Un pacco è lasciato cadere su un nastro trasportatore orizzontale. La massa del pacco è m, la velocità del nastro trasportatore è v e il coefficiente di attrito dinamico per il pacco sul nastro è $\mu d$. Per quanto tempo il pacco striscerà sul nastro? Qual è la distanza percorsa dal pacco durante l'intervallo di tempo calcolato nel punto precedente?
\subsubsection{Soluzione}
La forza di attrito si oppone allo scivolamento del pacco e, pertanto, trascina il pacco accelerandolo nel verso del moto del nastro. La forza di attrito è anche la risultante delle forze che agiscono sul pacco. Precisamente,
\begin{eqnarray*}
	m\overrightarrow{a}=\overrightarrow{F}_r=\overrightarrow{F}_{att}\\
	||\overrightarrow{F}_{att}||=\mu_dmg;&F_{att,x};&ma_x=\mu_dmg
\end{eqnarray*}
dove si è  preso come asse x quello corrispondente alla direzione del nastro, e come verso positivo quello corrispondente al moto del nastro, che è anche il verso del vettore accelerazione. Il pacco striscerà fino a quando raggiungerà la stessa velocità del nastro (il moto relativo diventa nullo). Pertanto, l’intervallo di tempo richiesto risulta
\begin{equation*}
	v=a:x\Delta t=\mu_dg\Delta t\to \Delta t=\frac{v}{\mu_dg}
\end{equation*}
La distanza percorsa si ricava usando le note relazioni della cinematica del moto con accelerazione costante
\begin{equation*}
	\Delta x=\frac{1}{2}\frac{v^2}{\mu_dg}
\end{equation*}
Si può risolvere il problema seguendo altri percorsi, tutti molto semplici. Ad esempio, si può studiare il problema nel sistema di riferimento del nastro. Supponiamo allora che il nastro si muova nel senso delle x negative. Rispetto al nastro (fermo) il pacco si muoverà con una velocità iniziale $v$ nel senso delle x positive. La forza di attrito, questa volta, ha componente negativa perché tendea frenare il moto del pacco rispetto al nastro ecc. ecc.
\subsection{Esercitazione 5}
Due vettori a e b hanno modulo uguale di 12,7 unità. Sono orientati come in
figura e la loro somma vettoriale è r. Trovare:
\begin{tasks}
	\task le componenti $x$ e $y$ di \textbf{r}
	\task il modulo di r;
	\task l'angolo che \textbf{r} forma con l'asse \textit{x}.
\end{tasks}
\begin{figure}[!ht]
	\centering
	\begin{tikzpicture}
		\node[] (pic) at (0,0) {\includegraphics[height=4cm]{img/esercizio 5
		im1.pdf}};
	\end{tikzpicture}
	\caption{figura 1}
\end{figure}
\begin{multicols}{2}
	\begin{equation*}
		\overrightarrow{r}=(r_x,r_y)=r_x \hat{i}+r_y*\hat{i}=(a_x+b_x, a_y+b_y)
	\end{equation*}
	\begin{equation*}
		|\overrightarrow{a}|=|\overrightarrow{b}|=12*7
	\end{equation*}
	\begin{equation*}
		\alpha=28.2^o
	\end{equation*}
	\begin{equation*}
		\beta=115^o
	\end{equation*}
	\begin{equation*}
		\overrightarrow{r}=\overrightarrow{a}+\overrightarrow{b}
	\end{equation*}
\end{multicols}
\begin{multicols}{2}
	\begin{equation*}
		a_x=|\bar{a}|\cos \alpha=12*7\cos 28.2^o=11.2
	\end{equation*}
	\begin{equation*}
		a_y=|\bar{a}|\sin \alpha=12*7\sin 28.2^o=6
	\end{equation*}
	\begin{equation*}
		\sigma=180^o-\alpha-\beta=46,8^o
	\end{equation*}
	\begin{equation*}
		b_x=-|\overrightarrow{b}|\cos \sigma=-8.7
	\end{equation*}
	\begin{equation*}
		b_y=|\overrightarrow{b}|\sin \sigma=9.3
	\end{equation*}
\end{multicols}
	\begin{equation*}
		\overrightarrow{r}=(11.2-8.7,6+9.3)=(2.5,15.3)
	\end{equation*}
	\begin{equation*}
		|\overrightarrow{r}|=\sqrt{r^2_x+r^2_y}=\sqrt{2.5^2+15.3^2}=15.5
	\end{equation*}



\chapter{Modelli atomici}
\section{Modello atomico di Bohr-Sommerfeld}
Il modello atomico proposto da Niels Bohr nel 1913, successivamente ampliato da Arnold Sommerfeld nel 1916, è la più famosa applicazione della quantizzazione dell'energia che, insieme alle spiegazioni teoriche sulla radiazione del corpo nero, sull'effetto fotoelettrico e sullo scattering Compton, e all'equazione di Schrödinger, costituiscono la base della meccanica quantistica.\\
Il modello, proposto inizialmente per l'atomo di idrogeno, riusciva anche a spiegare, entro il margine di errore statistico, l'esistenza dello spettro sperimentale. Bohr presenta così un modello dell'atomo, facendo intuire che gli elettroni si muovono su degli orbitali. \textit{Questo modello viene ancora utilizzato nello studio dei Semiconduttori.} 
\begin{center}
	By \href{https://it.wikipedia.org/wiki/Modello_atomico_di_Bohr}{Wikipedia}
\end{center}

\end{document}

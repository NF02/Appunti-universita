\chapter{Equazioni differenziali ordinarie}
\begin{defi}
  Si definisce equazione differenziale ordinaria un'equazione di tipo $g(x,y,y^\prime,
  y^{\prime\prime},\dots,y^{(n)})$ con $y$ funzione di $x$. {\color{red}Ordinaria:} Le
  derivate che compaiono nell'equazione sono di $y(x)$. L'ordine di un'equazione
  differenziale è l'ordine massimo delle derivate di $y(x)$ compaiono nell'equazione.
  Risolvere un'equazione differenziale significa trovare la funzione $y=y(x)$ che
  sostituta nell'equazione, con le sue derivate, rende l'uguaglianza un'identità.
  L'insieme delle soluzioni di un'equazione differenziale è una famiglia di funzioni ed è
  detto {\color{red}integrale generale}, che dipende da un parametro reale $(c)$.
  Attribuendo un valore alla costante si ha un {\color{red}integrale particolare}. Non
  sempre ogni soluzione dell'equazione differenziale è anche un integrale particolare.
  Ci sono casi di equazioni differenziali che ammettono anceh integrali singolari\footnote{integrali non ottenibili per nessun valore della costante $c$}
\end{defi}
\section{Equazioni differenziali a variabili separabili}
\begin{defi}
  Un'equazione differenziale del primo ordine è detta a variabili separabili se può
  essere scritta nella forma
  \begin{equation*}
    y^\prime =f(x)*g(x)
  \end{equation*}
\end{defi}
\begin{esempio}
  \begin{eqnarray*}
    y=4xy^2 & f(x)=4x & g(y)=y^2
  \end{eqnarray*}  
\end{esempio}
\begin{svol}
  Se $\exists y_0:g(y_0)=0\Rightarrow y=y_0$ è soluzione \\
  Se esiste una costante che annulla $g(y),y=y_0$ è soluzione
  \paragraph{Infatti}
  \begin{equation*}
    (y_0)^\prime=f(x)*\underbrace{g(y)}_{0 \text{ per ipotesi}}
  \end{equation*}
  derivata di costante
  \begin{eqnarray*}
    o=f(x)*o&o=o
  \end{eqnarray*}
  Se $g(y)\neq 0$ divido entrambi i membri per $g(y)$
  \begin{equation*}
    \frac{y^\prime}{g(y)}=f(x)
  \end{equation*}
  Poiché la derivata di una funzione si può esprimere come rapporto di differenziali
  \begin{eqnarray*}
    y^\prime=\frac{dy}{dx}&\to& \frac{dy}{g(y)}\frac{1}{dx}=f(x) \text{ da cui } \frac{dy}{g(y)}=f(x)dx
  \end{eqnarray*}
  Integrando membro a membro
  \begin{eqnarray*}
    \int \frac{dy}{g(y)}=\int f(x)dx & \text{se }& G(y)=\left(\frac{1}{g(y)}\right)^\prime \text{ e } F(x)=f^\prime (x)
  \end{eqnarray*}
  $G(y)=F(x)+c$ $c\in R$ -- Se $G(y)$ è invertibile $(\exists G^{-1}(y))$ posso ricavare
  $y$
  \begin{equation*}
    y=G^{-1}(F(x)=c)
  \end{equation*}
\end{svol}
\subsection{Proprietà generali delle equazioni differenziali lineari}
\begin{eqnarray*}
  \underbrace{y^{(n)}+a_{n-1}(x)y^{(n-1)}+\dots+a_1(x)y^\prime+a_0(x)y}_{\text{coefficineti}}=
  \underbrace{g(x)}_{\text{termine noto}} &\leftarrow&\text{ se } g(x)=0 \text{ si ha
                                                       un'equazione}\\
  &&\text{ differenziale omogenea}
\end{eqnarray*}
Possiamo scrivere le equazioni differenziali come
\begin{equation*}
  \underbrace{L(y)}_{\begin{matrix}
                       \text{\color{red}operatore}\\
                       \text{\color{red}lineare}
                     \end{matrix}}
                   =b(x)
\end{equation*}
\subsection{Equazioni differenziali lineari del primo ordine}
Integrale generale equazione differenziale del I ordine omogenea
\begin{equation*}
	y^\prime-a(x)y=0
\end{equation*}
\begin{eqnarray*}
	y_0(x)=ce^{-A(x)}&c\in R & A(x)\text{ primitiva di } a(x)
\end{eqnarray*}
\begin{proof}
	\begin{eqnarray*}
		y^\prime a(x)y=0 &\text{ moltiplico tuto per } e^{A(x)}
	\end{eqnarray*}
	\begin{equation*}
		\underbrace{e^{A(x)}y^\prime+a(x)e^{A(x)}y}_{\text{Derivata
		prodotto}}=0
	\end{equation*}
\end{proof}
\begin{eqnarray*}
	(e^{A(x)}y)^\prime=0 && \text{derivata prima nulla } \Rightarrow \text{la
	funzione da derivare è costante}\\
	e^{A(x)}y=c & q=ce^{-A(x)}
\end{eqnarray*}
\begin{itemize}
	\item Integrale equazioni differenziali lineari I ordine complete
		\begin{eqnarray*}
			&y^\prime+a(x)y=b(x)\\
			y_0(x)+\bar{y}(x) & y_0(x) & \text{Integrale generale equazione
			differenziale omogenea associata}\\
			&\bar{y}(x) &\text{Integrale equazione completa}
		\end{eqnarray*}
		\begin{equation*}
			\bar{y}(x)=e^{-A(x)}\int e^{-A(x)}b(x)dx
		\end{equation*}
\end{itemize}
\clearpage
\begin{proof}
	L'integrale paticolare di $y^\prime+a(x)y=b(x)$ è del tipo
	$\bar{y}=c(x)e^{-A(x)}$
	\begin{equation*}
		\bar{y}^\prime=c^\prime(x)e^{-A(x)}-c(x)*a(x)e^{-A(x)}
	\end{equation*}
	sostituisco $\bar{y}$ e $\bar{y}^\prime$ $c^\prime(x)e^{-A(x)}-
	\not{c(x)a(x)e^{-A(x)}}+\not{a(x)c(x)e^{-A(x)}}=b(x)$
	\begin{eqnarray*}
		c^\prime(x)e^{-A(x)}=b(x) & c^\prime(x)=e^{-A(x)}b(x)\\
		c(x)=\int e^{A(x)}b(x)dx & \bar{y}(x)=c(x)*e^{-A(x)}\int e^{A(x)}b(x)dx
	\end{eqnarray*}
	L'integrale generale di $y^\prime+a(x)y=b(x)$ è
	\begin{equation*}
		y_0(x)+\bar{y}(x)=ce^{-A(x)}+e^{-A(x)}\int
		e^{-A(x)}b(x)dx=e^{-A(x)}\left(\int e^{A(x)}b(x)dx+c\right)
	\end{equation*}
	\begin{esempio}
		\begin{eqnarray*}
			y^\prime =y &y^\prime-y=0 & a(x)=-1\text{ }A(x)=-x\\
			&y^{(x)}=ce^{-A(x)}=ce^x
		\end{eqnarray*}
	\end{esempio}
\end{proof}
\subsection{Equazioni differenziali di Bernoulli (non lineari, I ordine)}
Sono equazioni nella forma
\begin{eqnarray*}
	y^\prime + a(x)y=b(x)y^\alpha &\alpha\in \mathds{R} & \alpha\neq 0,1
\end{eqnarray*}
Se $\alpha >0$ $y=0$ è integrale singolare\\
Per $y\neq 0$ divido per $y^\alpha$
\begin{eqnarray}
	y^\prime y^{-\alpha}+a(x)y^{1-\alpha}=b(x)
\end{eqnarray}
Pongo $z(x)=y^{1-\alpha}(x)$ e risoluto l'equazione differenziale lineare in
$z(x)$ e poi passo a $y(x)$
\begin{esempio}
	\begin{eqnarray*}
		y^\prime-2y =e^{-x}y^2 & y=0 &\text{integrale singolare}
	\end{eqnarray*}
	per $y\neq 0$
	\begin{eqnarray*}
		\frac{y^\prime}{y^2}-\frac{2}{y}=-e^{-x} &
		z(x)=\frac{1}{y(x)}:z^\prime(x)=-\frac{1}{y^2}*y^\prime
	\end{eqnarray*}
	per cui si ha: $z^\prime+2z=+e^{-x}$
	\begin{eqnarray*}
		a(x)=2 & A(x)=2x\\
		b(x)=e^{-x}
	\end{eqnarray*}
	\begin{equation*}
		z(x)=e^{-A(x)}\left[\int e^{A(x)}b(x)dx+c\right]=e^{-2x}\left[\int
		e^{2x}e^{-x}dx+c\right]=e^{-2x}(e^x+c)
	\end{equation*}
	\begin{eqnarray*}
		z(x)=\frac{1}{y(x)} &
		y(x)=\frac{1}{z(x)}=\frac{1}{e^{-2x}}=\frac{1}{e^{-2x}(e^x+c)}
	\end{eqnarray*}
\end{esempio}
\clearpage
\subsection{Equazioni differenziali di Clairaut (Non lineare, I ordine)}
Sono equazione del tipo
\begin{eqnarray*}
	y=xy+g(y^\prime) &y^\prime \text{ compare dentro una funzione}
\end{eqnarray*}
Si risolvono nel segno modo:
\begin{enumerate}
	\item pongo $y^\prime =t(x)$ 
	\item derivo entrambi i termini 
		\begin{eqnarray*}
			y^\prime(x)=y^\prime(x)+xy^{\prime\prime}+g^\prime (y^\prime)
			y^{\prime\prime}(x)\\
			xy^{\prime\prime}(x)+g^\prime(y^\prime)y^{\prime\prime}(x)
		\end{eqnarray*}
		Con la sostituzione si ha:
		\begin{eqnarray*}
			xt^\prime(x)[x+g^\prime(t(x))] =0
		\end{eqnarray*}
		$t^\prime(x)=0\Rightarrow t(x)=c$ (costante)
		$t(x)=y^\prime(x)=c\Rightarrow y(x)=xy^\prime(x)+g(y^\prime(x))$\\
		L'integrale generale è famiglia di rette $y(x)=cx+g(c)$
		\begin{eqnarray*}
			x+g^{\prime}(t(x))=0 & \begin{cases}
				x=-g^\prime(t(x))\\
				y(x)=-g^\prime(t(x))t(x)+g(t(x))
			\end{cases} & \text{Integrale singolare}
		\end{eqnarray*}
		È una curva
		\begin{eqnarray*}
			\begin{cases}
				x=-g^\prime(t(x))\\
				y(x)=-g^\prime(t(x))t(x)+g(t(x))
			\end{cases} & \begin{matrix}
                                        \text{curva inviluppo della famiglia di rette}\\
                                        \text{Ogni retta $y(x)=cx+g(c)$ al variare di $c\in R$,}\\
                                        \text{è tangente alla curva inviluppo.}
			\end{matrix}
		\end{eqnarray*}
                \begin{esempio}
                  \begin{eqnarray*}
                    y=xy^\prime-\frac{1}{4}(y^\prime)^2 & t(x)=y^\prime(x)
                  \end{eqnarray*}
                  derivo $y^\prime=y^\prime+xy^{\prime\prime}-\frac{2}{4}y^\prime
                  y^{\prime\prime}$
                  \begin{eqnarray*}
                    xy^{\prime\prime}-\frac{y^\prime}{2}y^{\prime\prime}=0\\
                    xt^\prime (x)-\frac{t}{2}(x)*t^\prime(x)=0
                    &\Rightarrow&t^\prime(x)\left(x-\frac{t(x)}{2}\right)=0\Rightarrow
                                  t(x)=0\Rightarrow t(x)=y^\prime(x)=c, c\in R
                  \end{eqnarray*}
                  da cui $y(x)=cx-\frac{1}{4}c^2$ Integrale generale\\
                  \begin{eqnarray*}
                    x-\frac{t(x)}{2} & \begin{cases}
                                         x=\frac{t(x)}{2}\\
                                         y=\frac{t^2(x)}{2}-\frac{1}{4}t^2(x)=\frac{1}{4}t^2(x)
                                       \end{cases}
                                       \text{ in forma cartesiana } \begin{cases}
                                                                      t(x)=2x\\
                                                                      y=\frac{1}{4}*4x^2
                                                                    \end{cases}
                    & y=x^2   \begin{matrix}
                                \text{Integrale}\\
                                \text{singolare}
                              \end{matrix}
                  \end{eqnarray*}
                \end{esempio}
\end{enumerate}
\clearpage
\section{Problema di Cauchy \label{pcauchy}}
\begin{defi}
  Sia $f(x,y):D\to R$, con $D \subseteq R^2$, D aperto\\
  Problema di Cauchy
  \begin{eqnarray*}
    \begin{cases}
      y^\prime=f(x,y)\\
      y(x_0)=y_0
      \end{cases} & y(x_0)=y_0 &\text{condizione iniziale}
  \end{eqnarray*}
  $y=y(x)$ è detta {\color{red}soluzione locale}: esiste in un intorno di $x_0$, in cui
  $y(x)$ è derivabile e tale che
  \begin{eqnarray*}
    y^\prime=f(x,y(x))\\
    y(y_0)=x_0
  \end{eqnarray*}
\end{defi}
\begin{esempio}
  $\begin{cases}
     y-y\tan x=1\\
     y(0)=1
   \end{cases}$ l'integrale generale è $y(x)=\frac{1}{\cos x}(\sin x+c)$\\
   Determino l'integrale partcolare che soddisfi $y(0)=1$ $(x=0,y(x)=1)$
   \begin{eqnarray*}
     1=\frac{1}{\cos 0} (\sin 0 +c) \Rightarrow &c=1 & y(x)=\frac{1}{\cos x}(\sin x+1)
     \text{ } \begin{matrix}
         \text{Soluzione del problema}\\
         \text{di Cauchy}
       \end{matrix}
   \end{eqnarray*}
\end{esempio}
\subsection{Teorema di Peano (Esistenza di soluzioni)}
\begin{teorema}
  Dato il problema di Cauchy $\begin{cases} y^\prime=f(x,y)\\ y(x_0)=y_0\end{cases}$\\
  Con $f(x,y):D \to R, D \subseteq R^2$ aperto e $(x_0,y_0)\in D$\\
  Se $f(x,y)$ è continua in $(x_0,y_0)$ allora il problema di Cauchy ammette almeno una
  soluzione definita in un intorno di $x_0$
  \begin{eqnarray*}
    \begin{cases}
      y^\prime=x\sqrt{y}\\
      y(x)=1
    \end{cases}& f(x,y)=x\sqrt{y} \text{ è definita in } D=\{(x,y)\subset R^2:x\in R,
    y\geq 0\}
  \end{eqnarray*}
  per cui è continua in (0,1). Per il teorema di Peano esiste almeno una soluzione
  \begin{eqnarray*}
    y^\prime=x\sqrt{y} & y=0 & \text{Integrale singolare}\\
    \int \frac{dy}{\sqrt{y}} =\int xdx& \frac{1}{2} \sqrt{y}=\frac{x^2}{2}+c
                             & \Rightarrow \sqrt{y}=\frac{x^2}{4}+k \Rightarrow y=\left(
                               \frac{x^2}{4}+k\right)^2\\
    y(0)=1 & 1=(k)^2\text{ } k=\pm 1 & k=-1 \text{ non accettabile perché } \sqrt{y}=-1\\
    && k=1\\
    && y=\left(\frac{x^2}{4}+k\right)^2 \text{ soluzione del problema di Cauchy}
  \end{eqnarray*}
\end{teorema}
\subsection{Teorema di Cauchy (esistenza e unicità locale)\label{tcauchy}}
\begin{defi}
  Dato il problema di Cauchy $\begin{cases}y^\prime=f(x,y)\\ y(x_0)=y_0\end{cases}$\\
  Se
  \begin{description}
  \item[I)] f(x,y) è continua in un intorno $I\times J \in R\times R$ di $(x_0,y_0)$
    $(x_0\in I, y_0\in J)$\clearpage
  \item[II)] $f(x,y)$ è Lipschitziana rispetto a $y$ uniformemente per $x \in I$
    \begin{eqnarray*}
      \text{cioè } & \exists L>0:\abs{f(x,y_1)-f(x,y_2)}\leq L\abs{y_1-y_2}
      & \begin{matrix}
          \exists x\in I\\
          \exists y_1, y_2\in J
        \end{matrix}
    \end{eqnarray*}
    Allora esiste un intorno A di $x_0$ in cui $y=y(x)$ è un'unica soluzione del problema
    di Cauchy.
    \begin{eqnarray*}
      \exists \delta >0 & \underbrace{\exists !}_{\begin{matrix} \text{esite} \\
                                                    \text{una e}\\
                                                    \text{una sola} \end{matrix}}
      & y=y(x_0)
    \end{eqnarray*}
    Con $y: A= [x_0-\delta,x_0+\delta]\in R$ e derivabile in
    $A= [x_0-\delta,x_0+\delta]$\\
    L'interno può essere solo sinisto o destro. Il teorema di Cauchy rosove il problema
    ``in piccolo'' cioè in un intorno di $x_0$. 
  \end{description}  
\end{defi}
\subsubsection{Corollario del teorema di Cauchy}
\begin{defi}
  Sia $f(x,y)$ una funzione derivabile parzialmente rispetto a $y$.\\
  Se $\frac{\partial f}{\partial y}\in C_A^o$ allora $f(x,y)$ è lipschitziana rispetto a
  $y$ in A.\\
  Per cui
  \begin{center}
    se $f(x,y)$ continua in $(x_0,y_0)$\\
    e $\frac{\partial f}{\partial y}$ continua in $I\times J$ intorno di $(x_0,y_0)$\\
    allora $\exists! y=y(x)$ in $I\times J$ soluzone del problema di Cauchy,
  \end{center}
\end{defi}
\begin{esempio}
  \begin{eqnarray*}
    \begin{cases}
      y^\prime=\abs{y}\\
      y(0)=1
    \end{cases} & f(x,y)=\abs{y} \text{ continua in }(0,1)
  \end{eqnarray*}
  \begin{eqnarray*}
    \frac{\partial f}{\partial y}=\begin{cases}
                                    1 & y>0\\
                                    -1 & y<0
                                  \end{cases}
    & \text{ in } (0,1) \text{ vale } 1 \Rightarrow \frac{\partial f}{\partial y}
      continua \Rightarrow f(x,y)\text{ lipschitziana}
  \end{eqnarray*}
  Per il teorema di Cauchy in un intorno $I\times$ J di (0,1) $\exists! y=y(x)$ soluzione
  del problema di Cauchy.
\end{esempio}
\subsection{Equazioni differenziali lineari di ordine N}
Sia
\begin{align*}
  L(y)=b(x) & \text{ un'equazione differenziale lineare (L'operatore lineare)} \\
  L(y)=0 & \text{ equazione omogenea assocata}
\end{align*}
Se $y_0(x)$ è l'integrale generale dell'equazione omogenea associata e se $\bar{y}(x)$ è
un integrale particolare dell'equazione non omogenea, allora $y(x)=y_0(x)+\bar{y}(x)$ è
soluzione dell'equazone completa.
\subsubsection{Equazione completa}
\begin{equation}
  y^{(n)}+a_1(x)y^{(n-1)}+\dots+a_{n-1}(x)y^{\prime}+a_ny=b(x)
\end{equation}
\clearpage
\subsection{Equazioni differenziali lineari omogenee}
\begin{equation}
  y^{(n)}+a_1(x)y^{(n-1)}+\dots+a_{n-1}(x)y^{\prime}+a_ny=0
\end{equation}
L'integrale generale dell'equazione lineare omogenea è una combinazione lineare di $n$
soluzioni linearmenete indipendenti.
\begin{equation}
  y_0 (x)=c_1y_1(x)+c_2y_2(x)+\dots+c_ny_n(x)
\end{equation}
Sono integrali particolari dell'equazione linearmente indipendenti.\\
L'indipendenze delle soluzioni $y_1(x),y_2(x),\dots,y_n(x)$ è data dal non annullarsi per
$\forall x\in I$ del determinante.\\
{\color{red}Wronskiano}\footnote{è un determinante introdotto dal matematico 
polacco Josef Hoene-Wronski diffusamente utilizzato nello studio di equazioni
differenziali. Consente frequentemente di mostrare l'indipendenza lineare di 
un insieme di soluzioni.} così fatto:
\begin{multicols}{2}
\begin{equation*}
  det W(x)=\begin{vmatrix}
             y_1(x) & y_2(x) & \dots & y_n(x) \\
             y_1^{\prime} & y_2^{\prime} &\dots& y_n^{\prime}\\
             y_1^{\prime\prime} & y_2^{\prime\prime} & \dots & y_n^{\prime\prime}\\
             \vdots & \vdots && \vdots\\ 
             y_1^{n-1} & y_2^{n-1} & \dots & y_n^{n-1}
           \end{vmatrix}
\end{equation*}
\begin{description}
    \item[Prima riga] Soluzioni
    \item[Seconda riga] Derivate prime delle soluzioni
    \item[Terza riga] Derivate seconde delle soluzioni
    \item[ultima riga] Derivate alla $N-1$ delle soluzioni
\end{description}
\end{multicols}    
Se $det W(x)\neq 0$ $\forall x\in I$, allora $y_1(x),y_2(x),\dots,y_n(x)$ sono linearmente
indipendenti.\\
Come si determminano le soluzioni? \emph{Si introduce il polinomio caratteristico ad }
$y^2 \to \lambda^{n}$ 
\begin{eqnarray*}
y(x)=e^{\lambda x} & \text{soluzione per $\lambda$ redice del polinomio}.
\end{eqnarray*}
\subsection{Equazioni lineari complete}
\begin{enumerate}
\item Si determina l'integrale generale $y_0(x)$ dell'equazione omogenea associata
\item Si determina un integrale particolare $\bar{y}(x)$ dell'equazione non omogenea
  \item L'integrale dell'equazione completa è $y(x)=y_0(x)+\bar{y}(x)$
\end{enumerate}
Come determinare un integrale particolare dell'equazione non omogenea\\
Metodo di somiglianza: osserva la forma del termine noto $b(x)$
\begin{eqnarray*}
	b(x)=e^{\gamma x} P_n(x) & P_n(x) &\text{polinomio di grado n}
\end{eqnarray*}
Se $\gamma$ non è radice dell'equazione caratteristica
\begin{eqnarray*}
	\bar{y} (x)=e^{\gamma x} q_n(x) & q_n(x)& \text{polinomio di grado n}
\end{eqnarray*}
se $\gamma$ radice dell'equazione complessa, ne tengo conto e tengo conto delle
della sua molteplicità.
\begin{eqnarray*}
	\bar{y} (x)=e^{\gamma x} x^n q_n(x) & q_n(x) & \text{n molteplicità}
\end{eqnarray*}
\begin{esempio}
	\begin{equation*}
		y^{\prime\prime\prime}-4y^{\prime} =x
	\end{equation*}
	\begin{enumerate}
		\item equazione omogenea associata
			$y^{\prime\prime\prime}-4y^{\prime}=0 \Rightarrow \lambda^3
			-4\lambda =0 \Rightarrow \lambda(\lambda^2-4)=0 \Rightarrow
			\lambda_1=0,\lambda_2=2,\lambda_3=-2,y_1=e^{ox}=1,y^{-2x}$
	\end{enumerate}
	\begin{equation*}
		y_0(x)=c_1+c_2e^{2x}+c_3e^{-2x}
	\end{equation*}
\end{esempio}
\clearpage
\begin{eqnarray*}
	b(x)=x\to e^{\gamma x}=1 q_n(x) & e^{\gamma x}=1 \Rightarrow \gamma=0\\
	&q_n(x)=x \text{ grado 1}
\end{eqnarray*}
\begin{eqnarray*}
	\bar{y}(x)=x^ne^{\gamma x} q_n(x)=e^{0 x}(ax+b)=(ax+b)x\\
	\bar{y}(x)=ax^2+bx
\end{eqnarray*}
Ricavo $a$ e $b$
\begin{eqnarray*}
	y^\prime(x)=2ax+b &\Rightarrow y^{\prime\prime}(x)=2a+b &
	y^{\prime\prime\prime}(x)=0\\
	y^{\prime\prime\prime}-4y^\prime=x & 0-4(2ax+b)=x & -8ax-4b=x\\
	&\begin{cases}
		-8a=1\\
		-4b=0
	\end{cases}\begin{cases}
		a=-\frac{1}{8}\\
		b=0
	\end{cases} & \bar{y}(x)=-\frac{1}{8}x^3
\end{eqnarray*}
L'integrale generale è
$y(x)=y_0(x)+\bar{y}(x)=c_1+c_2e^{2x}+c_3e^{-2x}+\frac{1}{8}x^2$
\begin{center}
	\fbox
	{
	\begin{minipage}{0.95\textwidth}
		$b(x)=\sin \beta x$ $P_n(x)$ o $b(x)=e^{\alpha x}\cos \beta x$ 
		$P_n(x)$\\
		se $\alpha\pm i\beta$ non sono soluzione ell'equazione caratteristica
		\begin{eqnarray*}
			\bar{y}(x)=e^{\alpha x}[q_n(x)\sin \beta x +r_n(x)\cos\beta x]
		\end{eqnarray*}
		se $\alpha\pm i\beta$ sono soluzione dell'equazione caratteristica, ne
		tegno conto e tengo conto della sua molteplicità
		\begin{eqnarray*}
			\bar{y}(x)=e^{\alpha x}x^n[q_n(x)\sin \beta x +r_n(x)\cos\beta x]
		\end{eqnarray*}
	\end{minipage}
	}
\end{center}
\begin{esempio}
	$y^{\prime\prime}-3y^\prime+2y=xe^{3x}$
	\begin{enumerate}
		\item equazione omogenea associata
			$y^{\prime\prime}-3y^\prime+2y=0\Rightarrow \lambda^2-3\lambda
			+2=0$ $\lambda=\frac{2\pm\sqrt{9-8}}{2}\begin{cases}
				2\\
				1
			\end{cases}$
			\begin{eqnarray*}
				\lambda_1=1 & y_1=e^x\\
				\lambda_2=2 & y_2=e^{2x}
			\end{eqnarray*}
			$y_0(x)=c_1e^x+c_2e^{2x}$
			\begin{eqnarray*}
				b(x)=xe^{3x} & \Rightarrow e^{\gamma x} P_n(x) \Rightarrow
				\gamma =3 & P_n(x) \text{ grado 1}\\
				& \bar{y}(x)=e^{3x}q_n(x)=e^{3x}(ax+b)
			\end{eqnarray*}
			Trovare a e b
			\begin{eqnarray*}
				\bar{y}^\prime(x)=3e^{3x}(ax+b)+e^{3x}(a)\\
				y^{\prime\prime}=9e^{3x}(ax+b)+3e^{3x}a+3ae^{3x}
			\end{eqnarray*}
			\clearpage
			\begin{eqnarray*}
				y^{\prime\prime}-3y^\prime+2x=xe^{3x} & 9e^{3x}(ax+b)+6ae^{3x}-
				2(3e^{3x}(ax+b)+e^{3x}a)+(e^{3x}(ax+b))=xe^{3a}\\
				&9e^{3x}\not{(ax+b)}+6ae^{3x}-9e^{3x}\not{(ax+b)}-3e^{3x}a+2e^{3x}
				(ax+b)=xe^{3x}\\
				\begin{matrix}
					\bar{y}(x)=e^{3x}\left(\frac{1}{2}x-\frac{3}{4}\right)\\
					y(x)=c_1e^x+c_2e^{3x}\\+e^{3x}\left(\frac{1}{2}x-\frac{3}{4}
					\right)
				\end{matrix} & 3a+2(qx+b)=x \begin{cases}
					2a=1\\
					3a+2b=0
				\end{cases}\begin{cases}
					a=\frac{1}{2}\\
					b=-\frac{3}{4}
				\end{cases}
			\end{eqnarray*}
	\end{enumerate}
\end{esempio}
\subsubsection{Faccia un integrale particolare di $L(y)=b(x)$}
\begin{equation*}
	\bar{y}(x)=c_1(x)y_1(x)+\dots+c_n(x)y_n(x)
\end{equation*}
\begin{esempio}
	\begin{eqnarray*}
		y^{\prime\prime}+y=\frac{1}{\cos x} & \lambda^2+1=0 & \lambda =\pm 1\\
		& y_1=e^{ix}=\cos x\\
		& y_2=e^{-ix}=\sin x
	\end{eqnarray*}
	$y_0(x)=c_1\cos x +c_2 \sin x$
	\begin{eqnarray*}
		c_1 (x)\cos x +c_2(x) \sin x =0\\
		c_1^\prime (x)(-\sin x)+c_2(x)\cos x= \frac{1}{\cos x}
	\end{eqnarray*}
	\begin{equation*}
		W(x)=\begin{vmatrix}
			\cos x & \sin x\\
			-\sin x & \cos x
		\end{vmatrix} =\cos^2x+\sin^2x=1\neq 0 \forall x \in \mathds{R}
	\end{equation*}
	\begin{eqnarray*}
		c_1^\prime(x)=\frac{1}{W(x)}=\begin{vmatrix}
			0 &\sin x\\
			\frac{1}{\cos x} &\cos x
		\end{vmatrix}= -\frac{\sin x}{\cos x} &
		c_2^\prime=\frac{1}{W(x)}\begin{vmatrix}
			\cos x & 0\\
			-\sin x & \frac{1}{\cos x}
		\end{vmatrix} =1
	\end{eqnarray*}
	\begin{eqnarray*}
		c^\prime_i(x)=-\frac{\sin x}{\cos x} & c_1(x)=\int
		-\frac{\sin x}{\cos x} dx=\ln\abs{\cos x}\\
		&c_2(x)=\int dx=x
	\end{eqnarray*}
	\begin{eqnarray*}
		\bar{y}(x)=\cos x \ln\abs{\cos x}+x \sin x\\
		y(x)=c_1\cos x +c_2 \sin x +\cos \abs{\cos x}+x\sin x
	\end{eqnarray*}
\end{esempio}


\chapter{Derivate Parziali}
\section{Derivate parziali di primo grado}
\begin{defi}
  Sia $f(x,y)$ una funzione di due variabili definita in un punto interno ad $A$\\
  Consideriamo un interno circolare di $P(x_0,y_0),I(x_0,y_0), \delta$, in netto sulla retta $y=y_0$ e
  incrementa la $x_0$ passante da $x_0$ a $x_0+h$. Ho così un punto $P(x_0+h,y_0)\in A$.\\
  Definisco il rapporto di $f(x,y)$ nella sola $x$
  \begin{equation}
    \frac{f(x_0+h,y_0)-f(x_0,y_0)}{h}
  \end{equation}
  $f(x,y)$ si definisce {\color{red} derivabile parzialmente} se $\exists \lim\limits_{h\to 0}\frac{f(x_0+h,y_0)-f(x_0,y_0)}{h} = l\in R$ reale e finito.
  \begin{equation}
    \frac{\partial f}{\partial x} =fx=\lim\limits_{h\to 0}\frac{f(x_0+h,y_0)-f(x_0,y_0)}{h}
  \end{equation}
  Analogamente, considero un interno di $P(x_0,y_0), I(x_0,y_0),\delta$. Mi ruoto sulla retta $x=x_0$ e
  incremento la $y_0$ passando da $y_0$ a $y_0+k$. Ho così un punto $P(x_0,y_0+h)\in A$.\\
  Definisco il rapporto ingrementale di $f(x,y)$ nella sola $y$
  \begin{equation*}
    \frac{f(x_0+k,y_0)-f(x_0,y_0)}{k}
  \end{equation*}
  {\color{red} derivabile parzialmente} se $\exists \lim\limits_{h\to 0}\frac{f(x_0+h,y_0)-f(x_0,y_0)}{h} =
  l\in R$ reale e finito.\\
  Se in un punto ($x,y$) esistono entrambi le derivate parziale si dice che la funzione è {\color{red} derivabile}
  in (x,y) inoltre se $f$ è derivabile in ogni punto $(x,y)\in A$, si dice che f è derivabile in $A$.
\end{defi}
\subsection{Significato geometrico}
\begin{itemize}
\item Lo derivata prima par parziale in $P$ è $fx(x_0,y_0)$, è la tangente alla curva che si crea intersecando
  $f(x,y)$ con il piano $y=y_0$
\item La derivata prima parziale in $P$, $fy(x_0,y_0)$ è la tangente alla curva che si
  crea intersecando $f(x,y)$ con il piano $x=x_0$
\end{itemize}
Se esistono entrambe allora le due rette tangenti alle sezioni della funzione individuano il piano tangente al
solido nel punto $P(x_0,y_0,z)$
\section{Derivata parziale seconde}
\begin{defi}
  Sia $f(x,y)$ una derivabile e siano definite in un deminio le due derivate parziali
  \begin{equation*}
    \begin{matrix}
      f_x(x,y) & f_y(x,y)
    \end{matrix}
  \end{equation*}
  Tali funzioni passano a loro volta essere derivabili e si ottengono così le derivate seconde parziali di
  $f(x,y)$
  \begin{center}
    \Tree[.$f(x,y)$ [.$f_x(x,y)$ $f_{xx}(x,y)$ $f_{xy}(x,y)$ ] [.$f_y(x,y)$ $f_{yx}(x,y)$ $f_{yy}(x,y)$ ] ]
  \end{center} 
\end{defi}
\begin{multicols}{3}
  \begin{equation*}
    \begin{matrix}
      f_{yx}(x,y)\\
      \text{ derivata prima rispetto a}\\
      \text{ y poi rispetto a rispetto a x}
    \end{matrix}
  \end{equation*}
  \begin{equation*}
    \begin{matrix}
      f_{yx}(x,y)\\
      f_{yx}(x,y)
    \end{matrix}
    \text{ derivata seconde pure}
  \end{equation*}
  \begin{equation*}
    \begin{matrix}
      f_{yx}\\
      f_{yx}
    \end{matrix}
    \text{ derivata seconde resto}
  \end{equation*}
\end{multicols}
con $n$ variabili si hanno $n^2$ derivate seconde parziali -- Spesso le derivate seconde sono disposte in
una matrice quadrata, detta {\tt hessiana}, con il sinbolo $D^2$
\begin{equation}
  D^2f=\begin{bmatrix}
         f_{xx} & f_{xy}\\
         f_{yx} & f_{yy}
       \end{bmatrix}
       \text{n variabili} \to n*n
\end{equation}
Se esistono le quanto derivate di f, nel punto (x,y), si dice che $f$ è dirivabile due volte in $(x,y)$. Se
ciò accade $\forall (x,y)\in A$, $f$ è derivabile due volte nell'insieme A.
\subsection{Teorema di Schwarz (Dell'invertibilità dell'ordine di derivazione)\label{Schwarz}}
\begin{teorema}
  Sia $f(x,y)$ definita in $D$ e derivabile due volte $\forall (x,y) \in D$.\\
  Se le derivate seconde in $(x_0,y_0)$ $f_{xy}(x_0,y_0)$ e $f_{yx}(x_0,y_0)$ sono continue in ($x_0,y_0$) allora\\
  risulta $f_{xy}(x_0,y_0)=f_{yx}(x_0,y_0)$.
\end{teorema}
In generale se vale il teorema di Schwarz, la matrice Hessiana può essere scritta come
\begin{equation*}
  H=D^2f=\begin{bmatrix}
           f_{xx} & f_{xy}\\
           f_{xy} & f_{yy}
         \end{bmatrix}
         = \begin{bmatrix}
             f_{xx} & f_{yx}\\
             f_{yx} & f_{yy}
           \end{bmatrix}
\end{equation*}
$det H= f_{xx}*f_{yy}-(f_{xy})^2=f_{xx}*f_{yy}-(f_{yx})^2$
\section{Massimi e minimi relativi \label{minmaxrel}}
\begin{defi}
  Sia $f(x,y)$ una funzione definita in un insieme D, un punto $p_0(x_0,y_0)\in D$, si dice di {\color{red}
    massimo relativo} per la funzione se esiste intorno circolare di $P_0$ per cui il valore assunto della
  funzione nei punti dell'interno è minore o uguale a quello assunto in $P_0$.\\
  Analogamente un punto $P_0(x_0,y_0)$ si dice di {\color{red} minimo relativo} per la funzione se esiste un
  interno circolare di $P_0$ per cui il valore assunto dalla funzione nei punti dell'intorno è maggiore o uguale.
  \begin{equation*}
    \begin{matrix}
      \exists I_{(x,y),\delta}:\forall (x,y)\in I_{(x,y),\delta} & f(x_0,y_0)\geq f(x,y) & \text{Massimo relativo}\\
      \exists I_{(x,y),\delta}:\forall (x,y)\in I_{(x,y),\delta} & f(x_0,y_0)\leq f(x,y) & \text{Minimo relativo}
    \end{matrix}
  \end{equation*} 
\end{defi}
\subsection{Teorema di Fermat}
\begin{teorema}
  Sia $f(x,y)$ derinita in D e derivabile in un punto $P_0 (x_0,y_0)$\\
  Se in $P_0(x_0,y_0)$ $f(x,y)$ ha un massimo o un minimo relativo, allora le derivate prime
  parziali si annullano ($\nabla f=0$ gradiente nullo). La pendenza della tangente è zaro un
  massimo o minimo.
\end{teorema}
\subsubsection{Gradiente}
Sia $f(x,y)$ una funzione derivabile in un punto (x,y), cioè esistano in (x,y) le due derivate
parziali $f_x$ e $f_y$.\\
Si definisce {\color{red} gradiente} di f(x,y) nel punto (x,y): i vettore $\nabla f$ le cui componenti sono le derivate parziali di f(x,y).
\begin{equation}
  \nabla f(x,y) \equiv (f_x(x,y); f_y(x,y))
\end{equation}
\subsubsection{Massimi e minimi -- condizione necessaria}
\begin{defi}
  Se $P_0(x_0,y_0)$ è un punto di massimo/minimo relativo il gradiente è nullo. Così di massimo
  o minimo relativo interni al dominio della funzione f vanno ricercati tra i punti che annullano
  la funzione f. Pertanto un punto critico per una funzione derivabile e un punto in cui si
  annulla il gradiente della funzione.
\end{defi}

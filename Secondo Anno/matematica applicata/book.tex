\documentclass{book}

\usepackage[utf8]{inputenc}
\usepackage{titlesec}
\usepackage{easylist}
\usepackage{hanging}
\usepackage{hyperref}
\usepackage[a4paper,top=2.0cm,bottom=2.0cm,left=2.0cm,right=2.0cm]{geometry}
\usepackage{blindtext}
\usepackage{tipa}
\usepackage{epigraph}
\usepackage{enumerate}
\usepackage{longtable}
\usepackage{setspace}
\usepackage{verbatim}
\usepackage[T1]{fontenc}
\usepackage{graphicx}
\usepackage[italian]{babel}
\usepackage{amsmath}
\usepackage{pbox}
\usepackage{fancyhdr}
\usepackage{cancel}
\usepackage{tabularx}
\usepackage{booktabs}
\usepackage{multirow}
\usepackage{longtable}
\usepackage{tikz}
\usepackage{qtree}
\usepackage{tikz-qtree}
\usepackage{subfig}
\usepackage{xcolor}
\usepackage{amssymb}
\usepackage{mathrsfs}
\usepackage{textcomp}
\usepackage{dsfont}


\linespread{1.5} % l'interlinea

\frenchspacing

\newcommand{\abs}[1]{\lvert#1\rvert}

\usepackage{floatflt,epsfig}

\usepackage{multicol}
\newcommand\yellowbigsqcup[1][\displaystyle]{%
  \fboxrule0pt
  \ifx#1\textstyle\fboxsep-0.6pt\else\fboxsep-1.25pt\fi
  \mathrel{\fcolorbox{white}{yellow}{$#1\bigsqcup$}}}

\title{Appunti di Matematica Applicata}
\author{Nicola Ferru}
\begin{document}
\maketitle
\chapter{Introduzione}
La Matematica Applicata serve per svolgere problemi complessi, utilizzando degli algoritmi e calcolatori elettronici,
si utilizzano teoremi e strumenti.
\begin{enumerate}
	\item Introduzione;
	\item Richiami e complementi di algebra lineare;
	\item Trasformata di Fourier;
	\item Metodi diretti per la soluzione dei sistemi lineari;
	\item Matedi iterativi per la soluzione dei sistemi lineari;
	\item Metodi numerici per le equazioni differenziali ordinarie.
\end{enumerate}
\section{tipi di funzioni trattate}
\begin{enumerate}
	\item ODE $\to\text{ } y(x),y^\prime(x), \dots, y^{\prime\prime\prime}(x)$;
	\item PDE $\to w (x,y,z)$
\end{enumerate}
\section{Analisi di Reti complesse: algebra lineare numerica}
\begin{itemize}
	\item reti orientata -- ha un senso unico, e segue un verso, esempio Twitter.
	\item rete non orientata -- i rapporti sono coovalenti e non c'è un verso preciso, esempio Facebook.
\end{itemize}
\section{Definizione di analisi numerica}
La ``missione'' dell'analisi numerica è ottimizzare il software.
\chapter{spazio vettoriale}
uno spazio lineare o vettoriale reale è un insieme V su cui sono definite due operazioni con 10 proprietà
($\forall \alpha \in R$), ($\forall x, y, z \in V$)
\begin{equation*}
  \begin{matrix}
    +:V*V\to V & .=R*V\to V\\
    (x,y) \to x + y & (\alpha,x) \to \alpha x
  \end{matrix}
\end{equation*}
\begin{enumerate}
    \item $x+y\in V$ (chiusura risp. somma)
    \item $\alpha x \in V$ (chiuura risp. prodotto)
    \item $x+y=y+x$ (pr. commutativa)
    \item $(x+y)+z=x+(x+z)$ (pr. associativa)
    \item esiste $0\in V$ tale che $x+0=0$ (eledmento neutro)
    \item esiste $-x \in V$ tale che $x+(-x)=0$ (elemento addizionale)
    \item $\alpha(\beta x)=(\alpha \beta) x$ (pr. associativa)
    \item $\alpha(x+y)=\alpha x+\alpha y$ (pr. distributiva in $V$)
    \item $(\alpha + \beta)x=\alpha x+\beta x$ (pr. distributiva in $\mathds{R}$)
    \item $1x=x$ (elemento neutro)
\end{enumerate}
\section{Norme vettoriali}
\begin{enumerate}
    \item $||x||\geq 0$ (positività)
    \item $||\alpha x||=|\alpha|*||x||$ (omogeneità)
    \item $||x+y||\leq||x||+||y||$ (diseguaglianza triangolare)
\end{enumerate}
Da notare che, per la proprietà 2, $\abs{\abs{-x}}=\abs{\abs{x}}$, il che significa, com'è naturale aspettarsi,
che un vettore e suo opposto hanno la stessa lunghezza. È immediato dimostrare che una norma verifica anche la
disequalianza
\begin{equation}
  \abs{|x-y|}\geq \abs{||x||-||y||},\text{ }\forall x,y\in V.
\end{equation}
\section{Il metodo di ortogonalizzazione di Gram-Schmidt}


\end{document}

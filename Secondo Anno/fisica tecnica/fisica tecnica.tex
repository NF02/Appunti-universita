\documentclass{book}

\usepackage[utf8]{inputenc}
\usepackage{titlesec}
\usepackage{easylist}
\usepackage{hanging}
\usepackage{hyperref}
\usepackage[a4paper,top=2.0cm,bottom=2.0cm,left=2.0cm,right=2.0cm]{geometry}
\usepackage{blindtext}
\usepackage{tipa}
\usepackage{epigraph}
\usepackage{enumerate}
\usepackage{longtable}
\usepackage{setspace}
\usepackage{verbatim}
\usepackage[T1]{fontenc}
\usepackage{graphicx}
\usepackage[italian]{babel}
\usepackage{amsmath}
\usepackage{pbox}
\usepackage{fancyhdr}
\usepackage{cancel}
\usepackage{tabularx}
\usepackage{booktabs}
\usepackage{multirow}
\usepackage{longtable}
\usepackage{tikz}
\usepackage{qtree}
\usepackage{tikz-qtree}
\usepackage{subfig}
\usepackage{xcolor}
\usepackage{amssymb}
\usepackage{mathrsfs}
\usepackage{textcomp}
\usepackage{tasks}

\usepackage{amsthm}

\newtheorem{defi}{Definizione}
\linespread{1.2} % l'interlinea

\frenchspacing

\newcommand{\abs}[1]{\lvert#1\rvert}

\usepackage{floatflt,epsfig}

\usepackage{multicol}
\newcommand\yellowbigsqcup[1][\displaystyle]{%
  \fboxrule0pt
  \ifx#1\textstyle\fboxsep-0.6pt\else\fboxsep-1.25pt\fi
  \mathrel{\fcolorbox{white}{yellow}{$#1\bigsqcup$}}}

\title{Appunti fisica tecnica}
\author{Nicola Ferru}
\begin{document}
\maketitle
\chapter{Introduzione \label{intro}}
Lo scopo del corso è quello di introdurre e applicare i concetti necessari per arrivare a formulare
correttanente l'inquadramento termodinamico di un problema fisico in cui sono coinvolti scambi di calore, di
radiazione, di lavoro, di massa e materia. Tale formulazione deve avvenire in coerenza le approssimazioni e
relative stime, legate alla eventuali semplificazioni del caso in studio, e il fine è quello di stabilire
i rapporti quantitativi di causa ed effetto tra le entità di scambio, calore radiazione termica, lavoro
({\it in tutte le sue forme}) massa e materia e le coordinate termodinamiche ({\it che esprimono lo statto del
  sistema}). La conoscenza di tale legame, di causa ed effetto, permetterà di controllare o prevedere un
determinato processo o sarà utile semplicemente per conoscere le azioni necessarie affinché un sistema si
porti da uno stato iniziale (i) ad uno stato finale (f). 
\section{Il problema dell'inquadramento termico\label{pro.inq.term}}
La formulazione corretta dell'inquadramento termodinamico di un problema fisico è fontamentalmente basato su 4
concetti elementari e sul loro reciproco coordinamento. definiti come segue:
\begin{enumerate}
\item Sistema;
\item Ambiente circostante;
\item Entità di scambio;
\item coordinate termiche.
\end{enumerate}
\begin{defi}
  Il sistem è una porzione di materia su cui ricade il nostro interesse dal punto di vista termodinamico.
  Il sistema si trova in una regione di spazio che limita una qualsivoglia porzione di materia e che noi possiamo
  materialmente o concetualmente separare da tutto ciò che la circonda (un gas contenuto in un cilindro,
  oppure una particolare fluida, nella sua stessa individualità, circondata dal restante fluido a cui appartiene
  e in cui si trova immersa, un collettore solare che separa il fluido termovettore dell'ambiente esterno\dots)
  insomma qualsiasi regione di spazio che confina una porzione di materiale, nel momenbto in cui è oggetto di un
  interesse dal punto di vista termodinamico, individua SISTEMA termodinamico. Quindi il sistema è una
  qualsivoglia porzione di materiale che in un cetro istante $t_0$ occupa una certa regione di spazio e che può
  essere una separazione fisica oppure solo dentro la notra mente.  
\end{defi}
\begin{defi}
  L'Ambiente circostante è tutto ciò che non è sistema e che è esterno ad esso. Ma non è solo questo, orrorre
  prencisare che tutto ciò che non è sistema, per poter avere il suolo di {\it Ambiente Circostante} nei
  confronti del Sistema, deve avere anche un altro requisito. Deve essere in grado, in qualche modo e misura,
  di influinzzare/modificare direttamente lo stato termodinamico e il comportamento dello stesso. Questo
  significa che l'ambiente circostante è operativamente in grado di scambiare con il sistema deteminate grandezze
  che in seno all'inquadramento termodinamico vengono denominate ``Entità di scambio''. Le entità di cambio sono
  grandezze che possono essere colte e si manifestano all'atto dello scambio tra sistemi e ambiente circostante.
\end{defi}
\subsection{Entità di scambio}
\begin{tasks}
    \task Calore
    \task Radiazione terminca;
    \task Lavoro di tipo meccanico ed elettrico
    \task Massa e Materia.
\end{tasks}
\begin{center}
  Quindi una volta elencate queste grandezze quel'è l'ambiente circostante?
\end{center}
L'insieme di tutti i campi dell'universo, che scambiano con il sistema le entità
di scambio prima elencate, rappresenta {\it l'Ambiente Circostante.} L'insieme degli
altri corpi che non scambiano con gli altri sistemi nessuna entità di scambio allora
rapresenta, per il sistema in considerazione, il {\it resto dell'universo}.
\begin{defi}
  Le coordinate termodinamiche sono grandezze in cui valore numerico dipende da proprietà ({\it interne})
  attribuite alla massa/materia che costituisce il sistema stesso, dipendono dalla peculiarità della materia
  di cui è fatta la massa del sistema, e sono suscettibili di essere modificate dall'azione delle 4 entità
  di scambio definite precedentemente:
  \begin{tasks}(2)
    \task Calore
    \task Radiazione terminca;
    \task Lavoro di tipo meccanico ed elettrico
    \task Massa e Materia.
  \end{tasks}
  La definizione, in un dato istante, di una proprietà, coordinata termodinamiche, in quanto tale, tramite le
  sole condizioni che il sistema assume nell'istante considerato, senza dover quindi considerare quale grandezza
  o caratteristica delle condizioni del sistema. Una quantità o grandezza o caratteristica del sistema per
  poter essere determinato ``fotografia'' istantanea\footnote{Campionamento di un istante del sistema
    per poter avere un base di studio solida e delineata} del sistema senza conoscere in che modo il sistema
  è giunto in quella condizione.\\
  L'insieme dei valori assunti dalle proprietà, coordinate termodinamiche, in un dato istante, determinano lo
  stato del sistema in quell'instante, cioè definiscono dove sta il sistema dal sistema in quell'istante.\\
  Il problema che si pone è quello di determinare quali sono quindi le grandezze che dobbiamo introdurre per
  poter dire che stiamo dando una descrizione del comportamento del sistema del punto di vista termodinamico in
  un dato istante.\\
  Il problema che si pone allora è quello di teterminare quali sono le grandezze che dobbiamo introdurre
  per poter dire che stiamo dando una descrizione del comportamento del sistema dal punto di vista termodinamico.
  In questa fase non le possiamo ancora elencare e, in realtà, non sarebbe neanche utile stabilire a prioi un
  elenco universale\footnote{Visto che secondo il caso le grandezze in gioco possono variare}. Infatti, non è come
  la cinematica dove la velocità, accelerazione, posizione, il tempo, massa ed eventualmente. quantità di materia
  sono coordinate universali che descrivono il copro e a prescindere quindi da quali condizioni interne sia
  caratterizzato, stia attraversarsando. In termodinamica non si può prescindere dal cconoscere la natura del
  corpo e a prescindere quindi da quali condizioni interne sia caratterizzato, stia attraversando. {\bf In
    termodinamica non si può prescindere dal conoscere la natura del corpo che costituisce il sistema}.\\
  Non è sufficiente in un problema termodinamico sapere che la massa del sistema è di ``{\tt x kg}'', ma bisogna anche
  sapere di cosa sono fatti gli stessi, occorre sapere se sono kg di azoto o di ossigeno e in quale fase si
  trovano quei kg: gas, liguido, solido, ecc. Ogni sistema ha le sue coordinate termodinamiche che descrivono il
  comportamento del sistema e ne indicano lo stato.
\end{defi}
\clearpage
\subsection{Una sbarra in tensione e in compressione}
\begin{defi}
  gli elementi strutturali sottoposti a tensioni o a compressioni si trovano a pressione atmosferica.
  Inoltre, le variazioni di volume sono di molto spesso il volume e la pressione non vengono annoverate tra la
  coordinate termodinamiche necessarie per descrivere il sistema costituito da una sbarra soggetta a tensioni o
  a a compressione, in quanto non sono suscettibili nel loro impiego di variazioni significative.\\
  Le coordinate che invece tenute in considerazione sono:
  \begin{itemize}
  \item Sforzo $\sigma$ dell'elemento, definito come il carico per unità di sezione $\frac{kN}{m^2}$,
    $\frac{kg_f}{m^2}$, $\frac{lb_f}{m^2}$, positivo nel caso di una tensione e negativo nel caso di una
    compressione.
  \item Deformazione $\varepsilon$, quantità adimensionale che misura l'allugamento relativo dell'elemento
    rispetto alla dimensione originele:
    \begin{itemize}
    	\item $d\varepsilon = \frac{dL}{L}$, $d_\varepsilon$ risulta positivo nel caso di una tensione è
      		negativo nel caso di una tensione.
    \end{itemize}
    \item la temperatura $\theta$ di gas ideale.
  \end{itemize}
  
\end{defi}
\subsection{Cella elettromitica reversibile, cella Daniel}
\begin{tasks}
  \task Fem $\varepsilon$ della cella in volt;
  \task La carica $Q$ in Coulomb;
  \task La temperatura $\varepsilon$ di Gas ideale.
\end{tasks}
Come mostratto in tutti i casi precendenti, la rilevazione di carattere generale è il fatto che la temperatura
{\tt T}, la quantità di materia {\tt M}, la pressione {\tt p} e il volume specifico $v$, sono sicuramente
coordinate sempre della 


\end{document}

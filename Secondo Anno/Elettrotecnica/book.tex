\documentclass{book}

\usepackage[utf8]{inputenc}
\usepackage{titlesec}
\usepackage{easylist}
\usepackage{hanging}
\usepackage{hyperref}
\usepackage[a4paper,top=2.0cm,bottom=2.0cm,left=2.0cm,right=2.0cm]{geometry}
\usepackage{blindtext}
\usepackage{tipa}
\usepackage{epigraph}
\usepackage{enumerate}
\usepackage{longtable}
\usepackage{setspace}
\usepackage{verbatim}
\usepackage[T1]{fontenc}
\usepackage{graphicx}
\usepackage[italian]{babel}
\usepackage{amsmath}
\usepackage{pbox}
\usepackage{fancyhdr}
\usepackage{cancel}
\usepackage{tabularx}
\usepackage{booktabs}
\usepackage{multirow}
\usepackage{longtable}
\usepackage{tikz}
\usepackage{qtree}
\usepackage{tikz-qtree}
\usepackage{subfig}
\usepackage{xcolor}
\usepackage{amssymb}
\usepackage{mathrsfs}
\usepackage{textcomp}
\usepackage{amsthm}

% definizioni e teoremi
\newtheorem{defi}{Definizione}

\linespread{1.5} % l'interlinea

\frenchspacing

\newcommand{\abs}[1]{\lvert#1\rvert}

\usepackage{floatflt,epsfig}

\usepackage{multicol}
\newcommand\yellowbigsqcup[1][\displaystyle]{%
  \fboxrule0pt
  \ifx#1\textstyle\fboxsep-0.6pt\else\fboxsep-1.25pt\fi
  \mathrel{\fcolorbox{white}{yellow}{$#1\bigsqcup$}}}

\title{Appunti di Elettrotecnica}
\author{Nicola Ferru}
\begin{document}
\maketitle
\section{Argomenti}
l'elettrotecnica è la tecnica dell'energia elettrica, cioè le possibili applicazioni degli effetti prodotti
dalle cariche, ferme o in movimento.
\chapter{Circuiti magnetici}
\section{Introduzione}
\begin{defi}
  In elettromagnetismo si definisce la densità di corrente $J$ che misura la quantità di corrente
  che fluisce attraverso l'unità di superficie normale alla direzione del flusso di corrente.
  \begin{equation}
    i=\frac{dq}{dt}\left[\frac{C}{s}\right] = \frac{dq}{dt}][A]
  \end{equation}
\end{defi}
\subsection{Principi di conservazione delle cariche}
\begin{defi}
  Una carica non può essere creata né distrutta, è una legge neturale e la formula è
   \begin{equation}
	\nabla *j + \frac{\partial \rho}{\partial t}=0
   \end{equation}
   Densita di carica ({\it dipendono dalla coordinate spaziali})
   \begin{itemize}
   	\item Volumica: $\partial=\lim\limits_{\Delta v\to 0} \frac{\Delta q}{\Delta v}\left(\frac{C}{m^3}\right)$
     	\item Superficiale: $\partial=\lim\limits_{\Delta x\to 0} \frac{\Delta q}{\Delta s}\left(\frac{C}{m^2}\right)$
        \item Lineare: $\partial=\lim\limits_{\Delta l\to 0} \frac{\Delta q}{\Delta s}\left(\frac{C}{m}\right)$
   \end{itemize}
\end{defi}
\subsection{COSTRUZIONE DI UNA TEORIA}
\begin{itemize}
\item Definire le quantitò base
\item Postulare òe relazioni fondamentali
\item Specificare le regole di operazione
  ({\it cioè la Matematica})
\end{itemize}
\subsection{Teorema dei campi}
\begin{itemize}
\item Quantità basilari: Sorgenti, Campi (La sorgente di un
  campo elettromagnetico è invariabilmente una carica elettrica, a riposo o in moto);
\item Postulati Fondamentali: EQUAZIONI DI MAXWELL;
\item Regole Operative: Calcolo vettoriale.
\end{itemize}
\subsection{Equazioni di Maxwell}
\begin{table}[ht]
  \centering
    \begin{tabular}{|c|c|c|}
        \hline
        Forma Differenziale & forma Integrale &\\\hline
        $\nabla * E = rot E = - \frac{\partial B}{\partial t}$ & $\oint E*dl=-\int_S\frac{\partial D}{\partial t}*dS$ & L. Faraday \\
        $\nabla * H = \bar{J}+ \frac{\partial D}{\partial t}$
        & $\oint H*dl=I+\int_S\frac{\partial D}{\partial t} * dS$
                                                                                                                        & L. Ampére\\
        $\nabla*D=\rho$ & $\oint H*dl=Q$ & L. Gauss\\
        $\nabla*B=0$ & $\oint D*dS=Q$ & L. Gauss\\\hline
    \end{tabular}
\end{table}
\begin{itemize}
    \item Teorema di Stokes: $\int_S(\nabla* A)*dS = \oint A*\bar{Al}$
    \item Teorema della divergenza: $\int_V\nabla*A*dV=\oint_SA*dS$
\end{itemize}
\subsection{Quantità basilari nello studio dei campi}
\begin{table}[ht]
  \centering
    \begin{tabular}{|c|c|c|c|}
      \hline
      campo 	& 	    quantità 		& simbolo & unità \\\hline
      Elettrico & intensità di flusso elettrico &    E    & $\frac{V}{m}$ \\
                & densità di flusso elettrico 	&    D 	  & $\frac{C}{m^2}$\\
      Magnetico & densità di flusso magnetico 	&    B 	  & $T=V*s/m^2$\\
      		& intensità di campo magnetico  &    H	  & $A/m$\\hline
    \end{tabular}
\end{table}
\begin{itemize}
    \item \textit{E} è l'unico vettore necessario per lo studio del campo stazionario nel vuoto
    \item \textit{D} è utile nello studio del campo elettrico in mezzi materiali
    \item \textit{B} è l'unico vettore necessario per lo studio della magnetostatica nel vuoto
    \item \textit{H} è utile nello studio dei campi magnetici nei mezzi materiali.
\end{itemize}
\section {Campo elettrico}
\begin{itemize}
    \item $F=k\frac{Q*q}{r^2}\overrightarrow{r}$ Legge di Coulomb
    \item $E=\frac{F}{q}=k\frac{Q}{r^2}\overrightarrow{r}$ Campo Elettrico
    \item $dL=E*dl$ Lavoro Elementare
    \item $\int^B_A E*dl=V_B-V_A$ Differenza di potenziale
\end{itemize}
\begin{equation*}
  Q=\oint D*dS
\end{equation*}
D= Densità di Flusso Elettrico
\section{Campo Magnetico}

\end{document}
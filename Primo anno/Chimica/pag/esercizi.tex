\part{Esercizi}
\label{part:es}

\chapter{Esercizi Stechiometria}
\label{chap:esstechiometria}
In questo capitolo sono presenti una sequela di esercizi con tanto di
svolgimento allegato, spiegati e commentati. Ordinati in modo pratico
e veloce.
\begin{ess}
  Reazione tra idrogeno e ossigeno per formare l'acqua, dato:
  \begin{equation}
    \label{eq:esstechiometria1}
    2H_2+O_2\to 2H_2O
  \end{equation}
  Calcolare quanti di acqua si formano reagendo completamente con $4g$ di
  idrogeno.
\end{ess}
\begin{proof}[Svolgimento]
  Per svolgere questo esercizio è necessario in primo luogo, Determinare la
  massa molare di ciascuna sostanza:
  \begin{eqnarray*}
    H_2:2\times 1=2\frac{g}{mol} & H_2O: 2\times 1+16=18\frac{g}{mol}
  \end{eqnarray*}
  Fatto questo, è possibile calcolare le moli di $H_2$:
  \begin{eqnarray*}
    H_2=\frac{\text{massa di } H_2}{\text{massa molare di }H_2}=\frac{4}{2}=
    2\mol
  \end{eqnarray*}
  adesso, bisogna svolgere il rapporto molare dalla reazione chimica, infatti,
  da \[\ch{2 H2},\] si vede che 2 moli di \ch{H2} producono 2 moli di
  \ch{H2O}.
\end{proof}

\part{Esercizi}
\label{part:es}

\chapter{Esercizi Stechiometria}
\label{chap:esstechiometria}
In questo capitolo sono presenti una sequela di esercizi con tanto di
svolgimento allegato, spiegati e commentati. Ordinati in modo pratico
e veloce.
\begin{ess}
  \label{ess:esstechiometria1}
  Reazione tra idrogeno e ossigeno per formare l'acqua, dato:
  \begin{equation}
    \label{eq:esstechiometria1}
    2H_2+O_2\to 2H_2O
  \end{equation}
  Calcolare quanti di acqua si formano reagendo completamente con $4g$ di
  idrogeno.
\end{ess}
\begin{proof}[Svolgimento]
  Per svolgere questo esercizio è necessario in primo luogo, Determinare la
  massa molare di ciascuna sostanza:
  \begin{eqnarray*}
    H_2:2\times 1=2\nicefrac{g}{mol}, & H_2O: 2\times 1+16=18\nicefrac{g}{mol}
  \end{eqnarray*}
  Fatto questo, è possibile calcolare le moli di $H_2$:
  \begin{eqnarray*}
    \text{Moli di }H_2=\frac{\text{massa di } H_2}{\text{massa molare di }H_2}=\frac{4}{2}=
    2mol
  \end{eqnarray*}
  adesso, bisogna svolgere il rapporto molare dalla reazione chimica, infatti,
  da \[\ch{2 H2 + O2 -> 2 H2O},\] si vede che 2 moli di \ch{H2} producono 2 moli di
  \ch{H2O}. Per calcolare la massa di \ch{H2O} è necessario moltiplicare la mole di \ch{H2O} per la massa molare
  di \ch{H2O}:
  \begin{eqnarray*}
    \text{Massa di } H_2O=\text{moli di } H_2O \times \text{massa molare di } H_2O=2\times 18 =36g
  \end{eqnarray*}
  Quindi, si può dedurre che si formano $36g$ di acqua.
\end{proof}
\begin{ess}
  \label{ess:esstechiometria2}
  Reazione tra ferro e cloro per formare il cloro di ferro
  \begin{equation}
    \label{eq:esstechiometria2}
    2Fe+3Cl_2\to 2FeCl_{3}
  \end{equation}
  Calcolare quanti grammi di \ch{Cl2} sono necessari per reagire completamente con $11.2g$ di ferro.
\end{ess}
\begin{proof}[Svolgimento]
  Per svolgere questo esercizio è necessario in primo luogo, definire la massa molare di ciascuna sostanza:
  \begin{eqnarray*}
    Fe: 55.85\nicefrac{g}{mol} & Cl_2:2\times 35.45=70.9\nicefrac{g}{mol}
  \end{eqnarray*}
  Adesso bisogna calcolare la mole, facendo un rapporto massa molare di \ch{Fe} su massa di \ch{Fe}:
  \begin{eqnarray*}
    \text{Moli di } Fe=\frac{\text{massa di } Fe}{\text{massa molare di } Fe}=\frac{11.2}{55.85}\approx 0.2mol
  \end{eqnarray*}
  Un altro punto bisogna effetturare il rapporto molare dalla reazione chimica, Da \ch{2 Fe + 3 Cl2 -> 2 Fe Cl3}, si
  vede che 2 moli di \ch{Fe} reagiscono con 3 moli di \ch{Cl2}. Quindi:
  \begin{eqnarray*}
    \text{Moli di } Fe=\frac{\text{massa di } Fe}{\text{massa molare di } Fe} = \frac{11.2}{55.85} = 0.3\text{mol di }
    Cl_2
  \end{eqnarray*}
  e in fine basta calcolare la massa di \ch{Cl2}:
  \begin{eqnarray*}
    \text{Massa di } Cl_2=\text{moli di } Cl_2\times \text{massa molare di } Cl_2=0.3\times 70.9\approx 21.26g
  \end{eqnarray*}
  la conclusione è che ci sono circa $21.27g$ di \ch{Cl2}
\end{proof}

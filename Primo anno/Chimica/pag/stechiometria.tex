\chapter{Stechiometria}
\label{chap:stechiometria}
\begin{defi}
  La stechiometria è una branca della chimica che studia i rapporti quantitativi delle sostanze chimiche nelle
  reazioni chimiche. La stechiometria di reazione indica in che rapporto due o più sostanze reagiscono tra di loro.
  Essa viene rappresentata attraverso coefficienti, detti coefficienti stechiometrici.
\end{defi}
Per avere una rappresentazione quantitativa è fondamentale \underline{BILANCIALRE} correttamente l'equazione usando
gli opportuni \texttt{coefficienti stechiometrici}.
\begin{center}
 \ce{{\color{red}2}NH_3 \to N2 + {\color{red}3}H2}
\end{center}
I \textit{\color{red} coefficienti stechiometrici} sono numuri interi che indicano quante moli (o molecole, atomi
oppure ioni) reagiscono (se si tratta di un reagente) o si formano (se si tratta di un prodotto) durante la reazione
chimica.

\section{Bilanciamento di un'equazione chimica}
\label{sec:bildiuneqchimica}

\begin{defi}
  Le trasformazioni chimiche rispettano la \texttt{legge di conservazione della massa} o
  \texttt{legge di Lavoisier} (\ref{sec:legdiconse}). Per bilanciare un'equazione chimica si deve
  avere che:
  \begin{itemize}
  \item per ciascun elemento, gli atomi presenti tra i reagenti devono essere in numero
    \texttt{UGUALE} a quelli presenti tra i prodotti;
  \item se la reazione coinvolge specie ioniche, la somma delle cariche elettriche dei reagenti,
    deve essere uguale a quella dei prodotti
  \end{itemize}
\end{defi}

\subsection{Metodo per tentativi successivi}
\label{sec:metpertentsucc}

Si basa su una attenta osservazione dell'equazione non bilanciata:
\begin{center}
  \ce{Ca(OH)2 + H3PO4 \to Ca3(PO4)2 + H2O}
\end{center}
Si attribuiscono i coefficienti stechiometrici facendo in modo che il numero e il tipo degli atomi
presenti tra i reagenti, siano uguali a quelli presenti a destra della freccia:
\begin{center}
  \ce{{\color{red}3}Ca(OH)2 + {\color{red}2} H3PO4 \to Ca3(PO4)2 + {\color{red}6}H2O}
\end{center}
\begin{ess}
  Calcolare le moli e i grammi di \ce{NH3} che si formano quando si fanno reagire 42.00 grammi \ce{N2}
  con 9.09g di \ce{H2}.
  \begin{center}
    \ce{N2 + H2 \to NH3} Equazione da bilanciare\\
    \ce{N2 + {\color{red}3}H2 \to {\color{red}2}NH3} Equazione bilanciata
  \end{center}
  \begin{itemize}
  \item Massa molare di \ce{N2}: $2\cdot 14.00g\cdot mol^{-1}=28.00g\cdot mol^{-1}$ e quindi la mole di \ce{N2}: $\frac{42.00g}{28.00g}\cdot mol^{-1}=1.50 \text{ mol di \ce{N2}}$
  \item Massa molare \ce{H2}:  $2\cdot 1.01g\cdot mol^{-1}=2.02g\cdot mol^{-1}$ e quindi la mole di \ce{H2}: $\frac{9.09g}{2.02g}\cdot mol^{-1}=4.50$ mol di \ce{H2}
  \item Massa molare \ce{NH3}: $3\cdot 1.01g\cdot mol^{-1}=3.03g\cdot mol^{-1}+ 1\cdot 14.00g\cdot mol^{-1}= 14.00g\cdot mol^{-1} = MM_{\ce{NH3}}=3.03+14.00=17.03g\cdot mol^{-1}$
  \end{itemize}
  \begin{center}
    \begin{tabular}[th!]{lllllll}
      & \ce{N2} & + & \ce{{\color{red}3}H2} & \textrightarrow & \ce{{\color{red}2}NM3} & Equazione bilanciata\\
      moli iniziali & 1.50 && 4.50 && 0\\
      moli reagite & 1.50 && 4.50\\
      moli finali & 0.00 && 0.00 && 3.00
    \end{tabular}
  \end{center}
  Calcoliamo le moli di \ce{NH3} ottenute:
  \begin{equation*}
    n^0 \text{mole prodotto} = \frac{\text{coefficiente stechiometrico prodotto}}{\text{coefficiente stechiometrico reagente}} \times \text{moli effettivi di reagente}
  \end{equation*}
  Rispetto a \ce{N_2}: \ce{n^0} moli \ce{NH3}=$\frac{2}{1}\times 1.50=3.00$
  \begin{equation*}
    \text{grammi } \ce{NH3}= \frac{17.03g}{1mol}\times 3.00 \text{mol (moli ottenute)}=51.09g \text{di } \ce{NH3}
  \end{equation*}
\end{ess}
\begin{ess}
  Calcolare le moli di \ce{NH3} che si formano quando si fanno reagire 21.00 grammi \ce{N2} con 3.03g
  di \ce{H2}.
  \begin{center}
    moli \ce{N2}: $\frac{21.00g}{28.00g\cdot mol^{-1}}=0.75\text{mol di \ce{N2}}$\\
    moli \ce{H2}: $\frac{3.03g}{2.02g\cdot mol^{-1}}=1.50\text{mol di \ce{H2}}$\\
    \begin{tabular}[th!]{lllllll}
      &\ce{N2} & + & {\color{red}3}\ce{H2} &\textrightarrow & {\color{red}2}\ce{NH3}\\
      Moli iniziali & 0.75 && 1.50 && 0
    \end{tabular}
  \end{center}
  Per stabilire qual è il reagente limitante, calcoliamo quante moli di \ce{H2} servono per poter
  consumare 0.75 moli di \ce{N2}:
  \begin{eqnarray*}
    1\cdot (mol \ce{N2}):3\cdot (mol \ce{H2})= 0.75:X \text{ da cui } X = \frac{3\cdot 0.75}{1}= 2.25
  \end{eqnarray*}
  Quindi per far reagire completamente \ce{N2} è necessaria una mole di \ce{H2} pari a 2.25, dal
  momento che le moli di \ce{H2} disponibili sono solo 1.50, si deduce che:
  \begin{center}
    \ce{H2} è il \underline{reagente limitante}, \ce{N2} quello in \underline{eccesso}\footnote{Coefficiente stechiometrico}
  \end{center}
  Le moli di prodotti vanno calcolate considerando le moli del \underline{reagente limitante}.
  \begin{eqnarray*}
    3(mol \ce{H2}):2(mol \ce{NH3})=1.50:X \text{ da cui } X=\frac{2\cdot 1.50}{3}=1.00\text{mol di \ce{NH3}}
  \end{eqnarray*}
  Anche le moli non reagite del reagente in eccesso vanno calcolate considerando quelle del
  \underline{reagente limitante}.
  \begin{eqnarray*}
    3(mol \ce{H2}):1(mol \ce{NH3})=1.50:X \text{ da cui } X=\frac{1\cdot 1.50}{3}=0.50\text{mol di \ce{N2}}
  \end{eqnarray*}
  che hanno reagito con 1.50 moli di \ce{H2} -- Dopo aver calcolato ed analizzato la situazione possiamo dire e definire con una semplice sottrazione
  che la mole in eccesso è:
  \begin{eqnarray*}
    \text{moli in eccesso} = \text{moli effettive} - \text{moli reagente}= 0.75-0.50 = 0.25 mol
  \end{eqnarray*}
\end{ess}
\begin{ess}
  Calcolare le moli e i grammi di \ce{Al2S3} che si formano quando si fanno reagire 6.0 moli di
  \ce{Al} con 10.0 moli di \ce{S}.
  \begin{center}
    \begin{tabular}[th!]{llllllll}
      &\ce{Al} &+ & \ce{S} & \textrightarrow& \ce{Al2S3} & Equazione da bilanciare\\
      &2\ce{Al} & + & 3\ce{S} & \textrightarrow & \ce{Al2S3} & Equazione bilanciare\\
      moli iniziali&6.0 & 10.0 & & 0
    \end{tabular}
  \end{center}
  Per stabilire se c'è un reagente limitante, calcoliamo quante moli di \ce{S} servono per poter consumare 6.0 moli di \ce{Al}:
  \begin{eqnarray*}
    2\cdot (mol \ce{Al}):3\cdot (mol \ce{S}) = 6.0:X \text{ da cui } X= \frac{3\cdot 6.0}{2}=9.0
  \end{eqnarray*}
  Quindi bastano 9.0 moli di \ce{S} per far reagire completamente 6.0 moli di \ce{Al}. Dal momento che
  le moli di \ce{S} disponibili sono 10.0, l'alluminio è il reagente limitante. Ora, partendo dall'equazione bilanciata, possiamo ricavare i seguenti rapporti molari:
  \begin{equation*}
    \begin{matrix}
      \text{Moli iniziali} & 6.0 & 10.0 & 0.0\\
      \text{Moli reagente} & 6.0 & \nicefrac{1}{2}\cdot 6.0 = 3.0\\
      \text{Moli finali} & 0.0 & 10.0-9.0=1.0 & \nicefrac{1}{2}=3.0
    \end{matrix}
\end{equation*}
  Si formano 3.0 moli di \ce{Al2S3}. La massa molare di questo composto è di $150\nicefrac{g}{mol}$.
  Sono stati ottenuti:
  \begin{eqnarray*}
    (3.0\text{ mol di } \ce{Al2S3}) \cdot (150\nicefrac{g}{mol}) = 450g\text{ di \ce{Al2S3}}
  \end{eqnarray*}
\end{ess}

\section{Resa o rendimento di una reazione}
\label{sec:resaorendimentodiunarea}

Non sempre le reazioni procedono fino a \textit{completamento}, fino a quando i reagenti (almeno quello
limitente\footnote{Il reagente limitante è quello presente in minor quantità quindi limita la massima
  reazione effettuabile in un composto}) sono completamente convertiti nei prodotti. Inoltre, s possono
avere delle reazioni secondarie che portano i reagenti a trasformarsi in prodotti diversi da quelli
riportati nell'equazione.
\begin{center}
  \it In questi casi la quantità di prodotti ottenute, risultano inferiori alle quantità teoriche (stechiometriche).
\end{center}
\begin{defi}
  \label{defi:resarendimento}
  Si definisce \texttt{reso} o \texttt{rendimento percentuale} $(r)$ di una reazione chimica il rapporto
  (moltiplicato per cento) il rapporto tra la quantità di prodotto effettivo ottenuto \ce{Q_e} e quella
  che si sarebbe avuta se la reazione fosse andata a completamento (resa teorica \ce{Q_t}):
  \begin{equation}
    \label{eq:resa}
    r=\frac{Q_e}{Q_t}\cdot 100
  \end{equation}
\end{defi}
\begin{ess}
  Facendo gorgogliare un eccesso di \ce{Cl2} in una soluzione contenente 176.6g di \ce{MgBr2}, si
  ottiene 135.0g di \ce{Br2}. Qual'è la resa percentuale in \ce{Br2}?
  \begin{center}
    \ce{MgBr2 + Cl2 \to MgCl2 + Br2} L'equazione è bilanciata
  \end{center}
  \begin{description}
  \item[Peso molecolare \ce{MgBr2}:] $Mg: 1mol\cdot 24.3g\cdot mol^{-1}=24.3g$;\\
    $Br:2mol\cdot 79.9g\cdot mol^{-1}=159.8g$;\\
    $Totale: 24.3g+159.8g=184.1g$\\
    Massa molare = $184.1g\cdot mol^{-1}$
  \item[mole \ce{MgBr2}:] $\nicefrac{176.6g}{184,1g\cdot mol^{-1}}=0.9593\text{ mol di \ce{MgBr2}}$
  \item[grammi di \ce{Br2} corrispondenti a una resa del 100\%] $0.9593 mol \cdot 159.8g\cdot mol^{-1}=153.3g\text{ di \ce{Br2}}$ 
  \end{description}
  Ma visto che la resa del 100\% non esiste andiamo a calcolarci la resa reale del composto, andando ad
  utilizzare la formula vista in (\ref{eq:resa}), Nella reazione si ottengono 135.0g di \ce{Br2}, di
  conseguenza la resa percentuale è:
  \begin{equation*}
    \frac{135.0g}{153.3g}\cdot 100=88.1\%
  \end{equation*}
  Quindi abbiamo ottenuto una resa del 88.1\%
\end{ess}
\begin{ess}
  Si ricercano attivamente composti in grado di immagazzinare idrogeno da impiegare come combustibile
  per veicoli. Una delle reazioni studiate è:
  \begin{center}
    \ce{Li3N_{(s)} + H_{2(g)} \to LiNH_{2(s)} + LiH_{(s)}}
  \end{center}
  \begin{tasks}
    \task Quante moli di idrogeno sono presenti necessarie per reagire con 1.5mg di \ce{Li3N}?
    \task Calcolare la massa di \ce{Li3N} che produrrebbe 0.65 mol di \ce{LiH}
  \end{tasks}
  \begin{description}
  \item[Bilanciamo l'equazione:] \ce{Li3N_{(s)} + 2H_{2(g)} \to LiNH_{2(s)} + 2LiH_{(s)}}
  \item[Massa molare \ce{Li3N}:]
    \begin{eqnarray*}
      Li: 3 mol \cdot 6.94g\cdot mol^{-1}= 20.82g;\\
      N: 1mol \cdot 14.01g\cdot mol^{-1}=14.01g;\\
      Totale: 20.82g+14.01g=34.83g
    \end{eqnarray*}
    Massa molare= $34.83g\cdot mol^{-1}$
  \item[Massa Molare:]
    \begin{eqnarray*}
      H_2: 2.02g\cdot mol^{-1};\\
      LiNH_2: 22.97g\cdot mol^{-1};\\
      LiH: 7.95g\cdot mol^{-1}
    \end{eqnarray*}
  \item[Trasformiamo i grammi in moli:]
    \begin{eqnarray*}
      \text{$n^0$moli } \ce{Li3N}=\frac{\text{quantità di sostanza (g)}}{\text{massa molare } (g\cdot mol^{-1})}=
      \frac{1.5\cdot 10^{-3}g}{34.83g\cdot mol^{-1}}=4.3\cdot 10^{-5}mol
    \end{eqnarray*}
  \end{description}
  \begin{center}
    \begin{tabular}[ht!]{llllllllll}
      & \ce{Li3N_{(s)}} & + & 2\ce{H_{2(s)}} & \textrightarrow & \ce{LiNH_{2(s)}} & + & 2\ce{LiH_{(s)}}\\
      Moli iniziali & $4.3\cdot 10^{-5}$ && ? && 0 &&0\\
      Moli reagite & $4.3\cdot 10^{-5}$ && $2\cdot 4.3 \cdot 10^{-5}= 8.6\cdot 10^{-5}$\\
      Moli finali & 0.0 && 0.0 && $4.3\cdot 10^{-5}$ && $8.6\cdot 10^{-5}$ 
    \end{tabular}
  \end{center}
  Il rapporto stechiometrico tra \ce{Li3N} e \ce{LiH} è di 1 a 2, perciò per ottenere 0.65 mol di
  \ce{LiH} sono necessarie $\frac{0.65}{2}=0.32mol$ di \ce{Li3N}, che corrispondono a $0.32mol \cdot 34.83g\cdot mol^{-1}=11.15g$.
\end{ess}
\begin{ess}
  Il carburo di calcio, \ce{CaC2}, reagisce con acqua formando idrossido di calcio e il gas infiammabile
  etino.
  \begin{itemize}
  \item Qual'è il reagente limitante se 100.00g di acqua reagiscono con 100.00g di carburo di calco?
  \item Quale massa di etino è possibile produrre?
  \item Quale massa del reagente eccedente residuerebbe una volta completamenta la reazione?
  \item Quale sarebbe la resa percentuale della reazione se si ottenessero 23.20g di etino?
  \end{itemize}
  Come al solito il primo passo è proprio bilanciare l'equazione stechiometrica:
  \begin{center}
    \ce{CaC2 + 2H20_{(l)} \to Ca(OH)_{2(aq)} + C2H2_{(g)}}
  \end{center}
  \begin{description}
  \item[Massa molare \ce{CaC2}:]
    \begin{eqnarray*}
      Ca: 1mol \cdot 40.08g\cdot mol^{-1}= 40.08g;\\
      C: 2mol \cdot 12.01g\cdot mol^{-1}=24.02g;\\
      Totale: 40.08g+24.02g=64.10g
    \end{eqnarray*}
    Massa Molare = $64.10g\cdot mol^{-1}$
  \item[Massa molare:]
    \begin{eqnarray*}
      \ce{H2O}: 18.02g\cdot mol^{-1};\\
      \ce{Ca(OH)2}: 74.10g\cdot mol^{-1};\\
      \ce{C2H2}: 26.04g\cdot mol^{-1}
    \end{eqnarray*}
  \item[Trasformiamo i grammi in moli:]
    \begin{eqnarray*}
      n^omoli\text{ \ce{CaC2}}= \frac{\text{quantità di sostanza (g)}}{\text{massa molare }(g\cdot mol^{-1})} = \frac{100.00g}{64.10g\cdot mol^{-15}}=1.56mol\\
      n^omoli\text{ \ce{H2O}}=5.55mol
    \end{eqnarray*}
  \end{description}
  \begin{tabular}[ht!]{llllllllll}
    & \ce{CaC2_{(s)}} & + & 2\ce{H2O_{(l)}} & \textrightarrow & \ce{Ca(OH)2_{(aq)}} & + & \ce{C2H2_{(g)}}\\
    moli iniziali & 1.56 && 5.55 && 0 && 0\\
    moli reagite & 1.56 && $2\cdot 1.56 = 3.12$\\
    moli finali & 0.00 && 5.55-3.12= 2.43 && 1.56 && 1.56
  \end{tabular}
  \begin{center}
    \ce{CaC2} è il \underline{reagente limitante}, \ce{H2O} quallo in \underline{eccesso}
  \end{center}
  \begin{description}
  \item[Grammi di \ce{C2H2} corrispondenti a una resa del 100\%:]
    \begin{equation*}
      1.56mol\cdot 26.04g\cdot mol^{-1}=40.62g \text{ di \ce{C2H2}}
    \end{equation*}
    
  \item[Grammi di \ce{H2O} in eccesso:]
    \begin{equation*}
      2.43mol\cdot 18.02g\cdot mol^{-1}=43.79g \text{ di H2O}
    \end{equation*}
  \end{description}
  Se nel corso della reazione si ottenessero {\tt 23.20g di etino}, la resa percentuale sarebbe:
  \begin{equation*}
    \frac{23.20g}{40.62g}\cdot 100 = 57.11\%
  \end{equation*}
\end{ess}
\newpage
\begin{ess}
  Sulla base della seguente reazione (da bilanciare), stabilire qual'è il reagente limitante e calcolare
  quanti grammi di \ce{AgBr} e di \ce{Ca(No3)2} si formano per reazione di 8.60g di \ce{AgNo3} con 9.30g
  di \ce{CaBr2}.
  \begin{center}
    \ce{AgNO3 + CaBr2 \to Ca(NO3)2 + AgBr}
  \end{center}
  \begin{enumerate}
  \item Bilanciamo l'equazione:
    \ce{2AgNO3 + CaBr2 \to Ca(NO3)2 + 2AgBr}
  \item Calcoliamo la masse molari dei reagenti:
    \begin{eqnarray*}
      M.M._{\ce{AgNO3}}=107.87+14.01+3\cdot 16.00=169.88 \nicefrac{g}{mol}\\
      M.M._{\ce{CaBr2}}=199.81 \nicefrac{g}{mol}
    \end{eqnarray*}
  \item Calcoliamo le moli di reagenti:
    \begin{eqnarray*}
      \text{moli di \ce{AgNO3}} = \frac{8.60g}{169.88\nicefrac{g}{mol}}=0.051mol\\
      \text{moli di \ce{CaBr2}} = \frac{9.30g}{199.81\nicefrac{g}{mol}}=0.046mol
    \end{eqnarray*}
  \item Individuiamo il reagente limitante:
    \begin{description}
    \item[metodo a:] per ciascun reagente calcoliamo il rapporto tra le moli effettivamente presenti
      nell'ambiente della reazione e il coefficente stechiometrico
      \begin{eqnarray*}
        R_{\ce{AgNO3}}=\frac{moli_{\ce{AgNO3}}}{coefficente_{\ce{AgNO3}}}=\frac{0.051}{2}=0.026\\
        R_{\ce{CaBr2}} = \frac{moli_{\ce{CaBr2}}}{coefficente_{\ce{CaBr2}}}=\frac{0.046}{1}=0.046
      \end{eqnarray*}
      Confrontando i due rapporti risulta $R_{\ce{CaBr2}}>R_{\ce{AgNO3}}$; quasto significa che, rispetto
      ai rapporti stechiometrici, le moli di \ce{CaBr2} \underline{effettivamente presenti} sono in
      ECCESSO rispetto a quelle che sarebbero necessarie per far reagire completamente l'\ce{AgNO3},
      che pertanto risultano risulta essere il \underline{reagente limitante}.
    \item[metodo b:] dalla proporzione
      \begin{eqnarray*}
        2:1=0.051\text{mol di \ce{AgNO3}}:\text{x mol di \ce{CaBr2}}
      \end{eqnarray*}
      calcoliamo quante moli di \ce{CaBr2} sono necessarie per consumare $0.051mol$ di \ce{AgNO3}; si
      ottiene $x=\frac{0.051}{2}=0.026mol$. Dal momento che moli di \ce{CaBr2} disponibili sono 0.046,
      si deduce che il \ce{CaBr2} è presente in ECCESSO e che \ce{AgNO3} risulta essere il
      \underline{reagente limitante}.
    \end{description}
  \item Come in ogni caso andiamo a calcolare le moli finali
    \begin{center}
      \begin{tabular}[ht!]{llllllllll}
        & \ce{AgNO3} & + & 2\ce{CaBr2} & \textrightarrow & \ce{Ca(NO3)2} & + & 2\ce{2AgBr}\\
        moli iniziali & 0.051 && 0.046 && 0 && 0\\
        variazione & -0.051 && -0.026 && +0.026 && +0.051\\
        moli finali & 0.00 && 0.020 && 0.026 && 0.051 
      \end{tabular}
    \end{center}
    Le moli dei prodotti sono state calcolate partendo da quelle del reagente limitante e considerando i
    rapporti stechiometrici tra questo e ciascuno dei prodotti. Nel caso del \ce{Ca(NO3)2} abbiamo:
    \begin{eqnarray*}
        2:1=0.051\text{mol di \ce{AgNO3}}:\text{x mol di \ce{CaBr2}}
    \end{eqnarray*}
    da cui possimo calcolare moli di \ce{CaBr2} possiamo ottenere da 0.051 moli di \ce{AgNO3}:
    $x=\frac{0.051}{2}=0.026$ mol di \ce{CaBr2}. Analogamente, impostando la proporzione opportuna, può
    essere calcolato il numero di moli dell'altro reagente.
  \item Calcoliamo le masse molari dei prodotti:
    \begin{eqnarray*}
      M.M._{\ce{Ca(NO3)2}}=164.10 \nicefrac{g}{mol}\\
      M.M._{\ce{agbr}}=187.77 \nicefrac{g}{mol}
    \end{eqnarray*}
  \item Calcoliamo le masse in grammi dei prodotti:
    \begin{eqnarray*}
      \text{g di } \ce{Ca(NO3)2} = 0.026mol\cdot 164.10 \nicefrac{g}{mol}=4.3g\\
      \text{g di } \ce{AgNO3} = 0.051mol\cdot 187.77\nicefrac{g}{mol}=9.6g
    \end{eqnarray*}
  \end{enumerate}
\end{ess}

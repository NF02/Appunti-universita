\part{Analisi 2}
\chapter{Introduzione}

\section{limiti}
\begin{defi}
  il limite finito è il limite di f(x,y) per $(x,y)\to (x_0,y_0)$ $(\lim\limits_{(x,y)\to(x_0,y_0)} f(x,y)=L)$ se $\forall\xi >0: \forall (x,y)\in I(x_0,y_o)/\left\{x_0,y_0\right\}$
\begin{equation*}
  \abs{\underbrace{f(x,y)-L}_{\text{\shortstack{Distanza}}}}<\xi
\end{equation*}
se $(x_0,y_0)\xi D(f)$ $(\exists f(x_0,y_0))$
\begin{equation*}
  \boxed{\lim_{(x,y)\to (x_0,y_0)} f(x,y)\underbrace{f(x,y)-L}_{\text{\shortstack{Continua}}}}
\end{equation*}

\end{defi}
\begin{esempio}
  \begin{equation*}
    \begin{matrix}
      \lim_{(x,y)\to +\infty}(x^2+y^2)e^{x^2+y^2}
      \end{matrix}
  \end{equation*}
\end{esempio}
\subsection{Teoremi}
\begin{enumerate}
\item Unicità del limite
  \begin{equation*}
    \abs{l_2-l_1}\leq \abs{}
  \end{equation*}
\item Teorema del confronto
\item teorema della permanenza del segno
\item teorema della composizione delle funzioni continue\footnote{se si vanno a comporre delle funzioni utilizzando due o più funzioni continue si otterrà sempre una funzione continua}
\end{enumerate}
in generale i teoremi sulle operazioni
\begin{esempio}
  Dato $f(x,y)=\begin{cases} x^2+y^2\\ k&(x,y)=(0,0)\end{cases}$
  \begin{equation*}
    \begin{matrix}
      \lim_{(x,y)\to(0,0)}\frac{x^3}{x^2+y^2}=\begin{cases}
                                              x=\rho \cos y\\
                                              y=\rho \sin y
                                            \end{cases}\\
      \lim_{}
    \end{matrix}
  \end{equation*}
 
  
\end{esempio}
\chapter{Esercizi svolti}
\section{Teorema di Gauss e Stokes}
\begin{esercizio}
  Usando il teorema di Stokes, calcolare il flusso del rotore del campo vettoriale \textbf{F:}$\mathds{R}\to \mathds{R}^3$ definito da
  \begin{equation*}
	F(x,y,z)=\left(-x^2y, x^3+z^2,\arctan e^{x+y+s}\right)
  \end{equation*}
  attraverso la superficie
  \begin{equation*}
    \sum:\begin{cases}
           x^2+y^2+z^2=4\\
           z\geq 0
         \end{cases}
  \end{equation*}
  orientata secondo i versori uscenti dall'origine.
\end{esercizio}
\begin{svol}
  Il teorema di Stokes assicura che il flusso $\phi_\Sigma$ (rot $F$) del rotore di $F$ attraverso la calotta
  orientata $\sum$ coincide con il lavoro del campo $F$ lungo il bordo $\Gamma(\sum)$ di $\sum$ orientato
  coerentemente con $\sum$ (cioè secondo il verso di un osservatore che, disposto come il campo normale che
  orienta $\sum$, percorre $\Gamma(\sum)$ vedendo $\sum$ alla sua sinistra)\\
  La superfocoe $\Sigma$ è la semisfera di contro l'origine e raggio 2 contenuta nel semispazio $z\geq 0$ e
  dunque il suo bordo $\Gamma =\Gamma (\Sigma)$ è la circonferenza del piano $xy$ di centro l'origine e raggio 2,
  che ammette la rappresentazione parametrica
  \begin{multicols}{2}
    \begin{equation*}
      \Gamma:
      \begin{cases}
        x=2\cos t\\
        y=2\sin t, & t\in [0,2\pi],\\
        z=0
      \end{cases}
    \end{equation*}
  \end{multicols}
  Tale rappresentazione risulta coerente con l'oriantamento di $\sigma$, in quanto, al crescere di $t$,
  il punto $\gamma= (2\cos t, \sin t, 0)$ si muove lungo $\Gamma$ come in figura. Dunque, poiché
  \begin{equation*}
    F(\gamma(t))=F(2\cos t, 2\sin t,0)=(-8\cos^2t\sin t, 8\cos^3t, \arctan e^{2\cos t+2\sin t}),
    \gamma^\prime (t)=(2\sin t, 2\cos t, 0), 
  \end{equation*}
  si ha
  \begin{equation*}
    \phi(rot F)= \int_\Sigma F*dP=\int^{2x}_0F(\gamma(t)) *\gamma(t)dt=\int (16\cos^2t+16\cos^4t)dt =
    16\int^{2x}_{0}\cos^2tdt=16\left[\frac{t+\cos t \sin t}{2}\right]^{2x}_0=16x.
  \end{equation*}
\end{svol}
\clearpage
\begin{esercizio}
  Calcolare il lavoro del campo vettoriale
  \begin{equation*}
    F(x,y,z)=(\sin(x^2+z)- 2yz, 2xy+\sin(y^2+z),\sin(x^2+y^2))
  \end{equation*}
  lungo la circonferenza
  \begin{equation*}
    \begin{cases}
      x^2+y^2=1\\
        z=3
    \end{cases}
  \end{equation*}
  percorso in modo che la proiezione sul piano $xy$ giri in senso orario (rispetto ad un osservatore disposto come l'asse $z$).\\
\end{esercizio}
\begin{svol}
  Vista l'espressione del campo, il calcolo dipende dal lavoro richiesto non pare agevole. D'altra parte, risulta
  \begin{equation*}
    \begin{matrix}
	rot F= \begin{vmatrix}
            i & j & k\\
            \frac{\partial}{\partial x} & \frac{\partial}{\partial y} &  \frac{\partial}{\partial z}\\
            \sin(x^2+z)-2yz & 2xz+\sin(y^2+z) & \sin (x^2+y^2)
        \end{vmatrix}
    \end{matrix}
  \end{equation*}
    
\end{svol}

  
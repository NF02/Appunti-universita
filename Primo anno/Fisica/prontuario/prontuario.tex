\documentclass{article}

\usepackage[utf8]{inputenc}
\usepackage{titlesec}
\usepackage{easylist}
\usepackage{hanging}
\usepackage{hyperref}
\usepackage[a4paper,top=2.0cm,bottom=2.0cm,left=2.0cm,right=2.0cm]{geometry}
\usepackage{blindtext}
\usepackage{tipa}
\usepackage{epigraph}
\usepackage{enumerate}
\usepackage{longtable}
\usepackage{setspace}
\usepackage{verbatim}
\usepackage[T1]{fontenc}
\usepackage{graphicx}
\usepackage[italian]{babel}
\usepackage{amsmath}
\usepackage{pbox}
\usepackage{fancyhdr}
\usepackage{cancel}
\usepackage{tabularx}
\usepackage{booktabs}
\usepackage{multirow}
\usepackage{longtable}
\usepackage{tikz}
\usepackage{tikz-qtree}
\usepackage{subfig}
\usepackage{xcolor}
\usepackage{amssymb}
\usepackage{mathrsfs}
\usepackage{textcomp}

\usepackage{listings}
\usepackage{color}

\definecolor{mygreen}{rgb}{0,0.6,0}
\definecolor{mygray}{rgb}{0.5,0.5,0.5}
\definecolor{mymauve}{rgb}{0.58,0,0.82}

\lstset{ 
  backgroundcolor=\color{white},   % choose the background color; you must add \usepackage{color} or \usepackage{xcolor}; should come as last argument
  basicstyle=\footnotesize,        % the size of the fonts that are used for the code
  breakatwhitespace=false,         % sets if automatic breaks should only happen at whitespace
  breaklines=true,                 % sets automatic line breaking
  captionpos=b,                    % sets the caption-position to bottom
  commentstyle=\color{mygreen},    % comment style
  deletekeywords={...},            % if you want to delete keywords from the given language
  escapeinside={\%*}{*)},          % if you want to add LaTeX within your code
  extendedchars=true,              % lets you use non-ASCII characters; for 8-bits encodings only, does not work with UTF-8
  firstnumber=1000,                % start line enumeration with line 1000
  frame=single,	                   % adds a frame around the code
  keepspaces=true,                 % keeps spaces in text, useful for keeping indentation of code (possibly needs columns=flexible)
  keywordstyle=\color{blue},       % keyword style
  language=Octave,                 % the language of the code
  morekeywords={*,...},            % if you want to add more keywords to the set
  numbers=left,                    % where to put the line-numbers; possible values are (none, left, right)
  numbersep=5pt,                   % how far the line-numbers are from the code
  numberstyle=\tiny\color{mygray}, % the style that is used for the line-numbers
  rulecolor=\color{black},         % if not set, the frame-color may be changed on line-breaks within not-black text (e.g. comments (green here))
  showspaces=false,                % show spaces everywhere adding particular underscores; it overrides 'showstringspaces'
  showstringspaces=false,          % underline spaces within strings only
  showtabs=false,                  % show tabs within strings adding particular underscores
  stepnumber=2,                    % the step between two line-numbers. If it's 1, each line will be numbered
  stringstyle=\color{mymauve},     % string literal style
  tabsize=2,	                   % sets default tabsize to 2 spaces
  title=\lstname                   % show the filename of files included with \lstinputlisting; also try caption instead of title
}

\linespread{1.5} % l'interlinea

\frenchspacing

\newcommand{\abs}[1]{\lvert#1\rvert}

\usepackage{floatflt,epsfig}

\usepackage{multicol}
\newcommand\yellowbigsqcup[1][\displaystyle]{%
  \fboxrule0pt
  \ifx#1\textstyle\fboxsep-0.6pt\else\fboxsep-1.25pt\fi
  \mathrel{\fcolorbox{white}{yellow}{$#1\bigsqcup$}}}

\title{Prontuario Fisica 1}
\author{Nicola Ferru}
\begin{document}
\maketitle

\section{Vettori}
\label{sec:vect}

\subsection{Triangolo rettangolo: sin, cos e tan}
\label{sec:triangoretsincosetan}
\begin{eqnarray}
  \label{eq:sincostan}
  \cos \alpha = \frac{AB}{BC}\\
  \sin \alpha = \frac{AC}{BC}\\
  \tan \alpha = \frac{AC}{AB}
\end{eqnarray}

\subsection{Teorema di Carno sui triangoli}
\label{sec:carnoetriang}

\begin{eqnarray}
  \label{eq:teoremadicarnot}
  a^2=b^2+c^2-2bc\cdot \cos\alpha\\
  b^2=a^2+c^2-2ac \cdot \cos \beta\\
  c^2=b^2+a^2-2ba\cdot \cos \gamma
\end{eqnarray}

\su{teorema dei seni}
\label{teorseni}

\begin{eqnarray}
  \label{eq:sin}
  \frac{a}{\sin \alpha} = \frac{b}{\sin \beta} = \frac{c}{\sin \gamma}
\end{eqnarray}

\subsection{Algebra vettoriale}
\label{sec:algvett}

\begin{eqnarray}
  \label{eq:algvet}
  \text{Somma di vettori: } a+b=c
\end{eqnarray}

\subsubsection{Scomposizione di vettore }
\label{sec:scomposizione}

\begin{eqnarray}
  \label{eq:scompvetto}
  v_x=v\cos \alpha\\
  v_y=v\sin \alpha{}\\
  v=\sqrt{(v_x^2+v_y^2)}
\end{eqnarray}

\subsubsection{prodotti tra due vettori}
\label{sec:prodtraduevett}

\begin{eqnarray}
  \label{eq:prodtraduevett}
  \text{prodotto scalare: }a\cdot b=ab\cos \alpha{}\\
  \text{prodotto vettoriale:}a \times b=c
\end{eqnarray}

\section{Velocità}
\label{sec:velocità}

\subsection{velocità media}
\label{sec:velmedia}


\begin{eqnarray}
  \label{eq:velmedia}
  \bar{V} = \frac{\Delta x_{totale}}{\Delta t_{totale}}=\frac{x_2-x_1}{t_2-t_1}
\end{eqnarray}

\subsection{km/h a m/s}
\label{sec:km/hm/s}

\subsection{Accelerazione}
\label{sec:accelerarzione}

\begin{eqnarray}
  \label{eq:acc}
  \bar{a}=\frac{v_2-v_1}{t_2-t_1}=\frac{\Delta v}{\Delta t}\\
  v=v_0+at\\
  x=x_0+\bar{v} t
\end{eqnarray}

\subsection{Legge oraria}
\label{sec:leggeoraria}

\begin{eqnarray}
  \label{eq:leggeo}
  x=x_0+v_0t + \frac{1}{2} at^2\\
  x=vt+\frac{1}{2} at^2
\end{eqnarray}

\subsection{Moto in caduta libera}
\label{sec:motocadutalibera}

\begin{eqnarray}
  \label{eq:cadutalibera}
  y=v_0t-\frac{1}{2}gt^2
\end{eqnarray}

\end{document}

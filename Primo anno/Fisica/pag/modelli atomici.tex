\chapter{Modelli atomici}
\section{Modello atomico di Bohr-Sommerfeld}
Il modello atomico proposto da Niels Bohr nel 1913, successivamente ampliato da Arnold Sommerfeld nel 1916, è la più famosa applicazione della quantizzazione dell'energia che, insieme alle spiegazioni teoriche sulla radiazione del corpo nero, sull'effetto fotoelettrico e sullo scattering Compton, e all'equazione di Schrödinger, costituiscono la base della meccanica quantistica.\\
Il modello, proposto inizialmente per l'atomo di idrogeno, riusciva anche a spiegare, entro il margine di errore statistico, l'esistenza dello spettro sperimentale. Bohr presenta così un modello dell'atomo, facendo intuire che gli elettroni si muovono su degli orbitali. \textit{Questo modello viene ancora utilizzato nello studio dei Semiconduttori.} 
\begin{center}
	By \href{https://it.wikipedia.org/wiki/Modello_atomico_di_Bohr}{Wikipedia}
\end{center}
\chapter{La legge di Couloumb}
\section{Introduzione}
L'elettromagnetismo costituisce il fondamento su cui sono costruiti i computer,
le radio e televisori, le telecomunicazioni, illuminazioni ecc.
L'elettromagnetismo spiega come gli atomi siano tenuti insieme, come
avvengono i fulmini, le aurore e gli arcobaleni. Gli antichi filosofi greci
scoprirono che l'ambra strofinata attrae pagliuzze sottili e che pietre
magnetiche naturali attraggono pezzetti di ferro. Tra i tanti scienziati che
svilupparono l'elettromagnetismo moderno, notiamo il fisico sperimentale
\texttt{Michael Faraday} ed il teorico \texttt{James Clerk Maxwell}.
\subsection{La carica elettrica}
Una bacchetta di vetro strofinata con seta si \texttt{allontana} da un'altra
bacchetta di vetro strofinata con della seta.
\begin{enumerate}
	\item \textit{Forza repulsiva} - Una bacchetta di vetro strofinata con
		della seta si \textit{avvicina} ad una bacchetta di plastica strofinata
		con la pelle di camoscio.
	\item \textit{Forza attrattiva} - Le forze sono dovute alla \textbf{carica
		elettrica}.
\end{enumerate}
Esistono due tipi di carica:
\begin{enumerate}
	\item Carica positiva, contrassegnata con il segno +;
	\item Carica negativa, contrassegnata con il segno -
\end{enumerate}
Si definisce neutro un oggetto che ha le cariche positive e negative
perfettamente bilanciate.
Spostando la carica da un oggetto all'altro, si crea una carica in eccesso.
L'oggetto può scaricarsi con scintille oppure con l'umidità dell'aria.
\subsubsection{Le proprietà delle cariche}
\begin{enumerate}
	\item Le particelle cariche dello \textit{stesso segno} si respingono;
	\item Le particelle di carica opposta si attraggono;
	\item Se strofiniamo il vetro con un panno di seta risulta in una
		\textit{carica potenziale} nel vetro;
	\item Strofinando della plastica con della pelle di camoscio si ottiene una
		\textit{carica negativa} sulla stessa.
\end{enumerate}
\subsubsection{Conduttori e isolanti}
In natura esistono le seguenti tipologie di materiali:
\begin{tasks}{2}
	\task I conduttori - le cariche si muovono liberamente;
	\task Gli isolanti - le cariche non si muovono, per l'appunto restano
	isolate;
	\task I semiconduttori - le cariche si muovono, ma il materiale possiede un
	alta resistenza;
	\task I superconduttori - le cariche si muovono senza incontrare ostacoli
	di sorta.
\end{tasks}
\newtheorem{pcariche}{Particelle Cariche}
\begin{pcariche}
	La materia composte di atomi. Gli atomi hanno un \textbf{nucleo} con
	\begin{itemize}
		\item Protoni - cariche positive;
		\item Elettroni - carica negativa.
	\end{itemize}
	La carica dell'elettrone e del protone hanno la stessa intensità ma segno
	opposto. Gli elettroni sono \textbf{attratti verso il nucleo}. Nei
	conduttori, alcuni elettroni sono \textit{liberi di muoversi}, un isolante
	\textit{non ha elettroni liberi}.
\end{pcariche}
\subsection{Carica indotta}
Una carica negativa \textit{respinge} gli elettroni nel rame, risulta una
carica positiva indotta vicino alla carica esterna. Risulta una \textbf{forza
attrattiva} tra una carica negativa e un conduttore, Anche per una carica
positiva ed un conduttore la forza risulta \textbf{attrattiva}.
\section{Legge di Coulomb}
Tra due cariche puntiformi esiste una \textit{forza elettrostatica}. La forza è
diretta \textit{lungo la retta congiungente} le due cariche.
Se le cariche sono della stessa polarità le stesse si respingono, invece, se
sono di carica opposta, avviene un attrazione tra le cariche.
\subsubsection{Riassunto sui vettori}
\paragraph{Componenti:}
\begin{equation}
	F_x=F\cos 0;\text{ }F_y=F\sin 0
\end{equation}
\paragraph{Modulo e angolo:}
\begin{equation}
	F=\abs{\vec{F}}=\sqrt{F_x^2+F_y^2};\text{ } \tan 0 =\frac{F_y}{F_x}
\end{equation}
\paragraph{Versore:}
\begin{equation}
	\hat{a}=\frac{\vec{a}}{\abs{\vec{a}}}=\frac{\vec{a}}{a}
\end{equation}
\paragraph{Sommare:}
\begin{equation}
	\vec{F}=\vec{F}_1+\vec{F}_2\to F_x=F_{1x}+F_{2x};\text{ } F_y=F_{1y}+F_{2y}
\end{equation}
La forza di una carica $q_1$ in presenza di un'altra $q_2$ è:
\begin{equation}
	\vec{F}_{12}=k\frac{q_1q_2}{r^2}\hat{r}
\end{equation}
Dove $k=8,99*10^9Nm^2C^{-2}$ è la \textbf{costante di Coulomb} e $\vec{r}$ è il
vettore di lunghezza pare alla distanzia $q_2$ a $q_1$.
\begin{enumerate}
	\item Se $q_1$ e $q_2$ hanno la stessa polarità, il prodotto $q_1q_2$ è 
		\textbf{positivo} e la forza è \textit{repulsiva}.
	\item Se $q_1$ e $q_2$ hanno la polarità \textbf{opposta}, il
		prodotto $q_1q_2$ è \textbf{negativo} e la forza è \textit{attrattiva}.
\end{enumerate}
La forma è una coppia di azione-reazione: $\vec{F}_{21}=-\vec{F}_{12}$
\subsection{Unità do misura}
L'unità di carica nel SI è il \texttt{Coulomb} (\ref{}). La derivata del unità
fondamentale di corrente elettrica, \textbf{Ampere}. La corrente \textit{i} è
data dal rapporto $\frac{dq}{dt}$ con cui transita la carica \textit{q}: 
$i=\frac{dq}{dt}$.
Risulta $1C=1As$
\subsection{La costante dielettrica del vuoto}
La costante di \textit{Coulomb} viene anche espressa come
$k=\frac{1}{4\pi\xi_0}$ dove $\xi_0 = 8,85*10^{-12}C^2N^{-1}m^{-2}$ è la
\textbf{constante dielettrica del vuoto}.\\
Così scriviamo $\vec{F}=\frac{q_1q_2}{4\pi\xi_0r^2}\hat{r}$, o per ottenere il
modulo $F=\frac{\abs{q_1}\abs{q_2}}{4\pi\xi_0r^2}\hat{r}$
\subsection{Forze multiple}
Le forze elettrostatiche obbediscono al \textbf{principio di sovrapposizione}.
Se molte particelle sono vicine alla carica $q_1$, la forza netta è $\vec{F}_{1,net}=\vec{F}_{12}+\vec{F}_{14}+\dots+\vec{F}_{1n}$.
\paragraph{Attenzione:} \textit{somma vettoriale!}
\section{Teorema del guscio}
\begin{tasks}(2)
  \task Primo teorema del guscio:\\
  \textit{Una superficie sferica uniformemente carica attrae o respinge una carica esterna come se tutta la carico fasse concentroto
    nel suo centro}.
  \task Secondo teorema del guscio:\\
  \textit{Uno carico posto all'interno di uno superficie chiusa uniformemente carica non ne sente la foza}.
\end{tasks}
\section{La quantizzazione della carica}
Qualunque carica \textit{q} può essere scritta come $q=ne$ in cui $n=\pm 1,\pm 2, \pm 3, \dots$ ed è la carica elementare: $e = 1,602*10^{-19}C$
\begin{tasks}
  \task Il \textbf{protone} ha carica $+e$
  \task L'ettrone ha carica $-e$
\end{tasks}

Il valore di e è così piccolo che normalmente la granularità non appare nei fenomeni di larga scala. Attraverso un filo con corrente di 1A passano circa $6,2*10^{18}$ elettroni al secondo.

\section{La conservazione della carica}
La carica elettrica è conservata - Lo strofinamento del vetro con un panno di seta non crea carica positiva, ma trasferisce elettroni dal vetro alla seta. Anche nei processi nucleari la carica totale rimane invariata.
\section{Verifica}
\begin{enumerate}
\item Indicare il verso della forza che agisce sul protone centrale
\item Ordinare i tre casi secondo i valori decrescenti del modulo della forza netta sull'elettrone.
\end{enumerate}
\paragraph{Soluzione primo problema}
\begin{equation}
  q_1=+e,q_2=+2e, \text{ } R=2cm.
\end{equation}
Calcolo la forza $\vec{F}_{12}$
\begin{equation}
  F_{12}=k\frac{\abs{q_1}\abs{q_2}}{R^2}=k\frac{2e^2}{R^2}=\frac{8,99*10^9*2*(1,6*10^{19})}{R^2}=1,15*10^{-24}N
\end{equation}
Quindi il valore finale è $\vec{F}_{12}=-(1,15*10^{-24}N)\hat{x}$
\begin{equation}
  q_1=+e, q_2 = +2e,q_3=-2e, R=2cm.
\end{equation}
Calcolo la forza $\vec{F}_{1,net}$
\begin{equation}
 F_{13}=k\frac{2e^2}{\left(\frac{3}{4}R\right)^2}=2,05*10^{-24}N
\end{equation}
Quindi il valore che otteniamo è $F_{13}=(2,05*10^{-24}N)$
\begin{equation}
  \begin{matrix}
  \vec{F}_{1,net}=\vec{F}_{12}+\vec{F}_{13}=-(1,15*10^{-24}N)\hat{x}+(2,05*10^{-24}N)
    \hat{x}\\=(0,90*10^{24}N)\hat{y}=-(0,125*10^{-24}N)\hat{x}+(1,775*10^{-24}N)\hat{y}
  \end{matrix}
\end{equation}
Quindi il valore che otteniamo è $F_{1,net,x}=\sqrt{F^2_{1,net,x}+F^2_{1,net,y}}=1,78*10^{-24}N$
\paragraph{Soluzione secondo problema}
$q_1=8e,\text{ } q_2=-2e$. In che punto un protone è in equilibrio?
\begin{equation}
  \begin{matrix}
    \vec{F}_1+\vec{F}_2=0. \text{ } x>L.\text{ } \frac{kq_1e}{x^2}+\frac{kq_2e}{(x-L)^2}=0\\
    \to \left(\frac{x-L}{x}\right)=\frac{-q_2}{q_1}=\frac{1}{4}\to \frac{x-L}{x}=\frac{1}{2}\to x=2L
  \end{matrix}
\end{equation}


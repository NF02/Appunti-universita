\documentclass{book}

\usepackage[utf8]{inputenc}
\usepackage{titlesec}
\usepackage{easylist}
\usepackage{hanging}
\usepackage{hyperref}
\usepackage[a4paper,top=2.0cm,bottom=2.0cm,left=2.0cm,right=2.0cm]{geometry}
\usepackage{blindtext}
\usepackage{tipa}
\usepackage{epigraph}
\usepackage{enumerate}
\usepackage{longtable}
\usepackage{setspace}
\usepackage{verbatim}
\usepackage[T1]{fontenc}
\usepackage{graphicx}
\usepackage[italian]{babel}
\usepackage{amsmath}
\usepackage{pbox}
\usepackage{fancyhdr}
\usepackage{cancel}
\usepackage{tabularx}
\usepackage{booktabs}
\usepackage{multirow}
\usepackage{longtable}
\usepackage{tikz}
\usepackage{tikz-qtree}
\usepackage{subfig}
\usepackage{xcolor}
\usepackage{amssymb}
\usepackage{mathrsfs}
\usepackage{textcomp}
\usepackage{xfrac}
\usepackage{dsfont}

\usepackage{listings}
\usepackage{color}

\definecolor{mygreen}{rgb}{0,0.6,0}
\definecolor{mygray}{rgb}{0.5,0.5,0.5}
\definecolor{mymauve}{rgb}{0.58,0,0.82}

\lstset{ 
  backgroundcolor=\color{white},   % choose the background color; you must add \usepackage{color} or \usepackage{xcolor}; should come as last argument
  basicstyle=\footnotesize,        % the size of the fonts that are used for the code
  breakatwhitespace=false,         % sets if automatic breaks should only happen at whitespace
  breaklines=true,                 % sets automatic line breaking
  captionpos=b,                    % sets the caption-position to bottom
  commentstyle=\color{mygreen},    % comment style
  deletekeywords={...},            % if you want to delete keywords from the given language
  escapeinside={\%*}{*)},          % if you want to add LaTeX within your code
  extendedchars=true,              % lets you use non-ASCII characters; for 8-bits encodings only, does not work with UTF-8
  firstnumber=1000,                % start line enumeration with line 1000
  frame=single,	                   % adds a frame around the code
  keepspaces=true,                 % keeps spaces in text, useful for keeping indentation of code (possibly needs columns=flexible)
  keywordstyle=\color{blue},       % keyword style
  language=Octave,                 % the language of the code
  morekeywords={*,...},            % if you want to add more keywords to the set
  numbers=left,                    % where to put the line-numbers; possible values are (none, left, right)
  numbersep=5pt,                   % how far the line-numbers are from the code
  numberstyle=\tiny\color{mygray}, % the style that is used for the line-numbers
  rulecolor=\color{black},         % if not set, the frame-color may be changed on line-breaks within not-black text (e.g. comments (green here))
  showspaces=false,                % show spaces everywhere adding particular underscores; it overrides 'showstringspaces'
  showstringspaces=false,          % underline spaces within strings only
  showtabs=false,                  % show tabs within strings adding particular underscores
  stepnumber=2,                    % the step between two line-numbers. If it's 1, each line will be numbered
  stringstyle=\color{mymauve},     % string literal style
  tabsize=2,	                   % sets default tabsize to 2 spaces
  title=\lstname                   % show the filename of files included with \lstinputlisting; also try caption instead of title
}

\linespread{1.5} % l'interlinea

\frenchspacing

\newcommand{\abs}[1]{\lvert#1\rvert}

\usepackage{floatflt,epsfig}

\usepackage{multicol}
\newcommand\yellowbigsqcup[1][\displaystyle]{%
  \fboxrule0pt
  \ifx#1\textstyle\fboxsep-0.6pt\else\fboxsep-1.25pt\fi
  \mathrel{\fcolorbox{white}{yellow}{$#1\bigsqcup$}}}

\title{Formulario Fisica 1}
\author{Nicola Ferru}

% definizioni
\newtheorem{defi}{Definizione}[section]
\newtheorem{nota}{Nota}[section]

% stili grafici
\usepackage{tikzit}
\input{img/sstyle.tikzstyles}
\input{img/stile.tikzstyles}

\begin{document}
\maketitle
\tableofcontents

\chapter{Cinematica}
\label{sec:cin}

\section{Moto rettilineo uniforme}
\label{sec:motoretilineouniform}

\subsection{Legge oraria}
\label{sec:leggeoraria}
\begin{equation}
  \label{eq:leggeorrettilineouni}
  s=v(t-t_i)+s_i
\end{equation}

\subsection{Velocità}
\label{sec:velmotoretuni}

\begin{equation}
  \label{eq:velmotoretuni}
  v=\frac{s-s_i}{t-t_i}
\end{equation}

\subsection{Tempo}
\label{sec:tempomotretuni}
\begin{equation}
  \label{eq:tempomotretuni}
  t=\frac{s-s_i}{v}+t_i
\end{equation}

\subsection{Istante iniziale nullo}
\label{sec:istanteininullo}

\begin{equation}
  \label{eq:istanteinizialenullomotret}
  s=vt+s_0
\end{equation}

\subsection{Istante e posizione iniziali nulli}
\label{sec:isteposininulli}

\begin{equation}
  \label{eq:isteposininulli}
  s=vt
\end{equation}

\section{Moto uniformemente accelerato}
\label{sec:motouniacc}

Formula per calcolare la Velocità finale:
\begin{equation}
  \label{eq:vf}
  V_f=v_0+a\cdot t
\end{equation}

\begin{equation}
  \label{eq:motouniaccformbase}
  \begin{cases}
    v=v_i+a\Delta t\\
    s=\frac{1}{2}a(\Delta t)^2+v_i\Delta t+s_i
  \end{cases}
\end{equation}
o, in forma splicita
\begin{equation}
  \label{eq:motouniaccformbase2}
  \begin{cases}
    v=v_i+a(t-t_i)\\
    s=\frac{1}{2}a(t-t_i)^2+v_i(t-t_i)+s_i
  \end{cases}
\end{equation}
\subsection{Segmento percorso $s$ dopo il tempo $t$}
\label{sec:segPDTempt}
\begin{equation}
  \label{eq:segperdopot}
  s=s_0+v_0\cdot t \pm \frac{a}{2}\cdot t^2
\end{equation}

\subsection{Velocità}
\label{sec:velmotouniformacc}
\begin{eqnarray}
  \label{eq:velmotouniformacc}
  v=v_i+a(t-t_i)
\end{eqnarray}

\subsection{Equazione senza il tempo}
\label{sec:eqsenzailtempo}

\begin{equation}
  \label{eq:eqsenzailtempo}
  v^2=v_0^2+2a(s-s_0)
\end{equation}
\subsection{Corpo che cade}
\label{sec:corpochecade}
\begin{equation}
  \label{eq:altezzadicaduta}
  h=h_0+v_0\cdot t \pm \frac{g}{2}\cdot t^2
\end{equation}

\subsection{Caduta da $h_0$ con velocità iniziale nulla}
\label{sec:cadutadahconvelzero}
\begin{multicols}{2}
  \subsubsection{Tempo di caduta}
  \label{sec:tempcad}
  \begin{equation}
    \label{eq:tempcad}
    t_c=\sqrt{\frac{2h_0}{g}}
  \end{equation}
  \subsubsection{Velocità finale}
  \label{sec:velFin}
  \begin{equation}
    \label{eq:velfin}
    V_f=\sqrt{2gh}
  \end{equation}
\end{multicols}

\subsection{Lancio verso l'alto}
\label{sec:lancioversolalto}
Nel caso del lancio verso l'alto sono presenti queste due formule:
\begin{multicols}{2}
  \subsubsection{Altezza finale}
  \label{sec:altfinlancioversalt}
  \begin{equation}
    \label{eq:altfinlancioversalt}
    h=\frac{V_0^2}{2g}
  \end{equation}
  \subsubsection{Tempo finale}
  \label{sec:tempofinlancioveralt}
  \begin{equation}
    \label{eq:tempofinlancioveralt}
    t_h=\frac{V_0}{g}
  \end{equation}
\end{multicols}

\section{Moto circolare uniforme}
\label{sec:motCircUni}
  \begin{multicols}{2}
    \subsubsection{Accelerazione centripeta}
    \label{sec:acccent}
    \begin{equation}
      \label{eq:acccentripeta}
      a_c=\frac{V^2}{r}
    \end{equation}
    \subsubsection{Velocità angolare}
    \label{sec:velang}
    \begin{equation}
      \label{eq:velang}
      \omega=\sfrac{2\pi_{rad}}{T}=2\pi\cdot v
    \end{equation}
    \begin{equation}
      \label{eq:Velang}
      \omega = \frac{\Delta \alpha}{\Delta t}
    \end{equation}
  \end{multicols}


\subsection{Energia cinetica totale}
\label{sec:encintot}

\begin{equation}
  \label{eq:encintot}
  E=\frac{1}{2} mv^2
\end{equation}

\subsection{Forza centripeta e centrifuga}
\label{sec:forzcentecentr}

\begin{multicols}{2}
  \subsubsection{Forza centripeta}
  \label{sec:forzacent}
  \begin{equation}
    \label{eq:forzacent}
    F_{CP}=m\cdot \frac{v^2}{r}
  \end{equation}

  \subsubsection{Forza centrifuga}
  \label{sec:forzacentrifuga}
  \begin{equation}
    \label{eq:forzacentrifuga}
    F_{CF}=-m\cdot\frac{v^2}{r}
  \end{equation}
  \subsubsection{Velocità tangenziale}
  \label{sec:veltan}

  \begin{equation}
    \label{eq:veltangenziale}
    v=\omega \cdot r
  \end{equation}
  
  \subsubsection{Periodo}
  \label{sec:periodocentuni}
  \begin{equation}
    \label{eq:periodocentuni}
    t=\frac{1}{f}
  \end{equation}
\end{multicols}
\begin{equation}
  \label{eq:equalscrcrnt}
  \boxed{F_{CP}=-F_{CF}}
\end{equation}
\section{Moto circolare accelerato}
\label{sec:motoacc}

\subsection{Accelerazione tangenziale}
\label{sec:AccTanCircAcc}

\begin{equation}
  \label{eq:AccTanCircAcc}
  a_t=
  \begin{vmatrix}
    \frac{dv}{dt} 
  \end{vmatrix}\to a_t=\alpha r
\end{equation}

\subsection{Leggi orarie}
\label{sec:leggiorariecircacc}

\begin{eqnarray}
  \label{eq:leggiorariecircacc}
  \theta=\frac{1}{2}\alpha t^2 +\omega_0t+\omega_0 & \omega=\omega_0+\alpha t
\end{eqnarray}

\subsubsection{Equazione senza il tempo}
\label{sec:leggiorariecircaccsenzat}

\begin{equation}
  \label{eq:leggiorariecircaccsenzat}
  \omega^2=\omega^2_0+2\alpha (\theta-\theta_0)
\end{equation}

\subsection{Accelerazione totale}
\label{sec:acctotcircacc}
\begin{equation}
  \label{eq:acctotcircacc}
  \vec{a}_{tot}=\vec{a}_T+\vec{a}_C \to a_{tot} = \sqrt{a_T^2+a_C^2}
\end{equation}

\section{Somma dei vettori}
\label{sec:sommavect}
\begin{equation}
  \label{eq:sommavect}
  \abs{\vec{v}}=\sqrt{\abs{v_1}^2+\abs{v_2}^2+2\abs{v_1}\abs{v_2}+\log \alpha}
\end{equation}

\section{Prodotto tra vettori}
\label{sec:prodottotravect}

\subsection{Scalare}
\label{sec:scalare}

\begin{equation}
  \label{eq:scalare}
  a\cdot b=a\cdot\abs{b_p}
\end{equation}

\begin{equation}
  \label{eq:scalare2}
  a\cdot b= a\cdot b\cdot \cos \alpha
\end{equation}

\subsection{Vettoriale}
\label{sec:vectprod}
\begin{equation}
  \label{eq:prodvec}
  \begin{matrix}
    a\cdot b=a\cdot b\cdot \sin \alpha & a_n \dot b
  \end{matrix}
\end{equation}
di cui, $a_n$ è componente di $a \bot a b$.

\section{Moto con accelerazione variabile}
\label{sec:motoconaccevar}

\subsection{Velocità dopo un tempo $t$}
\label{sec:veldopotempt}
\begin{equation}
  \label{eq:veldopuntemptacvar}
  v=v_0+\int^t_{t_0}a(t)dt
\end{equation}


\subsection{Forza di attrito}
\label{sec:forzadiattr}
\begin{multicols}{2}
  \subsubsection{Attrito statico}
  \label{sec:attrstatico}
  \begin{equation}
    \label{eq:attrstatico}
    fs=\mu_sN
  \end{equation}
  
  \subsubsection{Attrito dinamico}
  \label{sec:attrdinamico}
  \begin{equation}
    \label{eq:attrdinamico}
    f_k=\mu_kN
  \end{equation}
\end{multicols}

\section{Piano inclinato}
\label{sec:pianoinclinato}

\begin{multicols}{2}
  \subsubsection{Accelerazione perpendicolare al piano}
  \label{sec:accperpalpiano}
  \begin{equation*}
    a_y=0
  \end{equation*}
  
  \subsubsection{Accelerazione parallela al piano}
  \label{sec:accparalpiano}
  \begin{equation*}
    a_x=g\cdot \sin(\alpha)
  \end{equation*}
  
  \subsubsection{Angolo d'altezza e lunghezza}
  \label{sec:angolodaltelung}
  \begin{equation*}
    \sin(\alpha)=\frac{h}{l}
  \end{equation*}
  
  \subsubsection{Angolo da base e lunghezza}
  \label{sec:angdabaseelungh}
  \begin{equation*}
    \cos(\alpha)=\frac{d}{l}
  \end{equation*}
  
  \subsubsection{Angolo d'altezza e base}
  \label{sec:angdaltebase}
  \begin{equation*}
    \tan(\alpha)=\frac{h}{d}
  \end{equation*}
\end{multicols}

\clearpage
\section{Moto Parabolico (Moto del proiettile)}
\label{sec:motoparabolico}

\begin{equation}
  \label{eq:componentixymotopar}
  \begin{cases}
    v_{0x}=v_0\cos(\alpha)\\
    v_{0y}=v_0\sin(\alpha)
  \end{cases}
\end{equation}
\begin{equation*}
  v=\sqrt{v_{0x}^2+v_{0y}^2}
\end{equation*}
\begin{equation*}
  \tan(\alpha)=\frac{v_{0x}}{v_{0y}}
\end{equation*}
\begin{equation}
  \label{eq:eqbasemotopar}
  \begin{cases}
    x=x_0+v_{0x}t\\
    y=-\frac{1}{2}gt^2+v_{0y}t+y_0
  \end{cases}
\end{equation}
\begin{equation}
  \label{eq:tempomotopar}
  t=\frac{v_0\sin(\alpha)\pm \sqrt{v_0^2\sin^2(\alpha)+2gy_0}}{g}
\end{equation}

\subsection{Traiettoria del moto parabolico}
\label{sec:traietmotopar}

\begin{eqnarray}
  \label{eq:traietdelmotopar}
  \begin{cases}
    t=\frac{x-x_0}{v_{0x}}\\
    y=-\frac{1}{2}gt^2+v_{0y}t+y_0
  \end{cases} & \text{(Oppure in forma esplicita)} & \begin{cases}
    t=\frac{x-x_0}{v_{0x}}\\
    y=-\frac{1}{2}g\left(\frac{x-x_0}{v_{0x}}\right)^2+v_{0y}\left(\frac{x-x_0}{v_{0x}}\right)+y_0
  \end{cases}
\end{eqnarray}

\chapter{Dinamica}
\label{chap:Dinamica}

\section{Lavora}
\label{sec:lavoro}

\subsection{Forza costante}
\label{sec:forcostlav}

\begin{equation}
  \label{eq:lavoro}
  L=(F\cdot \cos \alpha) \Delta s
\end{equation}

\subsection{Forza variabile}
\label{sec:forzavarlav}
\begin{equation}
  \label{eq:forzavarlav}
  L_{1,2}=\int_{x_1}^{x_2}F(x)dx
\end{equation}

\subsubsection{Unità di misura}
\label{sec:joule}

\begin{equation}
  \label{eq:joule}
  1N\cdot 1m=1J \textbf{ (Joule)}
\end{equation}

\subsection{Lavoro istantaneo}
\label{sec:lavoroistantaneo}

\begin{equation}
  \label{eq:lavista}
  dL=(F\cdot \cos \alpha)ds
\end{equation}

\section{Potenza}
\label{sec:potenza}

\begin{multicols}{2}
  \subsubsection{Potenza media}
  \label{sec:potmedia}
  \begin{equation}
    \label{eq:potmedia}
    <P>=\frac{\Delta L}{\Delta t}
  \end{equation}
  
  \subsubsection{Potenza istantanea}
  \label{sec:potistant}
  \begin{equation}
    \label{eq:potistant}
    P=\frac{dL}{dt}
  \end{equation}
\end{multicols}


\section{Energia cinetica}
\label{sec:energiacin}
\begin{equation}
  \label{eq:energiacin}
  k=\frac{1}{2} mv^2
\end{equation}
\subsection{Teorema Lavoro-Energia}
\label{sec:teoLavEn}

\begin{equation}
  \label{eq:teoLavEn}
  L=k-k_0
\end{equation}

\section{Energia Potenziale}
\label{sec:enpot}

\begin{equation}
  \label{eq:enpot}
  \Delta k=-\Delta U
\end{equation}
\begin{equation}
  \label{eq:differenzak}
  k_x-k_{x_0}=-(U_x-U_{x_0})
\end{equation}

\section{Energia meccanica}
\label{sec:enmecc}

\begin{equation}
  \label{eq:enmecc}
  E=k+U
\end{equation}

\subsection{Legge di conservazione dell'energia meccanica}
\label{sec:leggconsenmec}

\begin{equation}
  \label{eq:leggconsenmec}
  k_0+U_0=k_F+U_F
\end{equation}
\begin{multicols}{2}
  \begin{equation*}
    \Delta U=-\int_{x_0}^xF(x)dx
  \end{equation*}
  \begin{equation*}
    F(x)=-\frac{dU(x)}{dx}
  \end{equation*}
\end{multicols}

\subsection{Energia potenziale gravitazionale}
\label{sec:enpotgrav}
\begin{equation}
  \label{eq:enpotgrav}
  U_y-U_0=mgy\\
  U(y)=mgy
\end{equation}
\begin{equation*}
  U_y-U_0=\int_y^0F(y)dy=\int_y^0(-mg\cdot y)=mgy
\end{equation*}

\subsection{Energia potenziale elastica}
\label{sec:enerpotela}

\begin{equation*}
  U(x)=\int_x^0(-kx)dx=\frac{1}{2}kx^2
\end{equation*}

\subsection{Forza non conservative}
\label{sec:forzanoncon}

\begin{equation*}
  L_{non-cons.}=\Delta{}(k+U)=\Delta{}E
\end{equation*}

\subsection{Legge di conservazione dell'energia}
\label{sec:leggconsen}

\subsection{Centro di massa}
\label{sec:centdimassa}
\begin{equation}
  \label{eq:centdimassa}
  x_{cm}=\frac{\sum m_ix_i}{\sum m_i}
\end{equation}

\subsubsection{coordinate centro di massa (carpo esteso e di materia uniforme)}
\label{sec:coordcentdimassa}

\begin{equation}
  \label{eq:coordcentdimassa}
  x_{cm}=\lim\limits_{\Delta m_i\to 0}= \frac{\sum m_ix_i}{\sum m_i}
\end{equation}

\begin{equation*}
  x_{cm}=\frac{\int x dm}{\int dm}=\frac{1}{M}\cdot \int x dm
\end{equation*}

\subsubsection{Equazione vettoriale del centro di massa}
\label{sec:eqvettdelcentrodimassa}

\begin{equation}
  \label{eq:eqvettdelcentrodimassa}
  \vec{s}_{cm}=\frac{\int \vec{s}dm}{\int dm}
\end{equation}

\subsubsection{Accelerazione del centro di massa (di un sistema di particelle)}
\label{sec:accdelcentrodimass}
\begin{equation}
  \label{eq:accdelcentrodimass}
  M_{cm_x}=\sum F_i = \sum m_i\frac{dV_{i_x}}{dt}
\end{equation}

\subsubsection{Legge del moto traslatorio del centro di massa}
\label{sec:leggedelmottrasldelcndimassa}
\begin{equation}
  \label{eq:leggedelmottrasldelcndimassa}
  F_{est}=Ma_{cm}
\end{equation}

\subsection{Quantità di moto}
\label{sec:quantmoto}

\begin{equation}
  \label{eq:quantmoto}
  \vec{P}=mv
\end{equation}
\begin{equation*}
  \vec{F}=\frac{d\vec{P}}{dt}
\end{equation*}
\begin{equation*}
  P_{Tot}=M\cdot V_{cm}
\end{equation*}

\begin{equation*}
  \frac{d\vec{P}}{dt}=F_{ext}
\end{equation*}

\subsection{Teorema dell'impulso -- quantità di moto}
\label{sec:teodelimpquantmoto}
\begin{eqnarray}
  \label{eq:teoremadelimp}
  \vec{J}=\int_{t_1}^{t_2}F(t) dt=\Delta P
\end{eqnarray}

\subsection{Urto elastico a 2 dimensioni}
\label{sec:urtela2dim}
\begin{equation}
  \label{eq:urtela2dim}
  V_1+V_1=V_2+V_2
\end{equation}
\begin{equation}
  \label{eq:differenzatravel}
  V_1-V_2=V_2-V_1
\end{equation}
\begin{equation*}
  \frac{1}{2}m_1V_1^2+\frac{1}{2}m_2V_2^2=\frac{1}{2}m_1V_1^2+\frac{1}{2}m_2V_2^2
\end{equation*}

\subsubsection{Velocità urto completamente anelastico}
\label{sec:velurtocompan}
\begin{equation}
  \label{eq:velurtcompan}
  m_1V_1+m_2V_2=(m_1+m_2)V
\end{equation}

\section{Moto rotatorio}
\label{sec:motorot}

\subsection{Misura angolo in radianti}
\label{sec:angrad}
\begin{equation}
  \label{eq:angrad}
  \theta = \frac{s}{R}
\end{equation}

\subsection{Velocità angolare media}
\label{sec:velangmedia}
\begin{equation}
  \label{eq:velangmedia}
  <\omega>=\frac{\Delta \theta}{\Delta t}
\end{equation}

\subsection{Velocità angolare istantanea}
\label{sec:velangist}

\begin{equation}
  \label{eq:velangist}
  \omega(t)=\lim\limits_{\Delta t\to 0}\frac{\Delta \theta}{\Delta t}=\frac{d\theta}{dt}
\end{equation}

\subsection{Accelerazione angolare media}
\label{sec:accangmedia}

\begin{equation}
  \label{eq:accangmedia}
  \alpha=\frac{\Delta \omega}{\Delta t}
\end{equation}

\subsection{Moto con accelerazione angolare costante}
\label{sec:motconaccelangcost}
\begin{equation}
  \label{eq:motconaccelangcost}
  \begin{cases}
    \omega=\omega_0+\alpha t\\
    \theta= \frac{1}{2} (\omega_0+\omega)t\\
    \theta= \theta_0+\omega_0t+\frac{1}{2}\alpha t^2 & \text{equazioni orarie}
  \end{cases}
\end{equation}

\subsection{Velocità lineare di particella parte di un corpo rigido}
\label{sec:vellindipartcorprigid}
\begin{equation}
  \label{eq:vellindipartcorprigid}
  V=r\omega
\end{equation}

\subsection{Accelerazione lineare di particella parte di un corpo rigido}
\label{sec:acclindipartcorprigid}
\begin{equation}
  \label{eq:acclindipartcorprigid}
  a_T=r \alpha
\end{equation}

\subsection{Momento di inerzia del corpo rigido}
\label{sec:mominidelcorporigid}
\begin{equation}
  \label{eq:mominidelcorporigid}
  \begin{matrix}
    I=\sum(m_ir^2_i)\\
    \textdownarrow\\
    I=\int r^2dm
  \end{matrix}
\end{equation}

\subsection{Energia cinetica di un corpo in rotazione}
\label{sec:encindiuncorinrot}

\begin{equation}
  \label{eq:encindiuncorinrot}
  K_{tot}=\frac{1}{2}I\omega^2
\end{equation}

\subsection{Momento della forza}
\label{sec:momdellaforz}

\begin{equation}
  \label{eq:momdellaforz}
  \begin{matrix}
    \vec{r}=\vec{r}\times{}\vec{F}\\
    r=r F\sin \theta
  \end{matrix}
\end{equation}
\begin{equation*}
  \boxed{r=I\alpha}
\end{equation*}

\subsection{Momento di un particella}
\label{sec:momdiunpart}

\begin{equation}
  \label{eq:momdiunpart}
  \begin{matrix}
    \vec{L}=\vec{r}\times{}\vec{P}\\
    L=rP\sin \theta
  \end{matrix}
\end{equation}

\begin{equation*}
  \tau=\frac{dL}{dt}=\frac{d(I\omega)}{dt}
\end{equation*}
\begin{equation*}
  dL=\tau dt\Rightarrow \Delta L=\int \tau (t)dt
\end{equation*}

\subsubsection{Principio di conservazione del momento ongolare}
\label{sec:princdiconsdelmomang}
\begin{equation*}
  I\omega=\text{costante}
\end{equation*}

\subsection{Momenti di inerzia da ricordare}
\label{sec:momdiinerziadaric}
\begin{multicols}{2}
  
  \subsubsection{Rotazione rispetto all'asse del cilindro}
  \label{sec:rotrispalassdelcil}
  \begin{equation}
    \label{eq:rotrispalassdelcil}
    I=\frac{mR^2}{2}
  \end{equation}
  
  \subsubsection{Rotazione rispetto ad un diametro centrale}
  \label{sec:rotrispadundiamcent}
  \begin{equation}
    \label{eq:rotrispadundiamcent}
    I=\frac{mR^2}{4}+\frac{ml^2}{12}
  \end{equation}
  
  \subsubsection{Rispetto ad un asse perpendicolare al centro della lunghezza}
  \label{sec:risadunasperalcentrodellung}
  \begin{equation}
    \label{eq:risadunasperalcentrodellung}
    I=\frac{nl^2}{12}
  \end{equation}
  
  \subsubsection{Rotazione rispetto a un asse perpendicolare a un esterno}
  \label{sec:rotrispaunasperpaunestern}
  \begin{equation}
    \label{eq:rotrispaunasperpaunestern}
    I=\frac{ml}{2}
  \end{equation}
  
  \subsubsection{Sfera piena (rispetto a diametro qualunque)}
  \label{sec:sferapiena}
  \begin{equation}
    \label{eq:sperapiena}
    I=\frac{2nR^2}{5}
  \end{equation}
  
  \subsubsection{Superficie sferica (rispetto a diametro qualunque)}
  \label{sec:supersferica}
  \begin{equation}
    \label{eq:supersferica}
    I=\frac{2mR^2}{3}
  \end{equation}
\end{multicols}

\section{Equazione del moto di un oscillazione armonico}
\label{sec:eqdelmotodiunoscarm}

\begin{equation}
  \label{eq:eqdelmotodiunscarm}
  m\frac{d^{\prime\prime}x}{dt^2}+kx=0
\end{equation}


\subsection{Oscillatore armonico}
\label{sec:oscarm}


\subsection{Legge di Hook}
\label{sec:leggedihook}

\begin{equation}
  \label{eq:leggedihook}
  F=-kx
\end{equation}

\subsection{Energia potenziale}
\label{sec:enpotenz}
\begin{equation}
  \label{eq:enpotenz}
  U=\frac{1}{2}kx^2
\end{equation}

\subsection{Legge del moto armonico}
\label{sec:leggdelmotarmo}

\begin{equation}
  \label{eq:leggdelmotarmo}
  x(t)=A\cdot \cos(\omega t+\delta)
\end{equation}
\begin{equation}
  \label{eq:leggedelmotarmo2}
  (\omega t+\delta)=\text{fase del moto}
\end{equation}
\begin{equation*}
  -1<\cos x \leq 1 \Longrightarrow -A\leq x(t)\leq A
\end{equation*}
\begin{multicols}{2}
  \subsubsection{Periodo}
  \label{sec:periodoarm}
  \begin{equation*}
    T=\frac{2\pi}{\omega} = 2\pi\sqrt{\frac{m}{k}} \text{ Oppure } T=\frac{1}{f}
  \end{equation*}
  Dipende solo da $k$ e $m$
  \subsubsection{Frequenza}
  \label{sec:freqarm}
  \begin{equation*}
    f=\frac{1}{T}=\frac{\omega}{2\pi}=\frac{1}{2\pi}\sqrt{\frac{k}{m}}
  \end{equation*}
  È misurato in Hz
\end{multicols}

\subsubsection{Pulsazione}
\label{sec:pulsazione}

\begin{eqnarray}
  \label{eq:pulsazione}
  \omega=\frac{2\pi}{T}; \omega=2\pi f
\end{eqnarray}
\subsection{Velocità nel moto armonico}
\label{sec:velnelmotarmon}
\begin{equation}
  \label{eq:velnelmotarmon}
  V(t)=\frac{dx}{dt}=-\omega A\sin (\omega t +\delta)
\end{equation}

\subsection{Accelerazione nel moto armonico}
\label{sec:accnelmotarm}

\begin{equation}
  \label{eq:accnelmotarm}
  a(t)=\frac{d^{\prime\prime}x}{dt}=-\omega^2 A\cos (\omega t +\delta)
\end{equation}

\subsection{Velocità e Accelerazione massima}
\label{sec:velAccmass}

\begin{eqnarray}
  \label{eq:velAccmassArm}
  v_{max}=\omega A & a_{max}=A\omega^2
\end{eqnarray}
\subsection{Ricavare pulsazione e tempo con formule inverse o da legge oraria, velocità o accelerazione}
\label{sec:ricpulslegore}

\begin{eqnarray}
  \label{eq:ricpulslegore}
  A\cos(\omega t)=\frac{x}{A} & \to \omega t=\arccos(\cos(\omega t))\\
  \ t = \frac{\arccos(\cos(\omega t))}{\omega} & \omega=\frac{\arccos(\cos(\omega t))}{t} 
\end{eqnarray}
\subsection{Energia cinetica}
\label{sec:encin}

\begin{equation}
  \label{eq:encin}
  \begin{matrix}
    k=\frac{1}{2}mv^2=\underbrace{\frac{1}{2}k}_{\omega^2\cdot m}A^2\sin^2(\omega t*\delta) & \omega=\frac{k}{m}
  \end{matrix}
\end{equation}
\begin{multicols}{2}
  \subsubsection{Energia potenziale}
  \label{sec:enpotmotoarm}
  \begin{equation*}
    U=\frac{1}{2}kA^2\cos^2(\omega t +\delta)
  \end{equation*}
  
  \subsubsection{energia meccanica}
  \label{sec:enmecmotoarm}
  \begin{equation*}
    E=U+k=\frac{1}{2}kA^2
  \end{equation*}
\end{multicols}

\subsection{Moto armonico smorzato}
\label{sec:motarmsmorz}

\begin{multicols}{2}
  \begin{equation}
    \label{eq:motarmsmorz}
    m\frac{d^{\prime\prime}x}{dt^2}+b\frac{dx}{dt}+kx=0
  \end{equation}
  
  \subsubsection{Modulo forza d'attrito}
  \label{sec:modforzattr}
  \begin{equation}
    \label{eq:modforzattr}
    F_a=-b\frac{dx}{dt}
  \end{equation}
\end{multicols}

\subsubsection{legge del moto armonico smorzato}
\label{sec:leggdelmotarmsmorz}

\begin{equation}
  \label{eq:leggdelmotarmsmorz}
  x(t)=Ae^{-\frac{bt}{2m}}\cos(\omega^\prime t+\omega)
\end{equation}
dove
\begin{equation*}
  \omega^\prime = \sqrt{\frac{k}{m} - \left(\frac{b}{2m}\right)^2}
\end{equation*}

\subsection{Oscilazioni forzate}
\label{sec:oscillaforz}
\begin{equation}
  \label{eq:oscillaforz}
  m\frac{d^{\prime\prime}x}{dt^2}+b\frac{dx}{dt}+kx=\underbrace{F_m\cos(\omega^{\prime\prime}t)}_{\textsc{Forza esterna al sistema}}
\end{equation}

\subsection{Legge del moto armonico forzato}
\label{sec:leggedelmotoarmforz}

\begin{equation}
  \label{eq:leggedelmotoarmforz}
  x(t)=\left(\frac{F_m}{G}\right)\cdot \sin(\omega^{\prime\prime}-\alpha)
\end{equation}
dove:
\begin{eqnarray}
  \label{eq:leggedelmotoarmforz2}
  G=\sqrt{m^2(\omega^{\prime\prime}-\omega^2)^2+b^2\omega^{\prime\prime2}} &e& \alpha=\arccos\left(\frac{b\omega^{\prime\prime}}{G}\right)
\end{eqnarray}

\chapter{Pendoli}
\label{chap:pendoli}

\section{Pendolo semplice}
\label{sec:pensem}

\subsubsection{Periodo}
\label{sec:perpensem}
\begin{equation}
  \label{eq:pensem}
  T=2\pi\sqrt{\frac{l}{g}}
\end{equation}

\subsection{Componente attiva della forza peso}
\label{sec:componenteattdelforzapeso}

\begin{equation}
  \label{eq:componenteattdelforzapeso}
  F_{P,x}=-
  \begin{pmatrix}
    \frac{mg}{L}
  \end{pmatrix}x
\end{equation}

\section{Pendolo di torsione}
\label{sec:penditors}

\begin{equation}
  \label{eq:momedirich}
  \tau=-x\theta
\end{equation}

\subsection{Legge del moto}
\label{sec:leggedelmotopenditors}

\begin{equation}
  \label{eq:leggedelmotopenditors}
  \theta=\theta_m\cos(\omega t+\delta)
\end{equation}
È la soluzione all'equazione differenziale:

\begin{multicols}{2}
  \subsubsection{Soluzione integrale}
  \label{sec:leggedelmotopendditorsintegral}
  \begin{equation}
    \label{eq:leggedelmotopendditorsintegral}
    \frac{d^{\prime\prime}\theta}{dt^2}=-\frac{x}{I}\theta
  \end{equation}
  \subsubsection{Periodo}
  \label{sec:periodoleggedelmotopendditorsione}
  \begin{equation}
    \label{eq:periodoleggedelmotopendditorsione}
    T=2\pi\sqrt{\frac{I}{x}}
  \end{equation}
\end{multicols}

\subsection{Onde}
\label{sec:onde}


\subsection{Equazione di onda sinusoidale}
\label{sec:eqsin}
\begin{eqnarray}
  \label{eq:eqsin}
  y(x,t)=y_m\sin(kx-\omega t -\psi) & y_m=\text{ampiezza massima oscilazione}
\end{eqnarray}
Dove:
\begin{multicols}{2}
  \subsubsection{Numero d'onda}
  \label{sec:nonda}
  
  \begin{equation}
    \label{eq:nonda}
    k=\frac{2\pi}{\lambda}
  \end{equation}
  
  \subsubsection{Frequenza ongolare}
  \label{sec:fdonda}
  \begin{equation}
    \label{eq:fdonda}
    k=\frac{2\pi}{T}
  \end{equation}  
\end{multicols}

\subsubsection{Equazione per la teoria}
\label{sec:teoonde}

\begin{equation}
  \label{eq:teoonde}
  y(x,t)=y_m\sin2\pi\left(\frac{x}{\lambda}-\frac{t}{T}\right)
\end{equation}
\begin{itemize}
\item finito $
  \begin{matrix}
    t, & y(x)=y(x+k\lambda) & \forall k\in \mathds{Z}
  \end{matrix}
  $ (numeri interi)
\item finito $
  \begin{matrix}
    x, & y(t)=y(t+kT) & \forall k \in \mathds{Z} 
  \end{matrix}
  $
\end{itemize}

\subsubsection{Velocità di fase}
\label{sec:velfase}
\begin{equation}
  \label{eq:velfase}
  V=\frac{\lambda}{T}
\end{equation}

\begin{equation*}
  \lambda=VT=\frac{V}{f}
\end{equation*}


\subsection{Potenza}
\label{sec:potenza}

\begin{equation}
  \label{eq:potenza}
  <P>= 2\pi^2y_m^2f^2\mu V
\end{equation}

\subsection{Serie di Fourer}
\label{sec:fourer}

\begin{equation}
  \label{eq:fourer}
  \begin{matrix}
    y(t)=A_0+A_1\sin(\omega t)+A_2\sin(\omega t)+\dots+A_N\sin(\omega t) +\\
    B_1\cos(\omega t)+B_2 \cos(\omega t) +\dots + B_N\cos (\omega t)
  \end{matrix}
\end{equation}
dove $\omega=\frac{2\pi}{T}$

\subsection{Onda stazionaria}
\label{sec:ondaStazionaria}

\begin{equation}
  \label{eq:ondaStazionaria}
  y=2y_m\sin kx \cos \omega t
\end{equation}
data dalla massa di onda incidente e onda rilessa.

\subsection{Frequenze notivoli}
\label{sec:freqnot}

\begin{equation}
  \label{eq:freqnot}
  f=\underbrace{\frac{n}{2l}}_\lambda\underbrace{\sqrt{\frac{F}{\mu}}}_{v}
\end{equation}

\begin{equation}
  \label{eq:freqnot2}
  \begin{matrix}
    \lambda = \frac{2l}{n} \Leftrightarrow \sfrac{l}{\frac{\lambda}{2}}=n\\
    V=\sqrt{\frac{F}{\mu}}
  \end{matrix}
\end{equation}

\subsection{Onde sommarie}
\label{sec:ondesom}

Frequenza udibile dall'uomo: $20Hz\to 20.000Hz$

\subsubsection{Velocità di propagazione}
\label{sec:veldiprop}

\begin{equation}
  \label{eq:veldiprop}
  V=\sqrt{\frac{B}{\sigma_0}}
\end{equation}

\subsection{Equazione di un'onda sonora}
\label{sec:ondson}

\begin{equation}
  \label{eq:ondson}
  y=y_m\cos(kx-\omega t)
\end{equation}
\begin{eqnarray}
  \label{eq:ondson2}
  k=\frac{2\pi}{\lambda} & \omega = \frac{2\pi}{T}
\end{eqnarray}

\subsubsection{Variazione di pressione del mezzo (rispetto a un punto $P_0$)}
\label{sec:vardipresdelmezzo}
\begin{equation}
  \label{eq:vardipresdelmezzo}
  p=P\sin(kx-\omega t)
\end{equation}
dove $P$ è l'ampiezza di pressione:
\begin{equation}
  \label{eq:vardipresdelmezzo2}
  P=k\sigma_0V^2y_m
\end{equation}

\section{Grandezze acustiche Fondamentali}
\label{sec:granacustiche}

\subsection{Livello di intensità sonore}
\label{sec:livintson}

\begin{equation}
  \label{eq:livintson}
  L=10\cdot\log \left(\frac{I}{I_0}\right)
\end{equation}

\subsubsection{Ricavare intensità dal livello di intensità}
\label{sec:ricintdallint}

\begin{equation}
  \label{eq:ricintdallint}
  I=I_0\cdot 10^{\frac{L}{10}}
\end{equation}
\subsection{Potenza acustica}
\label{sec:potacust}

\begin{equation}
  \label{eq:potacust}
  P=I\cdot S
\end{equation}

\subsection{Dipendenza dalla velocità del erogatore}
\label{sec:dipenddalveldeler}
\begin{equation}
  \label{eq:avvidipdalvel}
  v_s=343\frac{m}{s}
\end{equation}
\subsubsection{Avvicinamento punto di riferimento}
\label{sec:avvi}

\begin{equation}
  \label{eq:avvicinamentopuntodiref}
  f_1=\frac{v_s}{v_s-v_e}\cdot f
\end{equation}

\subsubsection{Allontanamento punto di riferimento}
\label{sec:allpuntodirif}

\begin{equation}
  \label{eq:allpuntodiriferimento}
  f_2=\frac{v_2}{v_s+v_e}\cdot f
\end{equation}

\end{document}

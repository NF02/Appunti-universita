\documentclass{report}

\usepackage[utf8]{inputenc}
\usepackage{titlesec}
\usepackage{easylist}
\usepackage{hanging}
\usepackage{hyperref}
\usepackage[a4paper,top=2.0cm,bottom=2.0cm,left=2.0cm,right=2.0cm]{geometry}
\usepackage{blindtext}
\usepackage{tipa}
\usepackage{epigraph}
\usepackage{enumerate}
\usepackage{longtable}
\usepackage{setspace}
\usepackage{verbatim}
\usepackage[T1]{fontenc}
\usepackage{graphicx}
\usepackage[italian]{babel}
\usepackage{amsmath}
\usepackage{pbox}
\usepackage{fancyhdr}
\usepackage{cancel}
\usepackage{tabularx}
\usepackage{booktabs}
\usepackage{multirow}
\usepackage{longtable}
\usepackage{tikz}
\usepackage{tikz-qtree}
\usepackage{subfig}
\usepackage{xcolor}
\usepackage{amssymb}
\usepackage{mathrsfs}
\usepackage{textcomp}
\usepackage{tasks}
\usepackage{listings}
\usepackage{color}

\definecolor{mygreen}{rgb}{0,0.6,0}
\definecolor{mygray}{rgb}{0.5,0.5,0.5}
\definecolor{mymauve}{rgb}{0.58,0,0.82}

\lstset{ 
  backgroundcolor=\color{white},   % choose the background color; you must add \usepackage{color} or \usepackage{xcolor}; should come as last argument
  basicstyle=\footnotesize,        % the size of the fonts that are used for the code
  breakatwhitespace=false,         % sets if automatic breaks should only happen at whitespace
  breaklines=true,                 % sets automatic line breaking
  captionpos=b,                    % sets the caption-position to bottom
  commentstyle=\color{mygreen},    % comment style
  deletekeywords={...},            % if you want to delete keywords from the given language
  escapeinside={\%*}{*)},          % if you want to add LaTeX within your code
  extendedchars=true,              % lets you use non-ASCII characters; for 8-bits encodings only, does not work with UTF-8
  firstnumber=1000,                % start line enumeration with line 1000
  frame=single,	                   % adds a frame around the code
  keepspaces=true,                 % keeps spaces in text, useful for keeping indentation of code (possibly needs columns=flexible)
  keywordstyle=\color{blue},       % keyword style
  language=Octave,                 % the language of the code
  morekeywords={*,...},            % if you want to add more keywords to the set
  numbers=left,                    % where to put the line-numbers; possible values are (none, left, right)
  numbersep=5pt,                   % how far the line-numbers are from the code
  numberstyle=\tiny\color{mygray}, % the style that is used for the line-numbers
  rulecolor=\color{black},         % if not set, the frame-color may be changed on line-breaks within not-black text (e.g. comments (green here))
  showspaces=false,                % show spaces everywhere adding particular underscores; it overrides 'showstringspaces'
  showstringspaces=false,          % underline spaces within strings only
  showtabs=false,                  % show tabs within strings adding particular underscores
  stepnumber=2,                    % the step between two line-numbers. If it's 1, each line will be numbered
  stringstyle=\color{mymauve},     % string literal style
  tabsize=2,	                   % sets default tabsize to 2 spaces
  title=\lstname                   % show the filename of files included with \lstinputlisting; also try caption instead of title
}

\linespread{1.5} % l'interlinea

\frenchspacing

\newcommand{\abs}[1]{\lvert#1\rvert}

\usepackage{floatflt,epsfig}

\usepackage{multicol}
\newcommand\yellowbigsqcup[1][\displaystyle]{%
  \fboxrule0pt
  \ifx#1\textstyle\fboxsep-0.6pt\else\fboxsep-1.25pt\fi
  \mathrel{\fcolorbox{white}{yellow}{$#1\bigsqcup$}}}

\title{Esercizi}
\author{Nicola Ferru}
\begin{document}
\maketitle
\begin{enumerate}
\item In motociclista inizialmente vieggia per 3 minuti verso sud con una velocità di 20m/s. Nei successivi 2 minuti dirige verso ovest 25m/s poi un minuto a nord-overst per 30 m/s.
  \begin{itemize}
  \item il vettore spostamento totale;
  \item la velocià scalare media;
  \item il vetotre velocità media. si utilizzi un sistema di riferimento con assi x con positivo verso Est.
  \end{itemize}
  \begin{eqnarray}
    \label{eq:esercizio1}
    t_1=3.00min \to 180
  \end{eqnarray}
  \begin{eqnarray*}
    s_1=v_1\cdot t_1=3600m\\
    s_2=v_2\cdot t_2=3000m\\
    s_3=v_3\cdot t_3=1800m & s_3x = 18800\cdot \cos(45)=1272.78m
  \end{eqnarray*}
  Adesso sarà possibile calcolare lo spostamento totale in $x$ e $y$
  \begin{eqnarray*}
    \begin{cases}
      s_{tot}x= 52+s_3x=3000m+1272.73m=4272.79m \\
      s_{tot}y= s_1-s_3y= 3600m=2327.21m
    \end{cases}
  \end{eqnarray*}
  Ora, sarà possibile calcolare lo spazio totale
  \begin{eqnarray*}
    \vec{s}_{tot}=\vec{s}_{tot}x+\vec{s}_{tot}y\\
    s_{tot}=\sqrt{s_{tot}x^2+s_{tot}y^2}=4855.45m
  \end{eqnarray*}
  dopo aver fatto il calcolo dello spazio, adesso è necessario calcolare la velocità media:
  \begin{equation*}
    v_m=\frac{\Delta s}{\Delta t}=\frac{4855.46m}{(180+120+60)s}=13.52m/s
  \end{equation*}
\item un aventore lancia un boccale vuoto in sul bancone perché venga nuovamente riempito, il bancone è alto un 1.22m, esso non viene afferrato dal barista e cate a terra con una rotta parabolica di 1.40m.
  \begin{itemize}
  \item qual'è la velocità con cui ha lasciato il bancone?
  \item Qual'è la durezuibe della velocità del boccare poco prima di atterrare?
  \end{itemize}
  
  \subsubsection{Soluzione}
  \label{sec:soluzione2}
  \begin{eqnarray*}
    h=1.22m & x_1=1.40m\\
    v_0=?
  \end{eqnarray*}
  Partendo dal sistema base si può lavorare nel seguente modo:
  \begin{eqnarray*}
    \begin{cases}
      x=x_0+v_{0x}\cdot t\\
      y=-\frac{1}{2}
    \end{cases}\to
    \begin{cases}
      1.4m=v_0\cdot t & \to v_0=\frac{1.4}{t}=2.8\frac{m}{s}\\
      0=-\frac{1}{2}g\cdot t^2+1.22m
    \end{cases} \to t=\sqrt{\frac{1.22\cdot 2}{g}}=0.5s 
  \end{eqnarray*}
\item Un astronauta fa un salto con una velocità di $3m/s$ su un pianeta sconosciuta e atterra dopo 15m, quel'è la spinta gravitazionale?
  
\subsubsection{Soluzione}
\label{sec:sol3}

    \item Un punto materiale che si muove in senso orario una circonferenza di 2.5m, ad una accelerazione di $15m/s^2$ e conosciuamo un angolo $\beta = 30^o$.
    \begin{itemize}
    \item Determinale accelerazione centripeta;
    \item Modulo della velocità;
    \item Accelerazione tangenziale.
    \end{itemize}
    
\subsubsection{soluzione}
\label{sec:sol4}

    Determiniamo l'accelerazione cintripeta
    \begin{equation*}
      ac=a_{tot} \cdot \cos 30 = 13m/s^2
    \end{equation*}
    Ricaviamo il modulo della velocità:
    \begin{equation*}
      a_c=\frac{V^2}{\not{r}} *\not{r} \to v=\sqrt{a_c\cdot r} = 5.7m/s
    \end{equation*}
    Determiniamo l'accelerazione tangenziale:
    \begin{equation*}
      a_t=a\cdot \sin 30 = 7.5m/s^2
    \end{equation*}
    
  \item La ruota panoramica di un luna park ha un raggio di $15m$ e compie ogni minuto $5$ giri attorno al proprio asse
    orizzontale.
    \begin{tasks}
      \task Qual è il suo periodo di rotazione?
      \task Qual è il modulo la direzione e verso dell’accelerazione centripeta cui è sottoposto un passeggero nel punto più alto?
      \task Qual è il modulo direzione e verso dell’accelerazione centripeta quando il passeggero è nel punto più basso?
    \end{tasks}
    
\subsubsection{Soluzione}
\label{sec:sol5}
\begin{equation*}
  \begin{matrix}
    r=15m & f=5\frac{giri}{m}\\
    T=\frac{1}{f} \to \frac{1}{\frac{60}{5}}= 12s & a_c=\frac{V^2}{r}= 4.06m/s^2\to \underbrace{v=\omega*r}_{\omega=\frac{2\pi}{T}=0.52rad/s}=7.8m/s 
  \end{matrix}
\end{equation*}
\item Un cannone posizionato su un monte alto $1km$ spara un proiettile con un angolo di $35^o$ rispetto all’orizzontale. Il proiettile cade sulla vicina valle ad una distanza orizzontale $d=3km$. A quale velocità iniziale è stato sparato il proiettile? Qual è il tempo di volo?

\subsubsection{Soluzione}
\label{sec:sol6}

\begin{equation*}
  \begin{matrix}
    \theta=35^o & d=3km \to d=3000m & h=1km \to 1000m
  \end{matrix}
\end{equation*}
partendo da suddetti dati possiamo utilizzare la seguente formula parametrica:
\begin{eqnarray*}
  \begin{cases}
    x=x_0+v_{0x}t\\
    y=-\frac{1}{2}gt^2+v_{0y}t+y_0
  \end{cases}\to
  \begin{cases}
    3000m = V_0\cos \theta \cdot t\\
    0=1000m - \frac{1}{2} gt^2+v_0\sin\theta\cdot t
  \end{cases}\to
  v_0=\frac{x}{\cos\theta \cdot t}\\
  \to
  \begin{cases}
    0=y_0-\frac{1}{2}gt_2+\left(\frac{x}{\cos \theta\cdot \not{t}}\right)\cdot \sin \theta \cdot \not{t}
  \end{cases}
  \begin{cases}
    ...\\
    t=\sqrt{\left(y_0+\frac{x}{\cos\theta}\cdot \sin\theta\right)\frac{2}{g}}=25.14s
  \end{cases} \\
  \to \begin{cases}
    v_0=\frac{x}{\cos\cdot t} = 151.71m/s
  \end{cases}
\end{eqnarray*}
\item Una mazza da baseball colpisce una palla. Prima dell'impatto la palla va alla velocità $v_1$ di modulo $12m/s$ e angolo rispetto all'asse $x$ di $\theta_1=35^o$. Dopo ha velocità $v_2$ di modulo $10m/s$ e direzione perpendicolare all'asse $x$. L'asse $x$. L'evento dura $2ms$. 
\paragraph{Determinare}

\begin{tasks}
  \task l'intensità;
  \task la direzione della forza media che la mazza applica alla palla.
\end{tasks}

\subsubsection{Soluzione}
\label{sec:sol6}

\begin{equation*}
  \begin{matrix}
    v_1=12m/s & v_2=10m/s & \theta_1=30^o\\
    t=2\times 10^{-3}m/s
  \end{matrix}
\end{equation*}
Visto che il testo non esprime una massa, supponiamo che essa sia di 0.15kg.\\
Dopo aver supposto la massa sarà necessario ricavare i componenti di $v_0$:
\begin{eqnarray*}
  v_1x=v_1\cdot \cos \theta_1 = 9.83m/s\\
  v_1y=v_1\cdot \sin \theta_2 = 6.8m/s
\end{eqnarray*}
Mentre nel caso di $v_2$ sappiamo che risulta parallelo a $y$, quindi il risultato è:
\begin{eqnarray*}
  v_2x=0 (\perp x)\\
  v_2y=10
\end{eqnarray*}
quindi, andando a definire la quantità di moto:
\begin{eqnarray*}
  p=m\cdot v\\
  p_1=m\cdot v_1\\
  p_1x=m\cdot v_{1x} \to 1.47kg\cdot m/s\\
  p_1y=m\cdot v_{1y} \to 1.03kg\cdot m/s
\end{eqnarray*}
Quindi essendo la componente $x$ nell'istante finale nulla, la componente la suddetta componente dell'instante finale sarà altrettanto:
\begin{eqnarray*}
  p2=m\cdot v_2\\
  p2_x=0\\
  p2_y=m\cdot v_{2y}=1.5kg\cdot m/s
\end{eqnarray*}
Variazione della quantità di moto:
\begin{eqnarray*}
  \Delta p=p2-p1\\
  \Delta p_x=\underbrace{\not{p_2x}}_{0}-p_1x=-p1x=-1.47kg\cdot m/s \\
  \Delta p_y=p_2y-p_1y= 0.2 kg\cdot m/s
\end{eqnarray*}
Forza media:
\begin{eqnarray*}
  F_m=\frac{\Delta p}{\Delta t}
  \begin{cases}
    F_x=\frac{\Delta p_x}{\Delta t}=-711.9N\\
    F_y=\frac{\Delta p_y}{\Delta t}=226.2N
  \end{cases}
\end{eqnarray*}
Impostato questo sistema possiamo evincere che la componente $F_m$ è una somma sotto radice:
\begin{eqnarray*}
  F_m=\sqrt{F_x^2+F_y^2} \cong 746.99N
\end{eqnarray*}
Angolo (di rezione) $F_m$:
\begin{eqnarray*}
  \tan \phi = \frac{F_y}{F_x}\cong  - 0.3177 \\
  \phi = \tan^{-1}\left(- 0.3177\right) \cong - 17.6^o
\end{eqnarray*}
\item
  \begin{eqnarray*}
    F_{\parallel}=F_{p1}\cdot \sin \theta = 67.57N\\
    F_{\perp}
  \end{eqnarray*}
  Definire, componete per pendicolare, forza d'attrito (statica e dinamica), per valutare l'accelerazione devo consideraare tutti i componenti che agiscono nel moto (parallela, componente di attrito)
\item Un cannone posizionato su un monte alto 1km spara un proiettile con un
  proiettile cade sulla vicina valle ad una distanza orizzontale $d=3km$. A quale
  velocità iniziale è stato sparato il proiettile? Qual è il tempo di volo?
  
\subsubsection{Soluzione}
\label{sec:sol8}

\begin{equation*}
  \begin{matrix}
    y_0=1000m & d=3000m & \theta=35^o
  \end{matrix}
\end{equation*}
Primo passo è quello di definire le due equazioni del moto parabolico:
\begin{eqnarray*}
  \begin{cases}
    x=\underbrace{v_0\cos(\Theta)}_{v_0x}\cdot t\\
    y=h+\underbrace{v_0\sin\Theta}_{v_0y}-\frac{1}{2}gt^2
  \end{cases} \to
  \begin{cases}
    3000=v_0\cos\theta\cdot t\\
    0=1000+v_0\sin\theta t-\frac{1}{2}gt^2
  \end{cases}\to
  \begin{cases}
    t=\frac{3000}{v_0\cos\theta}=\\
    0=1000+v_0\sin\theta\cdot \frac{3000}{\not{v_0}\cos\theta}
    -\frac{1}{2}g\frac{3000^2}{v_0^2\cos^2\theta}
  \end{cases}\\
  \to
  \begin{cases}
    +\frac{1}{2}g\frac{3000^2}{\cos^2\theta}=1000+\tan\theta\cdot 3000 \to \frac{1}{2}g\frac{3000^2}{\cos^2\theta}=(1000+\tan \theta\cdot 3000)v_0^2
  \end{cases}\\
  \to
  \begin{cases}
    v_0^2=\sqrt{\frac{1}{2}g\frac{3000^2}{\cos^2\theta}\cdot \frac{1}{1000+3000\tan\theta}}=142.66\frac{M}{s}
  \end{cases}\\
  \begin{cases}
    t=\frac{3000}{v_0\cos\theta}=25.14s
  \end{cases}
\end{eqnarray*}
\item Un disco di massa $M=2kg$ perconferenza di raggio $r=20cm$ sul piavo privo di
  un tavolo e sostiene una mazza $M=3kg$ appesa ad un filo che possa attraverso un
  foro al centro del cerchio. Trovare a quale velocità deve muoversi $m$ per
  trattenere a riposo $M$.
  
\subsubsection{Soluzione}
\label{sec:sol10}
\begin{eqnarray*}
  m=2kg & r=20cm =0.2m & M=3kg
\end{eqnarray*}
\begin{eqnarray*}
  T=F_c\to T=F_{PM}\to F_{pm}=F_c\\
  M\cdot g=n\cdot \frac{v^2}{r}\\
  v=\sqrt{\frac{M\cdot g\cdot r}{m}}=1.71\frac{m}{s}
\end{eqnarray*}
\end{enumerate}
\end{document}

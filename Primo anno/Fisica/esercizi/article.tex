\documentclass{article}

\usepackage[utf8]{inputenc}
\usepackage{titlesec}
\usepackage{easylist}
\usepackage{hanging}
\usepackage{hyperref}
\usepackage[a4paper,top=2.0cm,bottom=2.0cm,left=2.0cm,right=2.0cm]{geometry}
\usepackage{blindtext}
\usepackage{tipa}
\usepackage{epigraph}
\usepackage{enumerate}
\usepackage{longtable}
\usepackage{setspace}
\usepackage{verbatim}
\usepackage[T1]{fontenc}
\usepackage{graphicx}
\usepackage[italian]{babel}
\usepackage{amsmath}
\usepackage{pbox}
\usepackage{fancyhdr}
\usepackage{cancel}
\usepackage{tabularx}
\usepackage{booktabs}
\usepackage{multirow}
\usepackage{longtable}
\usepackage{tikz}
\usepackage{tikz-qtree}
\usepackage{subfig}
\usepackage{xcolor}
\usepackage{amssymb}
\usepackage{mathrsfs}
\usepackage{textcomp}

\usepackage{listings}
\usepackage{color}

\definecolor{mygreen}{rgb}{0,0.6,0}
\definecolor{mygray}{rgb}{0.5,0.5,0.5}
\definecolor{mymauve}{rgb}{0.58,0,0.82}

\lstset{ 
  backgroundcolor=\color{white},   % choose the background color; you must add \usepackage{color} or \usepackage{xcolor}; should come as last argument
  basicstyle=\footnotesize,        % the size of the fonts that are used for the code
  breakatwhitespace=false,         % sets if automatic breaks should only happen at whitespace
  breaklines=true,                 % sets automatic line breaking
  captionpos=b,                    % sets the caption-position to bottom
  commentstyle=\color{mygreen},    % comment style
  deletekeywords={...},            % if you want to delete keywords from the given language
  escapeinside={\%*}{*)},          % if you want to add LaTeX within your code
  extendedchars=true,              % lets you use non-ASCII characters; for 8-bits encodings only, does not work with UTF-8
  firstnumber=1000,                % start line enumeration with line 1000
  frame=single,	                   % adds a frame around the code
  keepspaces=true,                 % keeps spaces in text, useful for keeping indentation of code (possibly needs columns=flexible)
  keywordstyle=\color{blue},       % keyword style
  language=Octave,                 % the language of the code
  morekeywords={*,...},            % if you want to add more keywords to the set
  numbers=left,                    % where to put the line-numbers; possible values are (none, left, right)
  numbersep=5pt,                   % how far the line-numbers are from the code
  numberstyle=\tiny\color{mygray}, % the style that is used for the line-numbers
  rulecolor=\color{black},         % if not set, the frame-color may be changed on line-breaks within not-black text (e.g. comments (green here))
  showspaces=false,                % show spaces everywhere adding particular underscores; it overrides 'showstringspaces'
  showstringspaces=false,          % underline spaces within strings only
  showtabs=false,                  % show tabs within strings adding particular underscores
  stepnumber=2,                    % the step between two line-numbers. If it's 1, each line will be numbered
  stringstyle=\color{mymauve},     % string literal style
  tabsize=2,	                   % sets default tabsize to 2 spaces
  title=\lstname                   % show the filename of files included with \lstinputlisting; also try caption instead of title
}

\linespread{1.5} % l'interlinea

\frenchspacing

\newcommand{\abs}[1]{\lvert#1\rvert}

\usepackage{floatflt,epsfig}

\usepackage{multicol}
\newcommand\yellowbigsqcup[1][\displaystyle]{%
  \fboxrule0pt
  \ifx#1\textstyle\fboxsep-0.6pt\else\fboxsep-1.25pt\fi
  \mathrel{\fcolorbox{white}{yellow}{$#1\bigsqcup$}}}

\title{Esercizi}
\author{Nicola Ferru}
\begin{document}
\maketitle
\begin{enumerate}
\item In motociclista inizialmente vieggia per 3 minuti verso sud con una velocità di 20m/s. Nei successivi 2 minuti dirige verso ovest 25m/s poi un minuto a nord-overst per 30 m/s.
  \begin{itemize}
  \item il vettore spostamento totale;
  \item la velocià scalare media;
  \item il vetotre velocità media. si utilizzi un sistema di riferimento con assi x con positivo verso Est.
  \end{itemize}
  \begin{eqnarray}
    \label{eq:esercizio1}
    t_1=3.00min \to 180
  
  \begin{eqnarray*}
    s_1=v_1\cdot t_1=3600m\\
    s_2=v_2\cdot t_2=3000m\\
    s_3=v_3\cdot t_3=1800m & s_3x = 18800\cdot \cos(45)=1272.78m
  \end{eqnarray*}
  Adesso sarà possibile calcolare lo spostamento totale in $x$ e $y$
  \begin{eqnarray*}
    \begin{cases}
      s_{tot}x= 52+s_3x=3000m+1272.73m=4272.79m \\
      s_{tot}y= s_1-s_3y= 3600m=2327.21m
    \end{cases}
  \end{eqnarray*}
  Ora, sarà possibile calcolare lo spazio totale
  \begin{eqnarray*}
    \vec{s}_{tot}=\vec{s}_{tot}x+\vec{s}_{tot}y\\
    s_{tot}=\sqrt{s_{tot}x^2+s_{tot}y^2}=4855.45m
  \end{eqnarray*}
  dopo aver fatto il calcolo dello spazio, adesso è necessario calcolare la velocità media:
  \begin{equation*}
    v_m=\frac{\Delta s}{\Delta t}=\frac{4855.46m}{(180+120+60)s}=13.52m/s
  \end{equation*}
\item un aventore lancia un boccale vuoto in sul bancone perché venga nuovamente riempito, il bancone è alto un 1.22m, esso non viene afferrato dal barista e cate a terra con una rotta parabolica di 1.40m.
  \begin{itemize}
  \item qual'è la velocità con cui ha lasciato il bancone?
  \item Qual'è la durezuibe della velocità del boccare poco prima di atterrare?
  \end{itemize}
  
  \subsubsection{Soluzione}
  \label{sec:soluzione2}
  \begin{eqnarray*}
    h=1.22m & x_1=1.40m\\
    v_0=?
  \end{eqnarray*}
  Partendo dal sistema base si può lavorare nel seguente modo:
  \begin{eqnarray*}
    \begin{cases}
      x=x_0+v_{0x}\cdot t\\
      y=-\frac{1}{2}
    \end{cases}\to
    \begin{cases}
      1.4m=v_0\cdot t & \to v_0=\frac{1.4}{t}=2.8\frac{m}{s}\\
      0=-\frac{1}{2}g\cdot t^2+1.22m
    \end{cases} \to t=\sqrt{\frac{1.22\cdot 2}{g}}=0.5s 
  \end{eqnarray*}
\item Un astronauta fa un salto con una velocità di $3m/s$ su un pianeta sconosciuta e atterra dopo 15m, quel'è la spinta gravitazionale?
  
\subsubsection{Soluzione}
\label{sec:sol3}


\end{enumerate}

\end{document}

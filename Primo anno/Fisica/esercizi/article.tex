\documentclass{report}

\usepackage[utf8]{inputenc}
\usepackage{titlesec}
\usepackage{easylist}
\usepackage{hanging}
\usepackage{hyperref}
\usepackage[a4paper,top=2.0cm,bottom=2.0cm,left=2.0cm,right=2.0cm]{geometry}
\usepackage{blindtext}
\usepackage{tipa}
\usepackage{epigraph}
\usepackage{enumerate}
\usepackage{longtable}
\usepackage{setspace}
\usepackage{verbatim}
\usepackage[T1]{fontenc}
\usepackage{graphicx}
\usepackage[italian]{babel}
\usepackage{amsmath}
\usepackage{pbox}
\usepackage{fancyhdr}
\usepackage{cancel}
\usepackage{tabularx}
\usepackage{booktabs}
\usepackage{multirow}
\usepackage{longtable}
\usepackage{tikz}
\usepackage{tikz-qtree}
\usepackage{subfig}
\usepackage{xcolor}
\usepackage{amssymb}
\usepackage{mathrsfs}
\usepackage{textcomp}
\usepackage{tasks}
\usepackage{listings}
\usepackage{color}

\definecolor{mygreen}{rgb}{0,0.6,0}
\definecolor{mygray}{rgb}{0.5,0.5,0.5}
\definecolor{mymauve}{rgb}{0.58,0,0.82}

\lstset{ 
  backgroundcolor=\color{white},   % choose the background color; you must add \usepackage{color} or \usepackage{xcolor}; should come as last argument
  basicstyle=\footnotesize,        % the size of the fonts that are used for the code
  breakatwhitespace=false,         % sets if automatic breaks should only happen at whitespace
  breaklines=true,                 % sets automatic line breaking
  captionpos=b,                    % sets the caption-position to bottom
  commentstyle=\color{mygreen},    % comment style
  deletekeywords={...},            % if you want to delete keywords from the given language
  escapeinside={\%*}{*)},          % if you want to add LaTeX within your code
  extendedchars=true,              % lets you use non-ASCII characters; for 8-bits encodings only, does not work with UTF-8
  firstnumber=1000,                % start line enumeration with line 1000
  frame=single,	                   % adds a frame around the code
  keepspaces=true,                 % keeps spaces in text, useful for keeping indentation of code (possibly needs columns=flexible)
  keywordstyle=\color{blue},       % keyword style
  language=Octave,                 % the language of the code
  morekeywords={*,...},            % if you want to add more keywords to the set
  numbers=left,                    % where to put the line-numbers; possible values are (none, left, right)
  numbersep=5pt,                   % how far the line-numbers are from the code
  numberstyle=\tiny\color{mygray}, % the style that is used for the line-numbers
  rulecolor=\color{black},         % if not set, the frame-color may be changed on line-breaks within not-black text (e.g. comments (green here))
  showspaces=false,                % show spaces everywhere adding particular underscores; it overrides 'showstringspaces'
  showstringspaces=false,          % underline spaces within strings only
  showtabs=false,                  % show tabs within strings adding particular underscores
  stepnumber=2,                    % the step between two line-numbers. If it's 1, each line will be numbered
  stringstyle=\color{mymauve},     % string literal style
  tabsize=2,	                   % sets default tabsize to 2 spaces
  title=\lstname                   % show the filename of files included with \lstinputlisting; also try caption instead of title
}

\linespread{1.5} % l'interlinea

\frenchspacing

\newcommand{\abs}[1]{\lvert#1\rvert}

\usepackage{floatflt,epsfig}

\usepackage{multicol}
\newcommand\yellowbigsqcup[1][\displaystyle]{%
  \fboxrule0pt
  \ifx#1\textstyle\fboxsep-0.6pt\else\fboxsep-1.25pt\fi
  \mathrel{\fcolorbox{white}{yellow}{$#1\bigsqcup$}}}

\title{Esercizi}
\author{Nicola Ferru}
\begin{document}
\maketitle
\begin{enumerate}
\item In motociclista inizialmente vieggia per 3 minuti verso sud con una velocità di 20m/s. Nei successivi 2 minuti dirige verso ovest 25m/s poi un minuto a nord-overst per 30 m/s.
  \begin{itemize}
  \item il vettore spostamento totale;
  \item la velocià scalare media;
  \item il vetotre velocità media. si utilizzi un sistema di riferimento con assi x con positivo verso Est.
  \end{itemize}
  \begin{eqnarray}
    \label{eq:esercizio1}
    t_1=3.00min \to 180
  \end{eqnarray}
  \begin{eqnarray*}
    s_1=v_1\cdot t_1=3600m\\
    s_2=v_2\cdot t_2=3000m\\
    s_3=v_3\cdot t_3=1800m & s_3x = 18800\cdot \cos(45)=1272.78m
  \end{eqnarray*}
  Adesso sarà possibile calcolare lo spostamento totale in $x$ e $y$
  \begin{eqnarray*}
    \begin{cases}
      s_{tot}x= 52+s_3x=3000m+1272.73m=4272.79m \\
      s_{tot}y= s_1-s_3y= 3600m=2327.21m
    \end{cases}
  \end{eqnarray*}
  Ora, sarà possibile calcolare lo spazio totale
  \begin{eqnarray*}
    \vec{s}_{tot}=\vec{s}_{tot}x+\vec{s}_{tot}y\\
    s_{tot}=\sqrt{s_{tot}x^2+s_{tot}y^2}=4855.45m
  \end{eqnarray*}
  dopo aver fatto il calcolo dello spazio, adesso è necessario calcolare la velocità media:
  \begin{equation*}
    v_m=\frac{\Delta s}{\Delta t}=\frac{4855.46m}{(180+120+60)s}=13.52m/s
  \end{equation*}
\item un aventore lancia un boccale vuoto in sul bancone perché venga nuovamente riempito, il bancone è alto un 1.22m, esso non viene afferrato dal barista e cate a terra con una rotta parabolica di 1.40m.
  \begin{itemize}
  \item qual'è la velocità con cui ha lasciato il bancone?
  \item Qual'è la durezuibe della velocità del boccare poco prima di atterrare?
  \end{itemize}
  
  \subsubsection{Soluzione}
  \label{sec:soluzione2}
  \begin{eqnarray*}
    h=1.22m & x_1=1.40m\\
    v_0=?
  \end{eqnarray*}
  Partendo dal sistema base si può lavorare nel seguente modo:
  \begin{eqnarray*}
    \begin{cases}
      x=x_0+v_{0x}\cdot t\\
      y=-\frac{1}{2}
    \end{cases}\to
    \begin{cases}
      1.4m=v_0\cdot t & \to v_0=\frac{1.4}{t}=2.8\frac{m}{s}\\
      0=-\frac{1}{2}g\cdot t^2+1.22m
    \end{cases} \to t=\sqrt{\frac{1.22\cdot 2}{g}}=0.5s 
  \end{eqnarray*}
\item Un astronauta fa un salto con una velocità di $3m/s$ su un pianeta sconosciuta e atterra dopo 15m, quel'è la spinta gravitazionale?
  
\subsubsection{Soluzione}
\label{sec:sol3}

    \item Un punto materiale che si muove in senso orario una circonferenza di 2.5m, ad una accelerazione di $15m/s^2$ e conosciuamo un angolo $\beta = 30^o$.
    \begin{itemize}
    \item Determinale accelerazione centripeta;
    \item Modulo della velocità;
    \item Accelerazione tangenziale.
    \end{itemize}
    
\subsubsection{soluzione}
\label{sec:sol4}

    Determiniamo l'accelerazione cintripeta
    \begin{equation*}
      ac=a_{tot} \cdot \cos 30 = 13m/s^2
    \end{equation*}
    Ricaviamo il modulo della velocità:
    \begin{equation*}
      a_c=\frac{V^2}{\not{r}} *\not{r} \to v=\sqrt{a_c\cdot r} = 5.7m/s
    \end{equation*}
    Determiniamo l'accelerazione tangenziale:
    \begin{equation*}
      a_t=a\cdot \sin 30 = 7.5m/s^2
    \end{equation*}
    
  \item La ruota panoramica di un luna park ha un raggio di $15m$ e compie ogni minuto $5$ giri attorno al proprio asse
    orizzontale.
    \begin{tasks}
      \task Qual è il suo periodo di rotazione?
      \task Qual è il modulo la direzione e verso dell’accelerazione centripeta cui è sottoposto un passeggero nel punto più alto?
      \task Qual è il modulo direzione e verso dell’accelerazione centripeta quando il passeggero è nel punto più basso?
    \end{tasks}
    
\subsubsection{Soluzione}
\label{sec:sol5}
\begin{equation*}
  \begin{matrix}
    r=15m & f=5\frac{giri}{m}\\
    T=\frac{1}{f} \to \frac{1}{\frac{60}{5}}= 12s & a_c=\frac{V^2}{r}= 4.06m/s^2\to \underbrace{v=\omega*r}_{\omega=\frac{2\pi}{T}=0.52rad/s}=7.8m/s 
  \end{matrix}
\end{equation*}
\item Un cannone posizionato su un monte alto $1km$ spara un proiettile con un angolo di $35^o$ rispetto all’orizzontale. Il proiettile cade sulla vicina valle ad una distanza orizzontale $d=3km$. A quale velocità iniziale è stato sparato il proiettile? Qual è il tempo di volo?

\subsubsection{Soluzione}
\label{sec:sol6}

\begin{equation*}
  \begin{matrix}
    \theta=35^o & d=3km \to d=3000m & h=1km \to 1000m
  \end{matrix}
\end{equation*}
partendo da suddetti dati possiamo utilizzare la seguente formula parametrica:
\begin{eqnarray*}
  \begin{cases}
    x=x_0+v_{0x}t\\
    y=-\frac{1}{2}gt^2+v_{0y}t+y_0
  \end{cases}\to
  \begin{cases}
    3000m = V_0\cos \theta \cdot t\\
    0=1000m - \frac{1}{2} gt^2+v_0\sin\theta\cdot t
  \end{cases}\to
  v_0=\frac{x}{\cos\theta \cdot t}\\
  \to
  \begin{cases}
    0=y_0-\frac{1}{2}gt_2+\left(\frac{x}{\cos \theta\cdot \not{t}}\right)\cdot \sin \theta \cdot \not{t}
  \end{cases}
  \begin{cases}
    ...\\
    t=\sqrt{\left(y_0+\frac{x}{\cos\theta}\cdot \sin\theta\right)\frac{2}{g}}=25.14s
  \end{cases} \\
  \to \begin{cases}
    v_0=\frac{x}{\cos\cdot t} = 151.71m/s
  \end{cases}
\end{eqnarray*}
\item Una mazza da baseball colpisce una palla. Prima dell'impatto la palla va alla velocità $v_1$ di modulo $12m/s$ e angolo rispetto all'asse $x$ di $\theta_1=35^o$. Dopo ha velocità $v_2$ di modulo $10m/s$ e direzione perpendicolare all'asse $x$. L'asse $x$. L'evento dura $2ms$. 
\paragraph{Determinare}

\begin{tasks}
  \task l'intensità;
  \task la direzione della forza media che la mazza applica alla palla.
\end{tasks}

\subsubsection{Soluzione}
\label{sec:sol6}

\begin{equation*}
  \begin{matrix}
    v_1=12m/s & v_2=10m/s & \theta_1=30^o\\
    t=2\times 10^{-3}m/s
  \end{matrix}
\end{equation*}
Visto che il testo non esprime una massa, supponiamo che essa sia di 0.15kg.\\
Dopo aver supposto la massa sarà necessario ricavare i componenti di $v_0$:
\begin{eqnarray*}
  v_1x=v_1\cdot \cos \theta_1 = 9.83m/s\\
  v_1y=v_1\cdot \sin \theta_2 = 6.8m/s
\end{eqnarray*}
Mentre nel caso di $v_2$ sappiamo che risulta parallelo a $y$, quindi il risultato è:
\begin{eqnarray*}
  v_2x=0 (\perp x)\\
  v_2y=10
\end{eqnarray*}
quindi, andando a definire la quantità di moto:
\begin{eqnarray*}
  p=m\cdot v\\
  p_1=m\cdot v_1\\
  p_1x=m\cdot v_{1x} \to 1.47kg\cdot m/s\\
  p_1y=m\cdot v_{1y} \to 1.03kg\cdot m/s
\end{eqnarray*}
Quindi essendo la componente $x$ nell'istante finale nulla, la componente la suddetta componente dell'instante finale sarà altrettanto:
\begin{eqnarray*}
  p2=m\cdot v_2\\
  p2_x=0\\
  p2_y=m\cdot v_{2y}=1.5kg\cdot m/s
\end{eqnarray*}
Variazione della quantità di moto:
\begin{eqnarray*}
  \Delta p=p2-p1\\
  \Delta p_x=\underbrace{\not{p_2x}}_{0}-p_1x=-p1x=-1.47kg\cdot m/s \\
  \Delta p_y=p_2y-p_1y= 0.2 kg\cdot m/s
\end{eqnarray*}
Forza media:
\begin{eqnarray*}
  F_m=\frac{\Delta p}{\Delta t}
  \begin{cases}
    F_x=\frac{\Delta p_x}{\Delta t}=-711.9N\\
    F_y=\frac{\Delta p_y}{\Delta t}=226.2N
  \end{cases}
\end{eqnarray*}
Impostato questo sistema possiamo evincere che la componente $F_m$ è una somma sotto radice:
\begin{eqnarray*}
  F_m=\sqrt{F_x^2+F_y^2} \cong 746.99N
\end{eqnarray*}
Angolo (di rezione) $F_m$:
\begin{eqnarray*}
  \tan \phi = \frac{F_y}{F_x}\cong  - 0.3177 \\
  \phi = \tan^{-1}\left(- 0.3177\right) \cong - 17.6^o
\end{eqnarray*}
\item
  \begin{eqnarray*}
    F_{\parallel}=F_{p1}\cdot \sin \theta = 67.57N\\
    F_{\perp}
  \end{eqnarray*}
  Definire, componete per pendicolare, forza d'attrito (statica e dinamica), per valutare l'accelerazione devo consideraare tutti i componenti che agiscono nel moto (parallela, componente di attrito)
\item Un cannone posizionato su un monte alto 1km spara un proiettile con un
  proiettile cade sulla vicina valle ad una distanza orizzontale $d=3km$. A quale
  velocità iniziale è stato sparato il proiettile? Qual è il tempo di volo?
  
\subsubsection{Soluzione}
\label{sec:sol8}

\begin{equation*}
  \begin{matrix}
    y_0=1000m & d=3000m & \theta=35^o
  \end{matrix}
\end{equation*}
Primo passo è quello di definire le due equazioni del moto parabolico:
\begin{eqnarray*}
  \begin{cases}
    x=\underbrace{v_0\cos(\Theta)}_{v_0x}\cdot t\\
    y=h+\underbrace{v_0\sin\Theta}_{v_0y}-\frac{1}{2}gt^2
  \end{cases} \to
  \begin{cases}
    3000=v_0\cos\theta\cdot t\\
    0=1000+v_0\sin\theta t-\frac{1}{2}gt^2
  \end{cases}\to
  \begin{cases}
    t=\frac{3000}{v_0\cos\theta}=\\
    0=1000+v_0\sin\theta\cdot \frac{3000}{\not{v_0}\cos\theta}
    -\frac{1}{2}g\frac{3000^2}{v_0^2\cos^2\theta}
  \end{cases}\\
  \to
  \begin{cases}
    +\frac{1}{2}g\frac{3000^2}{\cos^2\theta}=1000+\tan\theta\cdot 3000 \to \frac{1}{2}g\frac{3000^2}{\cos^2\theta}=(1000+\tan \theta\cdot 3000)v_0^2
  \end{cases}\\
  \to
  \begin{cases}
    v_0^2=\sqrt{\frac{1}{2}g\frac{3000^2}{\cos^2\theta}\cdot \frac{1}{1000+3000\tan\theta}}=142.66\frac{M}{s}
  \end{cases}\\
  \begin{cases}
    t=\frac{3000}{v_0\cos\theta}=25.14s
  \end{cases}
\end{eqnarray*}
\item Un disco di massa $M=2kg$ perconferenza di raggio $r=20cm$ sul piavo privo di
  un tavolo e sostiene una mazza $M=3kg$ appesa ad un filo che possa attraverso un
  foro al centro del cerchio. Trovare a quale velocità deve muoversi $m$ per
  trattenere a riposo $M$.
  
\subsubsection{Soluzione}
\label{sec:sol10}
\begin{eqnarray*}
  m=2kg & r=20cm =0.2m & M=3kg
\end{eqnarray*}
\begin{eqnarray*}
  T=F_c\to T=F_{PM}\to F_{pm}=F_c\\
  M\cdot g=n\cdot \frac{v^2}{r}\\
  v=\sqrt{\frac{M\cdot g\cdot r}{m}}=1.71\frac{m}{s}
\end{eqnarray*}
\item Una molla può esssere compressa di 2 cm da una forza di 270 N. Un bllcco di massa $m=12kg$, inizialmente fermo in cima ad un piano inclinato privo di attrito che forma un angolo di $30^o$ con il piano orizzontale, viene lasciato andare. Il blocco si ferma dopo aver compresso la molla di 5.5cm.
  \begin{tasks}
    \task in questo momento di quanto si è spostato lungo il piano inclinato?
    \task Qual'è la valocità del blocco quando arriva a toccare la molla?
  \end{tasks}
  
\subsubsection{Soluzione}
\label{sec:sol11}
\begin{eqnarray*}
  m=12kg & x_0=2cm\to 0.02cm & \alpha=30^o\\
  & x_1=5.5cm \to 0.055m 
\end{eqnarray*}
In questa situazione è il caso di applicare il principio di energia potenziale,
infatti, essendo una situazione composita, con la presenta di un piano inclinato e
di una molla è conveniente per via della legge della conservazione del energia.
\begin{eqnarray*}
  Ug=mgh\\
  h=l\sin30^o
\end{eqnarray*}
dopo aver stimato questi dati, andiamo a definire l'enegia potenziale elastica:
\begin{eqnarray*}
  Ue=\frac{1}{2}kx_1^2
\end{eqnarray*}
\begin{eqnarray*}
  F=kx\\
  k=\frac{F}{x}=\frac{270N}{0.02m}=13500N/m
\end{eqnarray*}
quindi una volta aver ricavato k possiamo tornatre al espressione $Ue$, facendo un ugualianza $Ug=Ue$, perché l'energia potenziale si trasparma in energia elastica. Sostituiamo le corrispettive espressioni con i parametri visti prima:
\begin{eqnarray*}
  mgh=\frac{1}{2}kx_1^2\to mg(l\sin30^o)=\frac{1}{2}(13500N/m)(0.055m)^2=mgl\sin30^o=\frac{1}{2}kx^2\\
  l=\frac{1}{2}x^2k\cdot \frac{1}{mg\sin 30} =0.347m
\end{eqnarray*}
mentre, il secondo punto lo si può svolgere nel seguente modo, andando a sotituire
all'interno della formula di $Ug$\footnote{formula dell'energia potenziale
  gravitazionale}, il valore di distanza $l$:
\begin{eqnarray*}
  Ug_2=mg\Delta h & \Delta h= (l-x)\sin30 & k=\frac{1}{2}mv^2\\
  mg(l-x)\sin 30 = \frac{1}{2}mv^2 & v=\sqrt{mg(l-x)\sin 30\cdot\frac{1}{\frac{1}{2}m}}=1.69\frac{m}{s}
\end{eqnarray*}
\item All'istante $t=0$ una sola forza $F$ d'intensità constanza
  comincia ad agire su un sasso che si muove nello spazio vuoto lungo
  l'asse $x$ verso le $x$ crescenti. Il sasso prosegue il suo moto in
  questa direzione.
  \begin{tasks}
    \task negli istanti successivi trovare quali delle seguenti funzioni
    $x(t)$ della posizione sono compatibili con il compatibili con il
    moto del sasso;
    \begin{enumerate}
    \item $x=4t-3$;
    \item $x=-4t^2+6t$;
    \item $x=4t^2+6t-3$.
    \end{enumerate}
    \task Per quale delle funzioni $F$ ha verso opposto a quello del moto
    iniziale del sasso?
  \end{tasks}
  
\subsubsection{Soluzioni}
\label{sol12}

Visto che si tratta di un oggetto sottoposto ad una forza, mi devo
aspettare un moto accelerato, per questo motivo, devo cercare l'equazione
di $x$ che risulti in una accelerazione non nulla. Per fare questo
risolvo la derivata di $\frac{dx}{dt}$, ottenendo quindi le 3
corrispettive volocità, ottenuto questi valori, derivo nuovamente rispetto al tempo, ottenendo quindi le accelerazioni.
\begin{eqnarray*}
  x(t)=4t-3 & v_1=\frac{dx_1}{dt}=4\frac{dx_1}{dt}=4 & a_1(t)=\frac{dv}{dt}=0
\end{eqnarray*}
L'accelerazione è nulla, di consequenza questa funzione di $x$ non può
corrispondere alla situazione.
\begin{eqnarray*}
  x(t)=-4t^2+6t-3 & v_2(t)=\frac{dx_2}{dt}=-8t+6 \\
  a_2(t)=\frac{dv_2}{dt}= -8 
\end{eqnarray*}
tratandosi di una funzione non nulla questa funzione può corrispondere
al quesito.
\begin{eqnarray*}
  x_3(t)=4t^2+6t-3 & v_3(t)=\frac{dx_3}{dt}=8t+6\\
  a_3(t)=\frac{dv_3}{dt}=8
\end{eqnarray*}
Anche questa funzione potrebbe essere qualla che descrive il quesito.

Possiamo affermare che la forza si oppone al movimento, per l'equazione
$x_2t$, in quanto il segno dell'accelerazione è meno. Questo è sintomo
che la funzione sia opposta al movimento.
\item (Onde sinosoidali)
  \begin{tasks}
    \task Ricavare un'equazione che descriva un'onda trasversale sinusoidali
    \task Qual è la velocità massama di un punto della corda?
    \task Calcolare la velocità di propagazione dell'onda.
  \end{tasks}
\subsubsection{Soluzione}
\label{sec:sol13}

\begin{eqnarray*}
  \lambda=0.15m & f=150Hz & A=47cm =0.47m
\end{eqnarray*}
\begin{tasks}
  \task $z(x,t)=A\sin(Kx+\omega t + \phi)$
  \begin{itemize}
  \item $k=\frac{2\pi}{\lambda}=41.89\frac{1}{m}$ ($N^o$ d'onda)
  \item $\omega=2\pi f=300\pi\cong 942.d8 \frac{rad}{s}$
  \end{itemize}
  $\to{} z_1(x,t)=0.47\sin(41.89x+300\pi t)$
  \task $Vz=\frac{dz}{dt}= A\omega \cos (kx+\omega t)$
  \begin{eqnarray*}
    v_{z_1max}=A\omega=0.47\cdot 300\pi = 443.22 m/s
  \end{eqnarray*}
  \task $z_2(x,t)=0.47\sin(41.89x-300\pi t)$
  \begin{eqnarray*}
    z(x,t)= z_1(x,t)+z_2(x,t)=\\
    =0.47\sin(41.89x+300\pi t)+0.47\sin(41.89x-300\pi t)=\\
    =0.47[2\sin (41.89x) \cos (300\pi t)]=\\
    =0.94\sin(41.89x)\cos(300\pi t)
  \end{eqnarray*}
  \task $v$ propagazione
  \begin{eqnarray*}
    v=\lambda f = 0.15\cdot 150=22.5\frac{m}{s}
  \end{eqnarray*}
\end{tasks} 
\item Un uomo colpisce con un martello una lunga barra di alluminio a una
  estremità. Una donna, all'altra estremità con l'orecchio vicino alla barra,
  sente il suono del colpo due volte (una attraverso l'aria e una attraverso la
  barra), con un intervallo di 0.12s. Sapendo la velocità del suono nella barra è
  15 volte maggiore rispetto a quella in aria, quanto è lungo la barra?
  
\subsubsection{Soluzione}
\label{sec:sol14}

  \begin{eqnarray*}
    \Delta t= 0.12s\\
    V_0=15\dot V_a\\
    L=?
  \end{eqnarray*}
  \begin{eqnarray*}
    V=\frac{s}{t}=v_a=\frac{L}{t_a} & t_a=\frac{L}{V_a} & t_b=\frac{L}{v_b}
                                                          =\frac{L}{V_a}\\
    \Delta t=t_a-t_b\\
    \Delta t=\frac{L}{v_a}-\frac{v}{v_b}=\frac{L}{v_a}-\frac{L}{15\cdot v_a}=\frac{L}{v+a}\left(1-\frac{1}{15}\right)\\
    v_a=343m/s\\
    L=\frac{\Delta t\cdot v_a}{\left(1-\frac{1}{15}\right)}= 44.1m
  \end{eqnarray*}
\item (piano inclinato) Un ragazzo trattiene una cassa di massa $m=15kg$ su di un pendio inclinato di
  $45^o$ applicando una forza $\vec{F}$.
  \begin{tasks}
  \task Trascurando l'attrito, quanto vale la forza che esercita il ragazzo?
  \task Se il ragazzo lascia andare la cassa, che arriva alla base del piano inclinato ad una velocità
    $v=4.5m/s$, quanto è lungo il tragitto percorso?
  \end{tasks}
  
\subsubsection{Soluzione}
\label{sec:sol15}
\begin{tasks}
  \task Scegliendo un sistema di opportuno, in cui l'asse delle $x$ è parallelo alla superficie del
  piano
inclinato e l'asse delle $y$ è perpendicolare ad essa.

Le forze che agiscono, oltre alla forza $\vec{F}$ esercitata dal ragazzo, sono la reazione vincolare
$\vec{N}$ esercitata del piano inclinato e la forza peso $\vec{P}$ della cassa.

La forza peso $\vec{P}$ deve essere scomposta nelle componenti $P_\parallel$ e $P_\bot$, ovvero le
proiettile di $\vec{P}$ lungo gli assi $x$ ed $y$, rispettivamente, rispetto al riferimento.

Dal momento che il corpo rimane fermo, la risultante $\vec{R}$ delle forza può essere uguagliato a 0:
\begin{equation*}
  \vec{F}+\vec{P}+\vec{N}=0
\end{equation*}
Esplicitiamo i segni e dividiamo l'equazione, una lungo l'asse $x$ ed una lungo l'asse $y$, ricordando
che prendiamo il segno $+$ se il vettore ha lo stesso verso dell'asse $x$ o $y$ ed il segno $-$ se ha
verso ha verso opposto:
\begin{equation*}
  \begin{matrix}
    (x):&F-P_\parallel =0\\
    (y):&N-P_\bot=0
  \end{matrix}
\end{equation*}
Esplicitiamo le espressioni di $P_\parallel$ e $P_\bot$
\begin{equation*}
  \begin{matrix}
    (x): & F-mg\cdot \sin(45^o)=0\\
    (y): & N-mg\cdot \cos(45^o)=0
  \end{matrix}
\end{equation*}
Sfruttando l'equazione lungo l'asse ($x$) otteniamo:
\begin{equation*}
  F=mg\cdot \sin(45^o)=104N
\end{equation*}
dove $g=9.81m/s^2$ è l'accelerazione gravitazionale.
\task Se il ragazzo andare la cassa, la lunguezza del tragitto percorso si può ottenere dalla relazione:
\begin{equation*}
  v^2_{fin} -v_{ini}^2=2a\Delta x
\end{equation*}
dove $\Delta x$ è la lungheza del tragitto percorso ed $a$ l'accelerazione della cassa.

Nel momento in cui la cassa inizia a scendere, la casse è soggetta alla sola componente $P_\parallel$
della forza peso:
\begin{equation*}
  mg\cdot \sin \theta=ma
\end{equation*}
da cui:
\begin{equation*}
  \Delta x=\frac{v^2_{fin}-v_{ini}^2}{1.5m}
\end{equation*}
\end{tasks}
\item Una cassa di massa $m=10kg$ viene lanciata dalla base di un angolo $\theta=30^o$ con una velocità
  $v_0=5.0m/s$ diretta verso l'alto. Sia $\mu_d=0.2$ il coefficiente di attrito dinamnico.
  \begin{tasks}
    \task Quanto vale la lunghezza del tratto percorso dalla cassa prima di fermarsi.
  \end{tasks}
  
\subsubsection{Soluzione}
\label{sec:sol16}

Scegliamo un sistema di riferimento opportuno, in cui l'asse delle $x$ è parallelo alla superficie del
piano inclinato e l'asse delle $y$ è perpendicolare ad essa.

Le forze che agiscono sono la forza di attrito $\vec{F}_a$, la reazione vincolare $\vec{N}$ esercitata
dal piano inclunato e la forza peso $\vec{P}$ della cassa.

La forza peso $\vec{P}$ deve essere scomposta nelle componenti $P_\parallel$ e $P_\bot$, ovvero le
proiezioni di $\vec{P}$ lungo gli assi $x$ ed $y$.

La risultante $\vec{R}$ delle forze è:
\begin{equation*}
  \vec{F}_a+\vec{P}+\vec{N}=m\vec{a}
\end{equation*}
Esplicitiamo i segni e dividiamo l'equazioni, una lungo l'asse $x$ ed una lungo l'asse $y$, ricordando
che prendiamo il segno $+$ se il vettore ha lo stesso verso dell'asse $x$ o $y$ ed il segno $-$ se ha
verso opposto
\begin{equation*}
  \begin{matrix}
    (x): & -F_a -P_\parallel=ma\\
    (y): & N-P_\bot =0
  \end{matrix}
\end{equation*}
dove $a$ compore solo lungo $x$ perché il corpo si muove solo lungo questo asse.

Esplicitiamo le espressioni di $F_a,P_\parallel$ e $P_\bot$ e mettiamo a sistema:
\begin{equation*}
  \begin{cases}
    -\mu_dN-mg\sin\theta=ma\\
    N-mg\cos\theta=0
  \end{cases}
\end{equation*}
Sostituiamo l’espressione di $N$ nell’equazione lungo $x$:
\begin{equation*}
  \begin{cases}
    -\mu_dg\cdot\cos\theta-g\cdot\sin\theta=a\\
    N=mg\cdot \cos\theta
  \end{cases}
\end{equation*}
da cui:
\begin{equation*}
  a=-6.6m/s^2
\end{equation*}
La lunghezza $L$ del tratto che la cassa riesce a percorrere prima di fermarsi $(v_{fin}=0m/s)$ si
ottiene dalla relazione:
\begin{equation*}
  v_{fin}^2-v_0^2=2aL
\end{equation*}
da cui:
\begin{equation*}
  L=\frac{-v_0^2}{2a}=1.9m
\end{equation*}
\item Una cassa di peso $400kg$ è lanciata con velocità $14.5m/s$ dalla cima di un piano inclinato
  liscio lungo $L=1.0m$ che forma un angolo di $\theta$ rispetto al piano orizzontale. dopo un
  tempo $t=1.5s$ la cassa arriva ala base del piano inclinato a velocità $v=m/s$.
  \begin{tasks}
    \task Quanto vale l'angolo $\theta$?
  \end{tasks}
  
\subsubsection{Soluzione}
\label{sec:sol17}

Scegliendo un sistema di riferimento opportuno, in cui l'asse delle $x$ è parallelo alla superficie del
$y$ è parallelo alla superficie del piano inclinato e l'asse delle $y$ è perpendicolare ad esso

Le forze che agiscono sono la reazione vincolare $\vec{N}$ esercitata dal piano inclinato e la forza
peso $\vec{P}$ della cassa.

La forza peso $\vec{P}$ deve essere scomposta nelle componenti $P_\parallel$ e $P_\bot$, ovvero le
proiezioni di $\vec{P}$ lungo gli assi $x$ ed $y$, del nostro sistema riferimento.

La risultante $\vec{R}$ delle forza è:
\begin{equation*}
  \vec{P}+\vec{N}=m\vec{a}
\end{equation*}
Esplicitiamo i segni e dividiamo l'equazione in due equazioni, una lungo l'asse $x$ ed una lungo l'asse
$y$, ricordando che prendiamo il segno $+$ se il vettore ha lo stesso verso dell'asse $x$ o $y$ ed il
segno $-$ se ha verso opposto
\begin{equation*}
  \begin{matrix}
    (x): & P_\parallel=ma\\
    (y): & N-P_\bot= 0
  \end{matrix}
\end{equation*}
dove $a$ compare solo nell'equazione lungo $x$ perché il corpo si muove solo lungo questo asse.

Esplicitiamo le espressioni $P_\parallel$ e $P_\bot$ e le si mette a sistema:
\begin{eqnarray*}
  \begin{cases}
    mg\cdot \sin\theta=ma\\
    N-mg\cdot\cos\theta=0
  \end{cases} &
                \begin{cases}
                  g\cdot \sin\theta = a\\
                  N=mg\cdot \cos\theta
                \end{cases}
\end{eqnarray*}
La cassa si muove di moto uniformemente accelerato. È possibile ricavare l'espressione di $a$ dalla
relazione:
\begin{eqnarray*}
  v=v_0+at\\
  a=\frac{v-v_0}{t}=1.53m/s^2
\end{eqnarray*}
Dall'espressione:
\begin{eqnarray*}
  g\cdot \sin\theta=a
\end{eqnarray*}
e si ottiene:
\begin{eqnarray*}
  \theta=\sin^{-1}\left(\frac{a}{g}\right)=9.0^o
\end{eqnarray*}
\item Due casse di masse $m_1=4.5kg$ ed $m_2=2.3kg$ sono collegate da una fune ideale. La massa $m_2$
  scivola senza attrito su un piano inclinato di $\theta=45^o$
  \begin{tasks}
    \task Calcola l'accelerazione $a$ delle due casse
  \end{tasks}
  
\subsubsection{soluzione}
\label{sec:sol18}
La forza peso $\vec{P}$ per la cassa $m_2$ è stata scomposta nei suoi componenti $P_\parallel$ e $P_\bot$.

Considerando la risultante delle forze (ovvero la somma di tutte le forze) lungo l'asse $x$ e lungo
l'asse $y$ per la cassa di massa $m_2$:
\begin{eqnarray*}
  \begin{cases}
    T-P_\parallel=m_2a_2\\
    N-P_\bot=0
  \end{cases}
\end{eqnarray*}
dove bisogna ricordare che la cassa si sta muovendo solo lungo l'asse $x$ ovvero rimane sempre a contato
con il piano senza mai staccarsi, per cui $a_y=0$ e $a_x=a$.

Per la cassa $m_1$ invece il moto avviene solo lungo l'asse $y$, quindi:
\begin{equation*}
  -T+P=m_1a_1
\end{equation*}
in questo caso $a_1=a_2=a$ e le tensioni ai capi della fune sono identiche dal momento che si tratta
di una fune ideale. Ricordando che $P_\parallel=m_2g\sin\theta$:
\end{enumerate}


\end{document}

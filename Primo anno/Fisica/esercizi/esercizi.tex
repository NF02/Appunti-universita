\documentclass{article}

\usepackage[utf8]{inputenc}
\usepackage{titlesec}
\usepackage{easylist}
\usepackage{hanging}
\usepackage{hyperref}
\usepackage[a4paper,top=2.0cm,bottom=2.0cm,left=2.0cm,right=2.0cm]{geometry}
\usepackage{blindtext}
\usepackage{tipa}
\usepackage{epigraph}
\usepackage{enumerate}
\usepackage{longtable}
\usepackage{setspace}
\usepackage{verbatim}
\usepackage[T1]{fontenc}
\usepackage{graphicx}
\usepackage[italian]{babel}
\usepackage{amsmath}
\usepackage{pbox}
\usepackage{fancyhdr}
\usepackage{cancel}
\usepackage{tabularx}
\usepackage{booktabs}
\usepackage{multirow}
\usepackage{longtable}
\usepackage{tikz}
\usepackage{tikz-qtree}
\usepackage{subfig}
\usepackage{xcolor}
\usepackage{amssymb}
\usepackage{mathrsfs}
\usepackage{textcomp}
\usepackage{tasks}

\usepackage{listings}
\usepackage{color}

\definecolor{mygreen}{rgb}{0,0.6,0}
\definecolor{mygray}{rgb}{0.5,0.5,0.5}
\definecolor{mymauve}{rgb}{0.58,0,0.82}

\lstset{ 
  backgroundcolor=\color{white},   % choose the background color; you must add \usepackage{color} or \usepackage{xcolor}; should come as last argument
  basicstyle=\footnotesize,        % the size of the fonts that are used for the code
  breakatwhitespace=false,         % sets if automatic breaks should only happen at whitespace
  breaklines=true,                 % sets automatic line breaking
  captionpos=b,                    % sets the caption-position to bottom
  commentstyle=\color{mygreen},    % comment style
  deletekeywords={...},            % if you want to delete keywords from the given language
  escapeinside={\%*}{*)},          % if you want to add LaTeX within your code
  extendedchars=true,              % lets you use non-ASCII characters; for 8-bits encodings only, does not work with UTF-8
  firstnumber=1000,                % start line enumeration with line 1000
  frame=single,	                   % adds a frame around the code
  keepspaces=true,                 % keeps spaces in text, useful for keeping indentation of code (possibly needs columns=flexible)
  keywordstyle=\color{blue},       % keyword style
  language=Octave,                 % the language of the code
  morekeywords={*,...},            % if you want to add more keywords to the set
  numbers=left,                    % where to put the line-numbers; possible values are (none, left, right)
  numbersep=5pt,                   % how far the line-numbers are from the code
  numberstyle=\tiny\color{mygray}, % the style that is used for the line-numbers
  rulecolor=\color{black},         % if not set, the frame-color may be changed on line-breaks within not-black text (e.g. comments (green here))
  showspaces=false,                % show spaces everywhere adding particular underscores; it overrides 'showstringspaces'
  showstringspaces=false,          % underline spaces within strings only
  showtabs=false,                  % show tabs within strings adding particular underscores
  stepnumber=2,                    % the step between two line-numbers. If it's 1, each line will be numbered
  stringstyle=\color{mymauve},     % string literal style
  tabsize=2,	                   % sets default tabsize to 2 spaces
  title=\lstname                   % show the filename of files included with \lstinputlisting; also try caption instead of title
}

\linespread{1.5} % l'interlinea

\frenchspacing

\newcommand{\abs}[1]{\lvert#1\rvert}

\usepackage{floatflt,epsfig}

\usepackage{multicol}
\newcommand\yellowbigsqcup[1][\displaystyle]{%
  \fboxrule0pt
  \ifx#1\textstyle\fboxsep-0.6pt\else\fboxsep-1.25pt\fi
  \mathrel{\fcolorbox{white}{yellow}{$#1\bigsqcup$}}}

\newtheorem{es}{Esercizio}
\newtheorem{sol}{Soluzione}

\title{Esercizi di fisica}
\author{Nicola Ferru}
\begin{document}
\maketitle

\section{Cinematica}
\label{sec:cinematica}

\subsection{Moto retilineo uniforme}
\label{sec:motoretuni}

\begin{es}
  Alla guida di un'automobile, dopo aver percorso una strada rettilinea per 8.4km a 70km/h, siate rimasti senza benzina. Avete quindi percorso a piedi, sempre nella stezza direzione, 2.9km fino al più vicino distributore, dove siete arrivati dopo 30 minuti di cammino.
  \begin{tasks}
    \task Qual'è stato il vestro spostamento complessivo dalla partenza in auto all'arrivo a piedi alla stazione di servizio?
    \task Qual'è l'intervallo di tempo $\Delta{}t$ relativo all'intero spostamento?
    \task Qual'è stata dunque la velocità vettoriale media della partenza in auto all'arrivo a piedi? Lo si trova sia numericamente sia graficamente.
    \task Supponiamo che, dopo le operazioni alla stazione di rifornimento, abbiate poi riportato il carburante fino alla macchina, impiegando nella sosta e nel viaggio di ritorno in totale 45 minuti. Qual'è stata la velocità scalare media per tutto il percorso, dalla partenza in auto fino all'arrivo a piedi alla macchina con il carburante?
  \end{tasks}
\end{es}
\begin{sol}
  Ora, il metodo migliore per svolgere questo esercizio è proprio quello di svolgerlo per punti, infatti, questo è uno dei casi in cui il testo ci da già la soluzione, per questo motivo anche essa sarà divisa in punti.
  \begin{tasks}
    \task In primo luogo andiamo a calcolare la distanza percorsa nel suo complessivo, cosa che è facilmente deducibele facendo una somma tra la prima distanza percorsa 8.4km e la seconda 2.9km quindi si può dedurre che il complessivo sia 11.3. Questo può essere anche espresso come il discriminante di $\Delta{}$.
    \begin{equation*}
      \Delta{}x=x_2-x_1=11.3km
    \end{equation*}
    \task Ora, bisogna calcolare l'intervallo di tempo $\Delta{}t$ relativo all'intero spostamento bisogna utilizzare la formula della velocità $\left(\frac{\Delta{}x}{\Delta{}t}\right)$ sia sul percorso fatto in auto, il percorso fatto a piedi e poi sul totale:
    \begin{eqnarray*}
      \vec{v}_{auto}=\frac{\Delta{}x_{auto}}{\Delta{}t_{auto}}\to \Delta{}t_{auto}=\frac{8.4km}{70km/h}=0.12h\\
      \Delta{}t_{tot}=\Delta{}t_{auto}+\Delta{}t_{piedi}=0.12h+0.5h=0.62h
    \end{eqnarray*}
    visto che a noi serve $\Delta{}t$ del percorso fatto in auto dobbiamo adoperare la formula inversa, mentre, nel caso del percorso fatto a piedi bisogna semplicemente convertire 30min in 0.5h per poter poi fare il calcolo di $\Delta{}t_{tot}$. 
    \task Per calcolare la velocità media dalla partenza in auto all'arrivo a piedi bisogna utlizzare la formula della velocità:
    \begin{equation*}
      \vec{v}=\frac{\Delta{}x}{\Delta{}t} = \frac{11.3km}{0.62h}=18.23km/h
    \end{equation*}
    \task Adesso per calcolare la velocità totale di tutto il percorso incluso il ritorno alla macchina con il carburante dobbiamo effetture il calcolo di $\Delta{}t_{total}$ e $\Delta{}x_{total}$ e poi si può calcolare la velocità.
    \begin{eqnarray*}
      \Delta{}t_{total} = 0.12h + 0.5h + 0.75h = 1.37h\\
      \Delta{}x_{total} = 8.4km + 2.9km + 2.9km =14.2km 
    \end{eqnarray*}
    Ora dopo aver ottenuto il valore delle variabili necessari a calcolare il $\vec{v}_{total}$ possiamo calcolarlo facilmente con la consueta formula:
    \begin{equation*}
      \vec{v}=\frac{\Delta{}x_{total}}{\Delta{}t_{total}} = \frac{14.2km}{1.37h}=10.36km/h \to 10.4km/h
    \end{equation*}
    visto che comunque il risultato lo riportiamo con una sola cifra decimale ho arrotondato per eccesso 10.36m/h a 10.4km/h.
  \end{tasks}
\end{sol}
\end{document}

\documentclass{book}
\usepackage[a4paper,top=2.0cm,bottom=2.0cm,left=3.0cm,right=3.0cm]{geometry}

%\documentclass[pdftex,10pt,a4paper]{book}
%\usepackage[paperwidth=19cm,
%paperheight=26cm, outer=2cm, 
%top=1.5cm, bottom=1.5cm]{ geometry}

\usepackage[english,italian]{babel} %l'ultima lingua è quella che legge per i titoli
\usepackage[utf8]{inputenc}
\usepackage[T1]{fontenc,url}
\usepackage{titlesec}
\usepackage{easylist}
\usepackage{hanging}

\usepackage[pdftex,colorlinks]{hyperref}
\hypersetup{
	colorlinks=true,
	linkcolor=black,
	filecolor=magenta,
	urlcolor=cyan,
}
\usepackage{hypcap}
\usepackage{blindtext}
\usepackage{tipa}
\usepackage{epigraph}
\usepackage{enumerate}
\usepackage{longtable}
\usepackage{setspace}
\usepackage{verbatim}
\usepackage{graphicx}
\usepackage{amsmath}
\usepackage{pbox}
\usepackage{fancyhdr}
\usepackage{cancel}
\usepackage{tabularx}
\usepackage{booktabs}
\usepackage{multirow}
\usepackage{longtable}
\usepackage{tikz}
\usepackage{tikz-qtree}
\usepackage{subfig}
\usepackage{xcolor}
\usepackage{amssymb}
\usepackage{amsmath}
\usepackage{mathrsfs}
\usepackage{textcomp}
\usepackage{circuitikz}
\usepackage{pifont}
\usepackage{imakeidx}
\usepackage{verbatim}
\usepackage{dsfont}
\usepackage{listings}
\usepackage{color}
\usepackage{upgreek}
\usepackage{tasks}
\usepackage{exsheets}
\usepackage{pgfplots}
\usepackage{amsthm}
\usepackage{wasysym}
\usepackage{dsfont}
\usepackage{thmtools}
\usepackage{enumitem}

% impostazioni grafici
\usepackage{tikzit}
\input{img/stile.tikzstyles}

\usepackage{showframe}
\renewcommand\ShowFrameLinethickness{0.15pt}
%\renewcommand*\ShowFrameColor{\color{red}}

%\usepackage{showkeys} %serve per mostrare le etichette "tag" o target, va tolta per la versione definitiva;

\SetupExSheets[question]{type=exam}

\definecolor{mygreen}{rgb}{0,0.6,0}
\definecolor{mygray}{rgb}{0.5,0.5,0.5}
\definecolor{mymauve}{rgb}{0.58,0,0.82}

\lstset{ 
  backgroundcolor=\color{white},   % choose the background color; you must add \usepackage{color} or \usepackage{xcolor}; should come as last argument
  basicstyle=\footnotesize,        % the size of the fonts that are used for the code
  breakatwhitespace=false,         % sets if automatic breaks should only happen at whitespace
  breaklines=true,                 % sets automatic line breaking
  captionpos=b,                    % sets the caption-position to bottom
  commentstyle=\color{mygreen},    % comment style
  deletekeywords={...},            % if you want to delete keywords from the given language
  escapeinside={\%*}{*)},          % if you want to add LaTeX within your code
  extendedchars=true,              % lets you use non-ASCII characters; for 8-bits encodings only, does not work with UTF-8
  firstnumber=1000,                % start line enumeration with line 1000
  frame=single,	                   % adds a frame around the code
  keepspaces=true,                 % keeps spaces in text, useful for keeping indentation of code (possibly needs columns=flexible)
  keywordstyle=\color{blue},       % keyword style
  language=Octave,                 % the language of the code
  morekeywords={*,...},            % if you want to add more keywords to the set
  numbers=left,                    % where to put the line-numbers; possible values are (none, left, right)
  numbersep=5pt,                   % how far the line-numbers are from the code
  numberstyle=\tiny\color{mygray}, % the style that is used for the line-numbers
  rulecolor=\color{black},         % if not set, the frame-color may be changed on line-breaks within not-black text (e.g. comments (green here))
  showspaces=false,                % show spaces everywhere adding particular underscores; it overrides 'showstringspaces'
  showstringspaces=false,          % underline spaces within strings only
  showtabs=false,                  % show tabs within strings adding particular underscores
  stepnumber=2,                    % the step between two line-numbers. If it's 1, each line will be numbered
  stringstyle=\color{mymauve},     % string literal style
  tabsize=2,	                   % sets default tabsize to 2 spaces
  title=\lstname                   % show the filename of files included with \lstinputlisting; also try caption instead of title
}

\frenchspacing

\newcommand{\abs}[1]{\lvert#1\rvert}

\usepackage{floatflt,epsfig}

\usepackage{multicol}
\newcommand\yellowbigsqcup[1][\displaystyle]{%
  \fboxrule0pt
  \ifx#1\textstyle\fboxsep-0.6pt\else\fboxsep-1.25pt\fi
  \mathrel{\fcolorbox{white}{yellow}{$#1\bigsqcup$}}}

\theoremstyle{definition}
\newtheorem{defi}{Definizione}[section]
\newtheorem{es}{Esempio}[section]
\newtheorem{teo}{Teorema}[section]
\newtheorem{oss}{Osservazione}[section]
\theoremstyle{plain}
\newtheorem{nota}{Nota}[section]
\newtheorem{prop}{Proposizione}[section]
\newtheorem{pro}{Proprietà}[section]
\newtheorem{corol}[section]{Corollario}
\title{Algebra e geometria}
\author{Nicola Ferru}
\begin{document}
\maketitle
\tableofcontents

\subsubsection{Prefazione}
\label{sec:pref}
Le modalità di utilizzo e distribuzione sono scritte nel file \href{https://github.com/NF02/Appunti-universita/blob/main/LICENSE}{LICENSE}.
\chapter{Vettori, coordinate e geometria}
\label{chap:vettcoordegeo}
Uno degli argomenti su cui il corso si basa sono proprio i \textit{vettori}. All'interno di questo capitolo saranno presenti nozioni e definizioni legate alla natura stessa di queste entità matematiche dai rudimenti ad alcuni spetti più avanzati.

\section{Vettori Geometrici}
\label{sec:vettorigeo}
\begin{defi}
  \label{def:vettorigeo}
  Un vettore geometrico applicato nel piano è un segmento orientato che va da un punto fisso O ``Origine'' verso un secondo punto $P$ del piano, come mostrato nella figira \ref{fig:vettorigeo}: 
  \begin{figure}[ht!]
    \centering
    \resizebox{3cm}{!}{
      \begin{tikzpicture}
	\begin{pgfonlayer}{nodelayer}
		\node [style=dot] (0) at (2, 1) {};
		\node [style=none] (1) at (-2, 1) {};
		\node [style=none] (2) at (-1, 5) {};
		\node [style=none] (3) at (4, 5) {};
		\node [style=none] (4) at (5, -3) {};
		\node [style=none] (5) at (-0.75, 5.5) {$S$};
		\node [style=none] (6) at (3.75, 5.5) {$P$};
		\node [style=none] (7) at (-2.25, 1.5) {$R$};
		\node [style=none] (8) at (5.25, -2.75) {$Q$};
		\node [style=none] (9) at (2.5, 1) {$O$};
	\end{pgfonlayer}
	\begin{pgfonlayer}{edgelayer}
		\draw [style=Rightarrow] (0) to (2.center);
		\draw [style=Rightarrow] (0) to (1.center);
		\draw [style=Rightarrow] (0) to (3.center);
		\draw [style=Rightarrow] (0) to (4.center);
	\end{pgfonlayer}
\end{tikzpicture}

    }
    \caption{Esempio vettori geometrici}
    \label{fig:vettorigeo}
  \end{figure}\\
  Analogamente, se il punto $P$ (\textit{e quindi il segmento}) è libero di variare in tutto lo spazio tridimensionale. In ambo i casi il vettore sarà denotato $\vec{OP}$ (\textit{si denota che il punto finale $P$ può anche uguale a $O$, ovvero il vettore può essere molto ravvicinato al punto $O$}).
\end{defi}
\begin{nota}
  \label{nota:vettorigeo}
  La direzione è indicata dalla simbolo freccia, graficamente la lunghezza e direzione del vettore implicano il modo in cui agisce nello spazio, ad esempio, se due vettori hanno direzioni opposte uno si sottrarrà potenzialmente al altro.
\end{nota}

\paragraph{Denotare che}
con $V_O^2$ l'unsieme dei vettori geometrici applicati in $O$ nel piano, e con $V_O^3$ l'insieme dei vettori geometrici applicati in $O$ liberi di variare in tutto lo spazio tridimensionale. I vettori orientati sono utilizzati infisica, dove vengono usati per rappresentare le forze applicate sul punto $O$.
\begin{es}
  Si può immaginare che in $O$ si trovi un oggetto sul quale viene esercitata una forza che lo ``trascina'' nella direzione e nel verso dati dalla freccia come evidenziato nella nota (\ref{nota:vettorigeo}), mentre l'intensità della forza esercitata è rappresentata dlla lunghezza del segmento.
\begin{figure}[ht!]
  \centering
  \resizebox{7.5cm}{!}{
      \begin{tikzpicture}
	\begin{pgfonlayer}{nodelayer}
		\node [style=none] (0) at (-2, 4) {};
		\node [style=none] (1) at (-4, 1) {};
		\node [style=none] (2) at (4, 4) {};
		\node [style=none] (3) at (2, 1) {};
		\node [style=none] (4) at (-2, 1) {};
		\node [style=none] (5) at (0, 4) {};
		\node [style=none] (6) at (2, 4) {};
		\node [style=none] (7) at (0, 1) {};
		\node [style=none] (8) at (-3.25, 2) {};
		\node [style=none] (9) at (2.5, 2) {};
		\node [style=none] (10) at (4.25, 4.25) {$P_3$};
		\node [style=none] (11) at (2.25, 0.5) {$P_1$};
		\node [style=none] (12) at (-4.5, 0.75) {$O$};
		\node [style=none] (13) at (-2, 4.25) {$P_2$};
		\node [style=none] (14) at (7, 4) {};
		\node [style=none] (15) at (5, 1) {};
		\node [style=none] (16) at (11, 1) {};
		\node [style=none] (17) at (13, 4) {};
		\node [style=none] (18) at (13.25, 4.25) {$P_3$};
		\node [style=none] (19) at (11.25, 0.5) {$P_1$};
		\node [style=none] (20) at (4.5, 0.75) {$O$};
		\node [style=none] (21) at (7, 4.25) {$P_2$};
	\end{pgfonlayer}
	\begin{pgfonlayer}{edgelayer}
		\draw [style=Rightarrow] (1.center) to (0.center);
		\draw [style=Rightarrow] (1.center) to (3.center);
		\draw [style=Dashedrightarrow] (4.center) to (5.center);
		\draw [style=Dashedrightarrow] (7.center) to (6.center);
		\draw [style=Rightarrow] (3.center) to (2.center);
		\draw [style=Rightarrow, bend left] (8.center) to (9.center);
		\draw [style=Rightarrow] (15.center) to (14.center);
		\draw [style=Rightarrow] (15.center) to (16.center);
		\draw [style=Rightarrow] (15.center) to (17.center);
		\draw [style=campitura] (17.center) to (16.center);
		\draw [style=campitura] (14.center) to (17.center);
	\end{pgfonlayer}
\end{tikzpicture}

    }
  \caption{Somma vettoriale}
  \label{fig:sommavect}
\end{figure}
Dal momento che $\vec{OP}_3$ rappresenta la forza totale esercitata la forza totale esercitata su $O$ quando si applicano contemporaneamente $\vec{OP_1}$ e $\vec{OP}_2$, il meccanismo più immediato è associare l'operazione ad una addizione, infatti, essa viene scritta come:
\begin{equation}
  \label{eq:sommavect}
  \vec{OP}_3=\vec{OP}_1+\vec{OP}_2
\end{equation}
La rappresentazione grafica è presente in figura \ref{fig:sommavect} definisce in modo in cui un'operazione di somma sull'insieme di vettori geometrici (del piono o dello spazio) viene rappresentata.
\end{es}
Per i vettori che non hanno la stessa direzione, si denota che $OP_3$ è la direzionale del parallelogramma che ha $OP_1$ e $OP_2$ come lati (infatti, viene definita anche come \textit{regola del parallelogramma}). Il motodo descrittivo funziona comunque anche per sommare due o più vettori che hanno la stessa direzione:
\begin{figure}[ht!]
  \centering
  \resizebox{9cm}{!}{
      \begin{tikzpicture}
	\begin{pgfonlayer}{nodelayer}
		\node [style=none] (0) at (0, 2) {};
		\node [style=none] (1) at (3, 2) {};
		\node [style=none] (2) at (5, 2) {};
		\node [style=none] (3) at (8, 2) {};
		\node [style=none] (4) at (1.5, 2.5) {};
		\node [style=none] (5) at (4.75, 2.5) {};
		\node [style=none] (6) at (1, 4.5) {};
		\node [style=none] (7) at (3.25, 3.25) {};
		\node [style=none] (8) at (8, 3.25) {};
		\node [style=none] (9) at (6, 4.5) {};
		\node [style=none] (10) at (6, 4) {$P_2$};
		\node [style=none] (11) at (1, 4) {$O$};
		\node [style=none] (12) at (2.75, 3.25) {$O$};
		\node [style=none] (13) at (8.25, 3.25) {$P_2$};
		\node [style=none] (14) at (5, 1.5) {$P_2$};
		\node [style=none] (15) at (3, 1.5) {$P_1$};
		\node [style=none] (16) at (0, 1.5) {$O$};
		\node [style=none] (17) at (10, 2) {};
		\node [style=none] (18) at (13, 2) {};
		\node [style=none] (19) at (15, 2) {};
		\node [style=none] (20) at (18, 2) {};
		\node [style=none] (21) at (15, 1.5) {$P_2$};
		\node [style=none] (22) at (13, 1.5) {$P_1$};
		\node [style=none] (23) at (10, 1.5) {$O$};
		\node [style=none] (24) at (18, 1.5) {$P_3$};
	\end{pgfonlayer}
	\begin{pgfonlayer}{edgelayer}
		\draw [style=Rightarrow] (0.center) to (1.center);
		\draw [style=Rightarrow] (1.center) to (2.center);
		\draw [style=Dashedrightarrow] (2.center) to (3.center);
		\draw [style=Rightarrow, bend left=75, looseness=2.75] (4.center) to (5.center);
		\draw [style=Dashedrightarrow] (6.center) to (9.center);
		\draw [style=Dashedrightarrow] (7.center) to (8.center);
		\draw [style=Rightarrow] (17.center) to (18.center);
		\draw [style=Rightarrow] (18.center) to (19.center);
		\draw [style=Rightarrow] (19.center) to (20.center);
	\end{pgfonlayer}
\end{tikzpicture}

    }
  \caption{Regola del parallelogramma}
  \label{fig:metparallelogramma}
\end{figure}\\
Anche in questo caso vale la formula \ref{eq:sommavect}, infatti, graficamente la $OP_3$ è chiaramente frutto di una somma tra il segmento $OP_1$ e $OP_2$. Un'altra operazione è il prodotto del vettore per un numero reale: nel contesto delle forze, il concetto è quella di rappresentare una variazione dell'intensità e eventualmente del verso della forza rappresentata dal vettore.\\
Più precisamente, dati un vettore geometrico $\vec{OP}$ e un numero releale $c\in\mathds{R}$, si può definire $c\vec{OP}$ come il vettore che sta sulla stessa retta a cui appartiene $\vec{OP}$, ma avente:
\begin{enumerate}
\item Stesso verso e lunghezza $c$ volte la lunghezza di $\vec{OP}$, se $c$ è positivo;
\item Verso opposto e lunghezza $-c$ volte quella di $\vec{OP}$, se $c$ è negativo;
\item Lunghezza ulla se c=0, cioè $0\vec{OP}=\vec{OO}$.
\end{enumerate}
\begin{figure}[ht!]
  \centering
  \resizebox{6cm}{!}{
      \begin{tikzpicture}
	\begin{pgfonlayer}{nodelayer}
		\node [style=none] (0) at (0, 1) {};
		\node [style=none] (1) at (2, 3) {};
		\node [style=none] (2) at (4, 5) {};
		\node [style=none] (3) at (-1, 0) {};
		\node [style=none] (4) at (-0.25, 1.25) {O};
		\node [style=none] (5) at (1.75, 3.25) {P};
		\node [style=none] (6) at (-1.25, -0.5) {$\frac{1}{2}\vec{OP}$};
		\node [style=none] (7) at (4.25, 5.5) {$2\vec{OP}$};
	\end{pgfonlayer}
	\begin{pgfonlayer}{edgelayer}
		\draw [style=Rightarrow] (0.center) to (1.center);
		\draw [style=Rightarrow] (1.center) to (2.center);
		\draw [style=Dashedrightarrow] (0.center) to (3.center);
	\end{pgfonlayer}
\end{tikzpicture}

    }
  \caption{Prodotto vettoriale}
  \label{fig:prodottovect}
\end{figure}
Nel contesto dei vettori, i numeri reali si chiamano anche \textit{scalari}.\\
Come si vedra nel ultima parte del capitolo, la nozione di vettore geometrico e le operazioni di somma tra vettori e prodotto di un vettore per un numero che appena definito saranno fornamentali per impostare e risolvere problemi geometrici nel piano e nello spazio. Per questo motivo, è necessario conoscere e mettere in evidenza le proprietà di cui godono tali operazionim che permettono di manipolare le espressioni e formule che coinvolgono i vettori. Si può verificare che valgono le seguenti:
\begin{enumerate}
\item La somma è \textit{associativa}
  \begin{equation}
    \label{eq:sommaassociativa}
    (\vec{OP}_1+\vec{OP}_2)+\vec{OP}_3=\vec{OP}_1+(\vec{OP}_2+\vec{OP}_3)
  \end{equation}
\item La somma è \textit{commutativa}
  \begin{equation}
    \label{eq:commutativa}
    \vec{OP}_1+\vec{OP}_2=\vec{OP}_2+\vec{OP}_1
  \end{equation}
\item Il vettore $\vec{OO}$ funge da elemento neutro per la somma:
  \begin{equation}
    \label{eq:sommaelementoneutro}
    \vec{OP}+\vec{OO}=\vec{OO}+\vec{OP}=\vec{OP}
  \end{equation}
\item Per ogni vettore $\vec{OP}$, il vettore $(-1)\vec{OP}$ (ovvero il vettore che si ottiene da $\vec{OP}$ basterà invertire il verso, senza modificare direzione e lunghezza) è il suo inverso additivo o opposto rispetto alla somma:
  \begin{equation}
    \label{eq:sommainversa}
    \vec{OP}+(-1)\vec{OP}=(-1)\vec{OP}+\vec{OP}=\vec{OO}
  \end{equation}
\item Dati due numeri reali $c_1$, $c_2$ e un vettore $\vec{OP}$, si ha
  \begin{equation}
    \label{eq:prodottoconduenumerireali}
    c_1(c_2\vec{OP})=(c_1c_2)\vec{OP}
  \end{equation}
  (\textit{Una situazione molto similare alla proprietà associativa del prodotto}).
\item Per ogni vettore $\vec{OP}$, si ha
  \begin{equation}
    \label{eq:perognivecOP}
    1\vec{OP}=\vec{OP}
  \end{equation}
  (\textit{ovvero il numero 1 funge da elemento neutro rispetto al prodotto per scalari}).
\item Dati due numeri reali $c_1$, $c_2$ ed un vettore $\vec{OP}$, si ha
  \begin{equation}
    \label{eq:numrealeVectOP}
    (c_1+c_2)\vec{OP}=c_1\vec{OP}+c_2\vec{OP}
  \end{equation}
\item Dati un numero reale $c$ e due vettori $\vec{OP}$, $\vec{OP}$ si ha
  \begin{equation}
    \label{eq:prodottoconduenumerirealiperunnumeroreale}
    c(\vec{OP}_1+\vec{OP}_2)=c\vec{OP}_1+c\vec{OP}_2
  \end{equation}
\end{enumerate}
Lo sviluppo suggerisce che valga la proprietà distributiva rispetto alla somma di numeri reale o rispetto alla somma di vettori.
\begin{oss}
  \label{oss:vettgeo1}
  Come esempio di applicazione delle proprietà appena elencate, è il caso di motrare che in un'uguaglianza tra vettori, esattamente come si fa in un'uguagliana tra numeri, si possono ``spostare i vettori'' da un membro all'altro cambiandoli di segno:
  \begin{equation*}
    \vec{OP}_1+\vec{OP}_2=\vec{OP}_3 \to \vec{OP}_1=\vec{OP}_3-\vec{OP}_2
  \end{equation*}
  Dove, come nel caso dei numeri lo spostamento dall'altra parte dell'uguaglianza comporta il cambiamento di segno scritto come $\vec{OP}_3-\vec{OP}_2$ che risulta essere la forma semplificata di $\vec{OP}_3+(-1)\vec{OP}_2$.\\
  Per vederlo, basterà sommare ad antrambi i membri di $\vec{OP}_1+\vec{OP}_2=\vec{OP}_3$ il vettore $(-1)\vec{OP}_2$:
  \begin{equation*}
    (\vec{OP}_1+\vec{OP}_2)+(-1)\vec{OP}_2=\vec{OP}_3+(-1)\vec{OP}_2
  \end{equation*}
  Applicando la propriatà associativa (\ref{eq:sommaassociativa}) a primo membro:
  \begin{equation*}
    \vec{OP}_!+\left[\vec{OP}_2+(-1)\vec{OP}_2\right]=\vec{OP}_3+(-1)\vec{OP}_2
  \end{equation*}
  Dopo aver fatto questo passaggio, sarà necessario applicare la proprietà (\ref{eq:sommainversa}) che afferma che $(-1)\vec{OP}_2$ è l'opposto di $\vec{OP}_2$:
  \begin{equation*}
    \vec{OP}_2+\vec{OO}=\vec{OP}_3+(-1)\vec{OP}_2
  \end{equation*}
  e infine va applicato la proprietà (\ref{eq:sommaelementoneutro}) che afferma che il vettore nullo funge da elemento neutro:
  \begin{equation*}
    \vec{OP}_1=\vec{OP}_3+(-1)\vec{OP}_2
  \end{equation*}
  e con questo è stata confermata l'affermazione iniziale.
\end{oss}

\section{Coordinate}
\label{sec:coordinate}

Considerando due vettori geometrici $\vec{OP}_1$ e $\vec{OP}_2$ nel piano, e si può supporre che $\vec{OP}_1$ e $\vec{OP}_2$ non abbiano la stessa dimensione. \\
Affermando che ogni vettore $\vec{OP}\in V_O^2$ può essere ottenuto sommando multipli opportuni di $\vec{OP}_1$ e $\vec{OP}_2$, ovvero:
\begin{equation*}
  \vec{OP}=c_1\vec{OP}_1+c_2\vec{OP}_2
\end{equation*}
dove $c_1$, $c_2$ sono opportuni numeri reali.
\clearpage
Infatti, questo può essere facilmente visto graficamente: come nel disegno seguente, prolungando i vettori $\vec{OP}_1$ e $\vec{OP}_2$ disegnando le due rette $r_1$ e $r_2$; proiettando quindi i punti $P$ su $r_1$ seguendo la direzione parallela a $\vec{OP}_2$, e chiamando il punto proiettato $Q_1$: e chiamandolo punto proiettato $Q_2$.
\begin{figure}[ht!]
  \centering
  \resizebox{6cm}{!}{
      \begin{tikzpicture}
	\begin{pgfonlayer}{nodelayer}
		\node [style=none] (0) at (0, 0) {};
		\node [style=none] (1) at (1, 2) {};
		\node [style=none] (2) at (2, 4) {};
		\node [style=none] (3) at (3.5, 7) {};
		\node [style=none] (4) at (2, 0) {};
		\node [style=none] (5) at (4, 0) {};
		\node [style=none] (6) at (7, 0) {};
		\node [style=none] (7) at (6, 4) {};
		\node [style=none] (8) at (2.75, 6.25) {$r_2$};
		\node [style=none] (9) at (1.5, 4) {$Q_2$};
		\node [style=none] (10) at (0.5, 2) {$P_2$};
		\node [style=none] (11) at (-0.5, -0.25) {O};
		\node [style=none] (12) at (2, -0.5) {$P_1$};
		\node [style=none] (13) at (4, -0.5) {$Q_1$};
		\node [style=none] (14) at (7, -0.5) {$r_1$};
		\node [style=none] (15) at (6.5, 4.25) {$P$};
	\end{pgfonlayer}
	\begin{pgfonlayer}{edgelayer}
		\draw [style=Rightarrow] (0.center) to (1.center);
		\draw [style=Rightarrow] (1.center) to (2.center);
		\draw (2.center) to (3.center);
		\draw [style=Rightarrow] (0.center) to (4.center);
		\draw [style=Rightarrow] (4.center) to (5.center);
		\draw (5.center) to (6.center);
		\draw [style=campitura] (5.center) to (7.center);
		\draw [style=campitura] (2.center) to (7.center);
		\draw [style=Rightarrow] (0.center) to (7.center);
	\end{pgfonlayer}
\end{tikzpicture}

    }
  \caption{Costruzione grafica $\vec{OP}=c_1\vec{OP}_1+c_2\vec{OP}_2$}
  \label{fig:costruvectgraph}
\end{figure}\\
Avendo costruito le due proiezioni parallelamente a $\vec{OP}_1$ e $\vec{OP}_2$ come lati e $\vec{OP}$ come diagonale, quindi per definizione di somma tra vettori geometrici si ha $\vec{OP}=\vec{OQ}_1+\vec{OQ}_2$.\\ Ma dal momento che $\vec{OQ}_1$ si trova sulla stessa retta di $\vec{OP}_1$ per come è definito il prodotto dei vettori per i numeri realim esisterà un numero reale $c_1$ tale che $\vec{OQ}_1=c_1\vec{OP}_1$ (dove $c_1$ dipende semplicemente dal rappotro tra la lunghezza di $\vec{OQ}_1$ e quella di $\vec{OP}_1$).\\
Si conclude che $\vec{OP}=c_1\vec{OP}_1+c_2\vec{OP}_2$.
Si noti che nella situazione considerata nel disegno, $c_1,c_2 > 0$ in quanto $\vec{OQ}_1$ e $\vec{OQ}_2$ hanno lo stesso verso di $\vec{OP}_1$ e $\vec{OP}_2$ rispettivamente. In generale, la stessa costruzione può essere effettuata per qualunque vettore $\vec{OP}$ del piano e i coefficienti $c_1$ e $c_2$ potranno anche essere negativi\footnote{Può essrere anche $c_1=0$ o $c_2=0$: nel primo caso, si ha $\vec{OP}=c_2\vec{OP}_2$, nel secondo $\vec{OP}=c_1\vec{OP}_1$, cioè $\vec{OP}$ non sta all'interno di uno dei quadranti in cui le rette $r_1,r_2$ dividono il piano, ma sta sulla retta $r_2$ (se $\vec{OP}=c_2\vec{OP}_2$) o sulla retta $r_1$ (se $\vec{OP}=c_1\vec{OP}_1$).} a seconda del quadrante nel quale si trova $\vec{OP}$, ovvero a seconda che la proiezione di $P$ sulle rette $r_1$, $r_2$ cada dalla stessa parte o dalla parte opposta dei punti $P_1$ e $P_2$.
\begin{figure}[ht!]
  \centering
  \resizebox{6cm}{!}{
      \begin{tikzpicture}
	\begin{pgfonlayer}{nodelayer}
		\node [style=none] (0) at (5, 4) {};
		\node [style=none] (1) at (3, 4) {};
		\node [style=none] (2) at (7, 4) {};
		\node [style=none] (3) at (1, 4) {};
		\node [style=none] (4) at (10, 4) {};
		\node [style=none] (5) at (11, 4) {};
		\node [style=none] (6) at (7, 8) {};
		\node [style=none] (7) at (4, 2) {};
		\node [style=none] (8) at (-1, 4) {};
		\node [style=none] (9) at (8, 10) {};
		\node [style=none] (10) at (3, 0) {};
		\node [style=none] (11) at (5, 8) {};
		\node [style=none] (12) at (9, 2) {};
		\node [style=none] (13) at (3.5, 1) {};
		\node [style=none] (14) at (-0.5, 1) {};
		\node [style=none] (15) at (5, 3.5) {O};
		\node [style=none] (16) at (7, 3.5) {$P_1$};
		\node [style=none] (17) at (9.25, 1.75) {$P$};
		\node [style=none] (18) at (6.5, 7) {};
		\node [style=none] (19) at (7, 7) {$P_2$};
		\node [style=none] (20) at (4.75, 8.25) {P};
		\node [style=none] (21) at (-1, 0.75) {$P$};
		\node [style=none] (22) at (0, 0) {$c_1<0$};
		\node [style=none] (23) at (0, -0.5) {$c_2<0$};
		\node [style=none] (24) at (10.75, 2) {$c_1>0$};
		\node [style=none] (25) at (10.75, 1.5) {$c_2<0$};
		\node [style=none] (26) at (3.25, 8.75) {$c_1<0$};
		\node [style=none] (27) at (3.25, 8.25) {$c_2>0$};
	\end{pgfonlayer}
	\begin{pgfonlayer}{edgelayer}
		\draw [style=Rightarrow] (0.center) to (2.center);
		\draw [style=Rightarrow] (2.center) to (4.center);
		\draw [style=Rightarrow] (0.center) to (1.center);
		\draw [style=Rightarrow] (1.center) to (3.center);
		\draw [style=Rightarrow] (0.center) to (7.center);
		\draw [style=campitura] (3.center) to (8.center);
		\draw [style=campitura] (4.center) to (5.center);
		\draw [style=campitura] (6.center) to (9.center);
		\draw [style=campitura] (1.center) to (11.center);
		\draw [style=campitura] (11.center) to (6.center);
		\draw [style=campitura] (4.center) to (12.center);
		\draw [style=campitura] (12.center) to (7.center);
		\draw [style=campitura] (3.center) to (14.center);
		\draw [style=campitura] (14.center) to (13.center);
		\draw [style=Rightarrow] (7.center) to (13.center);
		\draw [style=campitura] (13.center) to (10.center);
		\draw [style=Rightarrow] (0.center) to (11.center);
		\draw [style=Rightarrow] (0.center) to (12.center);
		\draw [style=Rightarrow] (0.center) to (14.center);
		\draw [style=Rightarrow] (0.center) to (18.center);
		\draw [style=Rightarrow] (18.center) to (6.center);
	\end{pgfonlayer}
\end{tikzpicture}

    }
  \caption{Condizione della formula $\vec{OP}=c_1\vec{OP}_1+c_2\vec{OP}_2$ in base ai reali $c_1,c_2$}
  \label{fig:condizionic1c2}
\end{figure}
\begin{defi}
  \label{defi:coppiaC1eC2talcheOp1}
  La coppia ($c_1,c_2$) di numeri reali tale che $\vec{OP}=c_1\vec{OP}_1+c_2\vec{OP}_2$ si dice la \textit{coppia delle coordinate} del vettore $\vec{OP}$ rispetto ai vettori base $\vec{OP}_1, \vec{OP}_2$.\\
  Le coordinate $c_1$ e $c_2$ di un vettore dipendono chiaramente dalla scelta dei vettori base $\vec{OP}_1$, $\vec{OP}_2$, ma una volta che essi sono stati fissati seriveremo $\vec{OP}\equiv (c_1,c_2)$, identificando di fatto il vettore con la coppia delle sua coordinate, e quindi l'insieme $\vec{V}_O^2$ con l'insieme $\mathds{R}^2$ delle coppie di numeri reali.
\end{defi}
\begin{oss}
  Bisognerebbe porsi il problema dell'\textit{unicità} di $c_1$ e $c_2$: se esistessero due modi diversi, diciamo $\vec{OP}=c_1\vec{OP}_1+c_2\vec{OP}_2$ e $\vec{OP}=c_1^\prime\vec{OP}_1+c_2^\prime\vec{OP}_2$, di decomporre $\vec{OP}$, non avremmo una e una spola coppia di numeri con cui identificarlo: in realtà, la costruzione grafica già suggerisce che l'unicità è garantita, ma si tornerà su tel questione nel paragrafo \ref{}.
\end{oss}
Un risultato analogo a quello visto per i vettori nel piano può essere ottenuto anche nell'insieme $V_O^3$ dei vettori geometri nello spazio tridimensionale. In questo non si deve però partire da una coppia di vettori non allineati ma da una terna di vettori $\vec{OP}_1$, $\vec{OP}_2$ e $\vec{OP}_3$ \textit{che non siano tutti e tre sullo stesso piano}: alloram, è semplice vedere graficamente, utilizzando proiezioni come fatto nel caso di due vettori nel piano, che ogni vettore $\vec{OP}\in V_O^3$ può essere scritto come combinazione $c_1\vec{OP}_1+c_2\vec{OP}_2+c_3\vec{OP}_3$.
\begin{figure}[ht!]
  \centering
  \resizebox{6cm}{!}{
      \begin{tikzpicture}
	\begin{pgfonlayer}{nodelayer}
		\node [style=none] (0) at (-1, 6) {};
		\node [style=none] (1) at (-1.5, 5) {};
		\node [style=none] (2) at (-2.25, 3.5) {};
		\node [style=none] (3) at (-4.5, -1) {};
		\node [style=none] (4) at (-1, -1) {};
		\node [style=none] (5) at (1.5, -1) {};
		\node [style=none] (6) at (3, 1) {};
		\node [style=none] (7) at (-2.5, 1) {};
		\node [style=none] (8) at (3.75, 4) {};
		\node [style=none] (10) at (4, 4.5) {$P$};
		\node [style=none] (11) at (-1.5, 6) {$r_3$};
		\node [style=none] (12) at (-7, 3) {};
		\node [style=none] (13) at (-10, -4) {};
		\node [style=none] (14) at (7, 3) {};
		\node [style=none] (15) at (4, -4) {};
		\node [style=none] (16) at (-2.5, 4) {$P_3$};
		\node [style=none] (17) at (-4.5, 2) {$c_3\vec{OP}_3$};
		\node [style=none] (18) at (-3.25, 0.25) {};
		\node [style=none] (19) at (-2.25, 1.5) {$c_2\vec{OP}_2$};
		\node [style=none] (20) at (3.5, 1.25) {$Q$};
		\node [style=none] (21) at (-5, -1.25) {O};
		\node [style=none] (22) at (-1, -1.5) {$P_1$};
		\node [style=none] (23) at (1.5, -1.5) {$c_1\vec{OP}_1$};
		\node [style=none] (24) at (-3.25, 0.5) {$P_2$};
	\end{pgfonlayer}
	\begin{pgfonlayer}{edgelayer}
		\draw [style=Rightarrow] (3.center) to (2.center);
		\draw [style=Rightarrow] (2.center) to (1.center);
		\draw [style=campitura] (1.center) to (0.center);
		\draw [style=Rightarrow] (3.center) to (4.center);
		\draw [style=Rightarrow] (4.center) to (5.center);
		\draw [style=Rightarrow] (3.center) to (6.center);
		\draw [style=campitura] (5.center) to (6.center);
		\draw [style=campitura] (7.center) to (6.center);
		\draw [style=campitura] (6.center) to (8.center);
		\draw [style=Rightarrow] (3.center) to (8.center);
		\draw [style=campitura] (2.center) to (8.center);
		\draw (13.center) to (12.center);
		\draw (12.center) to (14.center);
		\draw (14.center) to (15.center);
		\draw (15.center) to (13.center);
		\draw [style=Rightarrow] (3.center) to (18.center);
		\draw [style=Rightarrow] (18.center) to (7.center);
	\end{pgfonlayer}
\end{tikzpicture}

    }
  \caption{Vettori su spazio tridimensionale}
  \label{fig:vectspaztridim}
\end{figure}\\
Come rappresentato in figura \ref{fig:vectspaztridim}, si proietta il punto su cui stanno $\vec{OP}_1$ e $\vec{OP}_2$ seguendo la direzione $\vec{OP}_3$ e si individua così un punto $Q$; proiettando poi $P$ sulla retta $r_3$ parallelamente al vettore $\vec{OQ}$, risulta individuato un parallelogramma, che ci dice che $\vec{OP}$ si scrive come somma $\vec{OP}=\vec{OQ}+c_3\vec{OP}_3$ di $\vec{OQ}$ e di un opportuno multiplo $c_3\vec{OP}_3$ di $\vec{OP}_3$. A questo punto si osserva che $\vec{OQ}$, stando sul piano di $\vec{OP}_1$ e $\vec{OP}_2$ si scriverà come loro combinazione lineare $\vec{OQ}= c_1\vec{OP}_1+c_2\vec{OP}_2+c_3\vec{OP}_3$. In modo analogo a quato già fatto per i vettori geometrico del piano, si può dire che:
\begin{defi}
  \label{defi:ternadinumerireali}
  La terna ($c_1,c_2,c_3$) di numeri reali tale che $\vec{OQ}= c_1\vec{OP}_1+c_2\vec{OP}_2+c_3\vec{OP}_3$ si dice la \textit{terna delle coordinate} del vettore $\vec{OP}$ rispetto ai vettori di base $\vec{OP}_1,\vec{OP}_2, \vec{OP}_3$.
\end{defi}
Come osservato per i vettori del piano, le coordinate $c_1, \text{ } c_2, \text{ } c_3$ di un vettore dipendono chiaramente dalla scelta dei vettori base $\vec{OP}_1,\vec{OP}_2, \vec{OP}_3$, ma una volta che essi sono stati fissati si potrà scrivere $\vec{OP}\equiv (c_1,c_2,c_3)$, identificando di fatto il vettore con la terna delle sue coordinate, e quindi l'insieme $\vec{V}_O^2$ con l'insieme $\mathds{R}^3$ della terna di numeri reali.\\
L'importanza delle coordinate consiste nel fatto che esse, permattendoci di rappresentare i vettori mediamente coppie o terne di numeri, permettano di tradurre in calcolo tra vettori: questa è un'importante semplificazione, in quanto è più semplice lavorare con numeri che con costruzioni o dimostrazioni di geometria eoclidea che sarebbero altrimenti necessarie per lavorare con i vettori, che sono oggetti (entità) geometrici. Per dare un idea più chiara delle affermazioni esposte precedentemente è necessario stimare questo importante risultato:
\begin{prop}
  \label{prop:coordinate1}
  Sia $\vec{OP}_1$, $\vec{OP}_2$ una coppia di vettori base non allineati nell'insieme $V_O^2$. Le coordinate rispetto a $\vec{OP}_1$, $\vec{OP}_2$ hanno le seguenti proprietà:
  \begin{enumerate}
  \item Se $\vec{OP}$ e $\vec{OP}^\prime$ hanno coordinate rispettivamente $(x_1,x_2)$ e $(x^\prime_1,x^\prime_2)$, le coordinate di $\vec{OP}+\vec{OP}^\prime$ sono date dalla coppia $(x_1+x^\prime_1,x_2+x^\prime_2)$ ottenuta sommando componete per componente le coppie delle coordinate dei due vettori.
  \item Se $\vec{OP}$ ha coordinate $(x_1,x_2)$ e $c\in \mathds{R}$ è un numero reale, allora le coordinate di $c\vec{OP}$ sono date dalla coppia $(cx_1,cx_2)$ ottenuta moltiplicando per $c$ le coordinate di $\vec{OP}$.
  \end{enumerate}
\end{prop}
\begin{proof}
  Il fatto che $\vec{OP}$ abbia coordinate $(x_1,x_2)$ rispetto a $\vec{OP}_1$, $\vec{OP}_2$ significa per definizione che $\vec{OP}^\prime= x_1\vec{OP}_1+x_2\vec{OP}_2$, e analogamente il fatto che $\vec{OP}^\prime$ abbia coordinate $(x^\prime_1,x^\prime_2)$ significa che $\vec{OP}^\prime= x^\prime_1\vec{OP}_1+x^\prime_2\vec{OP}_2$. Ma allora
  \begin{equation*}
    \vec{OP}+\vec{OP}^\prime=(x_1\vec{OP}_1+x_2\vec{OP}_2)+(x^\prime_1\vec{OP}_1+x^\prime_2\vec{OP}_2)=
  \end{equation*}
  Riordinando gli addendi e raccogliendoli diversamente sfruttando le proprietà associativa e commutativa della somma tra vettori
  \begin{equation*}
    =(x_1\vec{OP}_1+x^\prime_1\vec{OP}_1)+(x_2\vec{OP}_2+x^\prime_2\vec{OP}_2)=
  \end{equation*}
  Sfruttando la proprietà \ref{eq:numrealeVectOP} sia nella prima parentesi che nella saconda, effettuato il raggruppamento mettendo in evvidenza nel caso della prima parentesi $\vec{OP}_1$, mentre, nel caso del secondo mettendo in evvidenza $\vec{OP}_2$, il risultato sarà
  \begin{equation*}
    =(x_1+x^\prime_1)\vec{OP}_1+(x_2+x^\prime_2)\vec{OP}_2
  \end{equation*}
  Ma questo, per definizione di coordinate, significa proprio che le coordinate di $\vec{OP}+\vec{OP}^\prime$ sono date dalla coppia $(x_1+x^\prime_1,x_2+x^\prime_2)$, come affermato nel punto 1 della Proposizione \ref{prop:coordinate1}.\\
  Per dimostrare la (2), bisogna partire sempre dal fatto che $\vec{OP}$ abbia coordinate $(x_1,x_2)$ significa per definizione che $\vec{OP}= x_1\vec{OP}_1+x_2\vec{OP}_2$. Allora
  \begin{equation*}
    c\vec{OP}=c(x_1\vec{OP}_1+x_2\vec{OP}_1)=
  \end{equation*}
  Applicando la proprietà (\ref{eq:prodottoconduenumerirealiperunnumeroreale}) otterremo la divisione in due gruppi di parentesi, con c messo in evidenza messi tra di loro in forma di addizione.
  \begin{equation*}
    =c(x_1\vec{OP}_1)+c(x_2\vec{OP}_2)=
  \end{equation*}
  Applicando la proprietà (\ref{eq:prodottoconduenumerireali}) a entrambi gli addendi si otterrà:
  \begin{equation*}
    =(cx_!)\vec{OP}_1+(cx_2)\vec{OP}_2
  \end{equation*}
  Ma questo, per definizione di coordinate, ci dice proprio che le coordinate di $c\vec{OP}$ sono date dalla coppia ($cx_1,cx_2$), come affermato nella (2) della  Proposizione \ref{prop:coordinate1}.
\end{proof}
\begin{es}
  \label{es:coordinate1}
  Per un esempio di quanto appena dimostrato, si prendano i vettori base $\vec{OP}_1$ e $\vec{OP}_2$ come nel disegno seguente, e si considerino i due $\vec{OQ}_1$ e $\vec{OQ}_2$
  \begin{figure}[ht!]
  \centering
  \resizebox{4cm}{!}{
      \begin{tikzpicture}
	\begin{pgfonlayer}{nodelayer}
		\node [style=none] (0) at (0, 3) {};
		\node [style=none] (1) at (0, 0) {};
		\node [style=none] (2) at (3, 0) {};
		\node [style=none] (3) at (8, 3) {};
		\node [style=none] (4) at (8, 0) {};
		\node [style=none] (5) at (0, 8) {};
		\node [style=none] (6) at (3, 8) {};
		\node [style=none] (7) at (-0.25, -0.25) {$O$};
		\node [style=none] (8) at (-0.75, 3) {$P_2$};
		\node [style=none] (9) at (3, -0.5) {$P_1$};
		\node [style=none] (10) at (8.25, 3.5) {$Q_1$};
		\node [style=none] (11) at (3, 8.5) {$Q_2$};
	\end{pgfonlayer}
	\begin{pgfonlayer}{edgelayer}
		\draw [style=Rightarrow] (1.center) to (0.center);
		\draw [style=Rightarrow] (1.center) to (2.center);
		\draw [style=Rightarrow] (1.center) to (6.center);
		\draw [style=Rightarrow] (1.center) to (3.center);
		\draw [style=DashedCampitura] (2.center) to (4.center);
		\draw [style=DashedCampitura] (4.center) to (3.center);
		\draw [style=DashedCampitura] (0.center) to (3.center);
		\draw [style=DashedCampitura] (2.center) to (6.center);
		\draw [style=DashedCampitura] (6.center) to (5.center);
		\draw [style=DashedCampitura] (5.center) to (0.center);
	\end{pgfonlayer}
\end{tikzpicture}

    }
  \caption{Rappresentazione grafica $OQ_1$ e $OQ_2$}
  \label{fig:coordinate1-1}
\end{figure}\\
Come si vede dalla figura (\ref{fig:coordinate1-1}), si ha $\vec{OQ}_1=2\vec{OP}_1+\vec{OP}_2$ e $\vec{OQ}_1=\vec{OP}_1+2\vec{OP}_2$, ovvero le coordinate $\vec{OP}_1$ sono date dalla coppia (2, 1).\\
Allora, in base alla (1) della Proposizione \ref{prop:coordinate1}, la somma $\vec{OQ}_1+\vec{OQ}_2$ ha coordinate (\textit{sempre rispetto a $\vec{OP}_1$ e $\vec{OP}_2$}) date da 
\begin{eqnarray*}
  \vec{OQ}_1=
  \begin{vmatrix}
    2\\
    1
  \end{vmatrix},\text{ } \vec{OQ}_2=
  \begin{vmatrix}
    1\\
    2
  \end{vmatrix} &\to& \vec{OQ}_1+\vec{OQ}_2=
                  \begin{vmatrix}
                    2{\color{red}+1}\\
                    1{\color{red}+2}
                  \end{vmatrix}=
                  \begin{vmatrix}
                    3\\
                    3
                  \end{vmatrix}= (3,3).
\end{eqnarray*}
ovvero si ha $\vec{OQ}_1+\vec{OQ}_2=3\vec{OP}_1+3\vec{OP}_2$. In effetti, questo può essere verificato graficamente costruendo con la regola del parallelogramma la somma $\vec{OQ}_1+\vec{OQ}_2$, come nella figura seguente
  \begin{figure}[ht!]
  \centering
  \resizebox{3.4cm}{!}{
      \begin{tikzpicture}
	\begin{pgfonlayer}{nodelayer}
		\node [style=none] (0) at (-3, 3) {};
		\node [style=none] (1) at (-3, 0) {};
		\node [style=none] (2) at (0, 0) {};
		\node [style=none] (3) at (-3, 6) {};
		\node [style=none] (4) at (-3, 9) {};
		\node [style=none] (5) at (0, 9) {};
		\node [style=none] (6) at (0, 6) {};
		\node [style=none] (7) at (0, 3) {};
		\node [style=none] (8) at (3, 9) {};
		\node [style=none] (9) at (3, 6) {};
		\node [style=none] (10) at (3, 3) {};
		\node [style=none] (11) at (3, 0) {};
		\node [style=none] (12) at (6, 9) {};
		\node [style=none] (13) at (6, 6) {};
		\node [style=none] (14) at (6, 3) {};
		\node [style=none] (15) at (6, 0) {};
		\node [style=none] (16) at (-0.75, 5.75) {$Q_2$};
		\node [style=none] (17) at (3.5, 2.5) {$Q_1$};
		\node [style=none] (18) at (6, 9.75) {$\vec{OQ}_1+\vec{OQ}_2$};
		\node [style=none] (19) at (-3.75, 3) {$P_2$};
		\node [style=none] (20) at (0, -0.5) {$P_1$};
		\node [style=none] (21) at (-3.25, -0.25) {O};
	\end{pgfonlayer}
	\begin{pgfonlayer}{edgelayer}
		\draw [style=DashedCampitura] (4.center) to (12.center);
		\draw [style=DashedCampitura] (12.center) to (15.center);
		\draw [style=DashedCampitura] (4.center) to (0.center);
		\draw [style=DashedCampitura] (2.center) to (15.center);
		\draw [style=DashedCampitura] (8.center) to (11.center);
		\draw [style=DashedCampitura] (5.center) to (2.center);
		\draw [style=DashedCampitura] (3.center) to (13.center);
		\draw [style=DashedCampitura] (14.center) to (0.center);
		\draw [style=Rightarrow] (1.center) to (0.center);
		\draw [style=Rightarrow] (1.center) to (2.center);
		\draw [style=Rightarrow] (1.center) to (12.center);
		\draw [style=Rightarrow] (1.center) to (6.center);
		\draw [style=Rightarrow] (1.center) to (10.center);
		\draw [style=Rightarrow] (6.center) to (12.center);
		\draw [style=Rightarrow] (10.center) to (12.center);
	\end{pgfonlayer}
\end{tikzpicture}

    }
  \caption{Rappresentazione grafica $\vec{OQ}_1+\vec{OQ}_2$}
  \label{fig:coordinate1-2}
\end{figure}\\
L'aspetto notevole è che si può dimostrare chi era il vettore $\vec{OQ}_1+\vec{OQ}_2$ (in coordinate) con un semplice conto aritmetico, anche prima di disegnarlo con la costruzione geometrica del parallelogramma.
\end{es}
\begin{oss}
  \label{oss:coordinate2}
  Affermazioni del tutto analoghe a quelle della Proposizione \ref{prop:coordinate1} valgono anche nel caso dei vettori nello spazio. Più precisamente, si ha che fissata una terna $\vec{OP}_1, \vec{OP}_2,\vec{OP}_3$ di vettori non complanari nell'insieme $V_O^3$ dei vettori dello spazio tridimensionale, allora le coordiante rispetto a tale terna di base hanno le seguenti proprietà:
  \begin{enumerate}
  \item Se $\vec{OP}$ e $\vec{OP}^\prime$ hanno coordinate rispettivamete $(x_1,x_2,x_3)$ e $(x_1^\prime,x_2^\prime,x_3^\prime)$, le coordinate di $\vec{OP}_1+\vec{OP}_1^\prime$ sono date dalla terna $(x_1+x_1^\prime,x_2+x_2^\prime,x_3+x_3^\prime)$ ottenuta sommando componente per componente le terne delle coordiante dei due vettori.
  \item Se $\vec{OP}$ ha coordinate $(x_1,x_2,x_3)$ e $c\in \mathds{R}$ è un numero reale, allora le coordinate, di $c\vec{OP}$ sono date dalla terna $(cx_1,cx_2,cx_3)$ ottenuta moltiplicando per $c$ le coordinate di $\vec{OP}$.
  \end{enumerate}
  La dimostrazione è perfettamente analoga a quella della Proposizione \ref{prop:coordinate1}.
\end{oss}
\section{Lunghezze e angoli}
\label{sec:lungeang}

Lavorare in coordiante rispetto a una base ci permette di tradurre numericamente costruzioni geometriche con i vettori e risolvere in modo più semplice problimi relativi ai vettori. Questo è quero qualunque sia la base scelta, tuttavia a seconda del problema specifico da risolvere, alcune basi possono essere più convenienti di altre, e in particolare quando si vuole rispondere, lavorando in coordinate, alle domande seguenti: ``Quel'è la lunghezza di un vettore dato? quel'è l'angolo tra due vettori dati?\\
In tal caso, le basi più convenienti da usare, come visto, sono quelle formate da (due nel caso del piano, tre nel caso dello spazio) vettori ctra loro ortogonali e di lunghezza 1 (\textit{rispetto a un'unità di misura scelta}). Tali basi si chiamano \textit{ortonormale}.\\
Infatti, considerando una tale base nel piano
\begin{figure}[ht!]
  \centering
  \resizebox{3cm}{!}{
      \begin{tikzpicture}
	\begin{pgfonlayer}{nodelayer}
		\node [style=none] (0) at (0, 4) {};
		\node [style=none] (1) at (0, 0) {};
		\node [style=none] (2) at (4, 0) {};
		\node [style=none] (3) at (0, 0.5) {};
		\node [style=none] (4) at (0.5, 0) {};
		\node [style=none] (5) at (-0.25, -0.5) {O};
		\node [style=none] (6) at (-0.75, 4) {$P_2$};
		\node [style=none] (7) at (4, -0.5) {$P_1$};
	\end{pgfonlayer}
	\begin{pgfonlayer}{edgelayer}
		\draw [style=campitura, bend left=45, looseness=1.50] (3.center) to (4.center);
		\draw [style=Rightarrow] (1.center) to (0.center);
		\draw [style=Rightarrow] (1.center) to (2.center);
	\end{pgfonlayer}
\end{tikzpicture}

    }
  \caption{Base del piano}
  \label{fig:basedelpiano}
\end{figure}\\
Ora, considerando un vettore $\vec{OP}$, di quale sono note le coordinate rispetto a tale base sono date da $(x_1,x_2)$ (ovvero, per definizione di coordinate, $\vec{OP}=x_1\vec{OP}_1+x_2\vec{OP}_2$): è possibile calcolare la lunghezza del vettore $\vec{OP}$ a partire dalle coordinate? Per rispondere a tale domanda, bisogna considerare le seguenti figure, nel quale è rappresentata la decomposizione $\vec{OP}=x_1\vec{OP}_1+x_2\vec{OP}_2$
\begin{figure}[ht!]
  \centering
  \resizebox{4cm}{!}{
      \begin{tikzpicture}
	\begin{pgfonlayer}{nodelayer}
		\node [style=none] (0) at (0, 4) {};
		\node [style=none] (1) at (0, 0) {};
		\node [style=none] (2) at (4, 0) {};
		\node [style=none] (3) at (0, 0.5) {};
		\node [style=none] (4) at (0.5, 0) {};
		\node [style=none] (5) at (-0.25, -0.5) {O};
		\node [style=none] (6) at (-0.75, 4) {$P_2$};
		\node [style=none] (7) at (4, -0.5) {$P_1$};
		\node [style=none] (8) at (0, 6) {};
		\node [style=none] (9) at (6, 0) {};
		\node [style=none] (10) at (0, 10) {};
		\node [style=none] (11) at (10, 0) {};
		\node [style=none] (12) at (6, 6) {};
		\node [style=none] (13) at (6, 0.5) {};
		\node [style=none] (14) at (5.5, 0) {};
		\node [style=none] (15) at (-1.25, 6) {$x_2\vec{OP}_2$};
		\node [style=none] (16) at (6, -0.5) {$x_1\vec{OP}_1$};
		\node [style=none] (17) at (10, -0.5) {$r_1$};
		\node [style=none] (18) at (6.25, 6.5) {$P$};
	\end{pgfonlayer}
	\begin{pgfonlayer}{edgelayer}
		\draw [style=campitura, bend left=45, looseness=1.50] (3.center) to (4.center);
		\draw [style=Rightarrow] (1.center) to (0.center);
		\draw [style=Rightarrow] (1.center) to (2.center);
		\draw [style=Rightarrow] (0.center) to (8.center);
		\draw [style=Rightarrow] (2.center) to (9.center);
		\draw [style=campitura] (8.center) to (10.center);
		\draw [style=campitura] (9.center) to (11.center);
		\draw [style=Rightarrow] (1.center) to (12.center);
		\draw [style=DashedCampitura] (8.center) to (12.center);
		\draw [style=DashedCampitura] (12.center) to (9.center);
		\draw [style=campitura, bend left=45, looseness=1.25] (14.center) to (13.center);
	\end{pgfonlayer}
\end{tikzpicture}

    }
  \caption{Base del piano con il vettore $\vec{OP}$}
  \label{fig:basedelpianoConVettOP}
\end{figure}\\
Dal momento che si è selto i vettori di base perpendicolari, quando si proietta $P$ sulla retta $r_1$ che contiene $\vec{OP}_1$ sequendo la direzione $\vec{OP}_2$, tale proiezione incontra $r_1$ con un angolo di $90^o$, e si viene quindi a formare un triangolo rettangolo (evidenziato nel figura \ref{fig:basedelpianoConVettOP}) avente come ipotenusa proprio $\vec{OP}$ e al quele possiamo quindi applicare il teorema di Pitagora per calcolare la lunghezza di $\vec{OP}$, che denoterà $\abs{\vec{OP}}$. \\
A quasto scopo, c'e da notare che il cateto orizzontale di tale triangolo è dato dal vettore $x_1\vec{OP}_1$, e quindi la sua lunghezza è data dal prodotto di $x_!$ per la lunghezza di $\vec{OP}_1$: ma avendo scelto i vettori di base di lunghezza unitaria, questo implica che la lunghezza di tale cateto sia semplicemente $x_1$; per quello che riguarda il cateto verticale, esso per costruzione ha la stessa lunghezza del vettore $x_2\vec{OP}_2$, ovvero $x_2$ (in quanto $\vec{OP}_2$ ha lunqhezza 1). Quindi il teorema di Pitagora dice che $\abs{\vec{OP}}^2=\sqrt{x_1^2+x_2^2}$,
\begin{equation}
  \label{eq:teoremadiPitapplicatoaAbsOP}
  \abs{\vec{OP}}=\abs{x}=\sqrt{x_1^2+x_2^2}
\end{equation}
che rappresenta la formula cercata, che ci dà la lunghezza di $\vec{OP}$ in funzione delle sue coordinate.\\
Si nota che nei ragionamenti svolti sono fontamentali per la scelta di una base fatta di vettori ortogonali (questo ha fatto comparire un triangolo rettangolo a cui viene applicato il teorema di Pitagora) e di lunghezza 1 (che ha permesso di esprimere le lunghezze dei cateti in funzione delle sole coordinate). \\
Dopo aver trattato del piano, adesso è necessario trattare lo spazio nella sua costruzione, infatti lo spazio trigonometrico è composto da una terna di vettori: $\vec{OP}_1,\vec{OP}_2,\vec{OP}_3$ appartenenti all'insieme $V_O^3$ dei vettori applicati nello spazio tridimensionale:
\begin{figure}[ht!]
  \centering
    \resizebox{4cm}{!}{
      \begin{tikzpicture}
	\begin{pgfonlayer}{nodelayer}
		\node [style=none] (0) at (0, 4) {};
		\node [style=none] (1) at (0, 0) {};
		\node [style=none] (2) at (4, 0) {};
		\node [style=none] (3) at (-3, -3) {};
		\node [style=none] (4) at (-0.25, -0.25) {};
		\node [style=none] (6) at (0, 0.25) {};
		\node [style=none] (7) at (0.25, 0) {};
		\node [style=none] (8) at (-0.5, 0.25) {O};
		\node [style=none] (9) at (-0.5, 4) {$P_3$};
		\node [style=none] (10) at (4, -0.5) {$P_2$};
		\node [style=none] (11) at (-3.5, -3) {$P_1$};
	\end{pgfonlayer}
	\begin{pgfonlayer}{edgelayer}
		\draw [style=Rightarrow] (1.center) to (0.center);
		\draw [style=Rightarrow] (1.center) to (2.center);
		\draw [style=Rightarrow] (1.center) to (3.center);
		\draw [bend left=45, looseness=1.50] (6.center) to (7.center);
		\draw [bend right=60, looseness=1.25] (6.center) to (4.center);
		\draw [bend right=60, looseness=1.25] (4.center) to (7.center);
	\end{pgfonlayer}
\end{tikzpicture}

    }
  \caption{Costruzione grafica base spazio}
  \label{fig:costbasespazio}
\end{figure}\\
Supponendo ora di avere un vettore $\vec{OP}$ e di volerne calcolare la lunghezza, si denota $\abs{\vec{OP}}$, in fuzione delle sue coordinate $x_1,x_2,x_3$ rispetto alla base $B$ scelta. Per definizione di coordintate, $\vec{OP}$ si decompone come somma $\vec{OP}=x_1\vec{OP}_1+x_2\vec{OP}_2+x_3\vec{OP}_3$, come in figura \ref{fig:basedelspazioConVettOP}.
\begin{figure}[ht!]
  \centering
  \resizebox{4cm}{!}{
      \begin{tikzpicture}
	\begin{pgfonlayer}{nodelayer}
		\node [style=none] (0) at (0, 2) {};
		\node [style=none] (1) at (0, 0) {};
		\node [style=none] (2) at (2, 0) {};
		\node [style=none] (3) at (-2, -2) {};
		\node [style=none] (6) at (1.75, 7) {};
		\node [style=none] (8) at (-0.5, 0.25) {O};
		\node [style=none] (9) at (-0.5, 2) {$P_3$};
		\node [style=none] (10) at (2.25, 0.5) {$P_2$};
		\node [style=none] (11) at (-2.75, -1.75) {$P_1$};
		\node [style=none] (12) at (1.75, -3) {};
		\node [style=none] (13) at (0, 9) {};
		\node [style=none] (14) at (4, 0) {};
		\node [style=none] (15) at (-3, -3) {};
		\node [style=none] (16) at (2, -3.5) {$Q$};
		\node [style=none] (17) at (4.75, 0) {$x_2\vec{OP}_2$};
		\node [style=none] (18) at (-4, -3) {$x_1\vec{OP}_1$};
		\node [style=none] (19) at (2, 7.5) {P};
		\node [style=none] (20) at (-1.25, 9) {$x_3\vec{OP}_3$};
		\node [style=none] (21) at (0, 8) {};
		\node [style=none] (22) at (1.75, 6) {};
		\node [style=none] (23) at (0, 7) {};
		\node [style=none] (24) at (1.75, 5) {};
		\node [style=none] (25) at (0, 6) {};
		\node [style=none] (26) at (1.75, 4) {};
		\node [style=none] (27) at (0, 5) {};
		\node [style=none] (28) at (1.75, 3) {};
		\node [style=none] (29) at (0, 4) {};
		\node [style=none] (30) at (0, 3) {};
		\node [style=none] (31) at (1.75, 2) {};
		\node [style=none] (32) at (1.75, 0) {};
		\node [style=none] (33) at (0, 1) {};
		\node [style=none] (34) at (1.75, -1.25) {};
		\node [style=none] (35) at (1.75, 1) {};
	\end{pgfonlayer}
	\begin{pgfonlayer}{edgelayer}
		\draw [style=Rightarrow] (1.center) to (0.center);
		\draw [style=Rightarrow] (1.center) to (2.center);
		\draw [style=Rightarrow] (1.center) to (3.center);
		\draw [style=Rightarrow] (1.center) to (12.center);
		\draw [style=Rightarrow] (0.center) to (13.center);
		\draw [style=Rightarrow] (1.center) to (6.center);
		\draw [style=DashedCampitura] (13.center) to (6.center);
		\draw [style=DashedCampitura] (6.center) to (12.center);
		\draw [style=Rightarrow] (2.center) to (14.center);
		\draw [style=DashedCampitura] (14.center) to (12.center);
		\draw [style=Rightarrow] (3.center) to (15.center);
		\draw [style=DashedCampitura] (15.center) to (12.center);
		\draw [style=campitura] (21.center) to (22.center);
		\draw [style=campitura] (23.center) to (24.center);
		\draw [style=campitura] (33.center) to (34.center);
		\draw [style=campitura] (0.center) to (32.center);
		\draw [style=campitura] (25.center) to (26.center);
		\draw [style=campitura] (27.center) to (28.center);
		\draw [style=campitura] (29.center) to (31.center);
		\draw [style=campitura] (30.center) to (35.center);
	\end{pgfonlayer}
\end{tikzpicture}

    }
  \caption{Base dello spazio con il vettore $\vec{OP}$}
  \label{fig:basedelspazioConVettOP}
\end{figure}\\
La decomposizione è stata ottenuta graficamente come segue: prima si proietta $P$ perpendicolarmente sul piano su cui stanno $P_1$ e $P_2$ ottenendo il punto $Q$ (l'angolo in $Q$ quindi è retto, come messo in evidenza nella figura) e si ottiene un rettangolo, come campitura in grigio nella figura, che dice che $\vec{OP}=\vec{OQ}+x_3\vec{OP}_3$; poi dal momento che $\vec{OQ}$ giace sul piano di $P_!$ e $P_2$ lo si può decomporre come $\vec{OQ}=x_1\vec{OP}_1+x_2\vec{OP}_2$ (sempre sul piano retti in quanto $\vec{OP}_1$ e $\vec{OP}_2$ sono perpendicolari), e quindi $\vec{OP}=\vec{OQ}+x_3\vec{OP}_3=x_1\vec{OP}_1+x_2\vec{OP}_2+x_3\vec{OP}_3$ come visto sopra.\\
Ora, essendo $\vec{OP}$ l'ipotenusa del triangolo $OPQ$ rettangolo in $Q$, per il teorema di Pitagora si avrà
\begin{equation}
  \label{eq:teopitapplicatinellospazio}
  \abs{OP}^2=\abs{OQ}^2+\abs{PQ}^2
\end{equation}
Ma da una parte, il segmento $PQ$, essendo un lato del rettangolo ombreggiato in figura, è lungo esattamente quanto il vettore $x_3\vec{OP_3}$, ovvere $x_3$ (in quanto $\vec{OP}$ ha lunghezza 1); dall'altra, $OQ$ è la diagonale del rettangolo che ha come lati i vettori $x_1\vec{OP}_1$ e $x_2\vec{OP}_2$ di lunghezza rispettivamente $x_1$ e $x_2$ (in quanto $\vec{OP}_1$ e $\vec{OP}_2$ hanno lunghezza 1), quindi sempre per il teorema di Pitagora si ha $\abs{OP}^2=x_1^2+x_2^2+x_3^2$, ovvero, se per la terna $x=(x_1,x_2,x_3)$ si utilizza la notazione $\abs{x}=\sqrt{x_1^2+x_2^2+x_3^2}$,
\begin{equation}
  \label{eq:teopitapplicatinellospazio2}
  \abs{\vec{OP}}=\abs{x}=\sqrt{x_1^2+x_2^2+x_3^2}
\end{equation}
che è la formula cercata, angolora della (\ref{eq:teoremadiPitapplicatoaAbsOP}), per la lunghezza di un vettore geometrico $\vec{OP}$ dello spazio in funzione delle sue coordinate rispetto alla base scelta.\\
Ora, bisogna porsi il problema di calcolare l'angolo tra due vettori non nulli $\vec{OP}, \vec{OQ}\in V_O^3$ una volta note le loro coordinate rispetto a una base ortonormale. Supponendo che tali coordinate siano rispettivamente ($x_1,x_2,x_3$) e ($y_1,y_2,y_3$).
\begin{figure}[ht!]
  \centering
  \resizebox{4cm}{!}{
      \begin{tikzpicture}
	\begin{pgfonlayer}{nodelayer}
		\node [style=none] (0) at (0, 3) {};
		\node [style=none] (1) at (0, 0) {};
		\node [style=none] (2) at (-1.5, -1.5) {};
		\node [style=none] (3) at (3.75, 0) {};
		\node [style=none] (4) at (2, 5.75) {};
		\node [style=none] (5) at (4, 4) {};
		\node [style=none] (6) at (1.75, 6) {$P$};
		\node [style=none] (7) at (4.25, 4.25) {$Q$};
		\node [style=none] (8) at (0.25, 0.75) {};
		\node [style=none] (9) at (0.5, 0.5) {};
		\node [style=none] (10) at (0.75, 1.25) {$\theta$};
		\node [style=none] (11) at (4.25, 0) {$P_2$};
		\node [style=none] (12) at (-0.5, 0) {$O$};
		\node [style=none] (13) at (-1.75, -1.75) {$P_1$};
		\node [style=none] (14) at (0, 3.5) {$P_3$};
	\end{pgfonlayer}
	\begin{pgfonlayer}{edgelayer}
		\draw [style=Rightarrow] (1.center) to (2.center);
		\draw [style=Rightarrow] (1.center) to (3.center);
		\draw [style=Rightarrow] (1.center) to (0.center);
		\draw [style=Rightarrow] (1.center) to (5.center);
		\draw [style=Rightarrow] (1.center) to (4.center);
		\draw [style=DashedCampitura] (4.center) to (5.center);
		\draw [style=DashedLine] (8.center) to (9.center);
	\end{pgfonlayer}
\end{tikzpicture}

    }
  \caption{Triangolo OPQ}
  \label{fig:triangoloOPQ1}
\end{figure}\\
Per un risultato di trigonometria, l'angolo $\theta$ tra $\vec{OP}$ e $\vec{OQ}$ è collegato alla lunghezza dei segmenti $OP,OQ,PQ$ dalla formuala\footnote{Si tratta di una sorta di ``teorema di Pitagora per triangoli qualunque'': infatti, se il trangolo è rettangolo in $O$, ovvero $\theta=\frac{\pi}{2}$, allora $\cos\theta=0$ e la formula si riduce a $\abs{\vec{PQ}}^2=\abs{OQ}+\abs{OQ}^2$, il classico teorema di Pitagora.}
\begin{equation}
  \label{eq:teopitapplicatinellospazio3}
  \abs{\vec{PQ}}^2+\abs{OP}^2+\abs{OQ}^2-2\cos\theta\abs{OP}\cdot\abs{OQ}
\end{equation}
Ora, per la (\ref{eq:teopitapplicatinellospazio2}), si ha $\abs{OP}=\sqrt{x_1^2+x_2^2+x_3^2}$ e $\abs{OQ}=\sqrt{y_1^2+y_2^2+y_3^2}$: per ricavare l'angolo $\theta$ tramte la formuala (\ref{eq:teopitapplicatinellospazio3}) resta da calcolare la lunghezza $\abs{PQ}$. Dal momento che la formuala (\ref{eq:teopitapplicatinellospazio2}) consente di calcolare la lunghezza solo dei vettori applicati in $O$, sarà possibile tracciare il seguente disegno
\begin{figure}[ht!]
  \centering
  \resizebox{4cm}{!}{
      \begin{tikzpicture}
	\begin{pgfonlayer}{nodelayer}
		\node [style=none] (0) at (0, 3) {};
		\node [style=none] (1) at (0, 0) {};
		\node [style=none] (2) at (-1.5, -1.5) {};
		\node [style=none] (3) at (3.75, 0) {};
		\node [style=none] (4) at (2, 5.75) {};
		\node [style=none] (5) at (4, 4) {};
		\node [style=none] (6) at (1.75, 6) {$P$};
		\node [style=none] (7) at (4.25, 4.25) {$Q$};
		\node [style=none] (8) at (0.25, 0.75) {};
		\node [style=none] (9) at (0.5, 0.5) {};
		\node [style=none] (10) at (0.75, 1.25) {$\theta$};
		\node [style=none] (11) at (4.25, 0) {$P_2$};
		\node [style=none] (12) at (-0.5, 0) {$O$};
		\node [style=none] (13) at (-1.75, -1.75) {$P_1$};
		\node [style=none] (14) at (0, 3.5) {$P_3$};
		\node [style=none] (15) at (2, -2) {};
		\node [style=none] (16) at (2.5, -2.5) {$R$};
	\end{pgfonlayer}
	\begin{pgfonlayer}{edgelayer}
		\draw [style=Rightarrow] (1.center) to (2.center);
		\draw [style=Rightarrow] (1.center) to (3.center);
		\draw [style=Rightarrow] (1.center) to (0.center);
		\draw [style=Rightarrow] (1.center) to (5.center);
		\draw [style=Rightarrow] (1.center) to (4.center);
		\draw [style=DashedCampitura] (4.center) to (5.center);
		\draw [style=DashedLine] (8.center) to (9.center);
		\draw [style=Rightarrow] (1.center) to (15.center);
		\draw [style=DashedCampitura] (5.center) to (15.center);
	\end{pgfonlayer}
\end{tikzpicture}

    }
  \caption{Triangolo OPQ e OQR}
  \label{fig:triangoloOPQeOQR}
\end{figure}\\
Il vettore $\vec{OR}$ parallelo al segmento $PQ$ e avente la sua stessa lunghezza, ovvero $\abs{PQ}=\abs{\vec{OR}}$.\\
Ora, essendo $\vec{OR}$ parallelo a $PQ$ e della stessa lunghezza, il quadrilattero di vertici $O,R,P,Q$ è un parallelogramma che ha $\vec{OR}$ e $\vec{OP}$ come lati e $\vec{OQ}$ come diagonale: quindi, dalla definizione di somma tra vettori applicati, si ha $\vec{OQ}=\vec{OR}+\vec{OP}$, ovvero $\vec{OR}=\vec{OQ}-\vec{OP}$.\\
Per le proprietà telle coordinate viste nell'Osservazione \ref{oss:coordinate2}, le coordinate di $\vec{OR}=\vec{OQ}-\vec{OP}$ sono date dalle coordinate di $\vec{OQ}$ meno le coordinate di $\vec{OP}$, ovvero $(y_1-x_1,y_2-x_2,y_3-x_3)$, e quindi, dalla (\ref{eq:teoremadiPitapplicatoaAbsOP}) si ha finalmente
\begin{equation}
  \label{eq:teopitapplicatinellospazio3}
  \abs{PQ}=\abs{\vec{OR}}=\sqrt{(y_1-x_1)^2+(y_2-x_2)^2+(y_3-x_3)^2}
\end{equation}
La formula (\ref{eq:teopitapplicatinellospazio2}) diventa allora
\begin{equation}
  \label{eq:teopitapplicatinellospazio4}
  \begin{matrix}
    (y_1-x_1)^2+(y_2-x_2)^2+(y_3-x_3)^2=y^2_1-x^2_1+y^2_2-x^2_2+y^2_3-x^2_3\\
    -2\cos\theta\sqrt{x^2_1+x^2_2+x^2_3}\cdot\sqrt{y^2_1+y^2_2+y^2_3}
  \end{matrix}  
\end{equation}
Poiché il primo membro, per la formula del quadrato di binomio\footnote{Lo sviluppo del quadrato di binomio è $(a\pm{}b)^2=a^2+b^2\pm2ab$}, è uguale a
\begin{equation*}
  x^2_1+y^2_1-2x_1y_1+x^2_2+y^2_2-2x_2y_2+x^2_3+y^2_3-2x_3y_3
\end{equation*}
semplificando con i quadrati a secondo membro rimane:
\begin{equation}
  \label{eq:teopitapplicatinellospazio5}
  -2x_1y_1+-2x_2y_2-2x_3y_3=-2\cos\theta\sqrt{x^2_1+x^2_2+x^2_3}\cdot\sqrt{y^2_1+y^2_2+y^2_3}
\end{equation}
ovvero, ricavando $\cos\theta$,
\begin{equation}
  \label{eq:teopitapplicatinellospazio6}
  \cos\theta = \frac{x_1y_1+x_2y_2+x_3y_3}{\sqrt{x^2_1+x^2_2+x^2_3}\cdot\sqrt{y^2_1+y^2_2+y^2_3}}
\end{equation} 
che è finalmente la formula cercata che esprime l'angolo tra due vettori in funzione delle loro coordinate rispetto alla base data.\\
Con un calcolo analogo nel piano (dove non cambia nulla delle costruzioni fatte e dei passaggi svolti, salvo il fatto che abbiamo due coordinate anziché tre) si ottiene la formula analoga
\begin{equation}
  \label{eq:teopitapplicatinellospazio7}
  \cos\theta = \frac{x_1y_1+x_2y_2}{\sqrt{x^2_1+x^2_2}\cdot\sqrt{y^2_1+y^2_2}}
\end{equation}
\begin{oss}
  \label{oss:teopitapplicatinellospazio}
  Una volta ricavato il valore del coseno dell'angolo mediante la formula (\ref{eq:teopitapplicatinellospazio6}) [la (\ref{eq:teopitapplicatinellospazio7}) nel caso del piano], all'interno dell'intervallo $\left[0,2\pi\right]$ avremo in generale \textit{due} possibili valori di $\theta$ corrispondenti: ad esempio, se $\cos\theta=\frac{1}{2}$ allora $\theta=\frac{\pi}{3}$ oppure $\theta=2\pi -\frac{\pi}{3}=\frac{5}{3}\pi$. Questo riflette il fatto geometroco ovvio, illutrato dall'immagine
  \begin{figure}[ht!]
    \centering
    \resizebox{3cm}{!}{
      \begin{tikzpicture}
	\begin{pgfonlayer}{nodelayer}
		\node [style=none] (0) at (0, 4) {};
		\node [style=none] (1) at (0, 0) {};
		\node [style=none] (3) at (2, 4) {};
		\node [style=none] (4) at (0, 1) {};
		\node [style=none] (5) at (0.5, 1) {};
		\node [style=none] (6) at (0, -0.75) {};
		\node [style=none] (7) at (0.5, 2) {$\leq \pi$};
		\node [style=none] (8) at (0, 0.5) {};
		\node [style=none] (9) at (0.25, 0.5) {};
		\node [style=none] (10) at (0, -1.25) {$\geq \pi$};
	\end{pgfonlayer}
	\begin{pgfonlayer}{edgelayer}
		\draw [style=Rightarrow] (1.center) to (0.center);
		\draw [style=Rightarrow] (1.center) to (3.center);
		\draw [bend left, looseness=1.50] (4.center) to (5.center);
		\draw [bend right=75, looseness=1.50] (8.center) to (6.center);
		\draw [bend right=75, looseness=1.50] (6.center) to (9.center);
	\end{pgfonlayer}
\end{tikzpicture}

    }
    \caption{Angolo $\theta$ in base al suo valore}
    \label{fig:angolotheta}
  \end{figure}\\
  che due vettori individuano due angoli, uno minore o uguale a $\pi$ e un altro maggiore o uguale a $\pi$. Per risolvere questa ambiguità, quando si parla di angolo tracciare due vettori d'ora in poi verrà inteso quello minore o uguale di $\pi$ (il cosiddetto \textit{angolo convesso}).
\end{oss}
\begin{es}
  Considerando ad esempio i vettori $\vec{OP}$ e $\vec{OQ}$ che rispetto a una terna ortonormale $\vec{OP}_1,\vec{OP}_2,\vec{OP}_3$ hanno coordintate rispettivamente (1,0,1) e (1,-1,0) (in base alla definizione di coordinate, sono quindi $\vec{OP}=1\vec{OP}_1+0\vec{OP}_2+1\vec{OP}_3=\vec{OP}_1+\vec{OP}_3$ e $\vec{OQ}=1\vec{OP}_1+(-1)\vec{OP}_2+0\vec{OP}_3=\vec{OP}_1-\vec{OP}_2$). Allora l'angolo tra $\vec{OP}$ e $\vec{OQ}$, in base alla formula (\ref{eq:teopitapplicatinellospazio5}), è dato da
  \begin{equation*}
   \cos \theta=\frac{1\cdot1+0\cdot(-1)+1\cdot 0}{\sqrt{1^2+0^2+1^2}\cdot\sqrt{1^2+(-1)^2+0^2}}=\frac{1}{\sqrt{2}\sqrt{2}}=\frac{1}{2}
  \end{equation*}
  ovvero, dalla trigonometria, $\theta=\frac{\pi}{3}$ (in gradi, $60^o$)\\
  Le formule (\ref{eq:teopitapplicatinellospazio6}) e (\ref{eq:teopitapplicatinellospazio7}) ci forniscono anche un criterioper verificare in coordinate se due vettori sono perpendicolari: infati, l'angolo $\theta$ è $\frac{\pi}{2}$ (ovvero 90 gradi) se e solo se $\cos\theta =0$, il che può essere vero solo se i numeratori della  (\ref{eq:teopitapplicatinellospazio6}) e della (\ref{eq:teopitapplicatinellospazio7}) sono nulli.\\
  Ad esempio, nello spazio, abbiamo che due vettori $\vec{OP}\equiv(x_1,x_2,x_3)$ e $\vec{OQ}\equiv(y_1,y_2,y_3)$ sono perpendicolari se e solo se si verifica
  \begin{equation}
    \label{eq:teopitapplicatinellospazio8}
    x_1y_1+x_2y_2+x_3y_3=0
  \end{equation}
  Ad esempio, i due vettori di coordinate (1,2,1) e (3,1,-5) sono perpendicolari in quanto
  \begin{equation*}
    1\cdot 2+2\cdot 1+1\cdot (-5)=3+2-5=0
  \end{equation*}
\end{es}
\begin{oss}
  \label{oss:teopitapplicatinellospazio2}
  In base al criterio (\ref{eq:teopitapplicatinellospazio8}), il vettore nullo $\vec{OO}$ risulta essere perpendicolare a qualunque altro vettore $\vec{OP}$, in quanto le sue coordinate sono $(0,0,0)$ e, qualunque siano le cordinate $(x_1,x_2,x_3)$ di $\vec{OP}$ si ottiene $x_1\cdot0+x_2\cdot 0+ x_3\cdot 0=0$.\\
  Tuttavia, si noti che le formule (\ref{eq:teopitapplicatinellospazio6}) e (\ref{eq:teopitapplicatinellospazio7}) sono applicabili per calcolrare un angolo solo se nessuno dei vettori è nulla, altrimenti una delle due radici a denominatore verrebbe $\sqrt{0^2+0^2+0^2}=0$, e come è noto non è possibile dividere per zero.\\
  Il numeratore che compare nella (\ref{eq:teopitapplicatinellospazio6}), o nella (\ref{eq:teopitapplicatinellospazio7}) nel caso del piano, può essere interpretato come una nuova operazione, una sorta di prodotto che date due terne (due coppie nel caso del piano) di numero reali, dà come risultato un numero reale: si denota $x=(x_1,x_2,x_3)$ e $y=(y_1,y_2,y_3)$\footnote{nel caso del piano, $x=(x_1,x_2)$ e $y=(y_1,y_2)$}, è possibile porre:
  \begin{equation}
    \label{eq:teopitapplicatinellospazio9}
    x\cdot y:=x_1y_1+x_2y_2+x_3y_3
  \end{equation}
  mentre, nel caso del piano sarà:
  \begin{equation*}
    x\cdot y:=x_1y_1+x_2y_2
  \end{equation*}
  La (\ref{eq:teopitapplicatinellospazio9}) è un esempio di \textit{prodotto scalare}, una nozione che vedremo più in generale in una dei prossimi capitoli (il nome viene dal fatto che si tratta di un'operazione il cui risultato è un numero reale, e come detto sopra nel contesto dei vettori i numeri reali si chiamano anche scalari).\\
  Si noti che anche le formule (\ref{eq:teoremadiPitapplicatoaAbsOP}) e (\ref{eq:teopitapplicatinellospazio}) per calcolare in coordinate della lunghezza di un vettore (rispettivamente nel piano e nello spazio) possono essere espresse in termini del prodotto scalare: infatti, ad esempio per la (\ref{eq:teopitapplicatinellospazio2}) si ha, facendo riferimento alla (\ref{eq:teopitapplicatinellospazio9}),
  \begin{equation*}
    \sqrt{x_1^2+x_2^2+x_3^2}=\sqrt{x_1\cdot x_1+x_2\cdot x_2+x_3\cdot x_3}=\sqrt{x\cdot x}
  \end{equation*}
\end{oss}
Il prodotto scalre rappresenta qundi una sorta di ``strumento di misura'' tramite il quale si esprimono le misure della lunghezza e degli angoli tra vettori quando si lavora in coordinate: quindi, per manipolare espressioni che riquardano lunquezza e angolo e, come verrà fatto in capitoli successivi, ricavare le formule che coinvolgono in qualche modo queste nozioni (ad esempio, riflessioni, proiezioni ortogonali, etc.) è necessario conoscerne le proprietà algebriche.\\
Le proprietà più importanti, limitando al cso di $\mathds{R}^3$ (tali proprietà saranno valide anche nel caso di $\mathds{R}^2$, dove si ricavano nello stesso modo e l'unica differenza è che nelle formule non compare la terza componente)
\begin{enumerate}
\item Il prodotto scalare gode della proprietà commutativa: infatti, si verifica immediatamente che
  \begin{equation*}
    x\cdot y:=x_1y_1+x_2y_2+x_3y_3=y_1x_1+y_2x_2+y_3x_3=y\cdot x
  \end{equation*}
\item Come visto nel pagrafo precedente che se due vettori geometrici $\vec{OP}$ e $\vec{OQ}$ nello spazio hanno coordinate date rispettivamente da due terne $x=(x_1,x_2,x_3)$ e $y=(y_1,y_2,y_3)$, allora la loro somma $\vec{OP}+\vec{OQ}$ ha coordinate dalla terna $(x_1+y_1,x_2+y_2,x_3+y_3)$ che si ottiene sommando le rispettive componenti delle due terne: questo definisce un'operazione di somma tra le $x$ e $y$, che prodotto scalare gode delle proprietà distributiva rispetto rispetto a tale somma, ovvero date tre terne $x=(x_1,x_2,x_3),y=(y_1,y_2,y_3),z=(z_1,z_2,z_3)$ valgono le
  \begin{eqnarray}
    \label{eq:teopitapplicatinellospazio10}
    x\cdot (y+z)=x\cdot y+x\cdot z, & (x+y)\cdot z=x\cdot z+y\cdot z
  \end{eqnarray}
  rispettivamente proprietà distributiva a destra e a sinistra.\\
  Infatti, essendo $y+z=(y_1+z_1,y_2+z_2, y_3+z_3)$, dalla (\ref{eq:teopitapplicatinellospazio9}) si ha
  \begin{equation*}
    x\cdot (y+z)=x_1(y_1+z_1)+x_2(y_2+z_2)+x_3(y_3+z_3)=
  \end{equation*}
  usando per ciascuno dei tre addendi la proprietà distributiva del prodotto di numeri reali rispetto alla somma
  \begin{equation*}
    \begin{matrix}
      =x_1y_1+x_1z_1+x_2y_2+x_2z_2+x_3y_3+x_3z_3=x_1y_1+x_2y_2+x_3y_3+x_1z_1+x_2z_2+x_3z_3=\\
      x\cdot y+x\cdot z
    \end{matrix}
  \end{equation*}
  come si voleva.\\
  Allo stesso modo (omettiamo quindi i dettagli) si verifica che vale anche la proprietà distributiva a sinistra, ovvero la seconda delle (\ref{eq:teopitapplicatinellospazio10}).
\item Visto nel paragrafo precedente che se un vettore geometrico $\vec{OP}$ nello spazio ha coordinate date dalla terna $x=(x_1,x_2,x_3)$, allora il prodotto $c\vec{OP}$ del vettore per un numero $c\in \mathds{R}$ ha coordinate date della terna $x$ per $c$: posto $cx=(cx_1,cx_2,cx_3)$, ci chiediamo come si comporta il prodotto scalare rispetto a questa operazione (che quindi non è nient'altro per un vettore), ovvero cosa se date due terne $x=(x_1,x_2,x_3)$ e $y=(y_1,y_2,y_3)$ e un numero $c\in \mathds{R}$ eseguiamo i prodotti scalare $(cx)\cdot y$ oppure $x\cdot (cx)$. Si verifica facilmente che si ha
  \begin{eqnarray}
    \label{eq:teopitapplicatinellospazio11}
    (cx)\cdot y=c(x\cdot y), & x\cdot (cy)=c(x\cdot y)
  \end{eqnarray}
  Infatti, essendo $cx=(cx_1,cx_2,cx_3)$, per la (\ref{eq:teopitapplicatinellospazio9}) si ha
  \begin{equation*}
    (cx)\cdot y=cx_1y_1+cx_2y_2+cx_3y_3=
  \end{equation*}
  mettendo in evidenza $c$ che compare in tutti e tre gli addendi
  \begin{equation*}
    =c(x_1y_1+x_2y_2+x_3y_3)=c(x\cdot y)
  \end{equation*}
  come volevamo. Allo stesso modo (omettiamo quindi i dettagli) si verifica la seconda delle (\ref{eq:teopitapplicatinellospazio11}).
\end{enumerate}
Vediamo ora che in $\mathds{R}^3$ è possibile introdurre anche un'altra operazione molto utile in geometria ma anche in altre ma anche in altre applicazioni (soprattutto in fisica), il \textit{prodotto vettoriale}, che date due terne di numeri reali dà come risultato non uno scalare (come nel caso del prodotto scalare) ma una nuova terna (che rappresenta in cordinate un nuovo vettore, da cui il nome). La definizione è la seguente: se $x=(x_1,x_2,x_3)$ e $y=(y_1,y_2,y_3)$ allora si pone
\begin{equation}
  \label{eq:teopitapplicatinellospazio12}
  x\wedge y := (x_2y_3-x_3y_2,x_2y_1-x_1y_3,x_1y_2-x_2y_1)
\end{equation}
Ad esempio, se $x=(1,2,3)$ e $y=(2,5,-1)$ si ha
\begin{equation*}
  x\wedge y := (2\cdot (-1)-3\cdot 5, 3\cdot 2-1,1\cdot 5-2\cdot 2) = (-17,7,1)
\end{equation*}
Il motivo di questa particolare definizione è che si vuole che la terna $x \wedge y$ rappresenti (in coordinate rispetto a una base ortonormale) un vettore che è perpendicolare sia al vettore rappresentato da $x$ che a quello rappresentato da $y$.\\
Per verificare che effevamente è così, basta usare il criterio di perpendicolarità visto nella (\ref{eq:teopitapplicatinellospazio8}), cioè moltiplicare le rispettive componenti di $x$ e $x\wedge y$ (la prima con la prima, la seconda con la seconda, la terza con la terza) e sommare:
\begin{eqnarray*}
  x_1(x_2y_3-x_3y_2)+x_2(x_3y_1-x_1y_3)+x_3(x_1y_2-x_2y_1)=\\
  x_1x_2y_3-x_1x_3y_2-x_2x_1y_3+x_3x_1y_2-x_3x_2y_1=0
\end{eqnarray*}
in quanto come si vede facilmente tutti i termini si semplificano.\\
Analogamente, per verificare che anche il vettore di coordiante $y$ è perpendicolare al vettore rappresentato dal prodotto vettoriale $x\wedge y$:
\begin{eqnarray*}
  y_1(x_2y_3-x_3y_2)+y_2(x_3y_1-x_1y_3)+y_3(x_1y_2-x_2y_1)=\\
  y_1x_2y_3-y_1x_3y_2-y_2x_1y_3+y_3x_1y_2-y_3x_2y_1=0
\end{eqnarray*}
come si voleva.\\
Come abbiamo già fatto per il prodotto scalare, vediamo le più importanti proprietà algebriche dell'operazione di prodotto vettoriale: iniziamo con il segnalare subito che esso \textit{non} è commutativo, ma si ha
\begin{equation*}
  x\wedge y=-y\wedge x
\end{equation*}
ovvero quando combiamo l'ordine dei fattori il risultato cambia di segno (ovvero otteniamo una terna con le componenti di segno opposto). Ad esempio, per i due vettori $x=(1,2,3)$ e $y=(2,5,-1)$ per cui sopra abbiamo già calcolato $x\wedge y=(-17,7,1)$, si ha
\begin{equation*}
  y\wedge x=(5\cdot 3+(-1)\cdot 2, -1\cdot 3, 2\cdot 3-5\cdot 1)=(17, -7, -1).
\end{equation*}
Ancora, nella manipolazione di espressioni e formule che coinvolgono il prodotto vettoriale è necessario fare attenzione al fatto che esso \textit{non} è neanche associativo, cioè in generale si ha
\begin{equation*}
  x\wedge (y\wedge z)\neq (x\wedge y) \wedge z
\end{equation*}
Ad esempio, se prendiamo $x=(1,0,0)$ e $y=z=(0,1,0)$, si vede facilmente che $x\wedge y = (0,0,1)$ e $(x\wedge y) \wedge z=(-1,0,0)$, mentre dall'altra si ha $y\wedge z=(0,0,0)$ e $x\wedge (y\wedge z)=(0,0,0)$.\\
Ancora, esattamente come abbiamo fatto nella (\ref{eq:teopitapplicatinellospazio10}) per il prodotto scalare, verifichiamo che anche il prodotto vettoriale gode di proprietà distributiva (\textit{sia a destra che a sinistra}) rispetto alla somma di terne definite $x+y=(x_1+y_1,x_2+y_2x_3+y_3)$, ovvero
\begin{eqnarray}
  \label{eq:teopitapplicatinellospazio13}
  x\wedge (y+z)=x\wedge y +x \wedge z, & (x+y)\wedge z= x\wedge z+y\wedge z
\end{eqnarray}
Ad esempio, verificando la prima (la seconda è analoga): essendo $y+x=(y_1+z_1,y_2+z_2,y_3+z_3)$, per definizione di prodotto vettoriale si ha
\begin{eqnarray*}
  x\wedge (y+z)= [x_2\cdot(y_3+z_3)-x_3\cdot(y_2+z_2),x_3\cdot(y_1+z_1)
  -x_1\cdot (y_3+z_3),x_1\cdot(y_2+z_2)\\-x_2(y_1+z_1)]=
\end{eqnarray*}
Svolgendo i calcoli per ognuna delle tre componenti
\begin{eqnarray*}
  =(x_2y_3+x_2z_3-x_3y_2-x_3z_2,x_3y_!+x_2z_1-x_1y_3-x_1z_3, x_1y_2+x_1z_2-x_2y_1-x_2z_1)=\\
  (x_2y_3-x_3y_2,x_3y_1-x_1y_3,x_1y_2-x_2y_1)+(x_2z_3-x_3z_2,x_3z_1-x_1z_3,x_1z_2-x_2z_1)
\end{eqnarray*}
ovvero proprio $x\wedge y+x\wedge z$, come voluto. Infine, si verifica che vale un'analoga della (\ref{eq:teopitapplicatinellospazio11}) anche per il prodotto vettoriale, ovvero
\begin{eqnarray}
  \label{eq:teopitapplicatinellospazio14}
  x\wedge (cy)=c(x\wedge y), & (cx)\wedge y =c(x\wedge y)
\end{eqnarray}
Ad esempio, si verifica la prima (la seconda è analoga): essendo $cy=(cy_!,cy_2,cy_3)$, per definizione di prodotto vettoriale si ha
\begin{eqnarray*}
  x\wedge (cy)=(x_2(cy_3)-x_3(cy_2),x_3(cy_1)-x_1(cy_3),x_1(cy_2)-x_2(cy_1))=
\end{eqnarray*}
mettendo in evidenza il $c$ in ognuna delle componenti
\begin{eqnarray*}
  =(c(x_2y_3-x_3y_2),c(x_3y_1-x_1y_3),c(x_1y_2-x_2y_1))=
\end{eqnarray*}
ovvero proprio $c(x\wedge y)$, come voluto.\\
È stato detto che il prodotto vettoriale $v \wedge y$ di due terne $x,y\in \mathds{R}^3$ dà le coordinate di un vettore perpendicolare a entrambi i vettori rappresentati da $x$ e da $y$, e che quindi trova sulla retta rappresentata nella figura seguente:
  \begin{figure}[ht!]
    \centering
    \resizebox{5cm}{!}{
      \begin{tikzpicture}
	\begin{pgfonlayer}{nodelayer}
		\node [style=dot] (0) at (0, 0) {};
		\node [style=none] (1) at (0, 2) {};
		\node [style=none] (2) at (0, 3) {};
		\node [style=none] (3) at (0, 4) {};
		\node [style=none] (4) at (0, 5) {};
		\node [style=none] (5) at (0, -1) {};
		\node [style=none] (6) at (0, -4) {};
		\node [style=none] (7) at (0, -5) {};
		\node [style=none] (8) at (0, -7) {};
		\node [style=none] (9) at (0, -8) {};
		\node [style=none] (10) at (-2, 1) {};
		\node [style=none] (11) at (1, -4) {};
		\node [style=none] (12) at (9, -4) {};
		\node [style=none] (13) at (6, 1) {};
		\node [style=none] (14) at (4, 0) {};
		\node [style=none] (15) at (1, -2) {};
		\node [style=none] (16) at (-0.75, 4.5) {$r$};
		\node [style=none] (17) at (5.25, 0) {$\vec{OQ}\equiv y$};
		\node [style=none] (18) at (1.5, -2.25) {$\vec{OP}\equiv x$};
	\end{pgfonlayer}
	\begin{pgfonlayer}{edgelayer}
		\draw [style=Rightarrow] (6.center) to (7.center);
		\draw [style=Rightarrow] (7.center) to (8.center);
		\draw [style=Rightarrow] (8.center) to (9.center);
		\draw [style=Rightarrow] (5.center) to (6.center);
		\draw [style=Rightarrow] (0) to (5.center);
		\draw [style=Rightarrow] (0) to (1.center);
		\draw [style=Rightarrow] (1.center) to (2.center);
		\draw [style=Rightarrow] (2.center) to (3.center);
		\draw [style=Rightarrow] (3.center) to (4.center);
		\draw (10.center) to (11.center);
		\draw (13.center) to (12.center);
		\draw (11.center) to (12.center);
		\draw (10.center) to (13.center);
		\draw [style=Rightarrow] (0) to (14.center);
		\draw [style=Rightarrow] (0) to (15.center);
	\end{pgfonlayer}
\end{tikzpicture}

    }
    \caption{Prodotto vettoriale  $v \wedge y$}
    \label{fig:prodvectvwedgey}
  \end{figure}\\
  Conoscendo quindi la direzione di tale vettore, per determinarlo completamente bisogna trovarne lunghezza e verso.\\
  Per quello che riguarda la lunghezza, per calcolarla basterà utilizzare la formula (\ref{eq:teopitapplicatinellospazio}). In base a tale formula e alla (\ref{eq:teopitapplicatinellospazio12}), si ha
  \begin{eqnarray*}
    \abs{x\wedge y}^2=(x_2y_3-x_3y_2)^2+(x_3y_1-x_1y_3)^2+(x_1y_2-x_2y_1)^2
  \end{eqnarray*}
Svolgendo i conti (omettendo i passaggi), non è difficile vedere che tale espressione è uguale a
  \begin{eqnarray*}
    (x_1^2+x_2^2+x_3^2)(y_1^2+y_2^2+y_3^2)-(x_1y_1+x_2y_2+x_3y_3)^2
  \end{eqnarray*}
  ovvero, ricordando la notazione di prodotto scalare introdotta nella (\ref{eq:teopitapplicatinellospazio9})
  \begin{eqnarray*}
    \abs{x}^2\cdot\abs{y}^2-(x\cdot y)^2
  \end{eqnarray*}
  Riscrivendo questa espressione come\footnote{Supponendo che sia $x$ che $y$ siano diverse dalla terna nulla (0,0,0), altrimenti non potrebbe porre $\abs{x}^2=x_1^2+x_2^2+x_3^2$ o $\abs{y}^2=y_1^2+y_2^2+y_3^2$ a denominatore. Del rsto, se $x$ o $y$ fossero uguali alla terna nulla, il problema di calcolare la lughezza di $x\wedge y$ non si porrebbe perché in quel caso dalla definizione di prodotto vettoriale si vedrebbe subito che anche $x\wedge y$ sarebbe la terna nulla, e quindi la lunghezza del vettore corrispondente sarebbe zero.}
  \begin{eqnarray*}
    \abs{x}^2\cdot\abs{y}^2\left(1-\frac{(x\cdot y)^2}{\abs{x}^2\cdot\abs{y}^2}\right)
  \end{eqnarray*}
  e ricordando che in alla (\ref{eq:teopitapplicatinellospazio6}) si ha $\cos\theta=\frac{x\cdot y}{\abs{x}\cdot\abs{y}}$ (dove $\theta$ è l'angolo formato dai vettori rappresentati da $x$ e $y$) concludendo che
  \begin{eqnarray*}
    \abs{x\wedge y}^2=\abs{x}^2\cdot\abs{y}^2(1-\cos^2\theta)=\abs{x}^2\cdot\abs{y}^2\sin^2\theta
  \end{eqnarray*}
  (dove viene utilizzata l'identità trigonometrica $\cos^2\theta+\sin^2\theta=1$), ovvero, estraendo la radice a entrambi i membri,
  \begin{eqnarray}
    \label{eq:teopitapplicatinellospazio15}
    \abs{x\wedge y}=\abs{x}\cdot\abs{y}\sin\theta
  \end{eqnarray}
  che è finalmente una formula semplice per la lunghezza vettoriale rappresentato in coordinate da $x\wedge y$, in funzione della lunghezza $\abs{x}$ del vettore rappresentato da $x$, della lunghezza $\abs{y}$ del vettore rappresentato da $y$ e dell'angolo $\theta$ formato da questi due vettori\footnote{Si noti che estraendo la radice è stato scritto $\sin\theta$, invece del vettore assoluto $\abs{\sin\theta}$ perché, supponendo che $\theta$ rappresenti l'angolo convesso tra i due vettori (si veda l'Osservazione \ref{oss:teopitapplicatinellospazio}), vale $\theta\in [0,\pi]$ e quindi $\sin \theta\leq 0$.}.\\ale formula dice ad esempio che $\abs{x\wedge y}=0$ (ovvero $x\wedge y=(0,0,0)$ rappresenta il vettore nullo $\vec{OO}$) esattamente quando $\sin\theta=0$ ovvero, come dice la trigonometria, quando $\theta=0$ o $\theta=\pi$ (180 gradi). Come si vede nella figura seguente
  \begin{figure}[ht!]
    \centering
    \resizebox{5cm}{!}{
      \begin{tikzpicture}
	\begin{pgfonlayer}{nodelayer}
		\node [style=none] (0) at (-3, 0) {};
		\node [style=none] (1) at (-2, 1) {};
		\node [style=none] (2) at (1, 4) {};
		\node [style=none] (3) at (2, 0) {};
		\node [style=none] (4) at (4, 2) {};
		\node [style=none] (5) at (6, 4) {};
		\node [style=none] (6) at (3.5, 1.5) {};
		\node [style=none] (7) at (4.5, 2.5) {};
		\node [style=none] (8) at (-1.5, 0.75) {$\theta=0$};
		\node [style=none] (9) at (3, 2.5) {$\theta=\pi$};
	\end{pgfonlayer}
	\begin{pgfonlayer}{edgelayer}
		\draw [style=Rightarrow] (4.center) to (5.center);
		\draw [style=Rightarrow] (4.center) to (3.center);
		\draw [style=Rightarrow] (0.center) to (1.center);
		\draw [style=Rightarrow] (1.center) to (2.center);
		\draw [bend left=75, looseness=1.50] (6.center) to (7.center);
	\end{pgfonlayer}
\end{tikzpicture}

    }
    \caption{Caso in cui $\theta=0$ e il caso in cui l'angolo $\theta = \pi$}
    \label{fig:theta0thetapi}
  \end{figure}\\
  questo equivale a dire che i vettori sono allineati. In altre parole, si deduce che $x\wedge y= (0,0,0)$ solo quando le terne $x$ e $y$ sono una multipla dell'altra (ovvero proporzionali).\\
  Conoscendo direttamente e lunghezza del vettore rappresentato da $x\wedge y$, per il verso sono possibili solo due possibilità:
  \begin{figure}[ht!]
    \centering
    \resizebox{5cm}{!}{
      \begin{tikzpicture}
	\begin{pgfonlayer}{nodelayer}
		\node [style=none] (0) at (-3, 4) {};
		\node [style=none] (1) at (-1.5, 0) {};
		\node [style=none] (2) at (3, 4) {};
		\node [style=none] (3) at (4.5, 0) {};
		\node [style=none] (4) at (-2, 3) {};
		\node [style=none] (5) at (-1.25, 1) {};
		\node [style=none] (6) at (1, 3) {};
		\node [style=none] (7) at (-2, 6) {};
		\node [style=none] (8) at (-2, 0) {};
		\node [style=none] (9) at (2, 3) {$\vec{OQ}\equiv y$};
		\node [style=none] (10) at (-1, 0.75) {$\vec{op}\equiv x$};
	\end{pgfonlayer}
	\begin{pgfonlayer}{edgelayer}
		\draw (0.center) to (1.center);
		\draw (1.center) to (3.center);
		\draw (0.center) to (2.center);
		\draw (2.center) to (3.center);
		\draw [style=Rightarrow] (4.center) to (5.center);
		\draw [style=Rightarrow] (4.center) to (6.center);
		\draw [style=Rightarrow] (4.center) to (7.center);
		\draw [style=Rightarrow] (4.center) to (8.center);
	\end{pgfonlayer}
\end{tikzpicture}

    }
    \caption{Vettore $OQ$ e $OP$ parallele ad $x$ e $y$}
    \label{fig:oqopparassinellospazio}
  \end{figure}\\
  In realtà il verso del vettore rappresentato da $x\wedge y$ non è determinabile in modo univoco, ma dipende da quale base ortonormale è stato scelto per tradurre i vettori in coordinate.
  
\section{Spazi vettoriali}
\label{sec:spazivect}

Come visto, i vettori geometrici sono degli oggetti che possono essere sommati tra loro e moltiplicati per un numero reale, ed è usando queste operazioni e le proprietà (\ref{eq:sommaassociativa})-(\ref{eq:prodottoconduenumerirealiperunnumeroreale}) che esse soddisfano che siano riusciti a introdurre importanti concetti, come quello di coordinate, ricavandone importanti proprietà.\\
Il fatto notevole è che in matematica e nelle sue applicazioni esistono molti altri insiemi, composti da elementi di natura molto diversa dai vettori geometrici, che si comportano tuttavia in modo analogo a questi ultimi, ovvero che possono essere in un certo senso sommati tra loro e moltiplicati per un numero reale, e che soddisfano proprietà analoghe a quelle viste nelle (\ref{eq:sommaassociativa})-(\ref{eq:prodottoconduenumerirealiperunnumeroreale}).\\
Ad esempio, si consideri l'insieme di tutte le funzioni $f:\mathds{R}\to \mathds{R}$: chiaramente, due funzioni possono essere sommate tra loro per ottenere una nuova funzione (es. se $f(x)=x^2$, e $g(x)=e^x$, la funzione che a ogni $x\in \mathds{R}$ associa $x^2+e^x$ costituisce una nuova funzione, che può essere pensata come la somma $f+g$), e una funzione può essere moltiplicata per un numero reale per ottenere una nuova funzione (ad esempio, data $f(x)=x^2$, la funzione che a ogni $x\in \mathds{R}$ associa $2x^2$ può essere pensata come la funzione $2f$).\\
Queste operazioni, come è facile verificare, soddisfano le proprietà analoghe alle (\ref{eq:sommaassociativa})-(\ref{eq:prodottoconduenumerirealiperunnumeroreale}) viste per i vettori geometrici: ad esempio, è chiaro che la somma di funzioni gode della proprietà communitativa (facendo riferimento all'esempio di sopra, $x^2+e^x=e^x+x^2$); o ancora, per quello che riguarda la proprietà (\ref{eq:sommaelementoneutro}), esiste una funzione che funge da elemento neutro per la somma (la funzione costante uguale a zero) e così via per tutte le altre proprità.\\
Un altro esempio di insieme che si comporta in modo analogo ai vettori geometrici, che è di fondamentale importanza in matematica e come si vedrà in particolare in questo corso, è quello dell'\textit{insieme delle n-uple di numeri reali.}\\
Dato un numero naturale positivo $n$, una $n$-uple $(x_1,x_2,\dots,x_n)$ è una sequenza ordinata di $n$ numero reali $x_1,x_2,x_n\in\mathds{R}$: ad esempio, per $n=2$ e $n=3$ si ottengono rispettivamente le coppie $(x_1,x_2)$ e le terne $(x_1,x_2,x_3)$, che abbiamo già introdotto parlando di coordinate di vettori geometrici già introdotto parlando di coordinate di vettori geometrici nel piano o nello spazio tridimensionale. Lungi dal rappresentare una generalizzazione astratta delle coppie o delle terne senza più significato concreto o utilità, le $n$-uple possono modellizzare oggetti e situazioni ``reali'' le più diverse tra loro: ad esempio, in fisica ogni evento dello spaziotempo è rappresentato da una 4-upla $(x_1,x_2,x_3,x_4)$, dove le prime tra componenti $x_1,x_2,x_3$ sono le coordinate del punto in cui avviene l'evento e l'ultima componete $x_4$ ci dice in quale istante di tempo esso avviene; ancora, la configurazione di un braccio meccanico con $n$ giunture può essere rappresentata da un $n$-upla in cui ogni componente ci dice l'angolo che formano i bracci nella giuntura corrispondente; oppure, se si avesse un mercato composto da 10 merci, la situazione dei prezzi in quel mercato può essere rappresentata da una 10-upla $(x_1,x_2,\dots,x_{10})$ in cui ciascuna componente indica il prezzo della merce corrispondente ($x_1$ della prima merce, $x_2$ della seconda, e così via).\\
Ora, sull'insieme delle $n$-uple di numeri reali, che si denota $\mathds{R}^n$, si può definire un'operazione di somma tra due $n$-uple $(x_1,x_2,\dots, x_n)$, $(y_1,y_2,\dots, y_n)$ sommando componente per componente
\begin{eqnarray}
  \label{eq:spaziovect1}
  (x_1,x_2,\dots, x_n)+(y_1,y_2,\dots, y_n):=(x_1+y_1,x_2,\dots,x_n+y_n)
\end{eqnarray}
e un'operazione di prodotto di un numero reale $c\in \mathds{R}$ per una $n$-upla $(x_1,x_2,\dots, x_n)$ moltiplicando per $c$ tutte le componenti della $n$-upla:
\begin{eqnarray}
  \label{eq:spaziovect2}
  c(x_1,x_2,\dots, x_n):=(cx_1,cx_2,\dots, cx_n)
\end{eqnarray}
nel caso delle coppie o delle terne, queste due operazioni sono proprio quelle che traducono in coordinate, come affermano la Proposizione \ref{prop:coordinate1} e l'Osservazione \ref{oss:coordinate2}, la somma e il prodotto per uno scalare di vettori geometrici.\\
Come è facile vedere, queste due operazioni verificano proprietà analoghe alle proprietà (\ref{eq:sommaassociativa})-(\ref{eq:prodottoconduenumerirealiperunnumeroreale}) che hanno somma e prodotto per un numero reale dei vettori geometrici. Ad esempio, sempre in riferimento alla proprietà (\ref{eq:sommaelementoneutro}), la n-upla che funge da elemento neutro per la somma è la $n$-upla $(0,0,\dots, 0)$ che ha tutte le componenti nulle, in quanto chiaramente
\begin{eqnarray}
  \label{eq:spaziovect3}
  (x_!,x_2,\dots, x_n)+(0,0,\dots,0)=(x_1+0,x_2+0,\dots,x_n+0)=(x_1,x_2,\dots,x_n).
\end{eqnarray}
In altre parole, la $n$-upla $(0,0,\dots,0)$ ha in $\mathds{R}^n$ lo stesso ruolo che il vettore $\vec{OO}$ ha nell'insieme dei vettori applicati o la funzione costante uguale a zero nell'insieme delle funzioni.\\
Queste analogie suggeriscono che si può dare una definizione generale, astratta, che comprenda come casi particolari gli esempi appena visti. Il vantaggio di tale impostazione è che può esser studiato una volta per tutte le proprietà di questi insiemi senza doverle vedere nei singoli casi: un teorema dimostrato in generale nel caso astratto risulta poi vero per tutti gli esempi di questo tipo di struttura.
\begin{defi}
  \label{def:spaziovect1}
  Uno \textit{spazio vettoriale reale (o $\mathds{R}-spazio$ vettoriale)} è un insieme su cui siano definite un'operazione di somma tra gli elementi di $V$ e un'operazione di prodotto tra numeri reali e elementi di $V$ in modo che siano soddisfatte le seguenti proprità:
  \begin{enumerate}
  \item La somma è $associativa$, cioè per ogni $v_1,v_2,v_3\in V$ si ha
    \begin{equation*}
      (v_1+v_2)+v_3=v_1+(v_2+v_3)
    \end{equation*}
  \item La somma e \textit{commutativa}, cioè per ogni $v_1,v_2\in V$ si ha
    \begin{equation*}
      v_1+v_2=v_2+v_1
    \end{equation*}
  \item Esiste un elemento di $V$, denotato $\bar{0}$ e chiamato \textit{vettore nullo}, tale che
    \begin{equation*}
      v+\bar{0}=\bar{0}+v=v
    \end{equation*}
    (ovvero $\bar{0}$ è l'elemento neutro per la somma data su $V$)
  \item Per ogni $v\in V$, l'elemento $(-1)v$ è il suo \textit{opposto rispetto alla somma} o inverso additivo:
    \begin{equation*}
      v+(-1)v= (-1)v+v=\bar{0}
    \end{equation*}
  \item Per ogni $c_1,c_2\in \mathds{R}$ e $v\in V$, vale
    \begin{equation*}
      c_1(c_2v)=(c_1c_2)v
    \end{equation*}
  \item Per ogni $c_1,c_2\in \mathds{R}$ e $v\in V$, vale
    \begin{equation*}
      c_1(c_2v)=(c_1c_2)v
    \end{equation*}
  \item Per ogni $c\in\mathds{R}$ e ogni $v_1,v_2\in V$, vale
    \begin{equation*}
      c(v_1+v_2)=cv_1+cv_2
    \end{equation*}
  \item Per ogni $v\in V$, si ha
    \begin{equation*}
      1v=v
    \end{equation*}
  \end{enumerate}
  Gli elementi di uno spazio vettoriale $V$ si chiamano \textit{vettori}; per contrapposizione,in questo contesto i numeri reali si chiamano anche \textit{scalari}.\\
  Un vettore $cv$ ottenuto moltiplicando $v$ per uno scalare $c$ si dice \textit{proporzionale a $v$} o multipo di $v$. Quindi sono spazi vettoriali reali gli insiemi $V_O^2$ e $V_o^3$ dei vettori geometrici rispettivamente limitati nel piano o liberi di variare in tutto lo spazio tridimensionale, l'insieme $\mathds{R}^n$ delle $n$-uple di numeri reali, e l'insieme di tutte le funzioni reali di variabile reale.
\end{defi}
\begin{oss}
  \label{oss:spaziovect1}
  Come già visto nel caso particolare dei vettori, grazie alla proprità (1)-(8) di sopra è possibile manipolare le espressioni contenenti vettori nel modo in cui manipolare solitamente le espressioni algebriche tra numeri, e in particolare ad esempio in uno spazio vettoriale si possono ``spostare i vettori'' da un membro all'altro di un'ugualianza cambiandoli di segno. Più precisamente, da un'espressione del tipo $v_1+v_2=v_3$ si può passare a $v_1=v_3-v_2$ nel modo seguente:\\
  sommando a entrambi i membri di $v_1+v_2=v_3$ in vettore $(-1)v_2$:
  \begin{equation*}
    (v_1+v_2)+(-1)v_2=v_3+(-1)v_2
  \end{equation*}
  Applicando la proprità associativa della somma (la 1 della Definizione \ref{def:spaziovect1}) a primo memebro:
  \begin{equation*}
    v_1+(v_2+(-1)v_2)=v_3+(-1)v_2
  \end{equation*}
  Applicando la proprità 4 che afferma che $(-1)v_2$ è l'opposto di $v_2$:
  \begin{equation*}
    v_1+\bar{0}=v_3+(-1)v_2
  \end{equation*}
  e infine applicando la 3 che definisce che il vettore nullo funge da elemento neutro:
  \begin{equation*}
    v_1=v_3+(-1)v_2.
  \end{equation*}
\end{oss}
Per un altro esempio di proprietà vera nel caso dei vettori geometrici e che in realtà vale in qualunque spazio vettoriale, consideriamo la $0\vec{OP}=\vec{OO}$, che discendeva dalla definizione stessa di prodotto di un numero reale per un vettore geometrico: infatti, dal momento che in generale $c\vec{OP}$ denota un vettore avente lunghezza uguale a $\abs{c}$ volte la lunghezza di $\vec{OP}$, questo implica che $0\vec{OP}$ abbia lunghezza zero, e quindi sia il vettore geometrico $\vec{OO}$ ``schiacciato'' sul punto $O$.\\
Ebbene, bisogna motrare che in realtà l'uguaglianza analoga $0v=\bar{0}$ vale per ogni vettore $v$ di un qualunque spazio vettoriale $V$: infatti, si ha
\begin{equation*}
  0v=(1+(-1))v=1v+(-1)v=v+(-1)v=\bar{0}
\end{equation*}
dove nella seconda uguaglianza è stato sfruttata la proprietà 6 della Definizione \ref{def:spaziovect1}, nella terza ugualianza invece è stata la proprietà 8 e nell'ultima uguaglianza la proprietà 4.\\
QUindi, il fatto che moltiplicando per 0 un vettore si ottenga il vettore nullo si rivela essere una proprietà che non dipende da come si definisce il prodotto nello specifico caso ma semplicemente dalle proprietà algebriche della definizione generale di spazio vettoriale.
\begin{oss}
  \label{oss:spaziovect2}
  Nella definizione di spazio vettoriale data sopra è stato supposto che gli elementi dello spazio $V$ possono essere moltiplicati per numeri reali, e per questo motivo si è parlato in termini di spazio vettoriale \textit{reale}.\\
  Analogamente, esistono gli spazi vettoriali \textit{complessi}, per i quali la definizione è identica a quella data nella Definizione \ref{def:spaziovect1} con l'unica differenza che i vettori possono essere moltiplicati per numeri complessi anziché reali.\\
  Ricordando che in numero complesso è un'espressione del tipo $a+bi$, essendo $a,b$ numeri reali e $i$ un nuovo numero, detto \textit{unità immaginaria}, con la proprietà (non soddisfatta da nussun numero reale) che $i^2=-1$.\\
  Ad esempio, $2+3i,\pi\sqrt{2}i$ sono numeri complessi; dato un numero complesso $z=a+bi$, il numero reale $a$ si dice \textit{parte reale} di $z$, mentre il numero reale $b$ (essendo il coefficente davanti all'unità immaginaria) si dice \textit{parte immaginaria} di $z$. La parte immaginaria $b$ può anche uguale a zero: in tale caso il numero complesso $a+bi$ coincide con il numero reale $a$ (quindi ogni numero reale può essere pensato come un particolare numero complesso con parte immaginaria nulla). L'insieme dei numeri complessi si denota $\mathds{C}$.\\
  I numeri complessi possono essere sommati semplicemente sommando le rispettive parti reali e immaginarie. Ad esempio
  \begin{equation*}
    (2+3i)+(4+5i)=(2+4)+(3+5)i=6+8i.
  \end{equation*}
  Per moltiplicare due numeri complessi, basta prima eseguire il prodotto come se si trattasse di un'espressione algebrica letterale in cui la $i$ è una indeterminata
  \begin{equation*}
    (2+3i)\cdot (4+5i)=2\cdot 4+2\cdot 5i+3i\cdot 4+3i\cdot 5i= 8+10i+12i+15i^2
  \end{equation*}
  e poi semplificarla ricordando che $i^2=-1$ e sommando i termini simili:
  \begin{equation*}
    8+10i+12i+15i^2=8+22i-15=-7+22i.
  \end{equation*}
  Non è difficile vedere che le operazioni di somma e prodotto così definite verificano le usuali propritàverificate da somma e prodotto di numeri reali: proprietà associativa e commutativa, esistenza di elemeni neutri (il numero 0, con la proprietà che $z+0=0+z=z$ per ogni $z\in \mathds{C}$, e il numero 1, con la proprietà che $z1=1z=z$ per ogni $z\in \mathds{C}$), proprietà distributiva.\\
  Inoltre, esattamente come succede per i numeri reali, ogni numero complesso $a+bi$ diverso da zero (ovvero per cui $a$ e $b$ non sono entrambi nulli) ammette un inverso moltiplicativo, ovvero un numero complesso che moltiplicato per $a+bi$ dà come risultato 1. Più precisamente, l'inverso di $a+bi$ e il numero complesso
  \begin{equation*}
    \frac{a}{a^2+b^2}-\frac{b}{a^2+b^2}i
  \end{equation*}
  Per verificare tale affermazione, è sufficiente moltiplicare tra loro $a+bi$ e $\frac{a}{a^2+b^2}-\frac{b}{a^2+b^2}i$ e verificare che il risultato sia uguale a 1. Riscrivendo $\frac{a}{a^2+b^2}-\frac{b}{a^2+b^2}i$ come $\frac{a+bi}{a^2+b^2}$ si vede subito che
  \begin{equation*}
    a+bi\cdot \frac{a+bi}{a^2+b^2}= \frac{(a+bi)(a-bi)}{a^2+b^2}=\frac{a^2-b^2i^2}{a^2+b^2}=\frac{a^2+b^2}{a^2+b^2}=1
  \end{equation*}
  (nella seconda uguaglianza è stata utilizzata l'identità notevole $(X+Y)(X-Y)=X^2-Y^2$, mentre nella terza il fatto che $i^2=-1$).\\
  Ad esempio, se $a+bi=2+3i$, ovvero $a=2,b=3$, si ha $a^2+b^2=2^2+3^2=4+9=13$ e quindi
  \begin{equation*}
    \frac{a}{a^2+b^2}-\frac{b}{a^2+b^2}i=\frac{2}{13}-\frac{3}{13}i
  \end{equation*}
  \clearpage
  è l'inverso di $2+3i$.\\
  Riassumendo, i numeri complessi hanno quindi in comune con i numeri reali le sequenti proprietà:
  \begin{enumerate}
  \item La somma e il prodotto godono entrambi delle proprietà associativa e communitativa
  \item Esiste un elemento neutro per le somme e un elemento neutro per il prodotto
  \item ogni elemento $a$ ammette un inverso additivo $-a$, tale che $a+(-a)=(-a)+a=0$
  \item ogni numero a \textit{diverso da} 0 ammette un inverso moltiplicativo $a^{-1}$, tale che $aa^{-1}=a^{-1}a=1$
  \item vale la proprietà distributiva
  \end{enumerate}
\end{oss}
Un qualunque insieme numerico le cui operazioni di somma e prodotto godano di queste proprietà si dice un \textit{campo numerico} (o semplicemente campo). Solitamente un campo si denota come la lettera $\mathds{K}$. Dal momento che per la maggior parte della nastra trattazione degli spazi vettoriali, che siano reali o complessi, si utilizzerà solo il fatto che $\mathds{R}$ e $\mathds{C}$ sono campi, ovvero hanno le proprietà dette, non c'è nessun motivo nelle dimostrazioni che verranno portate di distinguere tra il caso complesso e quello reale: si può tranquillamente parlare di \textit{spazio vettoriale definito su un campo $\mathds{K}$} e dimostrare le formule supponendo che gli scalari appartengano a $\mathds{K}$, che potrebbe essere  $\mathds{R}$ o $\mathds{C}$ senza che questo modifichi nulla rispetto alle dimostrazioni stesse.\\
L'esempio più importante di spazio vettoriale complesso è l'insieme  $\mathds{C}$ di tutte le $n$-uple\\ $(z_1,z_2,\dots,z_n)$ di numeri complessi $z_1,z_2,\dots,z_n\in \mathds{C}$, sul quale le operazioni di somma di $n$-uple e prodotto di una $n$-upla per uno scalare sono definite esattamente come in $\mathds{R}$
\begin{equation}
  \label{eq:spaziovect3}
  (z_1,z_2,\dots,z_n)+(w_1,w_2,\dots,w_n):=(z_1+w_1,z_2+w_2,\dots,z_n+w_n)
\end{equation}
\begin{equation}
  \label{eq:spaziovect4}
  c(z_1,z_2,\dots,z_n):=(cz_1,cz_2,\dots,cz_n)
\end{equation}
con l'unica differenza che ora lo scalare $c$ appartiene al campo dei numeri complessi.\\\\
Ora, verra mostrato come alcune delle più importanti nozioni viste per i vettori geometrici, e in particolare quella di coordinate, possono essere date in qualunque spazio vettoriale.\\
Ricordando che nello spazio $V^2_O$ dei vettori applicati nel piano, il punto di partenza della definizione di coordinate consiste nel mostrare che, fissati due vettori $\vec{OP}_1$ e $\vec{OP}_2$ non allineati, qualunque vettore $\vec{OP}\in V_O^2$ si può scrivere come loro combinazione $\vec{OP}=x_1\vec{OP}_1+x_2\vec{OP}_2$. Analogamente, nello spazio tridimensionale, per poter definire le coordinate si mostra che, fissati tre vettori $\vec{OP}_1$, $\vec{OP}_2$ e $\vec{OP}_3$ non appartenenti a uno stesso piano, qualunque vettore $\vec{OP}\in V_O^3$ può essere scritto come $\vec{OP}=x_1\vec{OP}_1+x_2\vec{OP}_2+v_3\vec{OP}_3$.\\
A parte il diverso numero di vettori che serve per ottenere le coordinate in $V_O^2$ e in $V_O^3$, in entrambi i casi il punto di partenza è la possibilità di ottenere qualunque vettore dello spazio combinando un numero finito di vettori dati.\\
Questo suggerisce la sequente definizione per un generico spazio vettoriale (definito su un qualunque campo $\mathds{K}$):
\begin{defi}
  \label{defi:spaziovect2}
  Sia $V$ un $\mathds{K}-$spazio vettoriale. Dei vettori $v_1,v_2,\dots,v_n\in V$ si dicono \textit{generatori di $V$} se ogni vettore $v\in V$ si può scrivere come
  \begin{equation*}
    v=x_1v_1+x_2v_2+\dots+x_nv_n
  \end{equation*}
  per certi coefficienti $x_1,x_2,\dots,x_n\in\mathds{K}$.\\
  Un'espressione del tipo $x_1v_1+x_2v_2+\dots+x_nv_n$ si dice \textit{combinazione lineare dei vettori $v_1,v_2,\dots,v_n$} sono generatori di $V$ se ongi vettore dello spazio si può scrivere come loro combinazione lineare. Si dice anche che $v_1,v_2,\dots,v_n$ \textit{generano} $V$. Quindi, nello spazio $V_O^2$ due vettori non allineati danno un insieme di generatori; nello spazio $V_O^3$ un insieme di generatori è invece dato da tre vettori che non stiano sullo stesso piano.
\end{defi}
La Definizione \ref{defi:spaziovect2} potrebbe far pansare che a questo punto si possano definire le coordinate di un vettore $v$ in uno spazio vettoriale $V$ rispetto a un insieme di generatori fissato $v_1,\dots, v_n$ semplicemente come i coefficienti $x_1,x_2,\dots,x_n$ che appaiono nella combinazione lineare $v=x_1v_1+x_2v_2+\dots+x_nv_n$, ovvero ricalcando la definizione \ref{def:vettorigeo} e \ref{defi:ternadinumerireali} date negli spazi $V_O^2$ e $V_O^3$.\\
In realtà, questo non è possibile in quanto sussiste un problema di unicità:\\
La Definizione \ref{defi:spaziovect2}, da sola, non garantisce che i coefficienti $x_1,x_2,\dots, x_n$ che servono per decomporre un vettore dato $v$ come $v=x_1v_1+x_2v_2+\dots +x_nv_n$ di $v_1,v_2,\dots, v_n$ siano univocamente determinati. In generale, infatti, un vettore può essere scritto in più modi diversi come combinazione di vettori dati: ad esempio, nello spazio $V=\mathds{R}^2$ delle coppie di numeri reali, consideriamo i vettori
\begin{eqnarray*}
  v_1=(1,0), & v_2=(0,1), & v_3=(1,1)
\end{eqnarray*}
e il vettore $v=(3,2)$.\\
Ad esempio, si hanno le seguenti, differenti decomposizioni di $v$ come combinazione lineare di $v_1,v_2,v_3$:
\begin{eqnarray*}
  (3,2)=2(1,0)+1(0,1)+1(1,1)\\
  (3,2)=4(1,0)+3(0,1)+(-1)\cdot(1,1)
\end{eqnarray*}
Il motivo per cui questo problema non si è verificato quando abbiamo definito le coordinate negli spazi $V_O^2$ e $V_O^3$ usando rispettivamente una coppia di vettori non allineati o a una terna di vettori non complanari, è che tali insiemi di generatori hanno una proprietà aggiuntiva rispetto alla Definizione \ref{defi:spaziovect2}: si tratta di \textit{insiemi di generatori minimali}, in un senso che ora bisogna precisare ed illugtrare.\\
Come si vedere nel disegno seguente, se dall'insieme di generatori di $V_O^2$ costituito da una coppia $\vec{OP}_1,\vec{OP}_2$ di vettori non allineati eliminando uno qualunque dei due vettori, il vettore rimanente non genera più lo spazio, in quanto con esso riusciamo a ottenere (prendendo i suoi multipli) sonlo i vettori che stanno sulla retta a cui esso appartiene. 
\begin{figure}[ht!]
  \centering
  \resizebox{5cm}{!}{
    \begin{tikzpicture}
	\begin{pgfonlayer}{nodelayer}
		\node [style=none] (0) at (-5, 0) {};
		\node [style=none] (1) at (-4, 2) {};
		\node [style=none] (2) at (-3.5, 3) {};
		\node [style=none] (3) at (-3, 0) {};
		\node [style=none] (4) at (-2, 0) {};
		\node [style=none] (5) at (0, 3) {};
		\node [style=none] (6) at (1, 0) {};
		\node [style=none] (7) at (2.5, 3) {};
		\node [style=none] (8) at (3, 4) {};
		\node [style=none] (9) at (3.5, 5) {};
		\node [style=none] (10) at (0, -2) {};
		\node [style=none] (12) at (-1, -4) {};
		\node [style=none] (13) at (2, 3) {$P^2$};
		\node [style=none] (14) at (0.25, 0) {$O$};
		\node [style=none] (15) at (-4.75, 2) {$P_2$};
		\node [style=none] (16) at (-3, -0.25) {$P_1$};
		\node [style=none] (17) at (-5.5, -0.25) {$O$};
		\node [style=none] (18) at (5, 0) {};
		\node [style=none] (19) at (6, 0) {};
		\node [style=none] (20) at (8, 0) {};
		\node [style=none] (21) at (10, 0) {};
		\node [style=none] (22) at (11, 0) {};
		\node [style=none] (23) at (12, 0) {};
		\node [style=none] (24) at (4.25, 0) {};
		\node [style=none] (25) at (12.75, 0) {};
		\node [style=none] (26) at (8, -0.5) {$O$};
		\node [style=none] (27) at (10, -0.5) {$P_1$};
	\end{pgfonlayer}
	\begin{pgfonlayer}{edgelayer}
		\draw [style=Rightarrow] (0.center) to (1.center);
		\draw [style=Rightarrow] (0.center) to (3.center);
		\draw (1.center) to (2.center);
		\draw (3.center) to (4.center);
		\draw [style=campitura] (4.center) to (5.center);
		\draw [style=campitura] (2.center) to (5.center);
		\draw [style=Rightarrow] (0.center) to (5.center);
		\draw [style=Rightarrow] (6.center) to (10.center);
		\draw [style=Rightarrow] (10.center) to (12.center);
		\draw [style=Rightarrow] (6.center) to (7.center);
		\draw [style=Rightarrow] (7.center) to (8.center);
		\draw [style=Rightarrow] (8.center) to (9.center);
		\draw [style=Rightarrow] (20.center) to (21.center);
		\draw [style=Rightarrow] (20.center) to (19.center);
		\draw [style=Rightarrow] (19.center) to (18.center);
		\draw [style=Rightarrow] (21.center) to (22.center);
		\draw [style=Rightarrow] (22.center) to (23.center);
		\draw [style=campitura] (24.center) to (18.center);
		\draw [style=campitura] (23.center) to (25.center);
	\end{pgfonlayer}
\end{tikzpicture}

  }
  \caption{Esempi di generatori di $V_O^2$}
  \label{fig:generatoriInVO2}
\end{figure}\\
Analogamente, nello spazio $V_O^3$ dei vettori geometrici liberi di variare in tutto lo spazio tridimensionale, se dall'insieme di generatori costituito da una terna $\vec{OP}_1,\vec{OP}_2,\vec{OP}_3$ di vettori non complanari eliminiamo anche un solo vettore, i due vettori rimanenti non generano più lo spazio: ad esempio, eliminando $\vec{OP}_3$, le combinazioni lineari dei due vettori restanti $\vec{OP}_1$ e $\vec{OP}_2$ ci danno solo i vettori $\vec{OP}$ che stanno sul piano $p$ del disegno
\begin{figure}[ht!]
  \centering
  \resizebox{5cm}{!}{
    \begin{tikzpicture}
	\begin{pgfonlayer}{nodelayer}
		\node [style=none] (0) at (0, 0) {};
		\node [style=none] (1) at (1, 4) {};
		\node [style=none] (2) at (1.25, 1) {};
		\node [style=none] (3) at (1.75, 3) {};
		\node [style=none] (4) at (4.25, 1) {};
		\node [style=none] (5) at (4.75, 3) {};
		\node [style=none] (6) at (8, 4) {};
		\node [style=none] (7) at (7, 0) {};
		\node [style=none] (8) at (2, 9) {};
		\node [style=none] (9) at (5.5, 11) {};
		\node [style=none] (10) at (10, 4) {};
		\node [style=none] (11) at (9, 0) {};
		\node [style=none] (12) at (16, 4) {};
		\node [style=none] (13) at (15, 0) {};
		\node [style=none] (14) at (11, 3) {};
		\node [style=none] (15) at (10.5, 1) {};
		\node [style=none] (16) at (14.5, 3) {};
		\node [style=none] (17) at (14, 1) {};
		\node [style=none] (18) at (12.75, 1) {};
		\node [style=none] (19) at (10.75, 2) {};
		\node [style=none] (20) at (10.25, 2.25) {$P_2$};
		\node [style=none] (21) at (12.5, 0.5) {$P_1$};
		\node [style=none] (22) at (10.25, 0.75) {$O$};
		\node [style=none] (23) at (14.75, 3.25) {$P$};
		\node [style=none] (24) at (15.5, 3.75) {$p$};
		\node [style=none] (25) at (7.5, 3.75) {$p$};
		\node [style=none] (26) at (1.75, 3.5) {$P_2$};
		\node [style=none] (27) at (4, 0.5) {$P_1$};
		\node [style=none] (28) at (0.75, 0.75) {$O$};
		\node [style=none] (29) at (6, 11) {$P$};
	\end{pgfonlayer}
	\begin{pgfonlayer}{edgelayer}
		\draw (0.center) to (1.center);
		\draw [style=Rightarrow] (2.center) to (3.center);
		\draw [style=Rightarrow] (2.center) to (4.center);
		\draw [style=campitura] (4.center) to (5.center);
		\draw [style=campitura] (3.center) to (5.center);
		\draw [style=campitura] (2.center) to (5.center);
		\draw [in=180, out=0] (1.center) to (6.center);
		\draw (0.center) to (7.center);
		\draw (7.center) to (6.center);
		\draw [style=Rightarrow] (2.center) to (8.center);
		\draw [style=campitura] (5.center) to (9.center);
		\draw [style=campitura] (8.center) to (9.center);
		\draw [style=Rightarrow] (2.center) to (9.center);
		\draw [style=Rightarrow] (15.center) to (19.center);
		\draw [style=Rightarrow] (15.center) to (18.center);
		\draw (12.center) to (13.center);
		\draw (11.center) to (13.center);
		\draw (11.center) to (10.center);
		\draw (10.center) to (12.center);
		\draw [style=campitura] (19.center) to (14.center);
		\draw [style=campitura] (14.center) to (16.center);
		\draw [style=campitura] (16.center) to (17.center);
		\draw [style=campitura] (17.center) to (18.center);
		\draw [style=Rightarrow] (15.center) to (16.center);
	\end{pgfonlayer}
\end{tikzpicture}

  }
  \caption{Esempi di generatori di $V_O^2$ e in $V_O^3$}
  \label{fig:generatoriInVO2VO3}
\end{figure}\\
È in questo senso che tali sistemi di generatori sono minimali: in essi, nessun vettore è superfluo, nessuno può esser eliminato senza perdere la proprietà generale tra poco che è esattamente questa proprietà che garantisce l'unicità dei coefficienti della combinazione lineare in cui si decompone un vettore dato (e che ci consente quindi di definire in modo univoco le coordinate), ma prima è necessario chiarere meglio il concetto di minimalità di un insieme generatori non minimale possono essere eliominati e quali no. Per questo è d'aiuto un esempio considerando nello spazio $V_O^3$ quattro vettori come nella figura seguente
\clearpage
\begin{figure}[ht!]
  \centering
  \resizebox{5cm}{!}{
    \begin{tikzpicture}
	\begin{pgfonlayer}{nodelayer}
		\node [style=none] (0) at (0, 0) {};
		\node [style=none] (1) at (1, 4) {};
		\node [style=none] (2) at (1.25, 1) {};
		\node [style=none] (3) at (1.75, 3) {};
		\node [style=none] (4) at (4.25, 1) {};
		\node [style=none] (5) at (3.25, 2.25) {};
		\node [style=none] (6) at (8, 4) {};
		\node [style=none] (7) at (7, 0) {};
		\node [style=none] (8) at (2, 9) {};
		\node [style=none] (26) at (4, 2.5) {$P_2$};
		\node [style=none] (27) at (4, 0.5) {$P_1$};
		\node [style=none] (28) at (0.75, 0.75) {$O$};
		\node [style=none] (29) at (2, 3.5) {$P_3$};
		\node [style=none] (30) at (2.5, 9) {$P_4$};
	\end{pgfonlayer}
	\begin{pgfonlayer}{edgelayer}
		\draw (0.center) to (1.center);
		\draw [style=Rightarrow] (2.center) to (3.center);
		\draw [style=Rightarrow] (2.center) to (4.center);
		\draw [style=Rightarrow] (2.center) to (5.center);
		\draw [in=180, out=0] (1.center) to (6.center);
		\draw (0.center) to (7.center);
		\draw (7.center) to (6.center);
		\draw [style=Rightarrow] (2.center) to (8.center);
	\end{pgfonlayer}
\end{tikzpicture}

  }
  \caption{Esempio di generatori non minimale in $V_O^3$}
  \label{fig:generatoriNonMinInVO3}
\end{figure}
in cui $OP_1,OP_2,OP_3$ appartengono a uno stesso piano, e $OP_4$ si trova invece fuori da questo piano.\\
Da una parte, si può vedere che un qualunque vettore di $V_O^3$ può essere scritto come combinazione di questi quattro vettori, che costituiscono quindi un insieme di generatori in $V_O^3$, dall'altra, non si trata di un insieme di generatori minimali nel senso spiegato sopra, in quanto alcuni vettori possono essere eliminati e i restanti continuano a generare lo spazio: ad esempio, è possibile eliminare $OP_1$ e gli altri tre continueranno ad esistere e a generare spazio; lo stesso accade con $OP_2$ e $OP_3$ mentre, se viene eliminato $OP_4$ i vettori restanti $OP_1,OP_2,OP_3$ non saranno più un insieme di generatori (trovandosi tutti su uno stesso piano, le kiri combinazioni darebbero solamente vettori che appartengono ancola a questo piano e non tutti quelli dello spazio).\\
Quindi, se un insieme di generatori non è minimale, è importante capire quali vettori possono essere effettivamente eliminati da esso. Il seguente risultato risponde proprio a questa domanda.
\begin{prop}
  \label{prop:spaziovect1}
  Siano $v_1,\dots,v_n\in V$ generatori dello spazio vettoriale V.\\
  L'insieme $\{v_1,\dots,v_{n-1}, v_{i+1},\dots,v_n$ è ancora un insieme di generatori se e solo se $v_i$ si può scrivere come combinazione dei rimanenti.
\end{prop}
\begin{proof}
  \label{proof:spaziovect1}
  è necessario dimostrare due implicazioni:
  \begin{enumerate}
  \item Se $v_i$ è combinazione di $v_1,\dots,v_{n-1}, v_{i+1},\dots,v_n$ allora bastano $v_1,\dots,v_{n-1}, v_{i+1},\dots,v_n$ per generare $V$.
  \item Se $v_1,\dots,v_{n-1}, v_{i+1},\dots,v_n$ sono generatori di $V$ allora $V_i$ si scrive come loro combinazione lineare.
  \end{enumerate}
  Bisona osservare che la seconda implicazione è ovvia, in quanto dire che $v_1,\dots,v_{n-1}, v_{i+1},\dots,v_n$ sono generatori di $V$ significa che ogni vettori di $V$ si scrive come loro combinazione lineare, e questo sarà in particolare vero per $v_i$.\\
  Per dimostrare invece la prima implicazione, bisogna supporre che $v_i$ si scriva come combinazione degli altri vettori di $\{v_1,\dots,v_n\}$, ovvero che esistano dei coefficienti $a_1,\dots,a_{i-1},a_{i+1},\dots,a_n\in \mathds{K}$ tali che
  \begin{equation}
    \label{eq:proofspaziovect1}
    v_1=a_1v_1+\dots+a_{i-1}v_{i-1}+a_{i+1}v_{i+1} +\dots+a_nv_n
  \end{equation}
  e per cercare di dimostrare che $v_1,\dots,v_{n-1}, v_{i+1},\dots,v_n$ generano $V$, ovvere ogni $v\in V$ si scrive come loro cambinazione lineare. Sapendo che tutti i vetotri $v_1,\dots,v_n$ (compreso $v_i$) generano $V$, ovvero che ogni vettore $v$ dello spazio $V$ si scrive come loro combinazione lineare:
  \begin{equation}
    \label{eq:proofspaziovect2}
    v=c_1v_1+\cdots+c_{i-1}v_{i-1}+c_iv_i+\cdot+c_nv_n
  \end{equation}
  Ma allora, sostituendo la (\ref{eq:proofspaziovect1}) nella (\ref{eq:proofspaziovect2}), si ottiene
  \begin{equation}
    \label{eq:proofspaziovect3}
    v=c_1v_1+\cdots+c_{i-1}v_{i-1}+c_i(a_1v_1+\cdots+a_{i-1}v_{i-1}+a_{i+1}v_{i+1} +\cdots+a_nv_n)+\cdots+c_nv_n
  \end{equation}
  ovvero, facendo i conti e mettendo in evidenza i vettori,
  \begin{equation}
    \label{eq:proofspaziovect4}
    v=(c_1+c_ia_1)v_1+\cdots + (c_{i-1}+c_ia_{i-1})v_{i-1}+(c_{i+1}+c_ia_{i+1})v_{i+1}+\cdots+(c_n+c_ia_n)v_n
  \end{equation}
  e cui bisogna vedere che ogni $v$ dello spazio si riesce a scrivere come combinazione di\\
  $v_1,\dots,v_{i-1},v_{i+1}m\dots,v_n$: questo dimostra che bastano tali vettori a generare lo spazio.
\end{proof}

\chapter{Sistemi di equazioni lineari e matrici}
\label{chap:eqlinematrici}

Per definire rigorosamente cosa si intenda per equazione lineare
(o equazione di $1^o$ grado) e scrivere il generico esempio di
equazione lineare, si trova prima una notazione conveniente per
denotare tali equazioni.\\
Infatti, per non avere limitazioni sul numero delle incognite,
non è possibile continuare a indicarle con le lettere
dell'alfabeto $x,y,z,etc.$, che sono in numero limitato, ma
verrà utilizzata sempre la stessa lettera, tradizionalmente
la $x$ con degli indicatori numerici che consentono di
quale incognita si tratta: $x_1,x_2,\dots,x_n$ con, dove $n$ è
un numero naturale. Facendo un esempio di strutturate in questo
modo, si otterrà una situazione di questo tipo
\begin{equation}
  \label{eq:eqlinematrici1}
  a_1x_1+a_2x_2+\cdots+a_nx_n=b
\end{equation}
dove $b,a_1,a_2,\dots,a_n$ sono elementi di un campo
(solitamente, il campo dei numeri reali o quello dei complessi)
che svolgono il ruolo rispettivamente di termine noto e
coefficienti delle incognite (per ogni incognita $x_i$, bisogna
denotare il suo coefficiente con una lettera, $a_i$ con lo stesso
indice dell'incognita).\\
Dare una soluzione dell'equazione (\ref{eq:eqlinematrici1})
significa trovare degli elementi del campo, ovvero dei numeri,
che sostituiti alle incognite rendano l'ugualianza vera.\\
Ad esempio, nell'equazione lineare in due incognite $x_1-x_2=1$
a coefficienti nel campo dei reali $\mathds{R}$, ponendo $x_1=2$
e $x_2=1$ si ottiene l'ugualianza vera $2-1=1$, mentre ad esempio
ponendo $x_1=1$ e $x_2=2$ si ottiene $1-2=1$ che è falsa.\\
Da questo semplice esempio si vede come dare una soluzione
dell'equazione $x_1-x_2=1$ significa non solo dare \textit{due}
valori numerici, da sostituire alle due incognite
dell'equazione, ma è necessario precisare quale valore vada
sostituito alla prima incognita e quale alla seconda, ovvero
specificare in quale ordine stiamo prendendo questi due
elementi.

La soluzione data di tale equazione può essere pensata e scritta
come una \textit{coppia ordinata} di numeri, che è possibile
denotare in (\ref{eq:eqlinematrici1}). La coppia (2,1) è una
soluzione dell'equazione $x_1-x_2=1$, mentre la caoppia (1,2) non
lo è.\\
Analogamente, per un'equazione con 3 incognite, una sua
soluzione sarà data da una terna ordinata (3,2,1) è una sua
soluzione, in quanto sostituendo $x_1=3,x_2=2,x_3=1$ si ottiene
l'uguaglianza vera $3-2+1=2$; la terna (2,1,3) invece, non è una
sua soluzione.

In generale, per equazioni con $n$ incognite si dovrebbe
utilizzare $n$-uple ordinate $(v_1,v_2,\dots,v_n)$: possiamo
allora dare la seguente:
\begin{defi}
  \label{defi:eqlinematrici1}
  Data un'equazione lineare $a_1x_1+a_2x_2+\cdots+a_nx_n=b$ in
  $n$ incognita a coefficienti in un campo $\mathds{K}$, si dice
  \textit{soluzione} dell'equazione una $n$-upla ordinata
  $(v_1,v_2,\dots,v_n)$ di elementi di $\mathds{K}$ tale che
  sostituendo $v_1$ al posto di $x_1,v_2$ al posto di $x_2$ etc.
  fino a $x_n$ l'equazione risulta verificata (ovvero
  l'ugualianza $a_1v_1+a_2v_2+\cdots+a_nv_n=b$ risulta vera).
\end{defi}
Ora, un \textit{sistema di equazioni lineari} è semplicemente un
insieme di equazioni lineari.\\
Per scrivere un generio tale sistema, per risolvere un problema
di notazione simile a quello affrontato quando è stato scritta
la generica equazione lineare, ovvero è necessaria una notazione
efficace per indicare i diversi coefficienti delle incognite
delle incognite nelle diverse equazioni del sistema.\\
A questo scopo, nell'espressione della generica equazione
lineare $a_1x_1+a_2x_2+\cdots+a_nx_n=b$, la seconda $a_{21}x_1+a_{22}x_2+\cdots+a_{2n}x_n=b_2$ e così via.\\
Allora, il generico sistema di equazioni lineari con $n$
incognite e $m$ equazioni (il numero di incognite può anche
essere diverso dal numero di equazioni, perciò li si indica con
due lettere diverse) sarà
\begin{eqnarray}
  \label{eq:eqlinematrici2}
  \begin{pmatrix}
    a_{11} & a_{12} & \cdots & a_{1n}\\
    a_{21} & a_{22} & \cdots & a_{2n}\\
    \vdots & \vdots & & \vdots\\
    a_{m1} &a_{m2} & &a_{mn}
  \end{pmatrix}
  \begin{pmatrix}
    x_1\\
    x_2\\
    \vdots\\
    x_n
  \end{pmatrix}=
  \begin{pmatrix}
    b_1\\
    b_2\\
    \vdots\\
    b_m
  \end{pmatrix}, & \text{oppure,} &
                  \begin{cases}
                    a_{11}x_1 + a_{12}x_2+\cdots+a_{1n}x_n=b_1\\
                    a_{21}x_1+a_{22}x_2+\cdots+a_{2n}x_n=b_2\\
                    \vdots\\
                    a_{m1}x_2+\cdots+a_{mn}x_n=b_m
                  \end{cases}
\end{eqnarray}
In (\ref{eq:eqlinematrici2}) è presente la soluzione dell'equazione,
sia in forma matriciale $Ax=b$ che in forma sistemica -- Visto ciò è
possibile dare la seguente
\begin{defi}
  \label{defi:eqlinematrici2}
  Una soluzione del sistema (\ref{eq:eqlinematrici2}) è una $n$-uple
  $(v_1,v_2\dots,v_n) \in \mathds{K}^n$ che è soluzione comune di tutte le
  equazioni del sistema.

  nel prossimo paragrafo varrà affrontato un algoritmo che consente di
  determinare tutte le soluzione di un sistema. In particolare, si
  scoprirà che possono verificarsi solo le seguenti tre
  possibilità\footnote{Questo è un fatto caratteristico delle equazioni
    lineari: per una generica equazione possono verificarsi anche altri
    casi, ad esempio l'equazione $x^2=9$ ha due soluzione, $x=3$ e
    $x=-3$.}:
  \begin{itemize}
  \item il sistema non ha nessuna soluzione
  \item il sistema ha una sola soluzione
  \item il sistema ha infinite soluzioni
  \end{itemize}
  Prima di entrare nei dettagli, è necessario vedere un esempio di
  ciascuna di queste possibilità, con l'obiettivo di iniziare a capire le
  ragioni per cui essere possono verificarsi.

  Non è difficile essibire un esempio di sistema con infinite soluzioni.
  Ad esempio, considerando il seguente sistema formato da una sola
  equazione in due incognite
  \begin{eqnarray*}
    \begin{cases}
      x_1+x_2=0.
    \end{cases} 
  \end{eqnarray*}
  Una soluzione del sistema è una coppia di numeri reali tali che la
  loro somma dà come risultato zero: questo significa che i numeri
  devono essere uno l'opposto dell'altro, e quindi scelto un qualunque
  $t\in \mathds{R}$, la coppia $(t,-t)$ è una soluzione: le soluzioni
  sono quindi infinite, tante quanti i numeri reali.\\
  Fatto ciò è possibile alla $x_1+x_2=0$ un'altra condizione, ottenendo
  quindi un sistema di due equazioni, ad esempio
  \begin{eqnarray}
    \label{eq:eqlinematrici3}
    \begin{cases}
      x_1+x_2=0 \\
      x_1+x_2=0
    \end{cases} 
  \end{eqnarray}
  Le soluzioni del sistema sono quindi le coppie che soddisfano non solo
  la prima equazione, cioè come detto tette quelle del tipo $(t,-t)$, ma
  anche la seconda, che afferma semplicemente che $x_1=x_2$, cioè i due
  elementi della coppia devono essere non solo opposti ma ache uguali tra
  loro. Ma l'unico numero reale uguale al suo opposto è lo zero, e quindi
  il sistema ha come unica soluzione la coppia (0,0). Questo esempio
  suggerisce che in generale più equazioni ci sono in un sistema,
  maggiori sono i vincoli che imponendo sulle incognite e quindi meno
  $n$-uple ci saranno che soddisfano tutte le condizioni siano sufficienti
  a ottenere una sola soluzione.

  Tuttavia, è facile fare un altro esempio che mostra che questa prima
  impressione non è del tutto esatta: considerando il sistema
  \begin{eqnarray}
    \label{eq:eqlinematrici4}
    \begin{cases}
      x_1+x_2=0 \\
      2x_1+2x_2=0
    \end{cases}
  \end{eqnarray}
  Ora, è immediato vedere che le soluzioni $(t,-t)$ della prima equazionde
  soddisfano tutte anche la seconda, quindi il sistema continua ad avere
  le infinite soluzioni $(t,-t)$. Quest accade perché la seconda equazione
  è in realtà del tutto equivalente alla prima [mettendo in evidenza il
  2, si può riscrivere $2x_1+2x_2=0$ come $2(x_1+x_2)=0$, ovvero,
  dividendo per 2, proprio la prima equazione] e non aggiunge nessun
  nuovo vincolo sulle incognite: si tratta di un'equazione superflua, la
  cui presenza o meno non cambia l'insieme delle soluzioni.\\
  Le equazioni superflue presenti in un sistema possono essere tuttavia
  molto meno evidenti che nel caso appena visto. Ad esempio, consideriando
  il sitema di due equazioni in tre incognite
  \begin{eqnarray}
    \label{eq:eqlinematrici5}
    \begin{cases}
      x_1+x_2+x_3=1\\
      2x_1+x_2+3x_3=2
    \end{cases}
  \end{eqnarray}
  Una qualunque terna $(x_1,x_2,x_3)$ che verifica le due equazioni
  soddisfa necessariamente anche l'ugualianza che si ottiene sommandole
  membro a membro, ovvero
  \begin{eqnarray*}
    (x_1+x_2+x_3) + (2x_1+x_2+3x_3) = 1+2
  \end{eqnarray*}
  cioè, svolgendo i conti,
  \begin{eqnarray*}
    3x_1+2x_2+4x_3=3
  \end{eqnarray*}
  Essendo tale equazione una conseguenza delle prime due, aggiungerla al
  sistema non modifica l'insieme delle soluzioni: in altre parole, il
  sistema
  \begin{eqnarray}
    \label{eq:eqlinematrici6}
    \begin{cases}
      x_1+x_2+x_3=1\\
      2x_1+x_2+3x_3=2\\
      3x_1+2x_2+4x_3=3
    \end{cases}
  \end{eqnarray}
  contiene un'equazione superflua, dipendente dalla altre, certamente meno
  evidente a pria vista che nel caso del sistema
  (\ref{eq:eqlinematrici4}).

  Naturalmente, equazioni superflue possono essere ottenute anche con
  combinazioni più complicate della somma delle prime due equazioni, ad
  esempio sempre in riferimento al sistema (\ref{eq:eqlinematrici5}), una
  terna che soddisfi le due equazioni necessariamente soddisfa anche
  l'ugualianza
  \begin{equation}
    \label{eq:eqlinematrici7}
    5(x_1+x_2+x_3)+(-3)(2x_1+x_2+3x_3)=5\cdot +(-3)\cdot 2
  \end{equation}
  cioè, svolgendo i conti,
  \begin{equation*}
    -x_1+2x_2-4x_3=-1
  \end{equation*}
  ovvero anche nel sistema
  \begin{equation}
    \label{eq:eqlinematrici8}
    \begin{cases}
      x_1+x_2+x_3=1\\
      2x_1+x_2+3x_3=2\\
      -x_1+2x_2-4x_3=-1
    \end{cases}
  \end{equation}
  la terza equazione è superflua, in un modo forse ancora meno evidente.

  Per quello che riguarda i sistemi senza soluzioni, è abbastanza semplice
  esivirne uno. Ad esempio, il sistema di due equazioni in due incognite
  seguente
  \begin{equation*}
    x_1+x_2=0\\
    x_1+x_2=1
  \end{equation*}
  è evidentemente privo di soluzioni, in questo se la somma di due numeri
  è uguale a 0 non può certamentet nello stesso tempo essere uguale a 1.
  In altre parole, le due equazioni del sono tra loro incompatibili,
  ovvero esprimono condizioni contraddittorie.
  Per questo motivo, un sistema che non ha soluzioni si dice
  \textit{incompatibile} (e per contro, si dirà \textit{compatibile} un
  sistema che ha almeno una soluzione).
  Per questo motivo, un sistema che non ha soluzioni si dice
  \textit{incompatibile} (e per contro, si dirà \textit{compatibile} un
  sistema che ha almeno una soluzione).
  Analogamente a quanto fatto sopra per le equazioni superflue, si possono
  costruire esempi di sistemi in cui l'incompatibilità di una equazione
  con le altre non e così evidente come nel semplice sistema precedente.
  Ad esempio, prendiamo sempre come punto di partenza il sistema
  (\ref{eq:eqlinematrici5}). Come visto sopra, una terna che soddisfi le
  due equazioni membro a membro.

  Ma allora, se modifichiamo solo il termine noto di quest'ultima
  uguaglianza, si ottiene una che è incompanibile con le altre due:
  ad esempio, il sistema
  \begin{eqnarray}
    \label{eq:eqlinematrici9}
    \begin{cases}
      x_1+x_2+x_3=1\\
      2x_1+x_2+3x_3=2\\
      3x_1+2x_2+4x_3=5
    \end{cases}
  \end{eqnarray}
  non ha soluzioni, perché per una qualunque terna che soddisfi le prime
  due equaioni si deve avere che $3x_1+2x_2+4x_3$ è uguale a 3, e non a 5.
\end{defi}
\begin{oss}
  \label{oss:eqlinematrici1}
  Un sistema di equazioni in cui i termini noti siano tutti uguali a zero
  (un tale sistema si dice \textit{omogeneo}) ha sempre almeno la
  soluzione $(0,0,\dots,0)$, in quanto ponendo tutte le incognite uguali
  a zero si ottengono uguaglianza vere. Quindi i sistemi omogenei sono
  sempre compatibili. Vedremo più avanti altre importanti caratteristiche
  dei sistemi omogenei che li distinguono dai sistemi non omogenei.
\end{oss}

\section{Matrice di un sistema lineare}
\label{sec:matricediunsistlineare}

Per conoscere un sistema è necessario conoscere, equazione per equazione,
quali sono i coefficienti che moltiplicano ogni singola incognita e i
termini noti. Quindi, se, dato un sistema, si scrive una tabella di
numeri disposti in righe e in colonne in modo che in ogni riga ci siano
i coefficienti delle incognite di una certa equazione (ordinati secondo
l'ordine scelto delle incognite) e il termine noto, tale tabella conterrà
tutte le informazioni che servono sul sistema
\begin{equation}
  \label{eq:matricediunsistlineare1}
  \begin{cases}
    x_1+3x_2=5\\
    2x_1-x_2=4
  \end{cases}
\end{equation}
può essere rappresentato dalla tabella
\begin{equation}
  \label{eq:matricediunsistlineare2}
  \begin{bmatrix}
    1 & 3 & 5\\
    2 & -1 & 4
  \end{bmatrix}
\end{equation}
questa viene chiamata in gergo, \textit{matrice completa del sistema}.
\begin{defi}
  \label{defi:matrice1}
  Una matrice A ad elementi reali è una tabella di numeri reali, detti le
  sue \textit{entrate}
  \begin{equation*}
    A=
    \begin{pmatrix}
      a_{11} & a_{12} & \dots & a_{1n}\\
      a_{21} & a_{22} & \dots & a_{2n}\\
      \vdots & \vdots & \vdots & \vdots\\
      a_{m1} & a_{m2} & \dots & a_{1n}\\
    \end{pmatrix}
  \end{equation*}
  scritti su righe e colonne: se la matrice ha $m$ righe e $n$ colonne,
  si dice che $A$ ha dimensione $m \times n$ oppure si può affermare che
  sia di tipo $m \times n$ o si può anche dire che appartiene a
  $\mathds{R}^{m \times n}$. Se la matrice è ad elementi complessi, si può
  affermare che $A$ appartiene a $\mathds{C}^{m \times n}$.

  Indicando gli elementi della matrice con $a_{ij}$ oppure $(A)_{ij}$
  utilizzando due indici in basso, dove $i$ è l'\textit{indice di riga}
  (dice in quale riga si trova e va da 1 a $m$) e $j$ è \textit{l'indice
    di colonna} (dice in quale colonna si trova e va da 1 a $n$). Si dice
  anche che $a_{ij}$ è l'entrata di posto $ij$.
\end{defi}
\begin{es}
  \label{es:matrice1}
  \textit{Matrice} con 3 righe e 4 colonne
  \begin{equation*}
    \begin{vmatrix}
      6 & -2 & \pi & 0\\
      10 & 3 & 0 & -1\\
      4 & \sqrt{2} & -3 & 1
    \end{vmatrix} \in \mathds{R}^{3 \times 4}
  \end{equation*}
\end{es}
\begin{defi}
  \label{defi:matrice2}
  Per \textbf{trasposta} della matrice $A\in \mathds{R}^{3\times 4}$ si
  intende, la matrice che si indica con $A^T\in \mathds{R}^{3\times 4}$ che
  si ottiene da $A$ scambiando ordinatamente le righe con le colonne
  \begin{equation*}
    (A^T)_{ij}=a_{ji}
  \end{equation*}
\end{defi}
\begin{es}
  \label{es:matrice2}
  Perndendo una matrice $A \in \mathds{R}^{2\times3}$, si otterrà un
  $A^T\in \mathds{R}^{3\times2}$
  \begin{eqnarray*}
    A=
    \begin{bmatrix}
      1 & -1 & 2\\
      2 & 0 & 3
    \end{bmatrix}, & A^T=
                     \begin{bmatrix}
                       1 & 2  \\
                       -1 & 0 \\
                       2 & 3
                     \end{bmatrix}
  \end{eqnarray*}
  (che appare come se fosse stata specchiata e ruotata di $90^o$)
\end{es}
\begin{defi}
  \label{defi:matrice3}
  Per \textbf{sottomatrice} $B\in\mathds{R}^{p\times q}$ di una matrice
  $\mathds{R}^{m \times n}$ si intende, la matrice in cui elementi
  appartengono a $p$ righe e $q$ colonne di $A$.
\end{defi}
\begin{es}
  \label{es:matrice3}
  \begin{equation*}
    \begin{bmatrix}
      6 & \pi & 0\\
      10 & 0 & -1
    \end{bmatrix}
  \end{equation*}
  è una sottomatrice della matrice dell'Esempio \ref{es:matrice1}
  scegliendo prima e seconda riga e prima, terza e quarta colonna.
\end{es}
\begin{oss}
  \label{oss:matrice1}
  se una matrice ha una sola dimensione ($n\times 1$) vengono definiti
  anche vettori.
\end{oss}
\begin{defi}
  \label{defi:matrice4}
  Una matrice di tipo $n\times n$, anche chiamata \textit{matrice
    quadrata} ed il numero $n$ prende il nome di \textit{ordine} della
  matrice. Gli elementi $a_{11},a_{12},\cdots,a_{nn}$ costituiscono la
  \textit{diagonale principale} della matrice.

  Una sottomatrice quadrata di $A\in\mathds{R}^{m\times n}$ viene anche
  definita \textbf{minore estratto da} $A$. Se $A\in\mathds{R}^{m\times n}$
  è quadrata, il \textbf{minore complementare} dell'elemento $a_{ij}$ di
  $A$ è il minore di ordine $n-1$ che si ottiene cancellando da A la
  riga e la colonna a cui appartiene $a_{ij}$ (cancellando riga $i$ e
  colonna $j$).\\
  Nell'ambito delle \textbf{matrici quadrate}, hanno particolare
  importanza i seguenti tipi matrici:
  \begin{itemize}
  \item \textit{simmetrica} se $a_{ij}=a_{ji}$, cioè $A=A^T$;
    \begin{es}
      \label{es:matrice4-1}
      \begin{equation*}
        \begin{bmatrix}
          3 & 7 & -1 & 2\\
          7 & -2 & 0 & 10\\
          -1 & 0 & -12 & 1\\
          2 & 10 & 1 & 6 
        \end{bmatrix}
      \end{equation*}
    \end{es}
  \item \textit{antisimmetrica} se $a_{ij}=-a_{ji}$; si noti che gli
    elementi della diagonale principale devono essere nulli, perché deve
    valere $a_{ii}=-a_{ii}$ ma se un numero reale è uguale al suo opposto
    deve essere per forza 0;
    \begin{es}
      \label{es:matrice4-2}
      \begin{equation*}
        \begin{bmatrix}
          \mathbf{0} & 7 & -2\\
          -7 & \mathbf{0} & 1\\
          2 & -1 & \mathbf{0}
        \end{bmatrix}
      \end{equation*}
    \end{es}
  \item \textit{triangolare superiore} se $a_{in}=0$ per $i>j$;
    \begin{es}
      \label{es:matrice4-3}
      \begin{eqnarray*}
        \begin{bmatrix}
          3 & 7 & -2\\
          \mathbf{\color{blue}0} & 11 & 1\\
          \mathbf{\color{blue}0} & \mathbf{\color{blue}0} & -5
        \end{bmatrix} & a_{21} = a_{31} = a_{32} = 0
      \end{eqnarray*}
    \end{es}
  \item \textit{triangolare inferiore} se $a_{ij}=0$ per $i<j$;
    \begin{es}
      \label{es:matrice4-4}
      \begin{eqnarray*}
        \begin{bmatrix}
          3 & \mathbf{\color{blue}0} & \mathbf{\color{blue}0}\\
          -7 & 5 & \mathbf{\color{blue}0}\\
          12 & -1 & 4
        \end{bmatrix} & a_{12}=a_{13}=a_{23}=0
      \end{eqnarray*}
    \end{es}
  \item \textit{diagonale} se $a_{ij}=0$ per $i\neq j$.
    \begin{es}
      \label{es:matrice4-5}
      \begin{equation*}
        \begin{bmatrix}
          4 & 0 & 0\\
          0 & -2 & 0\\
          0 & 0 & 9
        \end{bmatrix}
      \end{equation*}
      In particolare, se gli elementi diagonali sono uguali a 1, tale
      matrice si chiama \textbf{matrice identità} e si indica col simbolo
      $I$ (o $I_n$ se si vuole evidenziare il suo ordine)
      \begin{equation*}
        I_4=
        \begin{bmatrix}
          1 & 0 & 0 & 0\\
          0 & 1 & 0 & 0\\
          0 & 0 & 1 & 0\\
          0 & 0 & 0 & 1
        \end{bmatrix}
      \end{equation*}
      \begin{lstlisting}[caption=generare una matrice identità in GNU/Octave]
        > eye(4)
        ans =
        
        Diagonal Matrix

        1   0   0   0
        0   1   0   0
        0   0   1   0
        0   0   0   1
      \end{lstlisting}
    \end{es}
  \end{itemize}
\end{defi}
\begin{defi}
  \label{defi:matrice5}
  La \textbf{traccia} di una matrice quadrata $A$ è il numero dato dalla
  somma degli elementi sulla diagonale
  \begin{equation*}
    \mathrm{tr}(A)=a_{11}+a_{22}+\cdots+a_{nn}.
  \end{equation*}
\end{defi}
\begin{es}
  La traccia della matrice dell'Esempio \ref{es:matrice4-1} è
  $\mathrm{tr}(A)=3-2-12+6=-5$.
\end{es}
Considerando una matrice rettangolare $A\in\mathds{R}^{m\times n}$.
\begin{defi}
  \label{defi:matrice6}
  Una matrice viene detta \textbf{a gradini} se dalla prima all'ultima
  riga, il primo elemento non nullo di ogni riga compare con un indice di
  colonna sempre più grande. Il primo elemento non nullo di ogni riga è
  chiamato \textit{pivot}.
  \begin{es}
    \label{es:matrice6-1}
    \begin{equation*}
      \begin{bmatrix}
        7 & 1 & 1 & 3 \\
        0 & 4 & 3 & 5 \\
        0 & 0 & 0 & 6
      \end{bmatrix}
    \end{equation*}
    è una matrice a gradini. I suoi \textit{pivot} sono 7 nella prima
    seconda, 4 nella seconda, 6 nella terza, nell'ordine, sulla prima,
    seconda e quarta colonna (indice di colonna viene incrementato più si
    scende nella matrice).
  \end{es}
  \begin{es}
    \label{es:matrice6-2}
    \begin{equation*}
      \begin{bmatrix}
        7 & 1 & 1 & 3 \\
        0 & 4 & 3 & 5 \\
        0 & 2 & 0 & 6
      \end{bmatrix}
    \end{equation*}
    non è a gradini. Il primo elemento non nullo della terza riga sta
    nella stessa colonna del primo elemento nullo della seconda riga.
  \end{es}
  \begin{es}
    \label{es:matrice6-3}
    \begin{equation*}
      \begin{bmatrix}
        7 & 1 & 1 & 3 \\
        0 & 0 & 3 & 5 \\
        0 & 2 & 0 & 6
      \end{bmatrix}
    \end{equation*}
    non è a gradini. Il primo elemento non nullo della riga sta in una
    colonna di indice più piccolo del primo elemento non nullo della
    seconda riga.
  \end{es}
  \begin{es}
    \label{es:matrice6-4}
    Altri esempi di matrici a gradini sono le diagonali e le matrici
    triangolari superiori con gli elementi sulla diagonale principale
    diversi da zero.
  \end{es}
\end{defi}

\section{Operazioni tra matrici}
\label{sec:opmatrici}

\subsection{Somma di matrici}
\label{sec:somdimatrici}

Siano $A$ e $B$, due matrici dello stesso tipo $m\times n$. Gli elementi
della matrice $A+B$, anche detta \textbf{somma} di $A$ e $B$, si ottengono
sommando elementi aventi lo stesso posto in $A$ e $B$, cioè
\begin{eqnarray*}
  (A+B)_{ij}=a_{ij}+b_{ij} & i=1,\cdots,m, & j=1,\cdots,n.
\end{eqnarray*}
\begin{es}
  \label{es:sommatrice1}
  \begin{eqnarray*}
    A=
    \begin{bmatrix}
      \frac{7}{10} & -1 & \frac{1}{4}\\
      0 & 3 & \frac{1}{2}
    \end{bmatrix}, & B=
                    \begin{bmatrix}
                      0 & \frac{1}{2} & \frac{3}{4}\\
                      -2 & 1 & \frac{1}{2}
                    \end{bmatrix}, & A+B=
                                    \begin{bmatrix}
                                      \frac{7}{10} & -\frac{1}{2} & 1\\
                                      -2 & 4 & 1
                                    \end{bmatrix}
  \end{eqnarray*}
\end{es}

\paragraph{Proprietà:}

siano $A,B,C\in \mathds{R}^{m\times n}$, valgono le seguenti
\begin{description}
\item[Proprietà commutativa] $A+B=B+A$
\item[Proprietà associativa] $(A+B)+C=A+(B+C)$
\item[La matrice nulla $O$] (formata da tutti zeri) è tale che
  $A+O=O+A=A$ 
\item[La matrice di $-A$] (opposta di $A$) i cui elementi sono gli opposti
  dei relativi elementi di $A$ è tale che $A+(-A)=O$
\end{description}
La \textbf{differenza} tra matrice dello stesso tipo è definita da
\begin{equation*}
  A-B=A+(-B)
\end{equation*}

\subsection{Prodotto di uno scalare per una matrice}
\label{sec:prodmatrice}

Sia $A\in \mathds{R}^{m\times n}$ e $\lambda \in \mathds{R}$. Il
\textbf{prodotto} di $\lambda$ per $A$ è la matrice $\lambda A$ i quali
elementi sono ottenuti moltiplicando per $\lambda$ i corrispondenti
elementi di $A$, cioè
\begin{eqnarray*}
  (\lambda A)_{ij}=\lambda a_{ij} & i = 1,\dots,m, & j=1,\dots,n.
\end{eqnarray*}
\begin{es}
  \label{es:prodmatrice1}
  \begin{equation*}
    3
    \begin{bmatrix}
      -2 & 1 & 0\\
      4 & -1 & 2
    \end{bmatrix} =
    \begin{bmatrix}
      -6 & 3 & 0\\
      12 & -3 & 6
    \end{bmatrix}
  \end{equation*}
  
\end{es}

\paragraph{Proprietà}

siano $A,B\in \mathds{R}^{m\times n}$ e $\lambda, \mu \in \mathds{R}$, valgono le
seguenti
\begin{enumerate}
\item $\lambda (A+B)=\lambda A+\lambda B$
\item $(\lambda + \mu)A=\lambda A+\mu A$
\item $\lambda (\mu A)=(\lambda\mu) A$
\item $1A=A$
\end{enumerate}
\begin{oss}
  \label{oss:prodmatrice1}
  L'insieme delle matrici di tipo $m\times n$ dotato delle operazioni di
  somma di matrici e prodotto di uno scalare per una matrice è uno
  spazio vettoriale.
\end{oss}

\subsection{Prodotto di matrici (righe per colonne)}
\label{sec:prodmtxrigcol}

Siano $A\in \mathds{R}^{m\times n}$ e $B\in \mathds{R}^{n\times p}$ (il numero di
colonne in $A$ coincidono con il numero di righe in $B$) il \textbf{prodotto} delle
matrici $A$ e $B$ è la matrici
\begin{equation*}
  AB\in \mathds{R}^{m\times p}
\end{equation*}
il cui elemento generico è dato da
\begin{eqnarray}
  \label{eq:prodmtxrigcol1}
  (AB)_{ij}=\sum_{k=1}^na_{ik}b_{kj} & i=1,\cdots,m; & j = 1,\cdots,p
\end{eqnarray}
cioè la matrice prodotta ha numero di righe pari a quello della matrice di sinistra
e numero di colonne pari a quello della matrice di destra, e il suo generico elemento
è la somma dei prodotti degli elementi della riga di posto $i$ nella matrice $A$ per
i corrispondenti elementi della colonna di posto $j$ nella matrice B.
Ecco come diventa la formula (\ref{eq:prodmtxrigcol1}) se $A\in \mathds{R}^{2\times3}$
e $B\in \mathds{3\times 2}$, andando a scrivre gli elementi della matrice prodotto:
\begin{eqnarray}
  \label{eq:prodmtxrigcol2}
  \underbrace{\begin{bmatrix}
    a_{11} & a_{12} & a_{13}\\
    a_{21} & a_{22} & a_{23}
  \end{bmatrix}}_{A}
  \underbrace{\begin{bmatrix}
    b_{11} & b_{12} \\
    b_{21} & b_{22} \\
    b_{31} & b_{32}
  \end{bmatrix}}_{B}=
  \underbrace{\begin{bmatrix}
    c_{11} & c_{12}\\
    c_{21} & c_{22} 
  \end{bmatrix}}_{AB} &
                  \begin{matrix}
                    c_{11}=a_{11}b_{11} + a_{12}b_{21} + a_{13} b_{31}\\
                    c_{12}=a_{11}b_{12} + a_{12}b_{22} + a_{13} b_{32}\\
                    c_{21}=a_{23}b_{11} + a_{22}b_{21} + a_{23} b_{31}\\
                    c_{22}=a_{21}b_{12} + a_{22}b_{22} + a_{23} b_{32}
                  \end{matrix}
\end{eqnarray}
\begin{es}
  \label{es:prodmtxrigcol1}
  Prodotto tra una matrice $A$ di tupo $2\times 3$ una $B$ di tipo $3\times 2$
  \begin{eqnarray*}
    AB=
    \begin{bmatrix}
      4 & 5 & 3 \\
      2 & 3 & 1
    \end{bmatrix}\cdot
    \begin{bmatrix}
      2 & 3 \\
      4 & 1 \\
      9 & 2
    \end{bmatrix} =
    \begin{bmatrix}
      4\cdot 2 + 5 \cdot 3 + 3\cdot 9 & 4 \cdot 3 + 5 \cdot 1 + 3 \cdot 2\\
      2\cdot 2 + 5 \cdot 4 + 3\cdot 9 & 4 \cdot 2 + 3 \cdot 3 + 1 \cdot 3
    \end{bmatrix}
    =
    \begin{bmatrix}
      55 & 23\\
      25 & 11
    \end{bmatrix}
  \end{eqnarray*}
  La matrice prodotto è di tipo $2\times 2$. Riportando questa operazione su
  GNU/Octave:
  \begin{lstlisting}[caption=moltiplicazione riga per colonna]
    A = [ 4 5 3; 2 3 1 ];
    B = [ 2 3; 4 1; 9 2 ];
    C=A*B;
    disp (C);
  \end{lstlisting}
\end{es}
\begin{oss}
  \label{oss:prodmtxrigcol1}
  Se ha senseo calcolare $AB$, in generale non può avere senso calcolare $BA$.
  Nell'Esempio \ref{es:prodmtxrigcol1} ha senso calcolare $BA$. 
\end{oss}
\begin{oss}
  \label{oss:prodmtxrigcol2}
  Anche se entrambi i prodotti possono si possono eseguire, come avviene ad
  esempio $A$ e $B$ sono quadrate dello stesso ordine, in genere
  \begin{equation*}
    AB\neq BA
  \end{equation*}
  da questo si deduce che il prodotto tra matrici non gode della proprietà
  commutativa.
\end{oss}
\begin{es}
  \label{es:prodmtxrigcol2}
  \begin{eqnarray*}
    \begin{bmatrix}
      2 & -1\\
      3 & \frac{5}{4}
    \end{bmatrix}
    \begin{bmatrix}
      1 & 0 \\
      2 & -4
    \end{bmatrix}=
    \begin{bmatrix}
      0 & 4\\
      \frac{11}{2} & -5
    \end{bmatrix}\\
    \begin{bmatrix}
      1 & 0 \\
      2 & -4
    \end{bmatrix}
    \begin{bmatrix}
      2 & -1 \\
      3 & \frac{5}{4}
    \end{bmatrix}=
    \begin{bmatrix}
      2 & -1 \\
      -8 & -7
    \end{bmatrix}
  \end{eqnarray*}
  Si può dimostrare se $AB=BA$, qualunque sia la matrice $A$ di ordine $n$, allora
  $B=\lambda I_n$.
  
  \paragraph{Proprietà} purché le operazioni indicate abbiano senso (\textit{in base
    alle dimostrazioni delle matrici}), valgono le sequenti
  \begin{enumerate}
  \item $A(B+C)=AB+AC$, $(A+B)C=AC+BC$
  \item $A(BC)=(AB)C$
  \item $A(\lambda B)=\lambda (AB)$, $(\lambda A)B=\lambda(AB)$
  \end{enumerate}
\end{es}
\begin{oss}
  \label{oss:prodmtxrigcol2}
  Nel prodotto tra matrici non vale la \textit{legge di annullamento del
    prodotto}\footnote{Se due numeri $a$ e $b$ danno prodotto zero $ab=0$, allora
    almeno uno dei fattori è zero.}. Quindi si può ottenere la matrice nulla $AB=O$
  anche se $A$ e $B$ non sono matrici nulle. per esempio
  \begin{equation*}
    \begin{bmatrix}
      0 & 1 \\
      0 & 0
    \end{bmatrix}\cdot
    \begin{bmatrix}
      1 & 0 \\
      0 & 0
    \end{bmatrix} =
    \begin{bmatrix}
      0 & 0 \\
      0 & 0
    \end{bmatrix}
  \end{equation*}
\end{oss}
\begin{oss}
  \label{oss:prodmtxrigcol3}
  Se la matrice $A$ è quadrata, ha senso $A^2=AA, A^3=AAA, A^P=AA\cdot{}A$, detta
  \textit{potenza p-esima} della matrice A. Inoltre se $I_n$ è la matrice identità di
  ordine $n$, vale $AI_n=I_nA=A$.
\end{oss}
\begin{prop}
  \label{prop:prodmtxrigcol1}
  Siano $A\in\mathds{R}^{m\times n}$ e $B\in \mathds{R}^{m\times p}$, Allora vale
  \begin{equation*}
    (AB)^T=B^TA^T
  \end{equation*}
  \begin{proof}
    \begin{eqnarray*}
      ((AB)^T)_{ij}=(AB)_{ji}=\sum_{k=1}^na_{jk}b_{kj}\\
      (B^TA^T)=\sum_{k=1}^n(B^T)_{jk} (A^T)_{kj}=\sum_{k=1}^nb_{kj}a_{jk}
    \end{eqnarray*}
    e le due espressioni sono uguali perché prodotto di due scalari è communtativo.
  \end{proof}
\end{prop}

\section{Il determinante}
\label{sec:determinante}

Data una matrice quadrata di ordine $n$ a entrate in un campo $\mathds{K}$ (che può
essere $\mathds{R}$ o $\mathds{C}$)
\begin{equation*}
  A=
  \begin{bmatrix}
    a_{11} & a_{12} & \dots & a_{1n}\\
    a_{21} & a_{22} & \dots & a_{2n}\\
    \vdots & \vdots & \vdots& \vdots\\
    a_{n1} & a_{n2} & \dots & a_{nn}
  \end{bmatrix}
\end{equation*}
ad essa si associa un numero appartenente a $\mathds{K}$, detto \textbf{determinante}
della metrice, che è funzione delle sue entrate
\begin{equation*}
  \det(A)=
  \begin{bmatrix}
    a_{11} & a_{12} & \dots & a_{1n}\\
    a_{21} & a_{22} & \dots & a_{2n}\\
    \vdots & \vdots & \vdots& \vdots\\
    a_{n1} & a_{n2} & \dots & a_{nn}
  \end{bmatrix}
\end{equation*}
Se la matrice ha ordine $n\leq 3$, il suo determinente è così definito:
\begin{itemize}
\item per $n=1$, $\det(A)$ coincide con l'unico elemento della matrice;
\item per $n=2$, si pone
  \begin{equation*}
    \det(A)=
    \begin{pmatrix}
      a_{11} &a_{12}\\
      a_{21} & a_{22}
    \end{pmatrix}= a_{11}a_{22}-a_{12}a_{21}
  \end{equation*}
\item per $n=3$, si pone
  \begin{eqnarray*}
    \det(A)=
    \begin{pmatrix}
      a_{11} & a_{12} & a_{13}\\
      a_{21} & a_{22} & a_{23}\\
      a_{31} & a_{32} & a_{33}
    \end{pmatrix}= a_{11}
    \begin{bmatrix}
      a_{22} & a_{23}\\
      a_{32} & a_{33}
    \end{bmatrix} - a_{12}
    \begin{bmatrix}
      a_{21} & a_{23}\\
      a_{31} & a_{33}
    \end{bmatrix}+
    a_{13}
    \begin{bmatrix}
      a_{21} & a_{22}\\
      a_{31} & a_{32}
    \end{bmatrix}\\
    = a_{11}(a_{22}a_{33}-a_{23}a_{32})-a_{12}(a_{21}a_{33}-a_{23}a_{31})+a_{13}
    (a_{21}a_{32}-a_{22}a_{31})
  \end{eqnarray*}
\end{itemize}
Per estendere la definizione di determinante al caso $n$ generale, è necessaria una
premessa sulle permutazioni.

\section{Permutazioni}
\label{sec:perm}

Dato l'insieme $\{1,2,\cdots, n\}$ dei numeri naturali compresi tra 1 e n, una
funzione da questo insieme in se stesso associa ad ogni elemento di $\{1,2,\dots,n\}$
un immagine, scelta sempre all'interno di $\{1,2,\cdots,n\}$. Se le immagini sono
tutte diverse zenza ripetizioni, queste saranno ancora tutti gli elementi
$1,2,\dots,n$ semplicemente disposti in un altro ordine, ovvero permutati. Si tratta
allor di \textbf{permutazione di $n$ elementi}.
\begin{es}
  \label{es:perm1}
  Le seguenti rappresentano permutazioni di 4 elementi:
  \begin{eqnarray*}
    1 \to 1 && 1 \to 3\\
    2 \to 3 && 2 \to 4\\
    3 \to 2 && 3 \to 2\\
    4 \to 4 && 4 \to 1
  \end{eqnarray*}
  L'insieme delle permutazioni di $n$ eleemnti si denota $S_n$. Per ogni $n$, tale
  insieme contiene esattamente $n!:= n(n-1)(n-2)\cdots 2 \cdot 1$ (cioè $n$
  fattoriale) permutazioni: ad esempio per $n=2$ si ottiene $2\cdot 1=2$ permutazioni
  possibili, ovvero
  \begin{equation*}
    \begin{matrix}
      1\to 1 && 1\to 2\\
      2\to 2 && 2\to 1
    \end{matrix}
  \end{equation*}
  Tra le permutazioni, vi è sempre anche quella che associa a ogni elemento se stesso,
  detta \textit{permutazione identica}.
\end{es}
Per $n=3$ si ottiene invece $3! = 3\cdot 2\cdot 1 = 6$ permutazioni possibili,
ovvero
\begin{equation*}
  \begin{matrix}
    p_1 & p_2 & p_3 & p_4 & p_5 & p_6\\
    1 \to 1 & 1 \to 2 & 1 \to 1 & 1 \to 3 & 1 \to 2 & 1 \to 3 \\
    2 \to 2 & 2 \to 1 & 2 \to 3 & 2 \to 2 & 2 \to 3 & 2 \to 1 \\
    3 \to 3 & 3 \to 3 & 3 \to 2 & 3 \to 1 & 3 \to 1 & 3 \to 2
  \end{matrix}
\end{equation*}
Si noti che $p_2$, $p_3$ e $p_4$ scambiano tra loro due elementi fisso il terzo
($p_2$ scambia tra loro 1 e 2, $p_3$ scambia loro 2 e 3, $p_4$ scambia tra loro 1 e
3): in genere, una permutazione di questo tipo, che scambia tra loro due elementi
lasciando fissi tutti gli altri elementi presentata nell'esempio \ref{es:perm},
(scambia tra loro 2 e 3 lasciando fissi 1 e 4), mentre la seconda non lo è. Benché
non tutte le permutazioni sieno trasposizioni, qualunque realizzata eseguendo può
esssere ottenuta come composizione di trasposizioni, ovvero può essere permutazione
può essere ottenuta come composizioni di trasposizioni, ovvero può essere realizzata
eseguendo una sequenza di trasposizioni. Ad esempio, la permutazione $p_5$ di sopra,
che non è una trasposizione, può tuttavia essere ottenuta scambiando prima 1 e 2, e
poi 1 e 3, cioè componentdo 2 trasposizioni:
\begin{equation*}
  \begin{matrix}
    1 \to 2 \to 2\\
    2 \to 1 \to 3\\
    3 \to 3 \to 1
  \end{matrix}
\end{equation*}
In genere, se il numero di trasposizioni che servono per ottenere una permutazione
$p$ è pari, si dice che $p$ è una \textbf{permutazione pari}, se invece, il numero
di trasposizioni che servono per ottenere $p$ è dispari, si dice che $p$ sia una
\textbf{permutazione dispari}. Ad esempio, $p_5$ è una permutazione pari, in quanto
è stato ottenuto componendo 2 trasposizioni.
\begin{oss}
  \label{oss:perm1}
  Se una permutazione è già essa una trasposizione, allora essa è dispari (1 è un
  numero dispari).
\end{oss}
\begin{oss}
  \label{oss:perm2}
  Possono esserci più modi diversi di decomporre una permutazione come composizione
  di trasposizioni, ad esempio, la permutazione identica può essere vista o come
  risultato di $0$, oppure come risultato di 2 trasposizioni
  \begin{equation*}
    \begin{matrix}
      1 \to 2 \to 1\\
      2 \to 1 \to 2\\
      3 \to 3 \to 3
    \end{matrix}
  \end{equation*}
  Tuttavia, si può dimostrare che il numero di trasposizioni che servono per
  ottenere una permutazione data è o sempre pari o sempre dispari (nell'esempio, 0
  o 2, comunque pari).
\end{oss}
Si può allora definire il \textbf{segno} $s(p)$ di una permutazione $p$ come
\begin{itemize}
\item $s(p)=+1$ se $p$ è una permutazione pari;
\item $s(p)=-1$ se $p$ è una permutazione dispari.
\end{itemize}

\subsection{Determinante}
\label{sec:determinante}

\begin{defi}
  \label{defi:determinante1}
  Sia $A$ una matrice quadrata di ordine $n$ con entrate $a_{ij}$. Il determinante è
  definito da
  \begin{equation}
    \label{eq:determinante1}
    \det (A)=\sum_{p\in S_n}s(p)\cdot a_{1p(1)}a_{2p(2)}\cdots a_{np(n)}
  \end{equation}
  Il determinante è dato da una sommatoria che ha un addendo per ogni permutazione
  $p\in S_n$: ognuno di questi addendi è un prodotto di entrate di $A$ del tipo
  $a_{1p(1)}a_{2p(2)}\cdots a_{np(n)}$, con davanti un $+$ o $-$ a seconda che la
  permutazione $p$ sia pari o dispari. Si noti che l'espressione
  $a_{1p(1)}a_{2p(2)}\cdots a_{np(n)}$ è il prodotto di $n$ entrate scelte nella
  matrice, una per scambia gli undici $1,2,\dots,n$ senza ripetizioni, si sceglie
  un'entrata da ogni riga che le entrate scelte stiano anche su colonne diverse.
\end{defi}
Per chiarire la definizione, si prende i casi $n = 2$ e $n=3$.

Sia $n=2$ e $A=
\begin{bmatrix}
  a_{11} & a_{12}\\
  a_{21} & a_{22}
\end{bmatrix}
$. Dell'insieme $\{1,2\}$ ci sono 2 permutazioni (l'identità e la trasposizione 1
con 2), quindi nella sommatoria (\ref{eq:determinante1}) ci saranno solo due
addendi, del tipo $s(p)$ e $a_{1p(1)}a_{2p(2)}$:
\begin{itemize}
\item se $p$ è l'identità (permutazione pari) si ha $s(p)=+1$, l'addendo
  corrispondente sarà $+a_{12} a_{21}$.
\item se $p$ è la trasposizione che scambia 1 con 2 (permutazione dispari), si ha
  $s(p)=-1$ e l'addendo corrispondente sarà $-a_{12} a_{21}$.
\end{itemize}
Quindi il determinante risulta essere $\det(A)=a_{11}a_{22} -a_{12} a_{21}$.

Sia $n=3$ e $A=
\begin{bmatrix}
  a_{11} & a_{12} & a_{13}\\
  a_{21} & a_{22} & a_{23}\\
  a_{31} & a_{32} & a_{33}
\end{bmatrix}
$. Le permutazioni dell'insieme $\{1,2,3\}$ sono $3!=6$, quindi la sommatiria
(\ref{eq:determinante1}) avrà il addendi: per ognuna di queste permutazioni $p$
l'addendo corrispondente sarà del tipo $s(p)\cdot a_{1p(1)}a_{2p(2)}\cdots a_{np(n)}$.
Più precisamente si avranno gli addendi:
\begin{itemize}
\item $+a_{11}a_{22}a_{33}$ corrispondente alla permutazione $p(1)=1,p(2)=2, p(3)=3$
  (permutazione identica, che è una permutazione pari)
\item $-a_{11}a_{23}a_{32}$ corrispondente alla permutazione $p(1)=1,p(2)=3, p(3)=2$
  (una trasposizione, non per altro è una permutazione dispari)
\item $+a_{12}a_{21}a_{32}$ corrispondente alla permutazione $p(1)=2,p(2)=3, p(3)=1$
  (composizione di due trasposizioni, quindi permutazione pari)
\item $-a_{13}a_{21}a_{33}$ corrispondente alla permutazione $p(1)=2,p(2)=1, p(3)=3$
  (composizionde di due trasposizioni, non per altro è una permutazione dispari)
\item $+a_{13}a_{21}a_{32}$ corrispondente ala permutazione
  $p(1)=3,p(2)=1,p(3)=2$ (è una composizione di due trasposizioni, quindi
  una permutazione pari)
\item $-a_{13}a_{22}a_{31}$ corrisponde alla permutazione $p(1)=3, p(2)=2, p(3)=1$ (una trasposizione, quindi una permutazione dispari).
\end{itemize}
Quindi il determinante risulta essere
\begin{equation*}
  \det(A)=a_{11}a_{22}a_{33}-a_{11}a_{23}a_{32}+a_{12}a_{21}a_{32}-a_{13}a_{21}a_{33}+a_{13}a_{21}a_{32}-a_{13}a_{22}a_{31}
\end{equation*}
È necessario introdurre un metodo per calcolare il determiannte,
alternativo alla definizione, il cui utilizzo diretto richiederebbe di
scrivere una sommatoria che per una matrice di ordine $n$ a $n!$ addendi,
tanti quanti le permutazioni di $n$ elementi (si pensi che già per $n=4$
abbiamo $4!=24$ addendi). Allo scopo di calcolare il determinante,
verrà utilizzata la \textit{formula di Laplace}.

\subsection{Formula di Laplace}
\label{sec:formlaplace}
\begin{oss}
  \label{oss:formlaplace}
  Per calcolare il prodotto vettoriale tra due vettori $x$ e $y$ non è
  necessario studiare a emoria la formula, perché si può ricavare
  calcolando il determinante di una matrice $3\times 3$. Tale matrice si
  costruisce in questo modo:
  \begin{itemize}
  \item nella prima riga bisogna disporre le lettere $i,j,k$, che indicano
    i versori della base canonica in $\mathds{R}^3$;
  \item nella seconda riga le coordinate del vettore $x$;
  \item nella terza riga le coordinate del vettore $y$.
  \end{itemize}
  Calcolando il determinante (sviluppando Laplace secndo la prima riga),
  si ottiene
  \begin{equation*}
    \begin{bmatrix}
      \mathbf{i} & \mathbf{j} &\mathbf{k}\\
      x_1 & x_2 & x_3\\
      y_1 & y_2 & y_3
    \end{bmatrix}=(x_2y_3-x_3y_2)\mathbf{i}-(x_1y_3-x_3y_1)\mathbf{j}
    +(x_1y_2-x_2y_1)\mathbf{k}
  \end{equation*}
  che sono proprio le coordinate del vettore $x\wedge y=
  \begin{bmatrix}
    x_2y_3-x_3y_2\\
    x_1y_3-x_3y_1\\
    x_1y_2-x_2y_1
  \end{bmatrix}
  $
\end{oss}

\subsection{Proprietà del determinante}
\label{sec:prodeldet}
\begin{pro}
  \label{pro:prodeldet1}
  Il determiannte di una matrice è uguale a quello della sua trasposta
  \begin{equation*}
    \det(A)=\det(A^T)
  \end{equation*}
\end{pro}
\begin{es}
  \label{es:prodeldet1}
  \begin{eqnarray*}
    \begin{pmatrix}
      -2 & 3 & 1\\
      0 & 1 & -3\\
      1 & -3 & 5
    \end{pmatrix}=-2
    \begin{pmatrix}
      1 &-3\\
      2 & 5
    \end{pmatrix}= -2(5+6)=-22\\
    \begin{pmatrix}
      -2 & 0 & 0\\
      3 & 1 & 2 \\
      1 & -3 & 5
    \end{pmatrix}= -2
    \begin{pmatrix}
      1 & 2\\
      -3 & 5
    \end{pmatrix}=-2(5+6)=-22
  \end{eqnarray*}
\end{es}
\begin{pro}
  \label{pro:prodeldet2}
  Il determinante di una matrice triangolare (inferiore o superiore) è
  uguale al prodotto degli elementi della diagonale principale
  \begin{equation*}
    \begin{pmatrix}
      a_{11} & a_{12} & \cdots & a_{1n}\\
      0 & a_{22} & \cdots & a_{2n}\\
      \vdots & \vdots & \vdots & \vdots\\
      0 & 0 & \cdots & a_{nn}
    \end{pmatrix}=a_{11}a_{22}\dots a_{nn}
  \end{equation*}
  Il perticolare, anche il determiannte di una matrice diagonale è uguale
  al prodotto degli elementi della diagonale principale.
\end{pro}
\begin{es}
  \label{es:prodeldet2}
  \begin{equation*}
    \begin{bmatrix}
      2 & 16 & -50\\
      0 & 1 & 2022\\
      0 & 0 & -5
    \end{bmatrix}=-10
  \end{equation*}
\end{es}
\begin{pro}
  \label{pro:prodeldet3}
  Se gli elementi di una riga o di una collona sono moltiplicati per uno
  stesso numero $c\in\mathds{R}$, il determinante dato da $c\det (A)$
  \begin{equation*}
    \begin{bmatrix}
      ca_{11}& ca_{12} & ca_{13}\\
      a_{21} & a_{22} & a_{23}\\
      a_{31} & a_{32} & a_{33}
    \end{bmatrix}=c
    \begin{bmatrix}
      a_{11} & a_{12} & a_{13}\\
      a_{21} & a_{22} & a_{23}\\
      a_{31} & a_{32} & a_{33}
    \end{bmatrix}
  \end{equation*}
\end{pro}
\begin{es}
  \label{es:prodeldet3}
  \begin{equation*}
    \det(A)=
    \begin{bmatrix}
      1 & 0 & 0\\
      2 & 3 & 5\\
      6 & 1 & 2
    \end{bmatrix}=1
    \begin{bmatrix}
      3 & 5\\
      1 & 2
    \end{bmatrix}=6-5=1
  \end{equation*}
  Se viene moltiplicata la prima colonna per 2
  \begin{equation*}
    \begin{bmatrix}
      2 & 0 & 0\\
      4 & 3 & 5\\
      12 & 1 & 2
    \end{bmatrix}=2
    \begin{bmatrix}
      3 & 5\\
      1 & 2
    \end{bmatrix}=2(6-5)=2 =2\det(A)
  \end{equation*}
\end{es}
\begin{pro}
  \label{pro:prodeldet4}
  Se gli elementi di una riga o di una colonna sono somma di due
  addendi, il determinante è la somma dei determinanti delle due matrici
  che si ottengono da $A$ sostiduendo agli elementi della colonna in
  questione i primi o i secondi addendi (e lasciando fissi gli altri)
  \begin{equation*}
    \begin{pmatrix}
      a_{11}& a_{12} + b_{12} & a_{13}\\
      a_{21}& a_{22} + b_{22} & a_{23}\\
      a_{31}& a_{32} + b_{32} & a_{33}
    \end{pmatrix}=
    \begin{pmatrix}
      a_{11} & a_{12} & a_{13}\\
      a_{21} & a_{22} & a_{23}\\
      a_{31} & a_{32} & a_{33}
    \end{pmatrix}
    +
    \begin{pmatrix}
      a_{11} & a_{12} & a_{13}\\
      a_{21} & a_{22} & a_{23}\\
      a_{31} & a_{32} & a_{33}
    \end{pmatrix}
  \end{equation*}
\end{pro}
\begin{es}
  \begin{equation*}
    \begin{pmatrix}
      1 & 5\\
      -2 & 7
    \end{pmatrix}=
    \begin{pmatrix}
      1 & 2\\
      -2 & 6
    \end{pmatrix}+
    \begin{pmatrix}
      1 & 3 \\
      -2 & 1
    \end{pmatrix}
  \end{equation*}
  Gli elementi della seconda colonna sono somma di due addendi. Il
  determinante a sinitra è 17. Mentre a destra $10+7$.
\end{es}
\begin{pro}
  \label{pro:prodeldet5}
  Scambiando fra loro due righe o due colonne di una matrice, il
  corrispondente cambia di segno.
\end{pro}
\begin{es}
  \label{es:prodeldet5}
  \begin{equation*}
    \begin{pmatrix}
      1 & 0 & 0\\
      2 & 3 & 5\\
      6 & 1 & 2
    \end{pmatrix}=1
    \begin{pmatrix}
      3 & 5\\
      1 & 2
    \end{pmatrix}= 6-5=1
  \end{equation*}
  Scambio seconda e terza riga
  \begin{equation*}
    \begin{pmatrix}
      1 & 0 & 0 \\
      6 & 1 & 2 \\
      2 & 3 & 5
    \end{pmatrix}=1
    \begin{pmatrix}
      1 & 2\\
      3 & 5
    \end{pmatrix}=5-6=-1
  \end{equation*}
\end{es}
\begin{pro}
  \ref{pro:prodeldet6}
  Il determinante di una matrice con due righe o due colonne uguali
  è nullo. Infatti lo scambio di tali righe (o colonne) non altera il
  determinante, ma per la Proprietà \ref{pro:prodeldet5} deve essere
  $\det (A)=-\det(A)$, quindi $\det(A)=0$.
\end{pro}
\begin{pro}
  \ref{pro:prodeldet7}
  Se agli elementi di una riga si sommano gli elementi di un'altro riga
  moltiplicata per un numero, il determinante non cambia. In particolare
  se $n=3$
  \begin{equation*}
    A=
    \begin{bmatrix}
      a_{11} & a_{12} & a_{13}\\
      a_{21} & a_{22} & a_{23}\\
      a_{31} & a_{32} & a_{33}
    \end{bmatrix}
  \end{equation*}
  Se per esempio alla seconda riga sommiamo la terza riga moltiplicata per
  un numero $c$, si ottiene la matrice
  \begin{equation*}
    A^\prime =
    \begin{bmatrix}
      a_{11} & a_{12} & a_{13}\\
      a_{21}+ca_{31} & a_{22}+ca_{32} & a_{23}+ca_{33}\\
      a_{31} & a_{32} & a_{33}
    \end{bmatrix}
  \end{equation*}
  Quindi, applicacndo prima la Proprietà \ref{pro:prodeldet4} poi la
  Proprietà \ref{pro:prodeldet3}, si ottiene
  \begin{equation*}
    \det(A^\prime)=
    \begin{bmatrix}
      a_{11} & a_{12} & a_{13}\\
      a_{21} & a_{22} & a_{23}\\
      a_{31} & a_{32} & a_{33}
    \end{bmatrix} +
    \begin{bmatrix}
      a_{11} & a_{12} & a_{13}\\
      ca_{21} & ca_{22} & ca_{23}\\
      a_{31} & a_{32} & a_{33}
    \end{bmatrix}=\det(A)+c\begin{bmatrix}
      a_{11} & a_{12} & a_{13}\\
      a_{21} & a_{22} & a_{23}\\
      a_{31} & a_{32} & a_{33}
    \end{bmatrix}=\det(A)
  \end{equation*}
  tenuto conto che il determinante di una matrice con due righe uguali è
  nullo.
\end{pro}
\begin{pro}
  \label{pro:prodeldet8}
  Se sono nulli tutti gli elementi di una riga o di una colonna,
  $\det(A)=0$.
\end{pro}
\begin{es}
  \label{es:prodeldet6}
  Sviluppando la formula di Laplace sulla riga con tutti zeri, si vede
  che $\det(A)=0$
  \begin{equation*}
    \begin{bmatrix}
      1 & 2 & 3\\
      0 & 0 & 0\\
      -2 & 1 & -3
    \end{bmatrix} = -0
    \begin{bmatrix}
      2 & 3\\
      1 & -3
    \end{bmatrix}+ 0
    \begin{bmatrix}
      1 & 3\\
      -2 & -3
    \end{bmatrix}-0
    \begin{bmatrix}
      1 & 2 \\
      -2 & 1
    \end{bmatrix}= 0
  \end{equation*}
\end{es}
\begin{pro}
  \label{pro:prodeldet9}
  Il determinate di una matrice è nullo se e solo se una sue riga
  (o colonna) è combinazione lineare delle altre righe (o colonne).
  Segue dalle precedenti proprietà, infatti, supponendo che la terza riga
  sia combinazione lineare delle altre due
  \begin{eqnarray*}
    \begin{bmatrix}
      a_{11} & a_{12} & a_{13}\\
      a_{21} & a_{22} & a_{23}\\
      c_1a_{11}+c_2a_{21} & c_1a_{12}+c_2a_{22} & c_1a_{13}+c_2a_{23}
    \end{bmatrix}=\\
    \begin{bmatrix}
       a_{11} & a_{12} & a_{13}\\
      a_{21} & a_{22} & a_{23}\\
      c_1a_{11} & c_1a_{12} & c_1a_{13}
    \end{bmatrix}+\begin{bmatrix}
      a_{11} & a_{12} & a_{13}\\
      a_{21} & a_{22} & a_{23}\\
      c_2a_{21} & c_2a_{22} & c_2a_{23}
    \end{bmatrix}=\\c_1\cdot
    \begin{bmatrix}
      a_{11} & a_{12} & a_{13}\\
      a_{21} & a_{22} & a_{23}\\
      a_{31} & a_{32} & a_{33}\\
    \end{bmatrix}+c_2\cdot \begin{bmatrix}
      a_{11} & a_{12} & a_{13}\\
      a_{21} & a_{22} & a_{23}\\
      a_{31} & a_{32} & a_{33}\\
    \end{bmatrix}=0
  \end{eqnarray*}
  in quanto ci sono due righe uguali.
\end{pro}
\begin{es}
  \label{es:prodeldet8}
  Calcolatore il determinante della matrice, cercando di applicare le
  proprietà, in modo da semplificare il calcolo
  \begin{eqnarray*}
    \begin{bmatrix}
      ad+4 & c+7 & 4b+5 & 5a\\
      2b+1 & bc + 1 & b + 1 & a\\
      3d - 4 & -2d + 6 & bd+2 & 2a\\
      -2c & 3c & c & ac
    \end{bmatrix} & a,b,c,d \in \mathds{R}
  \end{eqnarray*}
\end{es}
\begin{proof}[Soluzione]
  \label{sol:prodeldet1}
  bisogna ricondursi al calcolo del determinante di una matrice
  triangolare
  \begin{equation*}
    abc^2
    \begin{bmatrix}
      ad & 1 & 4 & 5\\
      0 & b & 1 & 1 \\
      0 & 0 & d & 2\\
      0 & 0 & 0 & 1
    \end{bmatrix}=a^2b^2c^2d^2
  \end{equation*}
\end{proof}
Il concetto di matrice è di fondamentale importanza e compatirà in molti
contesti in questo corso. Nel contesto dei sistemi di equazioni lineari,
non solo la matrice completa costituisce una ``\emph{fotografia}'' fedele
di un sistemi e contiene tutte le informazioni necessarie a determinarlo,
ma sarà anche l'oggetto sul ci si focalizzerà all'interno del percorso.\\
Per motivi che verranno speficicati in seguito, sarà importante prendere
in considerazione anche la matrice che contiene solo i coefficienti delle
incognite, senza l'ultima colonna formata dai termini noti: si ottiene
così la cosiddetta \textit{matrice dei coefficienti del sistema}. Ad
esempio, la matrice dei coefficienti del sistema
(\ref{eq:matricediunsistlineare1}) è
\begin{equation}
  \label{eq:determinante2}
  \begin{pmatrix}
    1 & 3\\
    2 & -1
  \end{pmatrix}
\end{equation}
Ora, rappresentare un sistema mediante la sua matrice completa consente
di identiicare ogni sua equazione con una $n$-upla: la generica equazione,
diciarando la $i$-esima
\begin{equation}
  \label{eq:determinante3}
  a_{i1}x_1+a_{i2}x_2+\cdots+a_{in}x_n=b_i
\end{equation}
è rappresentata nella matrice completa dall'$i$-esima riga $R_i$
\begin{equation*}
  \begin{pmatrix}
    a_{i1} & a_{i2} & \cdots & a_{in} &b_i 
  \end{pmatrix}
\end{equation*}
e questa riga può a sua volta essere pensata come la $n+1$-upla
$\begin{pmatrix}
    a_{i1} & a_{i2} & \cdots & a_{in} &b_i 
\end{pmatrix}\in \mathds{K}^{n+1}$ (dove $\mathds{K}$ è campo dei
coefficienti delle equazioni). Queste corrispondenza è tale che qualunque
delle equazioni del sistema, corrisponde a una combinazione lineare delle
righe corrispondenti, viste come elementi di $\mathds{K}^{n+1}$, si
ottiene
\begin{equation*}
  c(a_{i1}x_1+a_{i2}x_2+\cdots+a_{in}x_n)=cb_{i}
\end{equation*}
ovvero, svolgendo i conti a primo membro, la nuova equazione
\begin{equation*}
  ca_{i1}x_1+ca_{i2}x_2+\cdots+ca_{in}x_n=cb_i
\end{equation*}
e tale equazione corrisponde alla $(x+1)-$upla
\begin{equation*}
  cR_i=(ca_{i1},ca_{i1_2},\cdots,ca_{in},cb_i)
\end{equation*}
ottenuta moltiplicando la $(n+1)-$uple
$R_i=(a_{i1},a_{i1_2},\cdots,a_{in},b_1)$ (che rappresenta l'equazione
originale) per $c$.\\
Allo stesso modo, se si sommano membro a membro l'equazione
(\ref{eq:determinante3}) per un'altra equazione
$a_{j1}x_1+a_{j2}+\cdots+a_{jn}x_{n}=b_j$ del sistema, si ottiene
\begin{equation*}
  a_{i1}x_1+a_{i2}+\cdots+a_{in}x_{n}+a_{j1}x_1+a_{j2}+\cdots+a_{jn}x_{n}
  =b_i+b_j
\end{equation*}
ovvero, raccoglendo gli addendi che contengono la stessa incognita, messa
in evidenza,
\begin{equation*}
  (a_{i1}+a_{j1})x_1+(a_{i2}+a_{j2})x_2+\cdots+(a_{in}+a_{jn})x_n=b_i+b_j
\end{equation*}
si ottiene una nuova equazione rappresentata dalla $n+1$-upla
\begin{equation*}
  R_i+R_j=(a_{i1}+a_{j1},a_{i2}+a_{j2},\cdots,a_{in}+a_{jn}b_i+b_j)
\end{equation*}
che si ottiene sommando le $n+1-$uple
$R_i=(a_{j1},a_{j2},\cdots,a_{jn},b_j)$ che rappresentavano le due
equazioni originali. Quindi, eseguendo più in generale una combinazione di
due o più equazioni di un sistema, rappresentate dalle righe
$R_1,R_2,\dots,R_m\in \mathds{K}^{n+1}$, l'equazione ottenuta
corrisponderà corrisponderà a una combinazione
\begin{equation*}
  c_1R_1+c_2R_2+\cdots+c_{in}R_{in}
\end{equation*}
delle righe corrispondenti.\\
Ad esempio, nel sistema (\ref{eq:eqlinematrici5}), se come visto in
(\ref{eq:eqlinematrici7}) moltiplicando (membro a membro) la prima
equazione per 5 e poi bisogna sommare alla seconda moltiplicata per $-3$,
ottenendo la nuova equazione $-x_2+2x_2+4x_3=-1$. Nella matrice completa
\begin{equation}
  \label{eq:determinante4}
  \begin{bmatrix}
    1 & 1 & 1 & 1\\
    2 & 1 & 3 & 2
  \end{bmatrix}
\end{equation}
l'operazione corrispondente non è nient'altro che la combinazione lineare
\begin{equation*}
  5R_1+(-3)R_2=5(1,1,1,1)+(-3)(2,1,3,2)=(-1,2,-4,-1)
\end{equation*}
delle sue due righe (viste come elementi di $\mathds{R}^4$).\\
Inoltre, se una terna $(x_1,x_2,x_3)$ soddisfa il sistema
(\ref{eq:eqlinematrici5}), essa soddisferà anche l'equazione
$-x_1+2x_2-4x_3=-1$, e in generale soddisferà tutte le possibile equazioni
che si corrispondono alle combinazioni lineari delle righe $R_1$ e $R_2$
della matrice completa.\\
In generale, dato un sistema di $m$ equazioni in $n$ incognite, con
matrice completa avente come righe $R_1,R_2,\dots,R_m$, una $n-$upla
$(x_1,x_2,\cdots,x_n)$ che soddisfa il sistema verifica anche tutte le
equazioni corrispondenti alle $(n+1)-$uple del tipo
$c_1R_1+c_2R_2+\cdots+c_mR_m$, ovvero quelle appartenenti al sottospazio
\begin{equation*}
  (R_1,R_2,\cdots,R_m)
\end{equation*}
generato dalle righe $R_1,R_2,\cdots,R_m$ (viste come elementi di
$\mathds{K}^{n+1}$), in quanto per definizione tale sottospazio è formato
proprio da tutte le combinazioni lineari $c_1R_1+c_2R_2+\cdots+c_mR_m$.\\
Da queste osservazioni si può dedurre il seguente riguente risultato, che
ci fornisce un criterio sufficiente perché due sistemi siano
\textit{equivalenti}, ovvero abbiano le stesse soluzioni:
\begin{prop}
  \label{prop:determinante1}
  Siano dati due sistemi di equazioni lineari in $n$ incognite, il
  primo con matrice completa formata dalle righe $R_1,R_2,\cdots,R_m$ e
  il secondo con matrice complate formata dalle righe
  $R_1,R_2,\cdots,R_i$. Se
  \begin{equation*}
    (R_1,R_2,\cdots,R_m)=(\bar{R_1},\bar{R_2},\cdots,\bar{R}_i)
  \end{equation*}
  allora i due sistemi sono equivalenti.
\end{prop}
\begin{proof}
  Se una $n-$upla $x=(x_1,x_2,\dots,x_n)$ è una soluzione del primo
  sistema, allora essa verifica tutte le sue equazioni, rappresentate
  dalle righe $R_1,R_2,\dots,R_m$ della sua matrice completa,
  rappresentate dalle sua matrice completa. Come osservato sopra, essa
  verifica allora anche tutte le equazioni corrispondenti alle
  $(n+1)$-upla uguale a $(\bar{R}_1,\bar{R}_2,\dots,\bar{R}_4)$, come
  affermato nell'ipotesi, $x$ verifica quindi tutte quindi tutte
  le equazioni corrispondenti alle $(n+1)-$uple del sottospazio
  $\left\langle \bar{R}_1,\bar{R}_2\dots,\bar{R}_4
\right\rangle$, e in
  particolare\footnote{All'interno del sottospazio $(v_1,v_2,\dots,v_n)$
    generato da un insieme di vettori e costituito come da tutte le
    combinazioni lineari $c_1v_1+c_2v_2+\dots+c_nv_n$ ci sono sempre
    anche i vettori $v_1,v_2,\dots,v_n$m stessi, in quanto ciascuno di
    loro può essere espresso come combinazione lineare:
    $v_1=1v_1+0v_2+\cdots+0v_n$, $v_2=0v_1+1v_2+\cdots+0v_n$, e così via.}
  $\bar{R}_1,\bar{R}_2,\dots,\bar{R}_4$, stesse, che rappresentano le
  equazioni del secondo sistema: quindi $x$ è soluzione anche del secondo
  sistema.\\
  Viceversa\footnote{Dimostrare che i due sistemi hanno le stesse
    soluzioni significa dimostrare che l'insieme delle soluzioni del
    primo è uguale all'insieme delle soluzioni del secondo, ovvero (come
    previsto dalla definizione di ugualianza di insiemi) che ogni
    $n$-upla che è soluzione del primo sistema è soluzione anche del
    secondo, e viceversa ogni $n$-upla soluzione del secondo sistema è
    anche soluzione del prima.}, se una $n$-upla $x=(x_1,x_2,\dots,x_n)$
  è una soluzione del secondo sistema, allora essa verifica tutte le sue
  equazioni, Quindi essa verifica allora anche tutte le equazioni
  corrispondenti alla $(n+1)-$uple contenute nel sottospazio $\left\langle
    \bar{R}_1,\bar{R}_2,\dots,R_t
  \right\rangle$. Essendo tale sottospazio uguale a $\left\langle
    R_1,R_2,\dots,R_m
  \right\rangle$, e in particolare $R_1,R_2,\dots,R_m$ stesse, che
  rappresentano le equazioni del primo sistema: quindi $x$ è soluzione
  anche del primo sistema.
\end{proof}
I metodo che useremo per risolvere un sistema, consiste proprio nel
trasformare il sistema dato in un sistema equivalente più semplice, nel
quale verranno eliminate tutte le equazioni superflue (che si ottengoo
come combinazione delle altre).

\section{L'algoritmo di eliminazione di Gauss-Jordan (o di riduzione a gradini)}
\label{sec:gauss-jordan}
Il metodo che vedremo ora per risolvere un qualunque sistema con $m$
equazioni lineari in $n$ incognite può essere spiegato come una
generalizzazione dei metodi tradizionalmente usati per la risoluzione dei
sistemi di due equazioni in due incognite. Per ricordare quali sono
questi metodi, prendiamo ad esempio il sistema
\begin{equation}
  \label{eq:gauss-jorda1}
  \begin{cases}
    x_1+x_2=0\\
    -x_1+x_2=1
  \end{cases}
\end{equation}
Solitamente, per risolvere tale sistema si ricava una delle incognite in
funzioni dell'altra usando una delle due equazioni, ad esempio dalla prima
equazione si trova $x_1=-x_2$, e si sostituisce l'espressione così
ottenuta nell'altra equazione:
\begin{equation*}
  -(-x_2)+x_2=1
\end{equation*}
ovvero
\begin{equation*}
  2x_2=1
\end{equation*}
In questo modo, è stato \emph{eliminato} la prima incognita dalla seconda
equazione che è diventata una semplice equazione di primo grado con una
sola incognita, che ha come soluzione $x_2=\frac{1}{2}$. A questo punto,
per ricavare $x_1$ basta sostituire il valore ottenuto di $x_2$ nella
prima equazione, ovvero
\begin{equation*}
  x_1+\frac{1}{2}=0 \to x_1=-\frac{1}{2}.
\end{equation*}
Quello che ha permesso di semplificare il sistema è stato quindi aver
ridotto il numero di incognite presenti in una delle equazioni. Allo
stesso risultato si può arrivare, equivalentemente, ad esempio sommando
membro a membro le due equazioni se $x_1+x_2=0$ e $-x_1+x_2=1$ allora
\begin{equation*}
  (x_1+x_2)+(-x_1+x_2)=0+1
\end{equation*}
ovvero, facendo i conti, si ottiene come sopra $2x_2=1$.\\
Questo secondo metodo, apparentemente più artificioso, in realtà si rivela
più semplice se si lavora sulla matrice completa del sistema invece, che
sulle equazioni. Infatti, la matrice completa del sistema
(\ref{eq:gauss-jorda1}) è
\begin{equation}
  \label{eq:gauss-jorda2}
  \begin{pmatrix}
    1 & 1 & 0\\
    -1 & 1 & 1
  \end{pmatrix}
\end{equation}
Sommare membro a membro le due equazioni equivale a sommare tra loro le
due: sostituendo poi tale somma alla seconda riga originale si ottiene,
senza dover maneggiare le incognite e dover fare sostituzioni o
semplificazioni.
\begin{equation}
  \label{eq:gauss-jorda3}
  \begin{pmatrix}
    1 & 1 & 0\\
    0 & 2 & 1
  \end{pmatrix}
\end{equation}
che corrisponde proprio al sitema ridotto
\begin{equation*}
  \begin{cases}
    x_1+x_2=0\\
    2x_2=1
  \end{cases}
\end{equation*}
risolvibile come visot sopra risolvendo prima l'equazione con una sola
incognita.\\
Questo stesso procedimento di eliminazione di incognite, realizzato
lavorando sulle righe della matrice completa, funziona in realtà per
risolvere qualunque sistema, qualunque sia il numero di equazioni e il
numero di incognite. Più precisamente, l'obbiettivo è avere il minor
numero numero di incognite possibile per garantire un risultato.\\
Se, per definire un criterio, si segne di eliminarle di seguito l'odine
$x_1,x_2,\dots, x_n$, questo significa che le righe della matrice completa
inizino con un numero sempre maggiore di zero.
\begin{equation*}
  \begin{pmatrix}
    1 & 1 & 1 & 1\\
    0 & 2 & 3 & 2\\
    0 & 0 & 4 & 5
  \end{pmatrix}
\end{equation*}
nella quale le righe iniziano con un numero sempre maggiore di zeri, ha
come sistema corrispondente
\begin{equation*}
  \begin{cases}
    x_1+x_2+x_3=1\\
    2x_2+2x_3=2\\
    4x_3=5
  \end{cases}
\end{equation*}
che ha la proprietà desiderata che le sue equazioni presentano un numero
decrescente di incognite.
Dopo questo è possibile rare la seguente
\begin{defi}
  \label{defi:gauss-jorda1}
  Una matrice si dice a gradini se, andando dalla prima all'ultima,
  ogni riga inizia con un numero sempre maggiore di zeri.\\
  Il primo elemento non nullo in ogni riga di una matrice a gradini si
  chiama \textit{pivot}.
\end{defi}
In altre parole, una matrice è a gradini se in ogni riga il primo elemento
non nullo compare con un indice di colonna sempre più grande. Ad esempio,
delle matrici seguenti
\begin{eqnarray*}
  \begin{pmatrix}
    7 & 1 & 1 & 3\\
    0 & 4 & 3 & 5\\
    0 & 0 & 0 & 6
  \end{pmatrix}, &
                   \begin{pmatrix}
                     7 & 1 & 1 & 3\\
                     0 & 1 & 3 & 5\\
                     0 & 2 & 0 & 6
                   \end{pmatrix}, &
                                   \begin{pmatrix}
                                     7 & 1 & 1 & 3\\
                                     0 & 0 & 3 & 5\\
                                     0 & 4 & 0 & 6
                                   \end{pmatrix}
\end{eqnarray*}
la prima è a gradini gradini perché i suoi pivot (7 nella prima riga,
4 nella seconda e 6 nella terza) si trovano, nell'ordine, sulla prima,
seconda e quarta colonna (indice di colonna sempre più grande), mentre
le altre no (nella seconda, il primo elemento non nullo della terza riga
sta nella stessa colonna del primo elemento non nullo della seconda riga;
nella terza matrice, il primo elemento non nullo della terza riga sta in
una colonna di indice più piccolo del primo elemento non nullo della
seconda riga). Un sistema si dice a gradini se la sua matrice completa è
una matrice a gradini.\\
Il procedimento che desscritto qui di seguito è chiamato \textit{metodo
  di riduzione a gradini} o, dal momento che consiste nell'eliminare
incognite, \textit{metodo di eliminazione di Gauss-Jordan}.\\
Il procedimento di riduione a gradoni, oltre a semplificare il sistema, fa
emergere anche le eventuali incompatibilità e le eventuali equazioni
superflue presenti nel sistema.\\
Per trasformare un sistema in un sistema a gradini, trasformeremo la sua
matrice completa in una matrice a gradini tramite le seguenti operazioni
slle sue righe, dette \textit{operazioni elementari di primo, secondo
  terzo tipo}:
\begin{description}
\item[primo tipo] Scambiare tra loro due righe della matrice (
  $R_i \leftrightarrow R_j$)
\item[secondo tipo] Moltiplicare una riga della matrice per un
  coefficiente non nullo ($R_i\to cR_i$, con $c\neq 0$)
\item[terzo tipo] Sommare a una riga della matrice un'altra riga
  moltiplicata per un mumero qualunque ($R_i=R_i+dR_j$)
\end{description}
Il fatto importante è che tali operazioni, che modificano le righe,
corrispondono a modificare le equazioni del sistema \textit{in modo però
  da non combiare l'insieme delle soluzioni,} come dimostra il seguente
risultato, corollario delle Proposizione \ref{prop:determinante1}:
\begin{prop}
  \label{prop:determinante2}
  Se viene effettuate operazioni elementari di primo, secondo e terzo
  tipo sulla matrice completa di un sistema, la matrice trasformeta è la
  matrice completa di un sistema \textit{equivalente} a quello iniziale
  (ovvero avente le stesse soluzioni del sistema iniziale).
\end{prop}
Dopo questo, verrà illustrato come mediante l'uso delle tre operazioni
elementari, ogni sistema possa essere trasformato in un sistema a gradini:
sia
\begin{equation}
  \label{eq:gauss-jorda4}
  A=
  \begin{pmatrix}
    a_{11} & a_{12} & \cdots & a_{1n} & b_1\\
    a_{21} & a_{22} & \cdots & a_{2n} & b_2\\
    a_{31} & a_{32} & \cdots & a_{3n} & b_3\\
    \vdots & \vdots & \vdots & \vdots\\
    a_{m1} & a_{m2} & \cdots & a_{mn} & b_m
  \end{pmatrix}
\end{equation}
La matrice completa del generico sistema (\ref{eq:eqlinematrici2}).\\
L'algoritmo inizia come segue: se la prima entrata $a_{11}$ della prima
riga è uguale a zero, le si scambia tra loro nelle due righe (applicando
quindi un'operazione elementare del primo tipo) in modo da garantire che
la nuova entrata $a_{11}$ sia diversa da zero. Fatto ciò, si applica alla
matrice (\ref{eq:gauss-jorda4}) le seguenti operazioni elementari (del
terzo tipo) sulle righe $R_2,\dots,R_m$ dalla seconda all'ultima:
\begin{equation*}
  \begin{matrix}
    R_2\to R_2-\frac{a_{21}}{a_{11}}R_1\\
    R_{3}\to R_3-\frac{a_{31}}{a_{11}}R_1\\
    \vdots\\
    R_{m}\to R_m-\frac{a_{m1}}{a_{11}}R_1
  \end{matrix}
\end{equation*}
(si noti che le operazioni si possono applicare proprio perché
$a_{11}\neq 0$). Queste trasformazioni rendono sicuramente uguale a zero
la prima entrata di ogni riga dalla secomnda in poi, e eventualmente
potrebbero aver annullato anche altre entrate, ovvero trasformano la
matrice (\ref{eq:gauss-jorda4}) in una matrice seguente tipo
\begin{equation}
  \label{eq:gauss-jorda5}
  \begin{pmatrix}
    a_{11} & \dots & \dots & \dots & \dots & a_{1n} & b_1\\
    0 & \dots & 0 & a_{2k}^\prime & \dots & a_{2n}^\prime & b_2^\prime\\
    0 & \dots & 0 & a_{3k}^\prime & \dots & a_{3n}^\prime & b_3^\prime\\
           & \cdots\\
    0 & \dots & 0 & a_{mk}^\prime & \dots & a_{mn}^\prime & b_m^\prime
  \end{pmatrix}
\end{equation}
si può supporre che la siconda rica sia quella che inizia con il minor
numero di zeri, con $a_{2k}^\prime\neq 0$. A questo punto, si ripete quanto
fatto nella prima parte della trasformazione, applicando stavolta alle
righe dalla terza in poi le trasformazioni elementari del terzo tipo
\begin{equation*}
  \begin{matrix}
    R_3\to R_3-\frac{a^\prime_{3k}}{a_{2k}^\prime}R_2\\
    \vdots\\
    R_m\to R_m-\frac{a^\prime_{mk}}{a_{2k}^\prime}R_2\\
  \end{matrix}
\end{equation*}
che sono tali da annullare la prima entrata non nulla dalla terza riga in
poi, ovvero da trasformare la matrice in una matrice del tipo
\begin{equation}
  \label{eq:gauss-jorda6}
  \begin{pmatrix}
    a_{11} & \dots & \dots & \dots & \dots & a_{1n} & b_1\\
    0 & \dots & 0 & a_{2k}^\prime & \dots & a_{2n}^\prime & b_2^\prime\\
    0 & \dots & 0 & 0 & \dots & a_{3n}^{\prime\prime} & b_3^{\prime\prime} \\
           &\dots\\
    0 & \dots & 0 & 0 & \dots & a_{mn}^{\prime\prime} & b_{m}^{\prime\prime}
  \end{pmatrix}
\end{equation}
In questo modo si può trasformare la matrice del sitema in una matrice
ogni riga inizia con un numero sempre maggiore di zeri, ovvero nella
matrice a gradini voluta, e il sistema corrispondente sarà equivalente al
sistema iniziale in quanto la trasmissione è stata effettuata con
operazioni elementari. Il modo migliore di capire questo procedimento è
con un esempio.
\begin{es}
  \label{es:gauss-jorda1}
  Sia
  \begin{equation}
    \label{eq:gauss-jorda1-1}
    \begin{cases}
      x_1+x_2+x_3=1\\
      -x_1+x_2-x_3=0\\
      -x_1+x_2+x_3=-3
    \end{cases}
  \end{equation}
  il sistema con matrice completa
  \begin{equation}
    \label{eq:gauss-jorda1-2}
    \left(
      \begin{array}{ccc|c}
        1 & 1 & 1 & 1\\
        -1 & 1 & -3 & 0\\
        -1 & 1 & 1 & -3
      \end{array}\right)
  \end{equation}
  Ora, è possibile trasformare tale matrice in una matrice a gradini
  usando le operazioni elementari, in modo da ottenere un sistema a
  gradini equavalente al sistema (\ref{eq:gauss-jorda1-1}).\\ Ricordando
  che, in base alla definizione di matrice a gradini, visto che il primo
  elemento $a_{11}$ della prima riga è diverso da zero, e sta nella prima
  colonna, i primi elementi diversi da zero della seconda e della terza
  riga non possono essere anche loro nella prima colonna: in altre parole,
  è mecessario trasformare la matrice in modo che $a_{21}$ e $a_{31}$
  siano uguali a zero.\\
  Ottenendo sicuramente lo scopo se si applicano anche le operazioni
  elementari del terzo tipo $R_2\to R_2+R_1$ e $R_3\to R_3+R_1$: infatti,
  \begin{eqnarray*}
    \left(
      \begin{array}{ccc|c}
        1 & 1 & 1 & 1\\
        -1 & 1 & -3 & 0\\
        -1 & 1 & 1 & -3
      \end{array}\right)& \overrightarrow{
        \begin{matrix}
          R_2\to R_2+R_1\\
          R_3\to R_3+R_1
        \end{matrix}
      }& \left(
        \begin{array}{ccc|c}
          1 & 1 & 1 & 1\\
          0 & 2 & -2 & 1\\
          0 & 2 & 2 & -2
        \end{array}\right)
  \end{eqnarray*}
  La matrice trasformata non è ancora una matrice a gradini in quanto il
  primo elemento non nullo della terza riga si trova in corrispondenza
  della stessa colonna (la seconda) del primo elemento non nullo nella
  seconda riga: è decessario che $a_{32} = 0$ (diventi nullo). A questo
  scopo, basta applicare l'operazione elementare $R_3\to R_3-R_2$:
  così facendo si ottiene
  \begin{eqnarray*}
    \left(
        \begin{array}{ccc|c}
          1 & 1 & 1 & 1\\
          0 & 2 & -2 & 1\\
          0 & 2 & 2 & -2
        \end{array}\right) & \overrightarrow{
                             R_3\to R_3-R_2} &
                                               \left(
        \begin{array}{ccc|c}
          1 & 1 & 1 & 1\\
          0 & 2 & -2 & 1\\
          0 & 0 & 4 & -3
        \end{array}\right)
  \end{eqnarray*}
  E alla fine di questo processo si otterà come sistema:
  \begin{equation}
    \label{eq:gauss-jorda1-3}
    \begin{cases}
      x_1+x_2+x_3=1\\
      2x_2+2x_3=1\\
      4x_3=-3
    \end{cases}
  \end{equation}
  corrispondente alla matrice trasformata è, equivalente al sistema
  originale (\ref{eq:gauss-jorda1-1}), quindi trovando la sua risoluzione
  trovando la soluzione del sistema (\ref{eq:gauss-jorda1-1}).
  \begin{equation*}
    2x_2-2x_3=1\to 2x_2=1+2x_3=1+2\left(-\frac{3}{4}\right)=1-\frac{3}{2}
    =-\frac{1}{2} \to x_2=-\frac{1}{4}
  \end{equation*}
  e analogamente, sostituendo i valori di $x_2$ e $x_3$ così ottenuti
  nella prima equazione si trova
  \begin{equation*}
    x_1+x_2+x_3=1\to x_1=1-x_2-x_3= 1 - \left(-\frac{1}{4}\right)-\left(-
      \frac{3}{4}\right)=2
  \end{equation*}
  Avendo quindi la terna $
  \begin{pmatrix}
    2, & -\frac{1}{4}, & -\frac{3}{4}
  \end{pmatrix}
  $ è l'unica soluzione del sistema (\ref{eq:gauss-jorda1-3}), ovvero
  del sistema iniziale (\ref{eq:gauss-jorda1-1}).
\end{es}
La riduzione a gradini non solo semplifica il sitema grazie alla
eliminazione di incognite, ma mette anche in evidenza eventuali
``equazioni superflue'' e incompatibilità tra le equazioni.
\begin{es}
  \label{es:gauss-jorda2}
  Considerando il sistema
  \begin{equation}
    \label{eq:gauss-jorda2-1}
    \begin{cases}
      x_1+x_2+x_3=1\\
      x_1-x_2-x_3=0\\
      x_1+3x_2+3x_3=1
    \end{cases}
  \end{equation}
  che ha come matrice completa
  \begin{equation}
    \label{eq:gauss-jorda2-2}
    \left(\begin{array}{ccc|c}
      1 & 1 & 1 & 1   \\
      1 & -1 & -1 & 0 \\
      1 & 3 & 3 & 1
    \end{array}\right)
  \end{equation}
  Come fatto per il sistema precedente, trasformando tale matrice in una
  matrice a gradini mediante operazioni elementari.
  \begin{eqnarray*}
    \left(\begin{array}{ccc|c}
      1 & 1 & 1 & 1   \\
      1 & -1 & -1 & 0 \\
      1 & 3 & 3 & 1
    \end{array}\right)& \overrightarrow{
        \begin{matrix}
          R_2\to R_2+R_1\\
          R_3\to R_3+R_1
        \end{matrix}
      }& \left(\begin{array}{ccc|c}
        1 & 1 & 1 & 1   \\
        1 & -2 & -2 & -1 \\
        0 & 2 & 2 & 0
      \end{array}\right)\\
    \left(\begin{array}{ccc|c}
        1 & 1 & 1 & 1   \\
        1 & -2 & -2 & -1 \\
        0 & 2 & 2 & 0
      \end{array}\right) & \overrightarrow{
                             R_3\to R_3-R_2} &
                              \left(\begin{array}{ccc|c}
                                1 & 1 & 1 & 1   \\
                                0 & -2 & -2 & -1 \\
                                0 & 0 & 0 & -1
                              \end{array}\right)
  \end{eqnarray*}
  Notare che la terza riga dalla matrice trasformata corrisponde
  all'equazione $0x_1+0x_2+0x_3=-1$, ovvero $0=-1$: poiché questa
  uguaglianza è falza, non esiste nessuna terna che soddisfi le tre
  condizioni del sistema ridotto corrispondente, ovvero tale sistema non
  ha soluzioni. Questo, in virtù dell'equivalenza tra il sistema originale
  e quello ridotto, questo indica che il sistema di partenza non ha
  soluzioni, ovvero è incompatibile.\\
  Evidentemente tra le equazioni del sistema di partenza vi era una
  incompatibilità non evidente che il procedimento di riduzione a gradini
  ha fatto emergere: infatti, se si moltiplica membro a membro la prima
  equazione per 2 e si sottrae la seconda equazione si ottiene
  \begin{eqnarray*}
    2(x_1+x_2+x_3)-(x_1-x_2-x_3)=2\cdot 1 - 0
  \end{eqnarray*}
  ovvero, svolgendo i calcoli, $x_1+3x_2+3x_3=2$. Quasta condizione, che è
  conseguenza delle prime due equazioni ed è quindi soddisfatta da
  qualunque terna le soddisfi, è chiaramente incompatibile con la terza
  equazione; il procedimento di riduzione a gradini ha messo alla luce
  questa incompatibilità trasformandola nell'incompatibilità evidente
  $0=-1$.\\
  Considerando ora come ultimo esempio il sistema
  \begin{equation}
    \label{eq:gauss-jorda2-3}
    \begin{cases}
      x_1+x_2+3x_3=1\\
      x_1-2x_2+x_3=0\\
      x_1-5x_2-x_3=-1
    \end{cases}
  \end{equation}
  che ha come matrice completa
  \begin{equation}
    \label{eq:gauss-jorda2-4}
    \left(
      \begin{array}{ccc|c}
        1 & 1 & 3 & 1\\
        1 & -2 & 1 & 0\\
        1 & -5 & -1 & -1
      \end{array}\right)
    \end{equation}
    Applicando operazioni elementari per ridurre a gradini,
    \begin{eqnarray}
      \label{eq:gauss-jorda2-4}
      \left(
      \begin{array}{ccc|c}
        1 & 1 & 3 & 1\\
        1 & -2 & 1 & 0\\
        1 & -5 & -1 & -1
      \end{array}\right) & \overrightarrow{
        \begin{matrix}
          R_2\to R_2+R_1\\
          R_3\to R_3+R_1
        \end{matrix}
      }& \left(
      \begin{array}{ccc|c}
        1 & 1 & 3 & 1\\
        0 & -3 & -2 & -1\\
        0 & -6 & -4 & -2
      \end{array}\right)\\
       \left(
       \begin{array}{ccc|c}
         1 & 1 & 3 & 1\\
         0 & -3 & -2 & -1\\
         0 & -6 & -4 & -2
       \end{array}\right) & \overrightarrow{
                             R_3\to R_3-R_2} &
                              \left(
      \begin{array}{ccc|c}
        1 & 1 & 3 & 1\\
        0 & -3 & -2 & -1\\
        0 & 0 & 0 & 0
      \end{array}\right)
    \end{eqnarray}
    In questo caso avviene un qualcosa di particolare, infatti, la terza
    equazione del sistema si annulla come, si evince dalla
    $0x_1+0x_2+0x_3 = 0$, ovvero $0 = 0$.\\
    Quindi trasformando da matrice a sistema
    \begin{equation}
      \label{eq:gauss-jorda2-5}
      \begin{cases}
        x_1+x_2+3x_3=1\\
        -3x_2-2x_3=-1
      \end{cases}
    \end{equation}
    Benché non sia rimasta un'equazione con una sola incognita come nel
    primo sistema appena risolto, è possibile comunque progedere
    nel seguente modo:\\
    Bisogna ricavare $x_2$ dalla seconda equazione:
    \begin{equation}
      \label{eq:gauss-jorda2-6}
      -3x_2-2x_3=-1\to -3x_2=2x_3-1=-\frac{2}{3}x_3+\frac{1}{3}
    \end{equation}
    e sostituendo l'espressione ottenuta nella prima equazione per
    ricavare $x_1$:
    \begin{equation}
      \label{eq:gauss-jorda2-7}
      x_1+x_2+3x_3=1\to x_1=1-x_2-x_3=1-\left(9\frac{2}{3}x_3+\frac{1}{3}
        \right)-3x_3=\frac{2}{3}-\frac{7}{3}x_3.
    \end{equation}
    Ora, qualunque valore $t\in \mathds{R}$ assegnando a $x_3$, la
    (\ref{eq:gauss-jorda2-6}) e la (\ref{eq:gauss-jorda2-7}) dice che se
    si pone $x_2=-\frac{2}{3}t+\frac{1}{3}$ e
    $x_1=\frac{2}{3}-\frac{7}{3}t$, le equazioni del sistema saranno
    soddisfatte, ovvero si otterrà una soluzione. Espresso in altri
    termini, le soluzioni del sistema sono esattamente tutte le terne
    del tipo $\left(\frac{2}{3}-\frac{7}{3}t, -\frac{2}{3}t+\frac{1}{3},t
    \right)$ al variare di $t\in \mathds{R}$: il sistema ha quindi
    infinite soluzioni.\\
    Più precisamente, dal momento che le infinite
    soluzioni del sistema dipendono da un solo parametro libero $t$, si
    dice che il sistema ha ``infinito alla uno'' (si scrive $\infty^1$)
    soluzioni.
\end{es}
In genere, è possibile dare la seguente
\begin{defi}
  \label{defi:gauss-jorda1}
  Un sistema di equazioni lineari ha $\infty^k$ soluzioni se
  l'espressione generale della sua soluzione dipende da $k$ parametri
  liberi.
\end{defi}
Nell'ultimo esempio esposto, la riduzione ha eliminato delle tre
equazioni del sistema riducendola all'identità $0=0$. In effetti, non è
difficile vedere che la terza equazione $x_1-5x_2-x_3=-1$ era un'equazione
``superflua'', o più precisamente dipendente dalle altre due: come si
vede nella matrice completa (\ref{eq:gauss-jorda2-4}), la terza riga, che
la rappresenta, è combinazione delle altre due:
\begin{equation*}
  (1,-5,-1,-1)=-(1,1,3,1)+2(1,-2,1,0)
\end{equation*}
In effetti, non è difficile vedere che una matrice a gradini, escluse le
righe nulle, non ha più righe dipendenti (e quindi le equazioni non
nulle di un sistema ridotto a gradini sono sicuramente indipendenti):
\begin{prop}
  \label{prop:gauss-jorda1}
  Le righe non nulle di una matrice ridotta a gradini sono
  linearmente indipendenti
\end{prop}
\begin{proof}
  Per definizione di matrice a gradini le sue righe saranno del tipo
  \begin{eqnarray*}
    R_1=(a_{11},\dots), & a_{11}\neq0\\
    R_2=(0,\dots, 0, a_{2k},\dots,), & a_{2k}\neq 0\\
    R_3=(0,\dots,0,0,\dots,0,a_{3j},\dots,), & a_{3j}\neq 0\\
    \vdots
  \end{eqnarray*}
  con $k>1,j>k$ etc, ovvero in ogni riga il primo elemento non nullo
  compare via via con secondo indice sempre più grande.\\
  Ora, per dimostrare che tali righe sono indipendenti basta supporre che
  tali righe sono indipendenti basta supporre che
  \begin{equation}
    \label{eq:gauss-jorda3}
    c_1(a_{11},\dots)+c_2(0,\dots,0,a_{2k},\dots,)+c_3(0,\dots,0,0,\dots,
    0,a_{3j},\dots,)+\dots=(0,\dots,0)
  \end{equation}
  e dimostrare che i coefficinti $c_1,c_2,c_3$, etc. Devono essere
  necessariamente tutte nulli.\\
  Andando a guardare cosa significa l'uguaglianza (\ref{eq:gauss-jorda3})
  vedendo che nella prima entrata rimane solo $c_1a_{11}=0$: ma, essendo
  per ipotesi $a_{11}\neq 0$, necessariamente deve essere $c_1=0$. Quindi,
  la (\ref{eq:gauss-jorda3}) si riduce a
  \begin{equation}
    \label{eq:guess-jorda4}
    c_2(0,\dots,0,a_{2k},\dots,)+c_3(0,\dots,0,0,\dots,a_{3j},\dots,)+
    \dots= (0,\dots,0)
  \end{equation}
  Ora, guardando la $k$-esima entrata di questa relazione (cioè la prima
  diversa da zero nella seconda riga): dal momento che tutte le righe
  successive alla seconda hanno la prima entrata diversa da zero con
  indice più alto, si ottiene $c_2a_{2k}=0$, che, essendo $a_{2k}\neq 0$,
  dice che $c_2=0$.\\
  Dunque la (\ref{eq:guess-jorda4}) si riduce a
  \begin{equation*}
    c_3(0,\dots,0,9,\dots,a_{3j},\dots,)+\dots=(0,\dots,0)
  \end{equation*}
  e, continuando a ragionare in questo modo, si vedranno tutti i
  coefficenti $c_i$ si devono annullare, e quindi non può esistere una
  combinazione lineare delle righe uguale al vettore nullo e con
  coefficienti non tutti nulli, ovvero le righe sono indipendenti.
\end{proof}
\begin{oss}
  \label{oss:guess-jorda1}
  Quando si effettua delle operazioni elementari sulle righe di una
  matrice, si può considerare anche le trasformazioni del tipo $R_i\to
  cR_i+dR_{j}$, purché il coefficiente $c$ per cui bisogna moltiplicare
  la riga $R_i$ in modo che sostituendo non sia zero: infatti, benché
  tale traormazione non sia una delle tre operazioni elementari, essa
  può essere pensata come il risultato dell'applicando alla riga $R_i$
  l'operazione elementare del secondo tipo $R_i\to cR_i$ (con $c\neq 0$
  come previsto) e poi applicando alla nuova riga $cR_i$ così ottenuta
  l'operazione elementatre del terzo tipo $cR_i=cR_i+dR_j$.
\end{oss}
Il procedimento di riduzione a gradini, che è stato utilizzato come
strumento di risoluzione di un sistema, in realtà è sostanzialmente un
motodo che stabilisce se dei vettori sono indipendenti. Infatti, più
precisamente, si riscontrano i seguenti fatti:
\begin{enumerate}
\item Il procedimento non modifica lo spazio generato dalle righe, ovvero
  se $R_1,R_2,\dots,R_m$ sono le righe della matrice iniziale, e
  $\bar{R}_1,\bar{R}_2,\dots,\bar{R}_m$ sono le righe della matrice
  trasformata, allora $(R_1,R_2,\dots,R_m)=(\bar{R}_1,\bar{R}_2,\dots,
  \bar{R}_m)$;
\item le righe non nulle alla fine del procedimento formano un insieme di
  vettori indipendenti.
\end{enumerate}
Quindi, se le righe non nulle dopo la riduzione sono le prime $l$, ovvero
$\bar{R}_1,\bar{R}_2,\dots,R_i$, dal fatto che queste sono indipendenti
deducendo che consituiscono una base del sottospazio da loro generato, e
quindi
\begin{equation*}
  \dim(\bar{R_1},\bar{R_2},\bar{R_i})=l;
\end{equation*}
ma poiché il sottospazio generato non cambia, si può concludere che
\begin{equation*}
  \dim(R_1,R_2,R_m)=l.
\end{equation*}
Il numero di righe non nulle rimaste dopo la riduzione a gradini ci dà
quindi la dimensione dello spazio generato dalle righe iniziali: in uno
spazio di dimensione $l$ ci sono al massimo $l$ vettori indipendenti, e
le restanti di righe non nulle rimaste dopo la riduzione dimostra che
quante righe indipendenti  aveva la matrice prima della riduzione. Questa
informazione, giustifica la seguente
\begin{defi}
  \label{defi:gauss-jorda2}
  Il massimo numero di righe indipendenti di una matrice $A$ si
  chiama il \textit{rango per righe} di $A$.\\
  Quindi la riduzione a gradini definisce un modo per calcolare la
  dimensione di uno spazio e per definizione di uno spazio e per
  verificare se delle $n$-uple date siano indipendenti.\\
  Ad esempio, considerando le seguenti 4-uple
  \begin{equation}
    \label{eq:gauss-jorda2-1}
    \begin{matrix}
      (1,1,2,1,), & (1,2,1,0), & (1,-1,4,3), & (2,1,1,0)
    \end{matrix} 
  \end{equation}
  vedendo di determinare se esse siano o meno indipendenti. Disponendo
  4-uple a formare le righe di una matrice
  \begin{equation*}
    \begin{bmatrix}
      1 & 1 & 2 & 1\\
      1 & 2 & 1 & 0\\
      1 & -1 & 4 & 3\\
      2 & 1 & 1 & 0
    \end{bmatrix}
  \end{equation*}
  riduciendo a gradini seguendo il procedimeto di eliminazione (come
  definito nei paragrafi precedenti)
  \begin{equation}
    \label{eq:gauss-jorda2-2}
    \begin{matrix}
      \begin{bmatrix}
        1 & 1 & 2 & 1\\
        1 & 2 & 1 & 0\\
        1 & -1 & 4 & 3\\
        2 & 1 & 1 & 0
      \end{bmatrix}
      \overrightarrow{
      \begin{matrix}
        R_2\to R_2-R_1\\
        R_3\to R_3-R_1\\
        R_4\to R_4-2R_1
      \end{matrix}
      }
      \begin{bmatrix}
        1 & 1 & 2 & 1\\
        0 & 1 & -1 & -1\\
        0 & -2 & 2 & 2\\
        0 & -1 & -3 & -2
      \end{bmatrix}
      \overrightarrow{
      \begin{matrix}
        R_3\to R_3+2R_2\\
        R_4\to R_4+R_2
      \end{matrix}
      }
      \begin{bmatrix}
        1 & 1 & 2 & 1\\
        0 & 1 & -1 & -1\\
        0 & 0 & 0 & 0\\
        0 & 0 & -4 & -3
      \end{bmatrix}\\
      \overrightarrow{
      R_4\leftrightarrow R_3}
      \begin{bmatrix}
        1 & 1 & 2 & 1\\
        0 & 1 & -1 & -1 \\
        0 & 0 & -4 & -3\\
        0 & 0 & 0 & 0
      \end{bmatrix}
    \end{matrix}
  \end{equation}
  l'ultimo scambio di righe è stato necessario per portare la matrice
  nella forma a gradini.\\
  Dopo la riduzione sono la riduzione sono rimaste 3 righe nulle, ovvero
  il rango della matrice è 3: nell'insieme iniziale vi erano allora 3
  righe indipendenti e una quarta dipendente dalle altre (quindi le
  4-uple non erano linearmente indipendenti).\\
  In particolare, è possibile affermare che il vettore da escludere se
  si vuole estrarre un insieme di vettori indipendenti dai quattro
  vettori dati era il terzo, \textit{corrispondente alla riga annullata
    dalla riduzione}. Infatti, per arrivare all'annullamento di tale riga
  in primo luogo è stato eseguito $R_2\to R_2-R_1$ e $R_3\to R_3-R_1$, e
  poi sommando alla (ottenuta) terza riga $R_3-R_1$ la (ottenuta) seconda
  riga $R_2-R_1$ moltiplicata per 2, ovvero
  \begin{equation*}
    (R_3-R_1)+2(R_2-R_1)=0
  \end{equation*}
  Svolgendo i conti, questa uguaglianza dice che $R_3-3R_1+2R_2=0$, ovvero
  \begin{equation*}
    R_3=3R_1-2R_2
  \end{equation*}
  che conferma che la terza riga è scrivibile come combinazione delle
  altre. Il fatto appena illustrato in questo è vero in genere: se una
  riga si annula in seguito all'algoritmo di riduzione a gradini allora
  essa era esprimibile come combinazione delle altre. Infatti, nel corso
  delle riduzione a gradini bisogna trasformare da prima tutte tutte le
  righe dalla seconda in poi combinandole con la prima
  \begin{equation*}
    \begin{matrix}
      R_2\to R_2+c_2R_1, & R_3\to R_3+c_3R_1,\dots,R_m \to R_m+c_mR_1
    \end{matrix}
  \end{equation*}
  Ne secondo passaggio, ogni riga così trasformata viene combinata con la
  seconda riga:
  \begin{equation*}
    R_k\to (R_k+c_kR_1)+c_k^\prime(R_2+c_2R_!)=R_k+(c_kc^\prime_kc_2)R_1
    +c_k^\prime R_2
  \end{equation*}
  e così via: a ogni passaggio la riga $R_k$ viene combinata con una in
  più delle righe precedenti, fino a che o non bisogna più modificarlo
  perché si inizia ad utilizzarla per ridurre le successive, oppure essa
  si annulla: in tal caso si arriva quindi a una relazione del tipo:
  \begin{equation*}
    R_k+d_1R_1+d_2R_2+\dots+d_jR_j=0
  \end{equation*}
  ovvero $R_k=-d_1R_1-d_2R_2-\dots-d_jR_j$, che dice che la riga $R_k$ che
  si è annullata era combinazione lineare delle altre righe.
\end{defi}
\begin{oss}
  \label{oss:gauss-jorda3}
  l'affermazione secondo cui le righe che si annullano sono combinazione
  lineare delle altre è vera quando si segue l'algoritmo di riduzione
  a gradini, ma in generale se una riga si annulla dopo una serie
  qualunque di operazioni elementari non è detto che sia combinazione
  lineare delle altre. Ad esempio, si consideri la seguente sequenza di
  operazioni elementari (che non segue i passi dell'algoritmo di
  riduzione a gradini)
  \begin{eqnarray*}
    \begin{bmatrix}
      1 & 0\\
      2 & 0\\
      0 & 1
    \end{bmatrix} \overrightarrow{
      R_3\to R_3+R_1
    }
    \begin{bmatrix}
      1 & 0 \\
      2 & 0 \\
      1 & 1
    \end{bmatrix}
    \overrightarrow{R_2\to R_2+R_3}
    \begin{bmatrix}
      1 & 0\\
      3 & 1\\
      1 & 1
    \end{bmatrix}\overrightarrow{R_3\to R_3-R_2}
    \begin{bmatrix}
      1 & 0 \\
      3 & 1 \\
      -2 & 0
    \end{bmatrix}\\
    \overrightarrow{R_3\to R_3+2R_1}
    \begin{bmatrix}
      1 & 0\\
      3 & 1\\
      0 & 0
    \end{bmatrix}
  \end{eqnarray*}
  a seguito della quale la terza riga si anulla, pur non essendo
  combinazione lineare delle altre: non c'è nessun modo di esprimere
  (0,1) come combinazione di (1,0) e (2,0)\footnote{si noti che non sono
    stati fatti scambi di riga}.\\
  La Definizione \ref{defi:gauss-jorda2} suggerisce che si può definire
  anche il \textit{rango per colonne} di una matrice come il numero
  massimo di colonne linearmente indipendenti\footnote{ovvero la
    dimensione dello spazio generato dalle colonne}. Tuttavia vale la
  seguente
\end{oss}
\begin{prop}
  \label{prop:gauss-jorda2}
  Per una qualunque matrice, il rango per righe coincide con il
  rango per colonne.\\
  Non dimostrando la proposizione \ref{prop:gauss-jorda2}, ma è possibile
  illustrarla con un semplice esempio: nella matrice
  \begin{equation*}
    \begin{bmatrix}
      1 & 1 & 1\\
      2 & 2 & 2\\
      3 & 4 & 5
    \end{bmatrix}
  \end{equation*}
  la seconda riga $R_2$ è evidentemente dipendente dalle altre, in quanto
  $R_2=2R_1$ (se si volesse far apparire anche la terza riga in questa
  relazione di dipendenza, si potra scrivere $R_2=2R_1+0R_3$).
  Per il risultato appena citato, allora anche una delle colonne della
  matrice deve essere dipendente dalle altre: in effetti, si ha
  \begin{equation*}
    \begin{bmatrix}
      1 \\
      2\\
      5
    \end{bmatrix}=2 \cdot
    \begin{bmatrix}
      1 \\
      2\\
      4
    \end{bmatrix}-
    \begin{bmatrix}
      1\\
      2\\
      3
    \end{bmatrix},
  \end{equation*}
  che era molto meno evidente della relazione di dipendenza esistente tra
  righe. Come ulteriore esempio, si considerino le stesse $4-$uple viste
  sopra in (\ref{eq:gauss-jorda2-1}): grazie all'uguaglianza del rango
  del rango per righe e per colonne, in effetti i vettori possono essere
  disposti sia in colona che in riga, l'importante è scegliere se tutti
  i vettori devono essere disposti in un o nell'altro modo.
  \begin{equation*}
    \begin{bmatrix}
      1 & 1 & 1 & 2\\
      1 & 2 & -1 & 1 \\
      2 & 1 & 4 & 1 \\
      1 & 0 & 3 & 0
    \end{bmatrix}
  \end{equation*}
\end{prop}
Il rango per righe di questa matrice, che possono essere calcolate con il
procedimento di riduzione a gradini, è quindi uguale al rango per colone
della precedente, ovvero deve sempre essere uguale a 3. Infatti
\begin{equation}
  \label{eq:guess-jorda5}
  \begin{matrix}
    \begin{bmatrix}
      1 & 1 & 1& 2\\
      1 & 2 & -1 & 1\\
      2 & 1 & 4 & 1\\
      1 & 0 & 3 & 0
    \end{bmatrix}
    \overrightarrow{
    \begin{matrix}
      R_2\to R_2- R_1\\
      R_3\to R_3-2R_1\\
      R_3\to R_4-R_1
    \end{matrix}
    }
    \begin{bmatrix}
      1 & 1 & 1 & 2\\
      0 & 1 & -2 & -1 \\
      0 & -1 & 2 & -3\\
      0 & -1 & 2 & -2
    \end{bmatrix}
    \overrightarrow{
    \begin{matrix}
      R_3\to R_3+R_2\\
      R_4\to R_4+R_2
    \end{matrix}
    }
    \begin{bmatrix}
      1 & 1 & 1 & 2\\
      0 & 1 & -2 & -1\\
      0 & 0 & 0 & -4\\
      0 & 0 & 0 & -3
    \end{bmatrix}\\
    \overrightarrow{
    R_4\leftrightarrow 4R_4-3R_3}
    \begin{bmatrix}
      1 & 1 & 1 & 2\\
      0 & 1 & -2 & -1\\
      0 & 0 & 0 & -4\\
      0 & 0 & 0 & 0
    \end{bmatrix}
  \end{matrix}
\end{equation}
Come previsto dall'uguaglianza del rango per righe o per colonne, come
ottenuto che il rango della matrice è 3.\\
Si noti che, in questo caso, a dire quale vettore è combinazione degli
altri è la posizione dei pivot nella matrice ridotta: poiché questi si
trovano in prima, seconda e quarta colonna, i vettori da tenere sono il
primo, il secondi e il quarto, mentre il terzo è da ecludere (come già
sapendo dalla riduzione fatta sui vettori disposti in riga).\\
Infatti, se si segue la riduzione guardando solo le prime due colonne,
dove si trovano i primi due pivot, si vede che il rango delle matrice è
2, il che dice che i primi due vettori sono indipendenti tra loro; se
si guarda cosa succede solo alle prime tre colonne, si può notare che
il rango è ancora d2 (in quanto la teza colonna non contiene pivot) e
questo dice che il terzo vettore era allora combinazione dei primi due:
è solo aggiungendo la quarta colonna, dove si trova il terzo pivot, che
si ottiene una matrice di rango 3, il che significa che è il quarto
vettore, contrariamente al terzo, ad essere indipendente dai primi due.\\
Facendo uso della nozione di rango, possono riassumere tutto quello che
è ormai noto sulla risolubilià di un sistema e sul numero delle sue
soluzioni nel seguente risultato, detto \textit{teorema di
  Rouché-Capelli}.
\begin{teo}
  \label{teo:gauss-jordan1}
  Un sistema di $m$ equazioni lineari in $n$ incognite è compatibile se
  e solo se il rango della matrice dei coefficienti dei coefficienti è
  uguale al rango della matrice completa, e in tal caso il sistema ha
  $\infty^{n-j}$ soluzioni (dove $r$ denota il rango della matrice). In
  particolare, il sistema ha un'unica soluzione se e solo se $r=n$.
\end{teo}
\begin{proof}
  Come noto, un sistema è incompatibile se e solo se in seguito alla
  riduzione a gradini compaiono righe del tipo $
  \left(\begin{array}{ccc|c}
    0&\dots&0&b
  \end{array}\right)
  $ con $b\neq 0$.\\
  Ma questo equivale a dire che nella matrice dei coefficienti si è
  annullata una riga in più che nella matrice completa, ovvero il rango
  (che, ricordando, è il numero di righe non nulle dopo la riduzione)
  della matrice dei coefficienti è diverso (in particolare, minore) del
  rango della matrice completa. Questo dimostra la prima affermazione del
  teorema.\\
  Per quello che riguarda la seconda affermazione, si sà che una volta
  ridotto il sistema si recava la matrice incognita che compare in ogni
  equazione non nulla in funzione delle rimanenti. Se il rango della
  matrice è $r$, ci sono proprio $r$ righe non nulle e quindi si
  ricavanti, che sono $n-r$ e fungono da parametri liberi. Quindi la
  soluzione generale si scrive in funzione di $n-r$ parametri, ovvero il
  sistema ha $\infty^{n-r}$ soluzioni.\\
  L'ultima affermazione del teorema discende dal fatto che la soluzione è
  unica quando non dipende da nessun paramentro libero, ovvero $n-r=0.$
\end{proof}
\begin{oss}
  \label{oss:guess-jorda2}
  So noti che il teorema affema che hanno un'unica soluzione i sistemi
  (compatibile) in cui il numero di incognite è ugual al numero di
  equazioni a patto che queste ultime siano indipendenti.\\\\
  Prima di vedere alcune applicazioni geometriche di tutte la teoria
  dei sistemi e della riduzione vista, concludendo questa parte con due
  importanti risultati che mostrano come i sistemi omogenei hanno delle
  importanti caratteristiche che li distinguono dai sitemi in genrale:
\end{oss}
\begin{prop}
  \label{prop:gauss-jorda3}
  Dato un sistema omogeneo di $m$ equazioni in $n$ incognite a
  coefficienti in un campo $\mathds{K}$, valgono le seguenti:
  \begin{enumerate}
  \item se $s=(s_1,\dots,s_n)$ e $s^\prime=(s_1^\prime,\dots,s_n^\prime)$
    sono soluzioni del sistema, lo è anche $s+s^\prime=(s_1+s_1^\prime,
    \dots,s_n+s_n^\prime)$;
  \item se $s=(s_1,\dots,s_n)$ è una soluzone del sistema e $c\in
    \mathds{K}$, allora lo è anche $cs=(cs_1,\dots,cs_n)$
  \end{enumerate}
  In altre parole, l'insieme delle soluzioni di un sistema omogeneo in
  $n$ incognite è un sottospazio vettoriale di $\mathds{K}^n$.
\end{prop}
\begin{proof}
  Sia $a_{j1}x_1+\cdots+a_{jn}x_n=0$ la generica equazione del sistema. Il
  fatto che $s=(s_1,\dots,s_n)$ e $s^\prime=(s_1^\prime,\dots,s_n^\prime)$
  sono soluzioni del sistema significa che $a_{j1}s_1+\cdots+a_{jn}s_n=0$
  e $a_{j1}s_{1}^\prime+\cdots+a_{jn}s_n^\prime=0$. Ma allora
  \begin{equation}
    \label{eq:gauss-jorda3-1}
    \begin{matrix}
      a_{j1} (s_1+s^\prime_1) +\cdots+a_{jn}(s_n+s_n^\prime)=a_{j1}s_1^\prime+
      \cdots+a_{jn} s_n+a_{jn}s_n^\prime=\\
      =a_{j1}s_1+\cdots+a_{jn}s_1^\prime+a_{j1}s_1^\prime+\cdots+a_{jn}s_n^\prime=0+0=0
    \end{matrix}
  \end{equation}
  ovvero anche $s+s^\prime=(s_1+s_1^\prime,\dots,s_n+s_n^\prime)$ è
  soluzione: questo dimostra il primo punto.
  Per dimostrare il secondo punto invece, bisogna supporre che
  $s=(s_1,\dots,s_n)$ sia soluzione del sistama, ovvero
  $a_{j1}s_1+\cdots+a_{jn}s_n=0$ per la generica equazione, e bisogna
  osserve che
  \begin{equation}
    \label{eq:gauss-jorda3-2}
    a_{j1}cs_1+\cdots+a_{jn}cs_n=c(a_{j1}s_1+\cdots+a_{jn}s_n)=c\cdot0=0
  \end{equation}
  ovvero anche $cs=(cs_1,\cdots,cs_n)$ è soluzione, come affermato
  nel secondo punto.
\end{proof}
Nel caso di sistemi non omogenei, ovvero quelli che hanno almeno
un'equazione $a_{j1}s_1+\cdots+a_{jn}x_n=b_j$ con $b_j\neq 0$, i passaggi
visti sopra non sono più applicabili in questi casi: ad esempio, il
calcolo (\ref{eq:gauss-jorda3-1}) diventerebbe
\begin{equation*}
  \begin{matrix}
    a_{j1}(s_1+s_1^\prime)+\cdots+a_{jn}(s_n+s_n^\prime)=a_{j1}s_1+a_{j1}
    s_1^\prime+\cdots+a_{jn}s_n+a_{jn}s_n^\prime=\\
    =a_{j1}s_1+\cdots+a_{jn}s_1^\prime+\cdots+a_{jn}s_n^\prime=b_j+b_j=2b_j
  \end{matrix}
\end{equation*}
e quindi $s+s^\prime=(s_1+s_1^\prime,\dots, s_n+s_n^\prime)$ non risolve più
l'equazione $a_{j1}x_1+\cdots+a_{jn}x_n=b$ ma l'equazione $a_{j1}x_1+\cdots
+a_{jn}b_n=cb_j$ (di nuovo, chiamate diversa se $b_j=0$).\\
Quindi per i sistemi non omogenei non è possibile affermare che l'insieme
delle soluzioni è un sottospazio vettoriale; tuttavia, vale il seguente
risultato che descrive comunque la struttura dell'insieme delle sue
soluzioni:
\begin{prop}
  \label{prop:gauss-jorda4}
  Dato un sistem non omegeneo di $m$ equazioni in $n$ incognite a
  coefficienti in un compo $\mathds{K}$, l'insieme delle sue soluzioni
  si ottiene sommando a una sua soluzione particolare le soluzioni del
  \textit{sistema omogeneo associato}\footnote{ovvero il sistema che si
  ottiene ponendo tutti i termini noti uguali a zero}.
\end{prop}
\begin{proof}
  Sia $s=(s_1,\cdots, s_n)$ una soluzione particolare del sistema non
  omogeneo e\\ $(w_1,\dots,w_n)$ una soluzione del sistema omogeneo
  associato. Questo significa che $a_{j1}s_1+\cdots+a_{jn}s_n=b_j$ e
  che $a_{j1}w_1+\cdots+a_{jn}w_{n}=0$. Ma allora
  \begin{eqnarray*}
    a_{j1}(s_1+w_1)+\cdots+a+a_{jn}(s_n+w_n)=a_{j1}s_1+a_{j1}w_1+\cdots+a_{jn}s_n+a_{jn}w_n=\\
    =a_{j1}s_1+\cdots+a_{jn}w_n+a_{jn}s_n+a_{j1}w_1+\cdots+a_{jn}w_n+b_j+0=b_j
  \end{eqnarray*}
  ovvero anche $s+w=(s_1+w_1,\cdots,s_n+w_n)$ è soluzione del sistema non
  omogeneo: questo mostra che sicuramente sommando a una soluzione
  particolare del sistema una soluzionedel suo sistema omogeneo associato
  si ottiene ancora una soluzione del sistema (non omogeneo): per concludere
  la dimostrazione, bisogna mostrare che in questo modo si ottengono
  \textbf{tutte} le soluzioni del sistema non omogeneo.
  \begin{eqnarray*}
    a_{j1}(s_1^\prime-s_1)+\cdots+a_{jn}(s_n^\prime-s_n)=a_{j1}s_1^\prime-a_{j1}s_1+\cdots+a_{jn}s_n^\prime
    -a_{jn}s_n=\\
    =a_{j1}s_1^\prime=\cdots+a_{jn}s_n^\prime-(a_{j1}s_1+\cdots+a_{jn}s_n)=b_j-b_j=0
  \end{eqnarray*}
  Questo mostra che $s^\prime-s+(s_1^\prime-s_1,\cdots,s_n^\prime s_n)$ è una
  soluzione del sistema omogeneo associato: poiché chiaramente
  $s^\prime=s+(^\prime-s)$, questo dice che qualunque soluzione $s^\prime$ del
  sistama (non omogeneo) si può scrivere come somma della soluzione particolare
  $s$ fissata fin dall'inizio più una soluzione del sistema omogeneo associato,
  come voluto.
\end{proof}
\begin{es}
  \label{es:gauss-jordan4}
  Si consideri il sistema omogeneo seguente:
  \begin{equation}
    \label{eq:gauss-jordan4-1}
    \begin{cases}
      x_1+x_2+x_3+x_4=0\\
      2x_1+x_2+x_3+x_4=0
    \end{cases}
  \end{equation}
  Con un solo passaggio
  \begin{equation*}
    \begin{vmatrix}
      1 & 1 & 1 & 1\\
      2 & 1 & 1 & -1
    \end{vmatrix}\overrightarrow{R_2\to R_2-2R_1}
    \begin{vmatrix}
      1 & 1 & 1 & 1\\
      0 & -1 & -1 & -3
    \end{vmatrix}
  \end{equation*}
  riducendo a gradini la sua matrice\footnote{Si noti che sono stati riportati
    i termini noti: infatti, essendo questi \textbf{tutti} uguali a zero,
    qualunque operazione elementare applicando alle righe rimarranno uguali a
    zero, quindi non è necessario scriverli.} e lo riducendo al sistema a gradini
  equivalente
  \begin{equation*}
    \begin{cases}
      x_1+x_2+x_3+x_4=0\\
      -x_2-x_3-3x_4=0
    \end{cases}
  \end{equation*}
  che è possibile risolvere dal basso: posto $x_3=t$ e $x_4=s$, la seconda
  equazione darà $x_2=-t-3s$, e sostituendo nella prima si otterrà la seguente
  equazione
  \begin{equation*}
    x_1=-x_2-x_3-x_4=t+3s-t-s=2s.
  \end{equation*}
  Le soluzioni del sistema sono quindi tutte e sole le 4-uple della tipologia:
  \begin{eqnarray*}
    (2s,-t-3s,t,s), & t,s\in \mathds{R}.
  \end{eqnarray*}
  e l'insieme di tali 4-uple costituisca un sottospazio vettoriale (di $\mathds{R}^4$)
  si può mettere in evidenza riscrivendo la soluzione generale nella forma seguente:
  \begin{eqnarray*}
    (2s,-t-3s,t,s)=t(0,-1,1,0)+s(2,-3,0,1)
  \end{eqnarray*}
  uguaglianza dalla quale si vede che la generica soluzione è combinazione lineare
  dei vettori\\ $(0,-1,1,0)$ e $(2,-3,0,1)$: in altre parole, l'insieme delle soluzioni
  coincide con il sotto sistema
  \begin{eqnarray*}
    \left<(0,-1,1,0),(2,-3,0,1)\right>
  \end{eqnarray*}
  generato da tali vettori.
\end{es}
Per un esempio di sistema non omogeneo, si consideri ad esempio il sistema
\begin{eqnarray*}
  x_1+x_2+x_3+x_4=3\\
  2x_1+x_2+x_3-x_4=5
\end{eqnarray*}
che ha copme sistema omogeneo associato proprio il \ref{eq:gauss-jordan4-1} appena
risolto. Anche qui, con un solo passaggio
\begin{eqnarray*}
  \begin{vmatrix}
    1 & 1 & 1 & 1 & 3\\
    2 & 1 & 1 & -1 & 5
  \end{vmatrix}\overrightarrow{R_2\to R_2-2R_1}
  \begin{vmatrix}
    1 & 1 & 1 & 1 & 3\\
    0 & -1 & -1 & -3 & -1
  \end{vmatrix}
\end{eqnarray*}
riducendo a gradini la sua matrice completa trasformandolo nel sistema a gradini equivalente
$
\begin{cases}
  x_1+x_2+x_3+x_4=3\\
  -x_2-x_3-3x_4=-1
\end{cases}
$, che è possibile risolvere dal basso: posto $x_3=t$ e $x_4=s$, la seconda equazione darà
$x_2=1-t-3s$, e sostituendo nella prima si ottiene $x_1=3-x_2-x_3-x_4=3-1+t+3s-t-s=2+2s$.
Le soluzioni del sistema sono quindi tutte e sole le $4-$uple del tipo $(2+2s, t-t-3s,t,s)$,
al variare di $t,s\in \mathds{R}$. Allora, si vede che tale soluzione generale può essere
decomposta come
\begin{eqnarray*}
  (2+2s,1-t-3s,t,s)=(2,1,0,0)+(2s,-t-3s,t,s)
\end{eqnarray*}
ovverom come previsto dalla Proposizione \ref{prop:gauss-jorda4}, come somma della soluzione
particolare $(2,1,0,0)$\footnote{quella che si ottiene ponendo $t=s=0$} più le 4-uple del tipo
(2s,-t-3s,t,s), che sono proprio le soluzioni del suo sitema omogeneo associato
(\ref{eq:gauss-jordan4-1}).\\
Per una questione di pura logica è giusto ripescare un concetto dalle rimembranze del esame
di Matematica Analisi 1, con la seguente
\begin{oss}
  \label{oss:gauss-jordan2}
  I teoremi appena visti hanno un analogo con quello che succede nel caso delle equazioni
  differenziali lineari, ovvero le equazioni del tipo
  \begin{eqnarray*}
    y^{(n)}+a_1y^{(n-1)}+\cdots+a_ny=f(t)
  \end{eqnarray*}
  Infatti, anche per tali equazioni si dimostra che la soluzione generale si trava
  sommando una soluzione particolare a una soluzione dell'equazione omogenea associata
  $y^{(n)}+a_1y^{(n-1)}+\cdots+a_ny=0$, il cui insieme delle soluzioni è, esattamente come
  accade per i sistemi omogenei, un sottospazio vettoriale\footnote{In questo caso dello
    spazio delle funzioni}.\\
  Ad esempio, l'equazione $y^{\prime\prime}+y=t^2$ ha come soluzione generale
  $y(t)=c_1\cos t+ c_2\sin t+t^2-2$, dove è facile vedere che $t^2-2$ è una sua soluzione
  particolare, mentre $c_1\cos t +c_2\sin t$ è, al variare di $c_1,c_2\in \mathds{R}$, la
  soluzione generale dell'equazione omogenea associata $y^{\prime\prime}+y=0$ (quindi, l'insieme
  delle soluzioni di tale equazione omogenea è il sottospazio vettoriale generato dalle
  funzioni $\cos t$ e $\sin t$).\\
  Esattamente come nel caso dei sistemi, quindi, si riesce a esprimere tutte le (infinite) soluzioni
  di un'equazione semplicemente usando una soluzione particolare e due soluzioni (indipendenti)
  dell'equazione omogenea.
\end{oss}
In genere, in uno spazio vettoriale V, un sottoinsieme $S\subseteq V$ ottenuto sommando un vettore
fissato $v^0$ a tutti i vettori di un sottospazio vettoriale $W$ dato (scrivendo in forma $S=v^0+W$)
si chiama anche \textit{sottospazio affine}.\\
Quindi, avendo dimostrato che l'insieme delle soluzioni di un sistema non omogeneo compatibile di
$m$ equazioni in $n$ incognite e coefficienti in $\mathds{K}$ è in sottospazio affine di $K^n$.
Un altro esempio di sotospazio affine, è quello dato dall'insieme $S$ dei vettori $\vec{OP}\in V^3_O$
il cui secondo estremo $P$ appartiene a un piano $\pi^\prime$ fissato \textit{non} passante per $O$.\\
Infatti, bisogna constatare (come da figura) che, fissato un vettore $\vec{OP}_0\in S$, qualunque altro
$\vec{OP}\in S$ si può scrivere come somma di $\vec{OP}_0$ e di un vettore $\vec{OQ}$ che giace sul
piano $\pi$ parallelo a $\pi^\prime$ e passante per $O$, ovvero $S=\vec{OP}_0+W$, dove $W$ è l'insieme
dei vettori giacciono su $\pi$ (che, come si evince, è un sottospazio vettoriale).
\begin{figure}[ht!]
  \centering
  \begin{tikzpicture}
	\draw (0,0) -- (5,0) -- (6,3) -- (1,3) -- cycle;
	\draw[->] (2,1) -- (3,5);
	\draw[->]  (2,1) -- (4,2);
	\draw[->] (2,1) -- (5,5.5);
	\draw[dotted] (3,5) -- (5,5.5) -- (4,2);
	\draw (1,4) -- (6,4) -- (7,7) -- (2,7) -- cycle;
	% etichette
	\node(origine) at (2,0.8) {O};
	\node (Q) at (5,6){$\vec{OP}_0+\vec{OQ}=\vec{OP}$};
	\node (pi) at (4.8,.3) {$\pi$};
	\node (pipr) at (5.8,4.3) {$\pi^\prime$};
	\node (P0) at (3,5.3) {$P_0$};
	\node (Q) at (4.3,2) {Q};
\end{tikzpicture}
  \caption{Dimostrazione grafica di $S=\vec{OP}_0+W$}
  \label{fig:OP0piuW}
\end{figure}

Con un ragionamento e un disegno analogo, si può vedere che l'insieme dei vettori $\vec{OP}$ il cui
secondo estremo $P$ sta su una retta non passante per $O$ è un sottospazio affine.\\
Ispirati da questi esempi, si dice spesso che un sottospazio affine è \textit{il traslato di un
  sottospazio vettoriale}.

\section{Qualche applicazione geometrica}
\label{sec:qualcheappgeo}

È il caso di esporre alcuni problemi geometrici che possono essere risolti grazie al procedimento
di riduzione a gradini e al concetto di rango.\\
Ad esempio, considerando due rette $r$ e $r^\prime$ di equazioni cartesiane
\begin{eqnarray}
  \label{eq:qualcheappgeo1}
  r:
  \begin{cases}
    x+y+z=2\\
    2x-y+z=5
  \end{cases}, & r^\prime:
  \begin{cases}
    x-y-2z=-2\\
    x+3y+2z=2
  \end{cases}
\end{eqnarray}
e supponendo di velor determinare se ese hanno punti in comune. Dal momento che i punti di una retta
espressa in equazioni cartesiane sono proprio le soluzioni del sistema formato dalle due equazioni,
i punti comuni alle due rette sono dati dalle soluzioni comuni a tutte e quattro le equazioni delle
due rette, ovvero le soluzioni del sistema
\begin{equation}
  \label{eq:qualcheappgeo2}
  r:
  \begin{cases}
    x+y+x=2\\
    2x-y+z=5\\
    x-y-2z=-2\\
    x+3y+2z=2
  \end{cases}
\end{equation}
Riducendo la matrice completa si ottine
\begin{eqnarray*}
  \left(
  \begin{array}[ht!]{ccc|c}
    1 & 1 & 1 & 2\\
    2 & -1 & 1 & 5\\
    1 & -1 & -2 & -2\\
    1 & 3 & 2 & 2
  \end{array}\right)
  \overrightarrow{
  \begin{matrix}
    R_2\to R_2-2R_1\\
    R_3\to R_3-R_1\\
    R_4\to R_4-R_1
  \end{matrix}
  }\left(
  \begin{array}[ht!]{ccc|c}
    1 & 1 & 1 & 2\\
    0 & -3 & -1 & 1\\
    0 & -2 & -3 & -4\\
    0 & 2 & 1 & 0
  \end{array}\right)\\ \overrightarrow{
  \begin{matrix}
    R_3\to 3R_3-2R_2\\
    R_4\to 3R_4+2R_2
  \end{matrix}
  }\left(
  \begin{array}[ht!]{ccc|c}
    1 & 1 & 1 & 2\\
    0 & -3 & -1 & 1\\
    0 & 0 & -7 & -14\\
    0 & 0 & 1 & 2
  \end{array}\right)\overrightarrow{R_4\to7R_4+R_3}
  \left(
  \begin{array}[ht!]{ccc|c}
    1 & 1 & 1 & 2\\
    0 & -3 & -1 & 1\\
    0 & 0 & -7 & -14\\
    0 & 0 & 0 & 0
  \end{array}
  \right)
\end{eqnarray*}
Quindi il sistema è compatibile e, essendosi annullata una riga, la matrice
ha rango 3, quindi avendo 3 incognite in base a quanto detto nel Teorema \ref{teo:gauss-jordan1}
è presente solo una soluzione, viene trovata risolvendo il sistema ridotto corrispondente
\begin{equation*}
  \begin{cases}
    x+y+z=2\\
    -3y-z=1\\
    -7z=-14
  \end{cases}
\end{equation*}
Dall'ultima equazione si ottiene $z=2$, che sostituito nella seconda dà
\begin{equation*}
  -3y=z+1=2+1=3
\end{equation*}
Quindi l'unica soluzione del sistema è data dalla terna $(1,-1,2)$, che sono le coordinate
del punto in cui si incontrano le due rette.
Supponendo invece che le rette siano
\begin{eqnarray}
  \label{eq:qualcheappgeo3}
  r:
  \begin{cases}
    x+y+z=0\\
    2x+y+-z=1
  \end{cases}, & r^\prime:
                 \begin{cases}
                   2x-y=3\\
                   x+y-z=1
                 \end{cases}
\end{eqnarray}
e supponendo anche di voler determinare se esse hanno punti in comune. Come sopra, si mette insieme
le quattro equazioni e si riduce la matrice completa del sistema così ottenuto:
\begin{equation}
  \label{eq:qualcheappgeo4}
  \begin{matrix}
    \left(
    \begin{array}{ccc|c}
      1 & 1 & 2 & 0\\
      2 & 1 & -1 & 1\\
      2 & -1 & 0 & 3\\
      1 & 1 & -1 & 1
    \end{array}\right)\overrightarrow{
    \begin{matrix}
      R_2\to R_2-2R_1\\
      R_3\to R_3-2R_1\\
      R_4\to R_4-R_1
    \end{matrix}
    }\left(
    \begin{array}{ccc|c}
      1 & 1 & 1 & 0\\
      0 & -1 & -3 & 1\\
      0 & -3 & -2 & 3\\
      0 & 0 & -2 & 1
    \end{array}\right) \overrightarrow{R_3\to R_3-3R_2}\\
    \left(
    \begin{array}{ccc|c}
      1 & 1 & 1 & 0\\
      0 & -1 & -3 & 1\\
      0 & 0 & 7 & 0\\
      0 & 0 & -2 & 1
    \end{array}
    \right) \overrightarrow{R_4\to7R_4+2R_3}
    \left(\begin{array}{ccc|c}
      1 & 1 & 1 & 0\\
      0 & -1 & -3 & 1\\
      0 & 0 & 7 & 0\\
      0 & 0 & 0 & 7
    \end{array}\right)
\end{matrix}
\end{equation}
Essendo il sistema incompatibile (l'ultima riga corrisponde all'uguaglianza falsa $0=7$) è possibile
dedurre che le due rette non hanno punti in comune.\\
Ora, mentre nel piano due rette non hanno punti in comune sono neccessariamente parallele, nello spazio
tridimensionale questo non è più vero: come si vede nella seguente figura, grazie alla dimensione extra
rispetto al piano, possono trovarsi su piani paralleli e quindi non incontrarsi pur avendo la stessa
direzione.
\clearpage
\begin{figure}[ht!]
  \centering
  \begin{tikzpicture}
	\draw (0,0) -- (4,0) -- (5,2) -- (1,2) -- cycle;
	\draw[fill=white] (-.5,1.7) -- (3.5,1.7) -- (4.5,3.5) -- (.5,3.5) -- cycle;
	\draw (1,1) -- (4,1);
	\draw (1.5,1.7) -- (2.5,3.5);

	\node (r) at (4.3,1) {$r$};
	\node (rp) at (2.2,2.4) {$r^\prime$};
\end{tikzpicture}
  \caption{Rette su piani paralleli}
  \label{fig:rettpar}
\end{figure}

In tal caso si dice che le rette sono \textit{sghembe}. Adesso è il caso di vedere come sia possibile
determinare se le rette sono \textit{sghembe} oppure \texttt{parallele} senza fare ulteriori conti, 
ma sfruttando la riduzione già svolta. Infatti, le due rette che non hanno punti in comune sono
parallele se e solo se quando le si trasla parallelamente a se stesse sull'origine esse risultano
coincidere (ovvero hanno infiniti punti in comune), mentre sono sghembe se e solo se quando le
si trasla parallelamente a se stesse sull'origine esse hanno in comune un solo punto, l'origine stessa:
\begin{figure}[ht!]
  \centering
  \begin{tikzpicture}
	\draw (1,1) -- (4,4);
	\draw (2,1) -- (5,4);
	\draw (3,1) -- (6,4);

\end{tikzpicture}
  \caption{Differenza tra parallele e sghembe}
  \label{fig:retteparesghembe}
\end{figure}
adesso è il caso di trattare il problema di capire se le due rette hanno
la stessa direzione nel problema di determinare un'intersezione (tra le
rette traslate). Ora, per traslare una retta espressa in equazioni
cartesiane, parallelamente a se stessa, basta modificare i termini noti
delle equazioni lasciando invariati i primi membri\footnote{Infatti,
  modificando solo il noto dell'equazione $ax+by+cz=d$ di un piano si
  ottiene un piano parallelo in quanto non si avrà cambiato la normale al
  piano, data dalla terna $(a,b,c)$. Poiché una retta è intersezione di due
  piani, modificando i termini noti delle due equazioni si sta muovendo
  parallelamente a se stessi i piani e quindi muovendo parallamente a se
  stessa la retta.}: in particolare, si ottiene la traslazione sull'origine
se si va apporre i termini noti uguali a zero (in quanto in tal caso le
le equazioni risultato soddisfatte dalla terna $x=0,y=0,z=0$, che sono le
coordinate dell'origine, il che significa che la retta traslata è
proprio quella che passa per l'origine).
Nel caso delle equazioni delle due rette $r$ e $r^\prime$ data da (\ref{eq:qualcheappgeo3}), le rette
trasla sull'origine sono rappresentate dalle equazioni
\begin{eqnarray}
  \label{eq:eq:qualcheappgeo5}
  \begin{cases}
    x+y+z=0\\
    2x+y-z=0
  \end{cases}, &
                 \begin{cases}
                   2x+y=0\\
                   x+y+z=0
                 \end{cases}
\end{eqnarray}
Per determinare se tali rette traslate hanno infiniti punti in comune o uno solo, ovvero come visto
sopra, se le rette di partenza erano rispettivamente parallele o sghembe, bisogna risolvere il sistema
che si ottiene mettendo insieme le quattro equazioni,
\begin{equation*}
  \begin{cases}
    x+y+z=0\\
    2x+y-z=0\\
    2x+y=0\\
    x+y+z=0
  \end{cases},
\end{equation*}
che ha come matrice associata
\begin{equation*}
  \left(
    \begin{array}{ccc|c}
      1 &  1 &  1 & 0\\
      2 &  1 & -1 & 0\\
      2 & -1 &  0 & 0\\
      1 &  1 & -1 & 0
    \end{array}
  \right)
\end{equation*}
Ora, per ridurre a gradini questa matrice verranno applicato esattamente le stesse operazioni usate
per ridurre la (\ref{eq:qualcheappgeo4}), che differisce da essa solo per il fatto di avere tutti i
termini noti uguali a zero: l'unica differenza sarà che i termini noti rimarranno sempre nulli qualunque
operazione elementare è necessario applicare, e quindi, senza dover riscrivere il processo prima visto,
sapendo che si arriverà alla stessa matrice ridotta ma con l'ultima colonna (quuella dei termini noti)
tutta nulla, ovvero
\begin{equation*}
  \left(
    \begin{array}{ccc|c}
      1 &  1 &  1 & 0\\
      0 & -1 & -3 & 0\\
      0 &  0 &  7 & 0\\
      0 &  0 &  0 & 0
    \end{array}
  \right)
\end{equation*}
Guardando questa matrice, che rappresenta ora un sistema compatibile con 3 incognite e rango 3, si
conclude che traslando le rette sull'origine si avrà solo una soluzione (l'origine stessa) e quindi le
rette di partenza non erano parallele.\\
Adesso, è il caso di vedere cosa potrebbe capitare se le rette sono pallele, prendendo il seguente
esempio
\begin{eqnarray}
  \label{eq:qualcheappgeo6}
  r:
  \begin{cases}
    x+y+z=1\\
    x-y+2z=0
  \end{cases}, & r^\prime:
                 \begin{cases}
                   2x+3z=3\\
                   x+3y=-3
                 \end{cases}
\end{eqnarray}
Allo scopo di controllare se le rette hanno punti in comune, come visto nel caso precedente, è
necessario andare a costituire un singolo sistema composto dai due sistemi di partenza $r$ e $r^\prime$,
il risultato sarà il seguente:
\begin{equation*}
    r_{tot}:
  \begin{cases}
    x+y+z=1\\
    x-y+2z=0\\
    2x+3z=3\\
    x+3y=-3
  \end{cases}
\end{equation*}
Da questo è possibile dedurre che la matrice completa di questo sistema sarà:
\begin{equation}
  \label{eq:qualcheappgeo7}
\begin{matrix}
  \left(
    \begin{array}{ccc|c}
      1 & 1 & 1 & 1\\
      1 & -1 & 2 & 0\\
      2 & 0 & 3 & 3\\
      1 & 3 & 0 & -3
    \end{array}\right)
  \overrightarrow{
     \begin{matrix}
       R_2\to R_2-R_1\\
       R_3\to R_3-2R_1\\
       R_4\to R_4-R_1
     \end{matrix}
   }
 \left(
   \begin{array}{ccc|c}
     1 & 1 & 1 & 1 \\
     1  & -2 & 1 & -1 \\
     0  & -2 & 1 & 1 \\
     0 & 2 & -1 & -4
   \end{array}
 \right) \overrightarrow{
   \begin{matrix}
     R_3\to R_3-R_2\\
     R_4\to R_4+R_2
   \end{matrix}
}\\
\left(
  \begin{array}{ccc|c}
    1 & 1 & 1 & 1 \\
    1 & -2 & 1 & -1 \\
    0 & 0 & 0 & 2 \\
    0 & 0 & 0 & -5
  \end{array}
  \right)\overrightarrow{R_4\to 2R_4+5R_3}\left(
  \begin{array}{ccc|c}
    1 & 1 & 1 & 1 \\
    1 & -2 & 1 & -1\\
    0 & 0 & 0 & 2\\
    0 & 0 & 0 & 0
  \end{array}\right)
  \end{matrix}
\end{equation}
COme si vede, da una partte le rette non hanno punti in comune in quanto la teza riga corrisponde
all'uguaglianza falsa $0 = 2$; dall'altra parte, il sistema formato dalle due rette traslate
sull'origine, ovvero con termini noti nulli, avrebbe come matrice ridotta la stessa matrice ottenuta
ora ma con l'ultima colonna composta da zeri, ovvero
\begin{equation*}
  \left(
    \begin{array}{ccc|c}
      1 & 1 & 1 & 0 \\
      1 & -2 & 1 & 0\\
      0 & 0 & 0 & 0\\
      0 & 0 & 0 & 0
    \end{array}
  \right)
\end{equation*}
Che rappresenta la matrice di un sistema ridotto compatibile con 3 incognite e rango 2, quindi infinite
soluzioni: questo rsignifica che le due rette, traslate sull'origine, hanno infiniti punti in comune,
ovvero coincidono, e quindi le due rette di partenza, prima della traslazione, erano parallele.
\begin{oss}
  \label{oss:qualcheappgeo3}
Per rette date in equazioni parametriche verificate se esse sono parallele o meno è immediata, in quento
come già definito in precedenza, nel equazioni parametriche
\begin{equation*}
  \begin{cases}
    x=x_0+lt\\
    y=y_0+mt\\
    z=z_0+nt
  \end{cases}
\end{equation*}
un vettore che rappresenta la direzione della retta è dato dalla terna $(l,m,n)$ dei coefficenti di
$t$ (questo è scrivibile maticamente anche in forma $\frac{l}{l^\prime}=\frac{m}{m^\prime}=
\frac{n}{n^\prime}$): basta quindi confrontare i due vettori così ottenuti per ognuna delle due rette,
che avranno la stessa direzione se tali vettori sono proporzionali. Nel caso in cui le rette siano date
in equazioni cartesiane, si può passare alle parametriche semplicmente risolvendo i sistemi dati dalle
cartesiane stesse. Ad esempio, per le rette viste sopra in (\ref{eq:qualcheappgeo6}), riducendo la
matrice completa delle cartesiane di $r$ si ottiene
\begin{equation*}
  \left(
    \begin{array}{ccc|c}
      1 & 1 & 1 & 1 \\
      1 & -1 & 2 & 0
    \end{array}
  \right)\overrightarrow{R_2\to R_2-R_1}
\left(
  \begin{array}{ccc|c}
    1 & 1 & 1 & 1 \\
      0 & -2 & 1 & -1
  \end{array}
\right)
\end{equation*}
ovvero il sistema ridotto $
\begin{cases}
  x+y+z=1\\
  -2y+z=-1
\end{cases}
$, da cui ponendo $z=t$ si ricava $-2y=-1-t$ (ovvero $y=\frac{1}{2}+\frac{1}{2}t$) e $x=1-y-z=
1-\left(\frac{1}{2}+\frac{1}{2}t\right)-t=\frac{1}{2}-\frac{3}{2}t$ quindi $r$ ha equazioni parametriche
  \begin{equation}
    \label{eq:qualcheappgeo8}
    \begin{cases}
      x=\frac{1}{2}-\frac{3}{2}t\\
      y=\frac{1}{2}+\frac{1}{2}t\\
      z=t
    \end{cases}
  \end{equation}
  Analogamente, riduzione la matrice completa delle cartesiane di $r^\prime$ ottenendo
  \begin{equation*}
    \left(
      \begin{array}{ccc|c}
        2 & 0 & 3 & 3 \\
        1 & 3 & 0 & -3
      \end{array}
    \right)\overrightarrow{R_2\to2R_2-R_1}\left(
      \begin{array}{ccc|c}
        2 & 0 & 3 & 1 \\
        0 & 6 & -3 & -9
      \end{array}
    \right)
  \end{equation*}
  ovvero il sistema ridotto $
  \begin{cases}
    2x+3z=3\\
    6y-3z=-9
  \end{cases}
  $, da cui ponendo $z = t$ si ricava $6y=-9+3t$ (ovvero $-\frac{3}{2}+\frac{1}{2}t$) e $2x=3-3z=3-3t$,
  ovvero $x=\frac{3}{2}-\frac{3}{2}t$: quindi $r$ ha equazioni parametriche
  \begin{equation}
    \label{eq:qualcheappgeo9}
    \begin{cases}
      x=\frac{3}{2}-\frac{3}{2}t\\
      y=-\frac{3}{2}+\frac{1}{2}t\\
      z=t
    \end{cases}
  \end{equation}
  Confrontando i coefficienti di $t$ nelle parametriche (\ref{eq:qualcheappgeo8}) e
  (\ref{eq:qualcheappgeo9}), si vede che le rette hanno entrambe direzione rappresentata dal vettore
  di coordinate $\left(-\frac{3}{2},\frac{1}{2},1\right)$, e quindi sono parallele\footnote{Si
    noti che le rette hanno la stessa direzione, che non esclude il caso in cui esse siano parallele
    coincidenti, ovvero che le equazioni cartesiane date rappresentassero in realtà la stessa retta.}.
\end{oss}
\begin{oss}
  \label{oss:qualcheappgeo4}
  Nel caso in cui si ovglia fare il passaggio inverso, ovvero passare da parametriche a cartesiane, nel
  caso della retta basta ricavare il parametro $t$ da una delle espressioni che compongono le
  parametriche e sostituirlo nelle altre. Ad esempio, se la retta ha equazioni parametriche
  \begin{equation*}
    \begin{cases}
      x=1-t\\
      y=2+t\\
      z=1-3t
    \end{cases}
  \end{equation*}
  si può ricavare $t=1-x$ dalla prima che, sostituita nelle altre, da $y=2+(1-x)=3-x$ e
  $1-3 (1-x)=-2+3x$. allora è anche possibile affermare che la retta ha equazioni cartesiane
  \begin{equation*}
    \begin{cases}
      x=1+t+s\\
      y=2+t-s\\
      z=3-2t-s
    \end{cases}
  \end{equation*}
  si può ricavare $t=x-1-s$ dalla prima che, sostituita nelle altre, dà $y=2+(x-1-s)-s=1+x-2s$ e
  $z=3-2(x-1-s)-s=5-2x+s$; ricavano poi $s$ da questa seconda espressione, ovvero $s=z+2x+s$; ricavano
  poi $s$ da questa seconda espressione, ovvero $s=z+2x-5$, è possibile sostituire nell'altra
  ottenendo $y=1+x-2(z+2x-5)=11-3x-2z$, ovvero $3x+y+2z=11$, che è l'equazione cartesiana del piano.
  Nel capicolo successivo verrà illustrato un metodo più elegante e efficiente per arrivare allo stesso
  risultato.\\
  Un altro problema geometrico che può essere risolto con l'aiuto delle tecniche viste a proposito della
  risoluzione dei sistemi è il seguente: supponendo di avere una retta data in equazioni cartesiane
  \begin{eqnarray}
    \label{eq:qualcheappgeo10}
    \begin{cases}
      Ax+By+Cz=D\\
      A^\prime x+B^\prime y+C^\prime z =D^\prime
    \end{cases}
  \end{eqnarray}
  Come visto nel capitolo precedente quando si tratta di ricavare tali equazioni, la
  (\ref{eq:qualcheappgeo10}) sta semplicemente dicendo che la retta data è intersezione del piano dato
  dall'equazione cartesiana $Ax+By+Cz=D$ e dal piano di equazione cartesiana $Ax+By+Cz=D$ e dal piano
  di equazione cartesiana $A^\prime x+B^\prime y+C^\prime z =D^\prime$: le due equazioni che compongono
  le cartesiane sono quindi le equazioni di due particolari piani che contengono la retta.
  Per determinare tutti i piani che contengiono la retta, bisogna esporre la seguente
\end{oss}
\begin{prop}
  \label{prop:qualcheappgeo4}
  La generica equazione cartesiana del piano che contiene la retta (\ref{eq:qualcheappgeo10}) è data
  da
  \begin{equation}
    \label{eq:qualcheappgeo11}
    \alpha (Ax+By+Cz-D)+\beta (A^\prime x +B^\prime y + C^\prime-D^\prime)=0
  \end{equation}
  al variare di $\alpha,\beta\in \mathds{R}$.
\end{prop}
\begin{proof}
  Bisogna iniziare con l'osservare che se un piano ha equazione della forma (\ref{eq:qualcheappgeo11}),
  allora esso contiene la reta. Infatti, dire che una retta è contenuta in un piano significa che se
  un punto appartiene anche al piano: ma se un punto appartiene alla retta, allora le sue coordinate
  $(x,y,z)$ soddisfano entrambe le equazioni $Ax+By+Cz=D$ e $A^\prime x+B^\prime+C^\prime z=D^\prime$ della
  retta, e quindi
  \begin{eqnarray*}
    \alpha(Ax+By+Cz-D)+\beta(A^\prime x+B^\prime y+C^\prime z-D^\prime) = \alpha \cdot 0+\beta \cdot 0=0
  \end{eqnarray*}
  ovvero $(x,y,z)$ soddisfa anche l'equazione (\ref{eq:qualcheappgeo11}), cioè il punto appartiene al
  piano rappresentato da tale equazione. Questo dimostra che, per ogni $\alpha, \beta\in \mathds{R}$,
  l'equazione (\ref{eq:qualcheappgeo11}) rappresenta un piano che contiene la retta.
  Viceversa, bisogna essere sicuri che qualche paiano che contenga la retta può essere rappresentato
  nella forma (\ref{eq:qualcheappgeo11}). Per vederlo, bisogna osservere che un generico piano di
  equazione $A^{\prime\prime}x+B^{\prime\prime}y+C^{\prime\prime}=D^{\prime\prime}$ contiene tutta la
  retta sta anche sul piano, ovvero se e solo se ogni terna che soddisfa le equazioni
  $Ax+By+Cz=D$ e $A^{\prime}x+B^{\prime}y+C^{\prime}=D^{\prime}$ soddisfa automaticamente anche
  l'equazione $A^{\prime\prime}x+B^{\prime\prime}y+C^{\prime\prime}=D^{\prime\prime}$ del piano.
  In altre parole, nel sistema
  \begin{eqnarray*}
    \begin{cases}
      Ax+By+Cz=B\\
      A^{\prime}x+B^{\prime}y+C^{\prime}=D^{\prime}\\
      A^{\prime\prime}x+B^{\prime\prime}y+C^{\prime\prime}=D^{\prime\prime}
    \end{cases}
  \end{eqnarray*}
  che si ottiene mettendo insieme tutte le cartesiane, la terza equazione è superfluo ovvero dipendente
  dalle altre. A livello della matrice completa
  \begin{eqnarray*}
    \begin{vmatrix}
      A & B & C & D\\
      A^\prime & B^\prime & C^\prime & D^\prime\\
      A^{\prime\prime} & B^{\prime\prime} & C^{\prime\prime} & D^{\prime\prime}
    \end{vmatrix}
  \end{eqnarray*}
  questo si traduce nel fatto che la terza riga deve essere combinazione lineare delle altre due,
  ovvero devono esistere $\alpha,\beta\in\mathds{R}$ tali che
  \begin{eqnarray*}
    \begin{vmatrix}
      A^{\prime\prime} & B^{\prime\prime} & C^{\prime\prime} & D^{\prime\prime}
    \end{vmatrix}=\alpha
    \begin{vmatrix}
      A & B & C & D
    \end{vmatrix} +\beta
    \begin{vmatrix}
      A^\prime & B^\prime & C^\prime & D^\prime
    \end{vmatrix}
  \end{eqnarray*}
  ovvero
  \begin{eqnarray*}
    \begin{matrix}
      A^{\prime\prime}=\alpha A +\beta A^\prime, & B^{\prime\prime}=\alpha B+\beta B^\prime,
      & C^{\prime\prime} =\alpha C + \beta C^\prime, & D^{\prime\prime}=\alpha D+\beta D^\prime
    \end{matrix}
  \end{eqnarray*}
  Quindi l'equazione $A^{\prime\prime}x+B^{\prime\prime}y+C^{\prime\prime}=D^{\prime\prime}$ si riscrive
  \begin{eqnarray*}
    (\alpha A+\beta A^\prime)x+(\alpha B+\beta B^\prime)y+(\alpha C+\beta C^\prime)z=\alpha D + \beta
    D^\prime
  \end{eqnarray*}
  che, si vede facilmente svolgendo i conti e confrontando, equivale proprio alla
  (\ref{eq:qualcheappgeo11}).
\end{proof}
\begin{es}
  \label{es:qualcheappgeo1}
  Data la retta
  \begin{equation*}
    r:
    \begin{cases}
      x+y+z=1\\
      x-y+2z=0
    \end{cases}
  \end{equation*}
  si determini il piano $\pi$ che contiene $r$ e passa per il punto $P_0$ di coordinate $(1,1,1)$.
  Determinando prima tutti i piani che contengono $r$, che secondo la Proposizione
  \ref{prop:qualcheappgeo4} sono dati al variare di $\alpha,\beta\in \mathds{R}$ dall'equazione
  \begin{equation}
    \label{eq:qualcheappgeo12}
    \alpha(x+y+z-1) +\beta(x-y+2z)=0.
  \end{equation}
  da cui $\alpha=-\beta$. Sostituendo questa relazione nella (\ref{es:qualcheappgeo1}), si ottiene
  \begin{equation*}
    -\beta(x+y+z-1)+\beta(x-y+2z)=0
  \end{equation*}
  ovvero, svolgendo i calcli,
  \begin{equation*}
    -2y+z+1=0.
  \end{equation*}
  Al variare del parametro $\beta$, queste equazioni rappresentano tutte lo stesso piano (il piano
  $\pi$ cercato) in quanto si tratta di equazioni proporzionali, tutte equivalenti: dividendo per
  il parametro $\beta$ (o, equivalentemente, scegliendo per esempio $\beta=1$), possiamo allora scrivere
  che $\pi$ ha equazione cartesiana
  \begin{equation*}
    -2y+z+1=0.
  \end{equation*}
  Se, invece del passaggio per il punto, nel caso in cui si impone per esempio che il piano, oltre a
  contenere $r$, fosse parallelo a un altro piano, ad esempio quello di equazione cartesiana
  $x+2y+3z=-1$, bisogna procesere come segue. Come visto nel capitolo precedente, due piani
  $Ax+By+Cz=D$ e $A^\prime x+B^\prime y+C^\prime z=D^\prime$ sono paralleli se e solo se le terne $(A,B,C)$
  e $(A^\prime,B^\prime,C^{\prime})$ sono proporzionali, in quanto rappresentano le coordinate di vettori
  normali (perpendicolari) ai piani. In realtà, paiché c'è la libertà di moltiplicare l'equazione di un
  piano per qualunque coefficiente senza che il piano venga modificato, si può sempre far che sia
  $(A,B,C)=(A^\prime,B^\prime,C^\prime)$. Allora, poiché svolgendo i calcoli nella
  (\ref{eq:qualcheappgeo12}), il generico piano che contiene $r$ è della forma
  \begin{equation}
    \label{eq:qualcheappgeo13}
    (\alpha+\beta)x+(\alpha-\beta)y+(\alpha+2\beta)z-\alpha = 0,
  \end{equation}
  la condizione di parallelism,o tra questo piano e il piano di equazione $x+2y+3z=-1$ è
  \begin{eqnarray*}
    (\alpha+\beta,\alpha-\beta,\alpha+2\beta)=(1,2,3)
  \end{eqnarray*}
  ovvero, trasponendo questo modello in forma sistemica ordinata:
  \begin{eqnarray*}
    \begin{cases}
      \alpha+\beta=1\\
      \alpha-\beta=2\\
      \alpha-2\beta=3
    \end{cases}
  \end{eqnarray*}
  Per trovare il piano dato, basta quindi risolvere tale sistema e
  sostituire valori di $\alpha$ e $\beta$ trovati nella equazione
  (\ref{eq:qualcheappgeo13}). In questo caso, si vede riducendo a gradini
  la sua matrice completa
  \begin{eqnarray*}
    \left(
    \begin{array}{cc|c}
      1 & 1 & 1 \\
      1&-1&2 \\
      1 & 2 &3
    \end{array}
    \right)\overrightarrow{
    \begin{matrix}
      R_1\to R_2-R_1\\
      R_3\to R_3-R_1
    \end{matrix}
    }\left(
    \begin{array}{cc|c}
      1 & 1 & 1 \\
      0 &-2&1 \\
      0&1&2
    \end{array}
    \right)
    \overrightarrow{R_3\to R_3+R_2}
    \left(
    \begin{array}{cc|c}
      1&1&1 \\
      0&-2&1 \\
      0&0&5
    \end{array}\right)
  \end{eqnarray*}
  il sistema è incompatible e quindi la condizione di parallelismo non
  può essere soddisfatta: tra i piani che contengono la retta $r$, non
  ne esiste nessuno che è parallelo al piano dato.
\end{es}
\begin{es}
  Date le due rette
  \begin{eqnarray*}
    r_1:
    \begin{cases}
      x+y+z=3\\
      x-2y+z=0
    \end{cases}, & r_2:
                   \begin{cases}
                     2x+y-z=2\\
                     x-y-z=-1
                   \end{cases}
  \end{eqnarray*}
  bisogna determinare, se esiste, il piano che le contiene.
  Come fatto precedente, il generico piano che contiene $r_1$ ha
  equazione
  \begin{eqnarray*}
    \alpha_1(x+y+z-3)+\beta_1(x-2y+z)=0
  \end{eqnarray*}
  ovvero, espresso in altra forma:
  \begin{equation}
    \label{eq:qualcheappgeo14}
    (\alpha_1+\beta_1)x+(\alpha_1-2B_1)y+(\alpha_1+\beta_1)z-3\alpha_1=0
  \end{equation}
  mentre il generico piano che contiene $r_2$ ha equazione
  \begin{eqnarray*}
    \alpha_2(2x+y+z-2)+\beta(x-y-z+1)=0
  \end{eqnarray*}
  ovvero
  \begin{equation}
    \label{eq:qualcheappgeo15}
    (2\alpha_2+\beta_2)x+(\alpha_2-\beta_2)y+(-2\alpha_2+\beta_2)z
    +(-2\alpha_2+\beta_2)=0.
  \end{equation}
  Per trovare, se siste, il piano che contiene entrambe le rette basta
  vedere se osistono valori di $\alpha_1,\beta_1,\alpha_2,\beta_2$ tali
  che la (\ref{eq:qualcheappgeo14}) e la (\ref{eq:qualcheappgeo15}) sono
  uguali: uguagliando i coeffcienti di $x,y,z$ e il temine noto in tali
  equazioni si ottiene
  \begin{eqnarray*}
    \begin{cases}
      \alpha_1+\beta_1=2\alpha_2+\beta_2\\
      \alpha_1-2\beta_1=\alpha_2-\beta_2\\
      \alpha_1+\beta_1=-\alpha_2-\beta_2\\
      -3\alpha_1=-1\alpha_2+\beta_2
    \end{cases}
  \end{eqnarray*}
  ovvero il sistema omogeneo di 4 equazioni in 4 incognite
  \begin{eqnarray*}
    \begin{cases}
      \alpha_1+\beta_1=2\alpha_2+\beta_2=0\\
      \alpha_1-2\beta_1=\alpha_2-\beta_2=0\\
      \alpha_1+\beta_1=-\alpha_2-\beta_2=0\\
      -3\alpha_1=-1\alpha_2+\beta_2=0
    \end{cases}
  \end{eqnarray*}
  Tale sistema ha sicuramente sempre la soluzione $\alpha_1=\beta_1
  =\alpha_2=\beta_2=0$, ma se si sostituiscono tali valori nella
  (\ref{eq:qualcheappgeo14}) e la (\ref{eq:qualcheappgeo15}) ottenendo
  $0=0$, che non è l'equazione di un piano: quindi per l'esistenza del
  piano che contiene entrambe le rette deve esistere una soluzione non
  nulla di tale sistema. Riducendo a gradini la sua matrice dei
  coefficienti si trova
  \begin{eqnarray*}
    \begin{bmatrix}
      1 & 1 & -2 & -1\\
      1 & -2 & -1 & 1\\
      1 & 1 & 1 & 1\\
      -3 & 0 & 2 & -1
    \end{bmatrix}\overrightarrow{
    \begin{matrix}
      R_2\to R_2-R_1\\
      R_1\to R_3-R_1\\
      R_4\to R_4-3R_1
    \end{matrix}
    }
    \begin{bmatrix}
      1 & 1 & -2 & -1\\
      0 & -3 & 1 & 2\\
      0 & 0 & 3 & 2\\
      0 & 3 & -4 & -4
    \end{bmatrix}
    \overrightarrow{R_4\to R_4+R_2}
    \begin{bmatrix}
      1 & 1 & -2 & -1\\
      0 & -3 & 1 & 2\\
      0 & 0 & 3 & 2\\
      0 & 0 & -3 & -2
    \end{bmatrix}\\\overrightarrow{R_4\to R_4+R_3}
    \begin{bmatrix}
      1 & 1 & -2 & -1\\
      0 & -3 & 1 & 2 \\
      0 & 0 & 3 & 2\\
      0 & 0 & 0 & 0
    \end{bmatrix}
  \end{eqnarray*}
  Poiché la matrice ha rango 3 il sistema ha sicuramente altre soluzioni
  oltre alla 4-uple nulla $(0,0,0,0)$, quindi esiste il piano che contiene
  le rette. Per trovarlo, basta risolvere il sistema, che ridotto
  precedentemente alla forma equivalente
  \begin{eqnarray*}
    \begin{cases}
      \alpha_1+\beta_1-2\alpha_2-\beta_2=0\\
      -3\beta_1+\alpha_2+3\beta_2=0\\
      3\alpha_2+2\beta_2=0
    \end{cases}
  \end{eqnarray*}
  Però non è necessario determinare completamente la soluzione del
  sistema: infatti, l'ultima equazione non nulla dà il valore di
  $\alpha_2$ (in funzione di $\beta_2$), mentre le prime due i valori
  $\alpha_1$ e $\beta_1$ (sempre in forma di $\beta_2$): se si sostituisce
  i valori $\alpha_1$ e $\beta_1$ così trovati nella
  (\ref{eq:qualcheappgeo14}) o quello di $\alpha$ nella
  (\ref{eq:qualcheappgeo15}) ottenendo lo stesso piano (il sistema
  esprimendo porprio la condizione che i due piani sono uguali), quindi
  per trovarlo basta determinare solo $\alpha_2$ dall'ultima equazione non
  nulle $2\alpha_2+2\beta_2=0$ senza dover risolvere le altre due.
  Questa equazione dà $\alpha_2=\frac{2}{3}\beta_2$ che sostituita nella
  (\ref{eq:qualcheappgeo15}) da
  \begin{eqnarray*}
    \left(-\frac{4}{3}\beta_2+\beta_2\right)x+\left(-\frac{2}{3}\beta_2
    -\beta_2\right)y+\left(\frac{2}{3}\beta_2-\beta_2\right)z+
    \left(\frac{4}{3}\beta_2+\beta_2\right)=0
  \end{eqnarray*}
  ovvero
  \begin{eqnarray*}
    -\frac{1}{3}\beta_2x+\frac{5}{3}\beta_2y-\frac{1}{3}\beta_2z+
    \frac{7}{3}\beta_2=0
  \end{eqnarray*}
  Dividendo per $\beta_2$ e moltiplicando per $-3$ si ottiene infine
  \begin{eqnarray*}
    x+5y+z-7=0
  \end{eqnarray*}
  che è l'equazione del piano cercato.
\end{es}
Il procedimento di riduzione a gradini è uno strumento non solo per
risolvere un sistema ma più in generale per scoprire se delle $n$-uple
sono linearmente indipendenti e in caso negativo, per individuare le
$n$-uple che sono combinazione delle altre. A sua volta questo contente
di determinare se sono indipendenti i vettori in un qualunque spazio
vettoriale: infatti, basta fissare una base, rappresentare i vettori
mediante le $n$-uple delle loro coordinate e poi controllare se sono
indipendenti tali $n$-uple mediante una semplice riduzione a gradini.
\begin{es}
  Fissata una base $\vec{OP}_1,\vec{OP}_2,\vec{OP}_3$ dello spazio
  tridimensionale dei vettori applicati $V_O^3$, si dica se i tre vettori
  applicati
  \begin{eqnarray*}
    v_1=\vec{OP}_1+\vec{OP}_2+\vec{OP}_3\\
    v_2=3\vec{OP}_1+4\vec{OP}_2+5\vec{OP}_3\\
    v_3=-\vec{OP}_1+2\vec{OP}_2+5\vec{OP}_3
  \end{eqnarray*}
  formano ancora una base.
  Tre vettori in uno spazio di dimensione 3 formano una base se e solo se
  sono indipendenti: per determinare l'indipendenza di $v_1,v_2,v_3$ basta
  considerare le terne $(1,2,3),(3,4,5),(-1,2,5)$ delle loro coordinate
  rispetto alla base di partenza $OP_1,OP_2,OP_3$ e determinare
  equivalentamente se esse sono indipendenti in $\mathds{R}$.
  \clearpage
  Costruendo la matrie che ha tali terne come righe
  \begin{eqnarray*}
    \begin{bmatrix}
      1 & 2 & 3\\
      3 & 4 & 5\\
      -1 & 2 & 5
    \end{bmatrix}
  \end{eqnarray*}
  e effetuando una riduzione a gradini:
  \begin{eqnarray*}
    \begin{bmatrix}
      1 & 2 & 3\\
      3 & 4 & 5\\
      -1 & 2 & 5
    \end{bmatrix}\overrightarrow{
    \begin{matrix}
      R_2\to R_2-3R_1\\
      R_3\to R_4+R_1
    \end{matrix}
    }
    \begin{bmatrix}
      1 & 2 & 3\\
      0 & -2 & -4\\
      0 & 4 & 8
    \end{bmatrix}
    \overrightarrow{R_3\to R_3+2R_2}
    \begin{bmatrix}
      1 & 2 & 3\\
      0 & -2 & -4\\
      0 & 0 & 0
    \end{bmatrix}
  \end{eqnarray*}
  Essendosi annullata la terza riga, significa che questa era
  combinazione delle prime due: quindi le tre terne non sono indipendenti
  e non lo sono neanche i vettori applicati $v_1,v_2,v_3$ da esse
  rappresentate (tali vettori giacevano quindi su uno stesso piano).
\end{es}

\section{Due integrazioni sul determinante}
\label{sec:dueintdet}

\subsection{Complanarità di rette in parametriche}
\label{sec:comdiretteinparametriche}

Siano $r$ e $r^\prime$ due rette, date in equazioni parametriche
\begin{eqnarray}
  \label{eq:dueintdet1}
  r:
  \begin{cases}
    x=x_0+at\\
    y=y_0+bt\\
    z=z_0+ct
  \end{cases}, & r^\prime:
                 \begin{cases}
                   x=x_0^\prime+a^\prime t\\
                   y=y^\prime_0+b^\prime t\\
                   z=z_0^\prime+c^\prime t
                 \end{cases}
\end{eqnarray}
Allora $r$ e $r^\prime$ sono complanari se e solo se
\begin{eqnarray}
  \label{eq:dueintdet2}
  \det
  \begin{bmatrix}
    x_0-x_0^\prime & y_0-y_0^\prime& z_0-z_0^\prime\\
    a & b & c\\
    a^\prime & b^\prime & c^\prime
  \end{bmatrix}=0
\end{eqnarray}
Infatti, due rette sono complanari solo se sono o parallele o incidenti
in un punto: nel caso in cui esse siano parallele, i loro vettori
direttori $v=(a,b,c)$ e $v^\prime=(a^\prime, b^\prime,
c^\prime)$\footnote{Che si vedono dai coefficienti di $t$ nelle
  parametriche} sono proporzionali, e quindi il determinante
(\ref{eq:dueintdet2}) è zero perché la seconda e la terza riga delle
matrice sono proporzionali e quindi dipendenti (e il determinante di una
matrice con le righe dipendenti, si annulla). Nel caso in cui le rette in
cui le rette siano incidenti in un punto, invece, dal momento che le
parametriche di una retta danno i punti della retta al variare del
parametro, esisteranno un valore di $t$ da sostituire nella prima delle
(\ref{eq:dueintdet1}) e un valore di $t^\prime$ da sostituire della
(\ref{eq:dueintdet1}) per i quali si ottiene lo stesso punto (quello in
comune alle due rette), ovvero per tali valori si ha
\begin{equation}
  \label{eq:dueintdet3}
  \begin{cases}
    x_0 +at=x_0^\prime+a^\prime t^\prime\\
    y_0 + bt = y_0^\prime + b^\prime t^\prime\\
    z_0+ct=z_0^\prime+c^\prime t^\prime
  \end{cases}
\end{equation}
È possibile riscrivere queste guaglianze come
\begin{eqnarray*}
  \begin{cases}
    x_0-x_0^\prime=-at+a^\prime t^\prime\\
    y_0-y_0^\prime=-bt+b^\prime t^\prime\\
    z_0-z_0^\prime=-ct+c^\prime t^\prime
  \end{cases}
\end{eqnarray*}
ovvero accorparle nella forma vettoriale
\begin{equation}
  \label{eq:dueintdet4}
  (x_0-x_0^\prime,y_0-y_0^\prime,z_0-z_0^\prime)=-(a,b,c)+t^\prime(a^\prime,
  b^\prime,c^\prime) 
\end{equation}
Qyesta uguaglianza sta allora dicendo che, nella matrice che compare nella
(\ref{eq:dueintdet2}), la prima riga è combinazione lineare delle altre
due: di nuovo, il determinante è allora zero in quanto il determinante di
una matrice con le righe dioendenti è nullo.\\
Questo conclude la dimostrazione del fatto che se le rette sono complanari
allora vale la \ref{eq:dueintdet2}. Viceversa, se il determinante si
annulla allora i casi sono due:
\begin{enumerate}
\item La seconda e la terza riga $(a,b,c)$ e $(a^\prime,b^\prime,c^\prime)$
  sono proporzionali, e le rette sono parallele, essendo tali righe i
  vettori direttori delle rette.
\item Se le rette non sono proporzionali, quando il determinante nullo e
  quindi le righe dipendenti necessariamentela prima riga
  $(x_0-x_0^\prime, y_0-y_0^\prime, z_0-z_0^\prime)$ deve essere
  combinazione lineare di $(a,b,c)$ e $(a^\prime,b^\prime,c^\prime)$:
  questo, ripetendo a ritroso i passaggi fatti per passare dalla
  (\ref{eq:dueintdet4}) alla (\ref{eq:dueintdet3}) mostra che le rette
  devono avere un punto in comune.
\end{enumerate}
  
\subsection{Regola di Sarrus}
\label{sec:regsarrus}

La regola di Sarrus è un metodo, alternativo allo svolgimento di Laplace,
per calcolare il determinante di una matrice quadrata di ordine 3
\begin{equation}
  \label{eq:dueintdet5}
  A=
  \begin{bmatrix}
    a_{11}&a_{12} & a_{13}\\
    a_{21} & a_{22} & a_{23}\\
    a_{31} & a_{32} & a_{33}
  \end{bmatrix}
\end{equation}
La regola di Sarrus funziona come segue: aggiungendo alla matrice data due
colonne ripetendo, dopo la terza, la sua prima e la sua seconda colonna:
\begin{equation}
  \label{eq:dueintdet6}
  \left[
  \begin{array}{ccccc}
    a_{11} & a_{12} & a_{13} & a_{11} & a_{12}\\
    a_{21} & a_{22} & a_{23} & a_{21} & a_{22}\\
    a_{31} & a_{32} & a_{33} & a_{31} & a_{32} 
  \end{array}\right]
\end{equation}
In questa matrice con cinque colonne è possibile mettere in evidenza tre
``diagonali''
\begin{equation}
  \label{eq:dueintdet7}
  \begin{matrix}
    \left[
    \begin{array}{ccccc}
      \mathbf{\color{blue}a_{11}} & a_{12} & a_{13} & a_{11} & a_{12}\\
      a_{21} & \mathbf{\color{blue}a_{22}} & a_{23} & a_{21} & a_{22}\\
      a_{31} & a_{32} & \mathbf{\color{blue}a_{33}} & a_{31} & a_{32} 
    \end{array}\right], & \left[
                          \begin{array}{ccccc}
                            a_{11} & \mathbf{\color{blue}a_{12}} & a_{13} & a_{11} & a_{12}\\
                            a_{21} & a_{22} & \mathbf{\color{blue} a_{23}} & a_{21} & a_{22}\\
                            a_{31} & a_{32} & a_{33} & \mathbf{\color{blue} a_{31}} & a_{32} 
                          \end{array}\right],\\ \left[
    \begin{array}{ccccc}
      a_{11} & a_{12} & \mathbf{\color{blue}a_{13}} & a_{11} & a_{12}\\
      a_{21} & a_{22} & a_{23} & \mathbf{\color{blue}a_{21}} & a_{22}\\
      a_{31} & a_{32} & a_{33} & a_{31} & \mathbf{\color{blue}a_{21}} 
    \end{array}\right],
  \end{matrix}
\end{equation}
e tre ``antidiagonali''
\begin{equation}
  \label{eq:dueintdet8}
  \begin{matrix}
    \left[
    \begin{array}{ccccc}
      a_{11} & a_{12} & \mathbf{\color{red} a_{13}} & a_{11} & a_{12}\\
      a_{21} & \mathbf{\color{red}a_{22}} & a_{23} & a_{21} & a_{22}\\
      \mathbf{\color{red}a_{31}} & a_{32} & a_{33} & a_{31} & a_{32} 
    \end{array}\right], & \left[
                          \begin{array}{ccccc}
                            a_{11} & a_{12} & a_{13}
                            & \mathbf{\color{red}a_{11}} & a_{12}\\
                            a_{21} & a_{22}
                            & \mathbf{\color{red}a_{23}} & a_{21} & a_{22}\\
                            a_{31} & \mathbf{\color{red}a_{32}} & a_{33} &  a_{31} & a_{32} 
                          \end{array}\right],\\ \left[
    \begin{array}{ccccc}
      a_{11} & a_{12} & a_{13} & a_{11} & \mathbf{\color{red} a_{12}}\\
      a_{21} & a_{22} & a_{23} & \mathbf{\color{red}a_{21}} & a_{22}\\
      a_{31} & a_{32} & \mathbf{\color{red}a_{33}} & a_{31} & a_{21} 
    \end{array}\right].
  \end{matrix}
\end{equation}
Il determinate della matrice risulta essere allora uguale al prodotto
degli elementi della prima diagonale più il prodotto degli elementi della
seconda diagonale più il prodotto degli elementi della terza diagonale,
meno il prodotto degli elementi della prima antidiagonale meno il prodotto
degli elementi della seconda antidiagonale meno il prodotto degli
elementi della terza antidiagonale:
\begin{eqnarray*}
  a_{11}a_{22}a_{33}+a_{12}a_{23}a_{31}+a_{13}a_{22}a_{32}-a_{13}a_{22}a_{31}
  -a_{11}a_{23}a_{32}-a_{12}a_{21}a_{33}
\end{eqnarray*}
Infatti, si potrebbe verificare che gli addendi di questa uguaglianza
coincidono con quelli previsti dalla definizione di determinante
mediante sommatoria e permutazioni (la regola di Sarrus dà quindi
semplicemente un modo pratico di ottenere quegli stessi addetti, con i
segni corretti, senza ricorrere alla definizione).\\
Bisogna sottolineare il fatto che la regola di Sarrus non può essere
applicata a matrici di ordine superiore a 3.


\chapter{Applicazioni lineari e prodotti di matrici}
\label{chap:appLinEprodotdimatrix}

\section{Applicazioni lineari: definizione ed esempi}
\label{sec:applindefes}

In questo paragrafo si parlerà di funzioni tra spazi vettoriali.
Ricordando che una funzione $f:X\to Y$ tra due insiemi $X$
(detto \textit{dominio}) e $Y$ (detto \textit{codominio}) è una legge che
associa a ogni $x\in X$ un ben preciso elemento di $Y$, detto
\textit{immagine di $x$} e denotato $f(x)$.
Tipicamente vengono trattate le funzioni $f:V\to W$ in cui dominio e
codominio sono, corrispettivamente, $V$ e $W$, spazi vettoriali su un
certo campo $\mathds{K}$, e in particolare si andrà a studiare quello che
soddisfa la seguente
\begin{defi}
  \label{defi:applindefes}
  Una funzione $f:V\to W$ tra spazi vettoriali si dice \textit{funzione
    lineare} (o \textit{applicazione lineare}) se verifica le due seguenti
  proprietà:
  \begin{equation}
    \label{eq:applindefes1}
    f(v+v^\prime)=f(v)+f(v^\prime) \text{ per ogni } v,v^\prime \in V 
  \end{equation}
  \begin{equation}
    \label{eq:applindefes2}
    f(cv)=cf(v) \text{ per ogni $v\in V$ e ogni scalare } c\in\mathds{K}
  \end{equation}
  Limitarsi, alla funzione lineare può essere molto restrittivo: ad
  esempio, si può vedere che se\footnote{Sapendo che $\mathds{R}^n$ è uno
    spazio vettoriale di dimensione $n$, in particolare per $n=1$ si
    ottiene $\mathds{R}^1=\mathds{R}$ (che risulta quindi uno spazio
    vettoriale di dimensione unitaria o $dim = 1$)} $V=W=\mathds{R}$, le
  uniche funzioni lineari $f:\mathds{R} \to \mathds{R}$ sono quelle del
  tipo $f(x)=ax$, con $a\in \mathds{R}$ fissato. Tuttavia, bisogna notare
  che tra le applicazioni lineari vi sono funzioni di grande importanzia
  e utilità in geometria e nelle sue applicazioni:
\end{defi}
\begin{es}
  \label{es:applindefes1}
  Dato lo spazio $V_O^2$ dei vettori geometrici applicati nel piano,
  considerando la funzione $f:V_O^2\to V_O^2$ che associa a ongi vettore
  $\vec{OP}$ il vettore che si ottiene ruotado $\vec{OP}$ di un angolo
  $\theta$ fissato in senso antiorario attorno all'origine $O$, come
  nella figura seguente
  \begin{figure}[ht!]
    \centering
    \resizebox{4cm}{!}{\begin{tikzpicture}
	\begin{pgfonlayer}{nodelayer}
		\node [style=dot] (0) at (-4, 0) {};
		\node [style=none] (1) at (-1.75, 0) {};
		\node [style=none] (2) at (1, 0) {};
		\node [style=none] (3) at (-1, 4.25) {};
		\node [style=none] (4) at (-0.25, -2.75) {};
		\node [style=none] (5) at (-4, 5) {};
		\node [style=none] (7) at (2, 0) {$f(\vec{OQ})$};
		\node [style=none] (8) at (-2.25, -1.25) {};
		\node [style=none] (9) at (-2.75, 1.75) {};
		\node [style=none] (10) at (-4, 2) {};
		\node [style=none] (11) at (-4.5, 0) {$O$};
		\node [style=none] (12) at (-1.5, -0.75) {$\theta$};
		\node [style=none] (13) at (-3.25, 2.5) {$\theta$};
		\node [style=none] (14) at (-4, 5.75) {$f(\vec{OP})$};
		\node [style=none] (15) at (-0.5, 4.5) {$P$};
		\node [style=none] (16) at (0.25, -3) {$Q$};
	\end{pgfonlayer}
	\begin{pgfonlayer}{edgelayer}
		\draw [style=Rightarrow] (0) to (2.center);
		\draw [style=Rightarrow] (0) to (5.center);
		\draw [style=Rightarrow] (0) to (4.center);
		\draw [style=Rightarrow] (0) to (3.center);
		\draw [style=ev. line, bend left] (5.center) to (3.center);
		\draw [style=ev. line, bend left=15, looseness=1.25] (2.center) to (4.center);
		\draw [style=Dashedrightarrow, bend right=15, looseness=1.25] (8.center) to (1.center);
		\draw [style=Dashedrightarrow, bend right=15, looseness=1.50] (9.center) to (10.center);
	\end{pgfonlayer}
\end{tikzpicture}
}
    \caption{Funzione $f:V_O^2\to V_O^2$ associata ai vettori $\vec{OP}$ e $\vec{OP}$}
    \label{fig:applindefes1}
  \end{figure}
  
  Come si vede nella figura seguente, dati due vettori $\vec{OP}$ e
  $\vec{OP}^\prime$, sommarli e poi ruotare il vettore risultante oppure
  prima ruotarli e poi sommare i vettori ruotati è equivalente, ovvero
  \clearpage
  \begin{figure}[ht!]
    \centering
    \resizebox{7cm}{!}{\begin{tikzpicture}
	\begin{pgfonlayer}{nodelayer}
		\node [style=none] (0) at (0.25, 1.75) {};
		\node [style=none] (1) at (1.75, 0) {};
		\node [style=none] (2) at (3.75, 0.75) {};
		\node [style=none] (3) at (1.75, 2.75) {};
		\node [style=none] (4) at (0, 4.25) {};
		\node [style=none] (5) at (4.25, -1) {};
		\node [style=none] (6) at (6, 0) {};
		\node [style=none] (7) at (2.5, -0.25) {};
		\node [style=none] (8) at (1, 0.75) {};
		\node [style=none] (9) at (1, 1.75) {};
		\node [style=none] (10) at (3.5, 0) {};
		\node [style=none] (11) at (2.75, 0.5) {};
		\node [style=none] (12) at (1.75, 1.25) {};
		\node [style=none] (13) at (0, 4.75) {$f(\vec{OP})+f(\vec{OP}^\prime)=f(\vec{OP}+\vec{OP}^\prime)$};
		\node [style=none] (14) at (3.75, 1.25) {$P^\prime$};
		\node [style=none] (15) at (4.25, -1.5) {$P$};
		\node [style=none] (16) at (7.25, 0) {$\vec{OP}+\vec{OP}^\prime$};
		\node [style=none] (17) at (1.25, -0.25) {$O$};
		\node [style=none] (18) at (2.5, 3) {$f(\vec{OP})$};
		\node [style=none] (19) at (-0.75, 1.75) {$f\vec{OP}^\prime$};
	\end{pgfonlayer}
	\begin{pgfonlayer}{edgelayer}
		\draw [style=Rightarrow] (1.center) to (3.center);
		\draw [style=Rightarrow] (1.center) to (0.center);
		\draw [style=Rightarrow] (1.center) to (4.center);
		\draw [style=DashedLine] (0.center) to (4.center);
		\draw [style=DashedLine] (4.center) to (3.center);
		\draw [style=Rightarrow] (1.center) to (6.center);
		\draw [style=Rightarrow] (1.center) to (2.center);
		\draw [style=Rightarrow] (1.center) to (5.center);
		\draw [style=DashedLine] (2.center) to (6.center);
		\draw [style=DashedLine] (6.center) to (5.center);
		\draw [style=Rightarrow, bend right=45, looseness=1.25] (7.center) to (8.center);
		\draw [style=Rightarrow, bend right=45, looseness=1.25] (11.center) to (12.center);
		\draw [style=Rightarrow, bend right=60] (10.center) to (9.center);
	\end{pgfonlayer}
\end{tikzpicture}
}
    \caption{Cosa accade se si ruota e sommano i vettori $\vec{OP}$ e $\vec{OP}^\prime$}
    \label{fig:applindefes2}
  \end{figure}
  Quindi vale la formula, già acclarata
  \begin{equation}
    \label{eq:applindefes3}
    f(\vec{OP})+f(\vec{OP}^\prime)=f(\vec{OP}+\vec{OP}^\prime)
  \end{equation}
  che dice che questa funzione soddisfa la proprietà
  (\ref{eq:applindefes1}). Analoramente, dato un vettore $\vec{OP}$ e un
  numero reale $c$, moltiplicare il vettore per $c$ e poi ruotarlo oppure
  prima ruotarlo e poi moltiplicarlo per $c$ è equivalente:
  \begin{figure}[ht!]
    \centering
    \resizebox{5cm}{!}{\begin{tikzpicture}
	\begin{pgfonlayer}{nodelayer}
		\node [style=none] (0) at (0, 0) {};
		\node [style=none] (1) at (0, 4) {};
		\node [style=none] (2) at (4, 0) {};
		\node [style=none] (3) at (0, 1.5) {};
		\node [style=none] (4) at (1.5, 0) {};
		\node [style=none] (5) at (0, 0.5) {};
		\node [style=none] (6) at (0.5, 0) {};
		\node [style=none] (7) at (0, 2.5) {};
		\node [style=none] (8) at (2.25, 0) {};
		\node [style=none] (9) at (1.5, -0.5) {$P$};
		\node [style=none] (10) at (-0.5, 1.5) {$f(\vec{OP})$};
		\node [style=none] (11) at (4, -0.5) {$c\vec{OP}$};
		\node [style=none] (12) at (0, 4.5) {$f(c\vec{OP})=cf(\vec{OP})$};
	\end{pgfonlayer}
	\begin{pgfonlayer}{edgelayer}
		\draw [style=Rightarrow] (0.center) to (4.center);
		\draw [style=Rightarrow] (4.center) to (2.center);
		\draw [style=Rightarrow] (0.center) to (3.center);
		\draw [style=Rightarrow] (3.center) to (1.center);
		\draw [style=Rightarrow, bend right=45, looseness=1.25] (6.center) to (5.center);
		\draw [style=Rightarrow, bend right=45] (8.center) to (7.center);
	\end{pgfonlayer}
\end{tikzpicture}
}
    \caption{Moltiplicazione tra il vettore $\vec{OP}$ e un numero reale $c$}
    \label{fig:applindefes3}
  \end{figure}
  
  Quindi si ha, l'equivalenza:
  \begin{equation}
    \label{eq:applindefes4}
    f(c\vec{OP})=cf(\vec{OP})
  \end{equation}
  che afferma, in modo abbastanza esplicito che, questa funzione anche
  la proprietà (\ref{eq:applindefes2}) della Definizione
  \ref{defi:applindefes}. E si conclude quindi, che le rotazioni attorno
  a $O$ sono applicazioni lineari dallo spazio vettoriale $V_O^2$ in se
  stesso.\\
  È possibile raggiungere la stessa conclusione anche per altre importanti
  trasformazioni geometriche; ad esempio, si consideri la riflessione
  rispetto a una retta $r$ passante per $O$, che manda ogni vettore
  $\vec{OP}\in V_O^2$ nel vettore sommetrico rispetto alla retta, come
  da figura
  \begin{figure}[ht!]
    \centering
    \resizebox{5cm}{!}{\begin{tikzpicture}
	\begin{pgfonlayer}{nodelayer}
		\node [style=none] (0) at (0, 0) {};
		\node [style=none] (1) at (0, 6) {};
		\node [style=none] (2) at (6, 0) {};
		\node [style=none] (3) at (-4, -4) {};
		\node [style=none] (4) at (7, 7) {};
		\node [style=none] (5) at (4, 6) {};
		\node [style=none] (6) at (6, 4) {};
		\node [style=none] (7) at (7.25, 7.25) {$r$};
		\node [style=none] (8) at (0, 6.5) {$f(\vec{OP})$};
		\node [style=none] (9) at (4, 6.5) {$f(\vec{OQ})$};
		\node [style=none] (10) at (6.5, 4.25) {Q};
		\node [style=none] (11) at (6.5, 0) {$R$};
		\node [style=none] (12) at (0, -0.5) {$O$};
	\end{pgfonlayer}
	\begin{pgfonlayer}{edgelayer}
		\draw (3.center) to (4.center);
		\draw [style=Rightarrow] (0.center) to (2.center);
		\draw [style=Rightarrow] (0.center) to (1.center);
		\draw [style=Rightarrow] (0.center) to (5.center);
		\draw [style=Rightarrow] (0.center) to (6.center);
	\end{pgfonlayer}
\end{tikzpicture}
}
    \caption{Metodo alternativo per dimostrare la funzione
      \ref{fig:applindefes1} mediante retta}
    \label{fig:applindefes4}
  \end{figure}

  Allora, Come già fatto per le rotazioni, si nota che, dati due vettori
  $\vec{OP}$ e $\vec{OP}^\prime$, sommarli e poi riflettere il vettore
  risultante oppure prima rifletterli e poi sommare i vettori riflessi è
  equivalente
  \clearpage
  \begin{figure}[ht!]
    \centering
    \resizebox{7cm}{!}{\begin{tikzpicture}
	\begin{pgfonlayer}{nodelayer}
		\node [style=none] (0) at (0.25, 1.75) {};
		\node [style=none] (1) at (1.75, 0) {};
		\node [style=none] (2) at (3.75, 0.75) {};
		\node [style=none] (3) at (1.75, 2.75) {};
		\node [style=none] (4) at (0, 4.25) {};
		\node [style=none] (5) at (4.25, -1) {};
		\node [style=none] (6) at (6, 0) {};
		\node [style=none] (13) at (0, 4.75) {$f(\vec{OP})+f(\vec{OP}^\prime)=f(\vec{OP}+\vec{OP}^\prime)$};
		\node [style=none] (14) at (3.75, 1.25) {$P^\prime$};
		\node [style=none] (15) at (4.25, -1.5) {$P$};
		\node [style=none] (16) at (7.25, 0) {$\vec{OP}+\vec{OP}^\prime$};
		\node [style=none] (17) at (1.25, -0.25) {$O$};
		\node [style=none] (18) at (2.5, 3) {$f(\vec{OP})$};
		\node [style=none] (19) at (-0.75, 1.75) {$f\vec{OP}^\prime$};
		\node [style=none] (20) at (-0.25, -2.25) {};
		\node [style=none] (21) at (6, 5) {};
		\node [style=none] (22) at (6.25, 5.25) {$r$};
	\end{pgfonlayer}
	\begin{pgfonlayer}{edgelayer}
		\draw [style=Rightarrow] (1.center) to (3.center);
		\draw [style=Rightarrow] (1.center) to (0.center);
		\draw [style=Rightarrow] (1.center) to (4.center);
		\draw [style=DashedLine] (0.center) to (4.center);
		\draw [style=DashedLine] (4.center) to (3.center);
		\draw [style=Rightarrow] (1.center) to (6.center);
		\draw [style=Rightarrow] (1.center) to (2.center);
		\draw [style=Rightarrow] (1.center) to (5.center);
		\draw [style=DashedLine] (2.center) to (6.center);
		\draw [style=DashedLine] (6.center) to (5.center);
		\draw (20.center) to (21.center);
	\end{pgfonlayer}
\end{tikzpicture}
}
    \caption{Metodo alternativo per dimostrare la funzione
      \ref{fig:applindefes2} mediante retta}
    \label{fig:applindefes5}
  \end{figure}
  e quindi anche in questo caso si otterà una formula identica alla
  \ref{eq:applindefes3}. Dato un vettore $\vec{OP}$ e un numero reale c,
  moltiplicare il vettore per $c$ e poi rifletterlo oppure prima
  rifletterlo e poi moltiplicarlo per $c$ è equivalente
  \begin{figure}[ht!]
    \centering
    \resizebox{7cm}{!}{\begin{tikzpicture}
	\begin{pgfonlayer}{nodelayer}
		\node [style=none] (0) at (0, 0) {};
		\node [style=none] (1) at (0, 4) {};
		\node [style=none] (2) at (4, 0) {};
		\node [style=none] (3) at (0, 1.5) {};
		\node [style=none] (4) at (1.5, 0) {};
		\node [style=none] (9) at (1.5, -0.5) {$P$};
		\node [style=none] (10) at (-0.5, 1.5) {$f(\vec{OP})$};
		\node [style=none] (11) at (4, -0.5) {$c\vec{OP}$};
		\node [style=none] (12) at (0, 4.5) {$f(c\vec{OP})=cf(\vec{OP})$};
		\node [style=none] (13) at (-1.75, -1.75) {};
		\node [style=none] (14) at (4.5, 4.5) {};
		\node [style=none] (15) at (5, 4.75) {$r$};
	\end{pgfonlayer}
	\begin{pgfonlayer}{edgelayer}
		\draw [style=Rightarrow] (0.center) to (4.center);
		\draw [style=Rightarrow] (4.center) to (2.center);
		\draw [style=Rightarrow] (0.center) to (3.center);
		\draw [style=Rightarrow] (3.center) to (1.center);
		\draw (13.center) to (14.center);
	\end{pgfonlayer}
\end{tikzpicture}
}
    \caption{Metodo alternativo per dimostrare la funzione \ref{fig:applindefes3} mediante retta}
    \label{fig:applindefes6}
  \end{figure}

  quindi, la formula è identica alla \ref{eq:applindefes4}, quindi
  è possibile concludere che anche la riflessione rispetto a una retta che
  passa per $O$, avendo le proprietà (\ref{eq:applindefes1}) e
  (\ref{eq:applindefes2}) richieste nella Definizione
  \ref{defi:applindefes}, è un'applicazione lineare $f:V_O^2\to V_O^2$.\\
  Come terzo esempio di applicazione lineare $V_O^2\to V_O^2$ bisogna
  citare la proiezione ortogonale, che proietta ortogonalmente i vettori
  su una retta fissata passante per $O$.
  \begin{figure}[ht!]
    \centering
    \resizebox{8cm}{!}{\begin{tikzpicture}
	\begin{pgfonlayer}{nodelayer}
		\node [style=none] (0) at (0, 0) {};
		\node [style=none] (1) at (4, 0) {};
		\node [style=none] (2) at (7.5, 0) {};
		\node [style=none] (3) at (10.25, 0) {};
		\node [style=none] (4) at (12.5, 0) {};
		\node [style=none] (5) at (7.5, 2) {};
		\node [style=none] (6) at (10.25, 2) {};
		\node [style=none] (7) at (5.25, 0) {};
		\node [style=none] (8) at (5.25, 0.75) {};
		\node [style=none] (9) at (6.25, 1.25) {};
		\node [style=none] (10) at (6.25, 0) {};
		\node [style=none] (11) at (7, 1.75) {};
		\node [style=none] (12) at (7, 0) {};
		\node [style=none] (13) at (5.5, 0.5) {};
		\node [style=none] (14) at (5.5, 0) {};
		\node [style=none] (15) at (6.5, 0.75) {};
		\node [style=none] (16) at (6.5, 0) {};
		\node [style=none] (17) at (8, 1.25) {};
		\node [style=none] (18) at (8, 0) {};
		\node [style=none] (19) at (8.75, 1.5) {};
		\node [style=none] (20) at (8.75, 0) {};
		\node [style=none] (21) at (9.5, 1.75) {};
		\node [style=none] (22) at (9.5, 0) {};
		\node [style=none] (23) at (4, -0.25) {$O$};
		\node [style=none] (24) at (7.5, -0.25) {$f(\vec{OP})$};
		\node [style=none] (25) at (10.25, -0.25) {$f(\vec{OQ})$};
		\node [style=none] (26) at (12.5, -0.25) {$r$};
		\node [style=none] (27) at (7.75, 2) {$P$};
		\node [style=none] (28) at (10.5, 2) {$Q$};
	\end{pgfonlayer}
	\begin{pgfonlayer}{edgelayer}
		\draw (0.center) to (1.center);
		\draw [style=Rightarrow] (1.center) to (2.center);
		\draw [style=Rightarrow] (2.center) to (3.center);
		\draw (3.center) to (4.center);
		\draw [style=Rightarrow] (1.center) to (5.center);
		\draw [style=Rightarrow] (1.center) to (6.center);
		\draw [style=Dashedrightarrow] (8.center) to (7.center);
		\draw [style=Rightarrow] (9.center) to (10.center);
		\draw [style=Dashedrightarrow] (11.center) to (12.center);
		\draw [style=Dashedrightarrow] (5.center) to (2.center);
		\draw [style=Dashedrightarrow] (13.center) to (14.center);
		\draw [style=Dashedrightarrow] (15.center) to (16.center);
		\draw [style=Dashedrightarrow] (17.center) to (18.center);
		\draw [style=Dashedrightarrow] (19.center) to (20.center);
		\draw [style=Dashedrightarrow] (21.center) to (22.center);
		\draw [style=Dashedrightarrow] (6.center) to (3.center);
	\end{pgfonlayer}
\end{tikzpicture}
}
    \caption{Retta con vettori proiettanti ortogonalmente per $O$}
    \label{fig:applindefes7}
  \end{figure}

  per la quale non è difficile vedere che valgono anche le proprietà
  (\ref{eq:applindefes1}) e (\ref{eq:applindefes2}).
  Analogamente a quanto visto per rotazioni, riflessioni e proiezioni nel
  piano, anche le corrispondenti trasformazioni $V_O^3\to V_O^3$ dello
  spazio tridimensionale $V_O^3$ sono applicazioni lineari; più
  precisamente, si può motrare che soddisfano le proprietà
  (\ref{eq:applindefes1}) e (\ref{eq:applindefes2}) della Definizione
  \ref{defi:applindefes} la rotazione di un angolo fissato $\theta$
  attorno a una retta data passante per $O$ (detta anche \emph{asse della
    rotazione}).
  \clearpage
  \begin{figure}[ht!]
    \centering
    \resizebox{3cm}{!}{\begin{tikzpicture}
	\begin{pgfonlayer}{nodelayer}
		\node [style=none] (0) at (3, 6) {};
		\node [style=none] (1) at (3, 0) {};
		\node [style=none] (2) at (3, 2) {};
		\node [style=none] (3) at (4.5, 4) {};
		\node [style=none] (4) at (1.5, 4) {};
		\node [style=none] (5) at (3.5, 4.5) {};
		\node [style=none] (6) at (3.25, 2) {$O$};
		\node [style=none] (7) at (3.5, 5) {$f(\vec{OP})$};
		\node [style=none] (8) at (5, 4) {$P$};
		\node [style=none] (9) at (3.25, 6) {$r$};
		\node [style=none] (10) at (3.75, 3) {};
		\node [style=none] (11) at (3.25, 3.25) {};
		\node [style=none] (12) at (3.5, 3) {$\theta$};
	\end{pgfonlayer}
	\begin{pgfonlayer}{edgelayer}
		\draw (0.center) to (1.center);
		\draw [style=Rightarrow] (2.center) to (3.center);
		\draw [style=ev. line, bend left=45] (4.center) to (3.center);
		\draw [style=ev. line, bend right=45] (4.center) to (3.center);
		\draw [style=Rightarrow] (2.center) to (5.center);
		\draw [style=Rightarrow, bend left=45] (11.center) to (10.center);
	\end{pgfonlayer}
\end{tikzpicture}
}
    \caption{La rotazione di un angolo fissato $\theta$ (asse della rotazione)}
    \label{fig:applindefes8}
  \end{figure}
  la riflessione rispetto a un piano passante per $O$
  \begin{figure}[ht!]
    \centering
    \resizebox{4cm}{!}{\begin{tikzpicture}
	\begin{pgfonlayer}{nodelayer}
		\node [style=none] (0) at (0, 0) {};
		\node [style=none] (1) at (1, 3) {};
		\node [style=none] (2) at (6, 0) {};
		\node [style=none] (3) at (7, 3) {};
		\node [style=none] (4) at (2.75, 1.5) {};
		\node [style=none] (5) at (5, 5) {};
		\node [style=none] (6) at (5, -3) {};
		\node [style=dot] (7) at (5, 1.5) {};
		\node [style=none] (8) at (5.25, 5) {$P$};
		\node [style=none] (9) at (5.5, -3) {$f(\vec{OP})$};
		\node [style=none] (10) at (2.5, 1.5) {$O$};
		\node [style=none] (11) at (7, 1.25) {$\pi$};
	\end{pgfonlayer}
	\begin{pgfonlayer}{edgelayer}
		\draw (0.center) to (1.center);
		\draw (1.center) to (3.center);
		\draw (3.center) to (2.center);
		\draw (2.center) to (0.center);
		\draw [style=Rightarrow] (4.center) to (5.center);
		\draw [style=Rightarrow] (4.center) to (6.center);
		\draw [style=DashedLine] (5.center) to (6.center);
	\end{pgfonlayer}
\end{tikzpicture}
}
    \caption{Riflessione rispetto a un piano passante per $O$}
    \label{fig:applindefes9}
  \end{figure}
  
  e la proiezione ortogonale su un piano passante per l'origine $O$:
  \begin{figure}[ht!]
    \centering
    \resizebox{5cm}{!}{\begin{tikzpicture}
	\begin{pgfonlayer}{nodelayer}
		\node [style=none] (0) at (0, 0) {};
		\node [style=none] (1) at (1, 2) {};
		\node [style=none] (2) at (6, 0) {};
		\node [style=none] (3) at (7, 2) {};
		\node [style=none] (4) at (2.75, 1) {};
		\node [style=none] (5) at (5, 3.75) {};
		\node [style=none] (6) at (5, 1) {};
		\node [style=none] (8) at (5.25, 4) {$P$};
		\node [style=none] (9) at (5.5, 1) {$f(\vec{OP})$};
		\node [style=none] (10) at (2.25, 1) {$O$};
		\node [style=none] (11) at (7, 1.25) {$\pi$};
	\end{pgfonlayer}
	\begin{pgfonlayer}{edgelayer}
		\draw (0.center) to (1.center);
		\draw (1.center) to (3.center);
		\draw (3.center) to (2.center);
		\draw (2.center) to (0.center);
		\draw [style=Rightarrow] (4.center) to (5.center);
		\draw [style=Rightarrow] (4.center) to (6.center);
		\draw [style=DashedLine] (5.center) to (6.center);
	\end{pgfonlayer}
\end{tikzpicture}
}
    \caption{Proiezione ortogonale su un piano passante per l'origine $O$}
    \label{fig:applindefes10}
  \end{figure}
\end{es}

\section{Matrice associata a un'applicazione lineare}
\label{sec:mtxAsaplin}

Una delle caratteristiche fondamentali di un'applicazione lineare
$f:V\to W$ è che, se gli spazi $V$ e $W$ hanno dimensione finita, allora
$f$ può essere rappresentata da una matrice.\\
I Dettagli: sia $f:V\to W$ un'applicazione lineare, e siano
$B_V=\{v_1,v_2,\dots,v_n\}$ e $B_W=\{w_1,w_2,\dots,w_m\}$ basi di $V$ e
$W$ rispettivamente. Allora, ogni vettore $v\to V$ può essere identificato
con un $n$-uple $(x_1,x_2,\dots,x_n)$, quella delle sue coordinate
rispetto alla base $B_V$, ovvero
\begin{equation}
  \label{eq:mtxAsaplin1}
  v=x_1v_1+x_2v_2+\cdots+x_nv_n),
\end{equation}
e analogamente ogni vettore $w\in W$ può essere identificato con una
$m$-upla $(y_1,y_2,\dots,y_m)$, quella delle sue coordinate rispetto alla
base $B_W$, ovvero
\begin{equation}
  \label{eq:mtxAsaplin2}
  w=y_1w_1+y_2w_2+\cdots+y_mw_m.
\end{equation}
Con queste identificazioni, la $f$ può essere pensata come una funzione
$\mathds{K}^n\to\mathds{m}$ che associa a ogni $n$-uple una $m$-upla.

L'obbiettivo a questo punto, è quello di esplicitare la funzione, a tale
scopo, sia $(x_1,\dots,x_n)$ la $n$-upla delle coordinate di un vettore
$v\in V$, ovvero come visto riportato prima, nella formula
(\ref{eq:mtxAsaplin1}). Allora la sua immagine $f(v)$ sarà
\begin{equation}
  \label{eq:mtxAsaplin3}
  f(v)=f(x_1v_1+\cdots+x_nv_n)=f(x_1v_1)+\cdots+f(x_nv_n)=x_1f(v_1)+\cdots+
  x_nf(v_n).
\end{equation}
Ora, ciascuno dei vettori $f(v_1),f(v_2),\dots,f(v_n)$ che compare nella
(\ref{eq:mtxAsaplin3}) appartiene al codominio $W$ della funzione, e
quindi potrà essere espresso come combinazione lineare dei vettori
$w_1,\dots,w_m$ della base $B_W$ fissata per $W$:
\begin{eqnarray}
  \label{eq:mtxAsaplin4}
  f(v_1)=a_{11}w_1+a_{21}w_2+\cdots+a_{m1}w_{m}
\end{eqnarray}
\begin{equation*}
  \vdots
\end{equation*}
\begin{equation}
  \label{eq:mtxAsaplin5}
  f(v_n)=a_{1n}w_1+a_{2n}w_2+\cdots+a_{nm}w_m
\end{equation}
Ora, sostituendo queste espressioni nella (\ref{eq:mtxAsaplin2}) si
ottiene
\begin{equation}
  \label{eq:mtxAsaplin6}
  \begin{matrix}
    f(v)=x_1(a_{11}w_1+a_{21}w_2+\cdots+a_{m1}w_m)+\cdots+x_n(a_{1n}w_1
    +a_{2n}w_2+\cdots+a_{mn}w_m)=\\
    =(a_{11}x_1+\cdots+a_{1n}x_n)w_1+\cdots+(a_{m1}x_1+\cdots+a_{mn}x_n)w_m
  \end{matrix}
\end{equation}
Questa uguaglianza sta dicendo che le coordinate del vettore $f(v)$
respetto alla base $B_W=\{w_1,w_2,\dots,w_m\}$ sono date da
$a_{11}x_1+\cdots+a_{1n}x_n,\dots,a_{m1}x_1+\cdots+a_{mn}x_n$ e quindi che,
tradotta in coordinate, la nostra applicazione lineare può essere
identificata con la funzione $\mathds{K}^n\to\mathds{K}^m$ che associa a
ogni $n$-upla $(x_1,\dots,x_n)$ la $m$-upla formata dai coefficienti che
appaiono nella (\ref{eq:mtxAsaplin6}), ovvero
\begin{equation}
  \label{eq:mtxAsaplin7}
  \begin{bmatrix}
    x_1\\
    x_2\\
    \vdots\\
    x_n
  \end{bmatrix}\to
  \begin{bmatrix}
    a_{11}x_1+a_{12}x_2+\cdots+a_{1n}x_n\\
    a_{21}x_1+a_{21}x_2+\cdots+a_{2n}x_n\\
    \vdots\\
    a_{m1}x_1+a_{m2}x_2+\cdots+a_{mn}x_n
  \end{bmatrix}
\end{equation}
I coefficienti che compaiono nella \ref{eq:mtxAsaplin7} formano una
matrice con $m$ righe e $n$ colonne
\begin{equation*}
  A=
  \begin{bmatrix}
    a_{11} & a_{12} & \cdots & a_{1n}\\
    a_{21} & a_{22} & \cdots & a_{2n}\\
           & \vdots\\
    a_{m1} & a_{m2} & \cdots & a_{mn}
  \end{bmatrix}
\end{equation*}
la \emph{matrice associata all'applicazione lineare rispetto alle basi}
$B_V$ e $B_W$. In base alle
(\ref{eq:mtxAsaplin6}),\dots,(\ref{eq:mtxAsaplin7}) tale matrice può
essere definita come \emph{la matrice ha sulle colonne le coordinate dei
  vettori\\ $f(v_1),\dots,f(v_n)$}\footnote{Ovvero le immagini dei vettori
  della base $B_v$ fissato nel dominio} \emph{rispetto alla base $B_W$
  fissata nel codominio}. Si denota che $M_{B_w B_v}(f)$ la matrice
associata a un'applicazione lineare $f:V\to W$ rispetto alle basi $B_V$
e $B_W$. Se dominio e codominio dell'applicazione coincidono, ovvero ha
una funzione lineare $f:V\to V$ (tali applicazioni si dicono
\emph{endomorfismi}), allora è possibile fissare la stessa base $B_V$
sia nel dominio che nel codominio, e calcolare la matrice associata
$M_{B_w B_v} (f)$: in tal caso, per brevità la si denota semplicamente
$M_{B_v} (f)$. In generale, data $A$, si richiamerà la \emph{funzione
  determinante $A$ la funzione definita dalla \ref{eq:mtxAsaplin7}}, e
la si denota come $L_A$.

La matrice associata a un'applicazione lineare dà tutte le informazioni
necessarie sull'applicazione, e usando gli strumenti imparati nei capitoli
precedenti (rango, determinante) sarà possibile denotare molte proprietà
della funzione data. Ma prima di esplorare questo aspetto è giusto, fare un
esempio di quello finora esposto.
\clearpage
\begin{es}
  \label{es:mtxAsaplin1}
  Questo esempio sarà esposto per punti per un semplice fattore di
  comodità, quindi\dots
  \begin{enumerate}
  \item Sia $f:V_O^2\to V_O^2$ la rotazione attorno a $O$ di un angolo
    fissato $\theta$, come già esposto nei paragrafi precedenti.\\
    Per calcolare la matrice associata, bisogna fissare una base $B$
    formata da due vettori $v_1=\vec{OP}_1$ e $v_2=\vec{OP}_2$ della
    stessa lunghezza e perpendicolari tra loro, come da figura
    \begin{figure}[ht!]
      \centering
      \resizebox{8cm}{!}{\begin{tikzpicture}
	\begin{pgfonlayer}{nodelayer}
		\node [style=none] (0) at (0, 0) {};
		\node [style=none] (1) at (0, 4) {};
		\node [style=none] (2) at (-2, 3) {};
		\node [style=none] (3) at (3, 2.75) {};
		\node [style=none] (4) at (4, 0) {};
		\node [style=none] (5) at (1.5, 0) {};
		\node [style=none] (6) at (1.25, 1) {};
		\node [style=none] (7) at (2, 0.5) {$\theta$};
		\node [style=none] (8) at (0, 1.5) {};
		\node [style=none] (9) at (-0.75, 1.25) {};
		\node [style=none] (10) at (-0.5, 2) {$\theta$};
		\node [style=none] (11) at (4.25, -0.5) {$P_1$};
		\node [style=none] (12) at (-0.5, 3.5) {$P_2$};
		\node [style=none] (13) at (3.5, 3) {$f(\vec{OP_1})$};
		\node [style=none] (14) at (-2.5, 2.75) {$f\vec{OP_2}$};
		\node [style=none] (15) at (-0.25, -0.5) {$O$};
	\end{pgfonlayer}
	\begin{pgfonlayer}{edgelayer}
		\draw [style=Rightarrow] (0.center) to (1.center);
		\draw [style=Rightarrow] (0.center) to (2.center);
		\draw [style=Rightarrow] (0.center) to (4.center);
		\draw [style=Rightarrow] (0.center) to (3.center);
		\draw [style=Dashedrightarrow, bend right, looseness=1.25] (5.center) to (6.center);
		\draw [style=DashedLine, bend right=15, looseness=1.25] (1.center) to (2.center);
		\draw [style=Dashedrightarrow, in=45, out=165] (8.center) to (9.center);
		\draw [style=DashedLine, bend right] (4.center) to (3.center);
	\end{pgfonlayer}
\end{tikzpicture}
}
      \caption{Base $B$ formata da due vettori $v_1$ e $v_2$ formati da $\vec{OP}$}
      \label{fig:mtxAsaplin1}
    \end{figure}
    e si determina la matrice $M_B(f)$.\\
    A questo scopo, come afferma la definizione di matrice associata, bisogna
    trovare le coordinate di $f(v_1)$ e $f(v_2)$ rispetto a $B$, ovvero esprimere
    $f(\vec{OP}_1)$ e $f(\vec{OP}_2)$ come combinazone lineare $x_1\vec{OP}_1+
    x_2\vec{OP}_2$ dei vettori della base $B$. Partendo con $f(\vec{OP}_1)$: come
    si vede dalla figura
    \begin{figure}[ht!]
      \centering
      \resizebox{8cm}{!}{\begin{tikzpicture}
	\begin{pgfonlayer}{nodelayer}
		\node [style=none] (0) at (0, 5) {};
		\node [style=none] (1) at (0, 0) {};
		\node [style=none] (2) at (5, 0) {};
		\node [style=none] (3) at (7, 0) {};
		\node [style=none] (4) at (0, 7) {};
		\node [style=none] (5) at (5, 5) {};
		\node [style=none] (6) at (1.5, 1.5) {};
		\node [style=none] (7) at (1.75, 0) {};
		\node [style=none] (8) at (5.75, 5.5) {$R_1$};
		\node [style=none] (9) at (7, -0.5) {$P_1$};
		\node [style=none] (10) at (5, -0.5) {$A_1$};
		\node [style=none] (11) at (0, -0.5) {$O$};
		\node [style=none] (12) at (2.25, 1) {$\theta$};
		\node [style=none] (13) at (-0.5, 5) {$B_1$};
		\node [style=none] (14) at (-0.75, 7) {$P_2$};
	\end{pgfonlayer}
	\begin{pgfonlayer}{edgelayer}
		\draw [style=Rightarrow] (1.center) to (0.center);
		\draw [style=Rightarrow] (1.center) to (2.center);
		\draw [style=Rightarrow] (2.center) to (3.center);
		\draw [style=Rightarrow] (0.center) to (4.center);
		\draw [style=Rightarrow] (1.center) to (5.center);
		\draw [style=DashedLine] (0.center) to (5.center);
		\draw [style=DashedLine] (5.center) to (2.center);
		\draw [bend left=15] (5.center) to (3.center);
		\draw [bend right] (7.center) to (6.center);
	\end{pgfonlayer}
\end{tikzpicture}
}
      \caption{Sviluppo di $f(\vec{OP}_1)$}
      \label{fig:mtxAsaplin2}
    \end{figure}

    si ha quindi $f\left(\vec{OP}_1\right)=\vec{OP_1}=\vec{OA_1}+\vec{OB_1}$,
    essendo $A_1$ e $B_1$ le proiezioni ortogonali di $R_1$ sui vettori di
    base. Ora, chiaramente $\vec{OA}_1=x_1\vec{OP}_1$ e $\vec{OB_1}=x_2\vec{OP}_2$,
    dove $x_1$ è dato dal rapporto $\frac{\abs{\vec{OA}_1}}{\abs{\vec{OP}_1}}$ tra
    la lunghezza di $\vec{OA}_1$ e quella di $\vec{OP_2}$. Ma essendo la lunghezza
    di $\vec{OP_1}$ uguale alla lunghezza di $\vec{OR}_1=f(\vec{OP}_1)$, è possibile
    affermare che $x_1$ è uguale al rapporto tra la lunghezza del cateto $\vec{OA}_1$
    e quella dell'ipotenusa $\vec{OP}_1$ del triangolo rettangolo $OR_1A_1$, ovvero,
    $x_1=\cos\theta$.\\
    In modo analogo, poiché $\vec{OP}_2$ ha la stessa lunghezza di $\vec{OP_1}$ e
    quindi di $\vec{OR}_1$, ha la stessa lunghezza del segmento $A_1R_1$, si ha
    che
    \clearpage
    \begin{eqnarray*}
      x_2=\frac{\abs{\vec{OB}_1}}{\abs{\vec{OP_2}}}=\frac{\abs{A_1R_1}}{\abs{OR_1}},
    \end{eqnarray*}
    ovvero, $x=\sin\theta$. Riassumento,
    \begin{eqnarray}
      \label{eq:mtxAsaplin8}
      f(\vec{OP_1})=\vec{OR}_1=\vec{OA}_1+\vec{OB_1}=\cos{\theta}\vec{OP_1}+\sin\theta
      \vec{OP}_2
    \end{eqnarray}
    Per svolgere $f(\vec{OP_2})$, basterà fare l'analogo ragionamento
    \begin{figure}[ht!]
      \centering
      \resizebox{8cm}{!}{\begin{tikzpicture}
	\begin{pgfonlayer}{nodelayer}
		\node [style=none] (0) at (0, 4) {};
		\node [style=none] (1) at (0, 0) {};
		\node [style=none] (2) at (-4, 0) {};
		\node [style=none] (3) at (4, 0) {};
		\node [style=none] (4) at (-4, 4) {};
		\node [style=none] (5) at (-5, 0) {};
		\node [style=none] (6) at (0, 5) {};
		\node [style=none] (7) at (-4.75, 4) {$R_2$};
		\node [style=none] (8) at (-0.75, 0.75) {};
		\node [style=none] (9) at (0, 1) {};
		\node [style=none] (10) at (-4, 0) {};
		\node [style=none] (11) at (-0.5, 1.25) {$\theta$};
		\node [style=none] (12) at (0.5, 4) {$B_2$};
		\node [style=none] (13) at (0.5, 5) {$P_2$};
		\node [style=none] (14) at (-4, -0.5) {$A_2$};
		\node [style=none] (15) at (0, -0.5) {$O$};
		\node [style=none] (16) at (4, -0.5) {$P_1$};
	\end{pgfonlayer}
	\begin{pgfonlayer}{edgelayer}
		\draw [style=Rightarrow] (1.center) to (0.center);
		\draw [style=Rightarrow] (0.center) to (6.center);
		\draw [style=Rightarrow] (1.center) to (2.center);
		\draw [style=Rightarrow] (1.center) to (3.center);
		\draw [style=Rightarrow] (1.center) to (4.center);
		\draw [style=DashedLine] (4.center) to (0.center);
		\draw [style=DashedLine] (2.center) to (4.center);
		\draw [bend left=15] (4.center) to (6.center);
		\draw [bend left=345, looseness=1.25] (9.center) to (8.center);
		\draw (10.center) to (5.center);
	\end{pgfonlayer}
\end{tikzpicture}
}
      \caption{Sviluppo di $f(\vec{OP}_2)$}
      \label{fig:mtxAsaplin3}
    \end{figure}

    si ha $f(\vec{OP_2})=\vec{OP}_2=\vec{OA_2}+\vec{OB_2}$. chiaramente,
    $\vec{OA_2}=-x_1\vec{OP}_1$, dove $x_1$ è dato dal rapporto $\frac{\abs{\vec{OA}_2}}
    {\abs{\vec{OP}_1}}$ tra la lunghezza di $\vec{OA}_2$ e quella di $\vec{OP}_1$ e
    $\vec{OB_2}=x_2\vec{OP_2}$, dove $x_2$ è dato dal rapporto $\frac{\abs{\vec{OB}_2}}
    {\abs{\vec{OP_2}}}$ tra la lunghezza di $\vec{OB}_2$ e quindi $\vec{OR_2}=f(\vec{OP_2})$,
    mentre, la lunghezza di $\vec{OA_2}$ è uguale alla lunghezza del segmento $R_2B_2$, è
    possibile affermare che $x_1$ è uguale al rapporto tra la lunghezza del cateto $R_2B_2$
    e quella dell'ipotenusa $OR_2$ del triangolo rettangolo $OR_2B_2$, ovvero, $x_2=\cos\theta$.
    
    Analogamente, poiché $\vec{OP}_2$ ha la stessa lunghezza di
    $\vec{OR}_2$, si ha che $x_2=\frac{\abs{\vec{OB}_2}}
    {\abs{\vec{OP}_2}}$, che è il rapporto tra il cateto e l'ipotenusa
    del triangola rettangolo $OB_2R_2$, ovvero, $x_2=\cos\theta$.

    Riassumendo,
    \begin{equation}
      \label{eq:mtxAsaplin9}
      f(\vec{OP_1})=\vec{OR}_2=\vec{OA_1}+\vec{OB}_1=\sin\theta
      \vec{OP}_1+\cos\theta\vec{OP}_2
    \end{equation}
    Quindi, (\ref{eq:mtxAsaplin8}) e (\ref{eq:mtxAsaplin9}) dicono
    che la matrice associata a $f$ rispotto a $B$ avrà sulla prima
    colonna $(\cos\theta, \sin\theta)$ e sulla seconda colonna
    $(-\sin \theta, \cos\theta)$, ovvero
    \begin{equation}
      \label{eq:mtxAsaplin10}
      M_B(f)=
      \begin{vmatrix}
        \cos\theta & -\sin\theta\\
        \sin\theta & \cos \theta
      \end{vmatrix}
    \end{equation}
    In base a quanto visto nella (\ref{eq:mtxAsaplin7}), si ottiene
    allora che la rotazione, in cordinate, si traduce in funzione di
    $f:\mathds{R}^2\to\mathds{R}^2$ data da
    \begin{equation}
      \label{eq:mtxAsaplin11}
      (x_1,x_2)\to (\cos\theta{}x_1-\sin\theta{}x_2,\sin\theta{}x_1
      +\cos\theta{}x_2)
    \end{equation}
    Ad esempio, scegliendo un angolo $\theta=\frac{\pi}{4}$ e si
    sostituisce in (\ref{eq:mtxAsaplin10}) e (\ref{eq:mtxAsaplin11}),
    considerando che $\cos\frac{\pi}{4}=\sin\frac{\pi}{4}=\frac{\sqrt{2}}{2}$,
    si ottengono corrispettivamente;
    \begin{eqnarray}
      \label{eq:mtxAsaplin12}
      M_B(f)=
      \begin{pmatrix}
        \frac{\sqrt{2}}{2} & -\frac{\sqrt{2}}{2}\\
        \frac{\sqrt{2}}{2} & \frac{\sqrt{2}}{2}
      \end{pmatrix} & e & (x_1,x_2)\to
                          \begin{pmatrix}
                            \frac{\sqrt{2}}{2}x_1-\frac{\sqrt{2}}{2}x_2,
                            \frac{\sqrt{2}}{2}x_1+\frac{\sqrt{2}}{2}x_2
                          \end{pmatrix}
    \end{eqnarray}
    Per illustrare come questa semplice funzione $\mathds{R}^2\to \mathds{R}^2$
    rappresenti effettivamente la rotazione di $\frac{\pi}{4}$, è il caso di
    utilizzare un esempio, prenendo il vettore $v=v_1+v_2$, che come si vede
    dalla figura seguente, coincide con la diagonale del quadrato che ha $v_1$ e
    $v_2$ come lati:
    \clearpage
    \begin{figure}[ht!]
      \centering
      \resizebox{5cm}{!}{\begin{tikzpicture}
	\begin{pgfonlayer}{nodelayer}
		\node [style=none] (0) at (0, 5) {};
		\node [style=none] (1) at (0, 0) {};
		\node [style=none] (2) at (5, 0) {};
		\node [style=none] (3) at (5, 5) {};
		\node [style=none] (4) at (6, 5) {$v_1+v_2$};
		\node [style=none] (5) at (5, -0.5) {$v_1$};
		\node [style=none] (6) at (0, -0.5) {$O$};
		\node [style=none] (7) at (-0.5, 5) {$v_2$};
	\end{pgfonlayer}
	\begin{pgfonlayer}{edgelayer}
		\draw [style=DashedLine] (0.center) to (3.center);
		\draw [style=DashedLine] (3.center) to (2.center);
		\draw [style=Rightarrow] (1.center) to (0.center);
		\draw [style=Rightarrow] (1.center) to (2.center);
		\draw [style=Rightarrow] (1.center) to (3.center);
	\end{pgfonlayer}
\end{tikzpicture}
}
      \caption{Quadrato composto da $v_1$ e $v_2$}
      \label{fig:mtxAsaplin4}
    \end{figure}
    Tale vettore ha quindi come coordinate rispettoa $B$ la coppia $(x_1,x_2)=(1,1)$.
    In base alla (\ref{eq:mtxAsaplin12}), le coordinate di $f(v)$ rispetto a $B$ sono
    quindi date da
    \begin{equation*}
      \begin{pmatrix}
        \frac{\sqrt{2}}{2}\cdot 1 - \frac{\sqrt{2}}{2} \cdot 1,
        \frac{\sqrt{2}}{2} \cdot 1 + \frac{\sqrt{2}}{2}\cdot 1
      \end{pmatrix}
      =(0,\sqrt{2})
    \end{equation*}
    ovvero deve essere $f(v)=0v_1+\sqrt{2}v_2=\vec{2}v_2$.

    In effetti, tale risultato ottenuto analiticamente in coordinate è confermato
    dall'analisi grafica, che dice: il vettore $f(v)$ che si ottiene ruotando la
    sua lunghezza è proprio $\sqrt{2}$ volte la lunghezza di $v_2$:
    \begin{figure}[ht!]
      \centering
      \resizebox{5cm}{!}{\begin{tikzpicture}
	\begin{pgfonlayer}{nodelayer}
		\node [style=none] (0) at (0, 5) {};
		\node [style=none] (1) at (0, 0) {};
		\node [style=none] (2) at (5, 0) {};
		\node [style=none] (3) at (5, 5) {};
		\node [style=none] (4) at (6, 5) {$v_1+v_2=v$};
		\node [style=none] (5) at (5, -0.5) {$v_1$};
		\node [style=none] (6) at (0, -0.5) {$O$};
		\node [style=none] (7) at (-0.5, 5) {$v_2$};
		\node [style=none] (8) at (0, 7) {};
		\node [style=none] (9) at (-0.5, 7.25) {$f(v)$};
	\end{pgfonlayer}
	\begin{pgfonlayer}{edgelayer}
		\draw [style=DashedLine] (0.center) to (3.center);
		\draw [style=DashedLine] (3.center) to (2.center);
		\draw [style=Rightarrow] (1.center) to (0.center);
		\draw [style=Rightarrow] (1.center) to (2.center);
		\draw [style=Rightarrow] (1.center) to (3.center);
		\draw [style=Rightarrow] (0.center) to (8.center);
		\draw [bend left=15] (8.center) to (3.center);
	\end{pgfonlayer}
\end{tikzpicture}
}
      \caption{Quadrato composto da $v_1$ e $v_2$: analisi grafica}
      \label{fig:mtxAsaplin4-1}
    \end{figure}
    
    e con questo il primo punto è concluso.
  \item Sia $V=V_O^2$ lo spazio vettoriale dei vettori applicati in un punto $O$
    nel piano e sia $V_O^2\to V_O^2$ la proiezione ortogonale su una retta fissata
    $r$ passante per $O$ (in modo similare in quanto visto nella Figura \ref{fig:applindefes6}).

    Per calcolarne la matrice associata $M_B(f)$, si fissa una base $B=\{v_1,v_2\}$ di $V_O^2$
    come nella seguente figura
    \begin{figure}[ht!]
      \centering
      \resizebox{5cm}{!}{\begin{tikzpicture}
	\begin{pgfonlayer}{nodelayer}
		\node [style=none] (0) at (0, 4) {};
		\node [style=none] (1) at (0, 0) {};
		\node [style=none] (2) at (4, 0) {};
		\node [style=none] (3) at (-4, -4) {};
		\node [style=none] (4) at (4, 4) {};
		\node [style=none] (5) at (-0.5, 0) {$O$};
		\node [style=none] (6) at (4, -0.5) {$v_1$};
		\node [style=none] (7) at (-0.5, 4) {$v_2$};
		\node [style=none] (8) at (4.25, 4.25) {$r$};
	\end{pgfonlayer}
	\begin{pgfonlayer}{edgelayer}
		\draw (3.center) to (4.center);
		\draw [style=Rightarrow] (1.center) to (0.center);
		\draw [style=Rightarrow] (1.center) to (2.center);
	\end{pgfonlayer}
\end{tikzpicture}
}
      \caption{Calcolo della matrice associata $M_B(f)$ con $B=\{v_1,v_2\}$ di $V_O^2$}
      \label{fig:mtxAsaplin5}
    \end{figure}

    Poi, bisogna proiettare $v_1$ ortogonalmente su $r$, ottenendo un vettore $v$ che
    sta sulla retta ed è lungo come metà dela diametro del quadrato in cui lati sono
    $v_1$ e $v_2$: essendo tale diagonale, per definizione di somma tra vettori,
    coincidente con $v_1+v_2$, quindi si ha che $f(v_1)=\frac{1}{2}(v_1+v_2)=
    \frac{1}{2}v_1+\frac{1}{2}v_2$; analogamente, come si vede dalla figura, anche
    proiettando $v_2$ sulla retta si ottiene lo stesso vettore $v$, quindi si ha
    anche $f(v_2)=\frac{1}{2}(v_1+v_2)=\frac{1}{2}v_1+\frac{1}{2}v_2$.
    \begin{figure}[ht!]
      \centering
      \resizebox{5cm}{!}{\begin{tikzpicture}
	\begin{pgfonlayer}{nodelayer}
		\node [style=none] (0) at (0, 4) {};
		\node [style=none] (1) at (0, 0) {};
		\node [style=none] (2) at (4, 0) {};
		\node [style=none] (3) at (-4, -4) {};
		\node [style=none] (4) at (7, 7) {};
		\node [style=none] (5) at (4, 4) {};
		\node [style=none] (6) at (2, 2) {};
		\node [style=none] (7) at (0, 3.25) {};
		\node [style=none] (8) at (1.75, 1.75) {};
		\node [style=none] (9) at (0, 2.25) {};
		\node [style=none] (10) at (1.25, 1.25) {};
		\node [style=none] (11) at (0, 1.25) {};
		\node [style=none] (12) at (0.75, 0.75) {};
		\node [style=none] (13) at (-0.5, 0) {$O$};
		\node [style=none] (14) at (-0.5, 4) {$v_2$};
		\node [style=none] (15) at (4, -0.25) {$v_1$};
		\node [style=none] (16) at (7.25, 7.25) {$r$};
		\node [style=none] (17) at (4.5, 4) {$v_1+v_2$};
	\end{pgfonlayer}
	\begin{pgfonlayer}{edgelayer}
		\draw (3.center) to (4.center);
		\draw [style=Rightarrow] (1.center) to (2.center);
		\draw [style=Rightarrow] (1.center) to (0.center);
		\draw [style=DashedLine] (0.center) to (5.center);
		\draw [style=DashedLine] (2.center) to (5.center);
		\draw [style=DashedLine] (0.center) to (2.center);
		\draw [style=Rightarrow] (1.center) to (6.center);
		\draw [style=Rightarrow] (7.center) to (8.center);
		\draw [style=Rightarrow] (9.center) to (10.center);
		\draw [style=Rightarrow] (11.center) to (12.center);
		\draw [style=Rightarrow] (0.center) to (6.center);
	\end{pgfonlayer}
\end{tikzpicture}
}
      \caption{Proiezione $v_1$ e $v_2$ nella forma $v_1+v_2$ }
      \label{fig:mtxAsaplin6}
    \end{figure}
    
    Si vede quindi che le coordinate di $f(v_1)$ rispetto a $B$ sono $\left(\frac{1}{2},
      \frac{1}{2}\right)$, e anche le coordiante di $f(v_2)$ rispetto a $B$ sono
    $\left(\frac{1}{2},\frac{1}{2}\right)$: disponendo tali coordinate rispettivamente
    sulla prima e sulla seconda colonna, come previsto dalla definizione di matrice associata,
    si ottiene
    \begin{equation*}
      M_B(f)=
      \begin{pmatrix}
        \frac{1}{2} & \frac{1}{2}\\
        \frac{1}{2} & \frac{1}{2}
      \end{pmatrix}
    \end{equation*}
    e la funzione $\mathds{R}^2 \to \mathds{R}^2$, corrisponde a
    \begin{equation}
      \label{eq:mtxAsaplin13}
      (x_1,x_2)\to
      \begin{pmatrix}
        \frac{1}{2}x_1+\frac{1}{2}x_2, \frac{1}{2}x_1+\frac{1}{2}x_2
      \end{pmatrix}
    \end{equation}
    Quindi dà una rappresentazione in coordinate della proiezione.

    Ad esempio, il vettore $v=-v_1+v_2$, che ha coordinate $(-1,1)$ rispetto
    a $B$, viene mandata in base alla (\ref{eq:mtxAsaplin13}) nel vettore di
    coordinate
    \begin{eqnarray*}
      \begin{pmatrix}
        \frac{1}{2}\cdot (-1)+\frac{1}{2}\cdot 1, \frac{1}{2}\cdot(-1)+\frac{1}{2}\cdot 1
      \end{pmatrix}=(0,0)
    \end{eqnarray*}
    ovvero nel vettore nullo $\vec{OO}$. infatti, Come si vede nella figura seguente, tale vettore
    appartiene alla retta passante per $O$ e ortogonale a $r$, e i vettori che giacciono su questa
    retta vengono chiaramente proiettati sul vettore nullo $\vec{OO}$.
    \begin{figure}[ht!]
      \centering
      \resizebox{5cm}{!}{\begin{tikzpicture}
	\begin{pgfonlayer}{nodelayer}
		\node [style=none] (0) at (-5, 5) {};
		\node [style=none] (1) at (-5, 0) {};
		\node [style=none] (2) at (0, 5) {};
		\node [style=none] (3) at (0, 0) {};
		\node [style=none] (4) at (5, 0) {};
		\node [style=none] (5) at (5, -5) {};
		\node [style=none] (6) at (7, -7) {};
		\node [style=none] (7) at (9, -9) {};
		\node [style=none] (8) at (-7, 7) {};
		\node [style=none] (9) at (-9, 9) {};
		\node [style=none] (10) at (-10, 10) {};
		\node [style=none] (11) at (10, -10) {};
		\node [style=none] (12) at (-7, -7) {};
		\node [style=none] (13) at (8, 8) {};
		\node [style=none] (14) at (0, -0.5) {$O$};
		\node [style=none] (15) at (8.25, 7.5) {$r$};
		\node [style=none] (16) at (0.5, 5) {$v_2$};
		\node [style=none] (17) at (-4.5, 5.5) {$-v_1+v_2$};
		\node [style=none] (18) at (5, -0.5) {$v_1$};
	\end{pgfonlayer}
	\begin{pgfonlayer}{edgelayer}
		\draw (12.center) to (13.center);
		\draw [style=Rightarrow] (3.center) to (0.center);
		\draw [style=Rightarrow] (0.center) to (8.center);
		\draw [style=Rightarrow] (8.center) to (9.center);
		\draw [style=Rightarrow] (3.center) to (5.center);
		\draw [style=Rightarrow] (5.center) to (6.center);
		\draw [style=Rightarrow] (6.center) to (7.center);
		\draw (7.center) to (11.center);
		\draw (9.center) to (10.center);
		\draw [style=Rightarrow] (3.center) to (2.center);
		\draw [style=Rightarrow] (3.center) to (1.center);
		\draw [style=Rightarrow] (3.center) to (4.center);
		\draw [style=DashedLine] (1.center) to (0.center);
		\draw [style=DashedLine] (0.center) to (2.center);
	\end{pgfonlayer}
\end{tikzpicture}
}
      \caption{proiezione di $-v_1+v_2$ e del vettore nullo $\vec{OO}$}
      \label{fig:mtxAsaplin7}
    \end{figure}
  \item Come ultimo esempio, si prende $f:V_O^2\to V_O^2$ la riflessione rispetto a una retta
    fissata $r$ passate per $O$, ovvero l'endomorfismo che associa ogni vettore $\vec{OP}$ il
    suo simmetrico rispetto a $r$.

    Per calcolarne la matrice associata $M_B(f)$, considerando la stessa base $B=\{v_1,v_2\}$
    usata nell'esempio precedente
    \clearpage
    \begin{figure}[ht!]
      \centering
      \resizebox{5cm}{!}{\begin{tikzpicture}
	\begin{pgfonlayer}{nodelayer}
		\node [style=none] (0) at (0, 4) {};
		\node [style=none] (1) at (0, 0) {};
		\node [style=none] (2) at (4, 0) {};
		\node [style=none] (3) at (-4, -4) {};
		\node [style=none] (4) at (4, 4) {};
		\node [style=none] (5) at (-0.5, 0) {$O$};
		\node [style=none] (6) at (4, -0.5) {$v_1$};
		\node [style=none] (7) at (-0.5, 4) {$v_2$};
		\node [style=none] (8) at (4.25, 4.25) {$r$};
	\end{pgfonlayer}
	\begin{pgfonlayer}{edgelayer}
		\draw (3.center) to (4.center);
		\draw [style=Rightarrow] (1.center) to (0.center);
		\draw [style=Rightarrow] (1.center) to (2.center);
	\end{pgfonlayer}
\end{tikzpicture}
}
      \caption{Calcolo della matrice associata $M_B(f)$ con $B=\{v_1,v_2\}$ di $V_O^2$}
      \label{fig:mtxAsaplin8}
    \end{figure}
    e si nota che quando si riflette $v_1$ rispetto a $r$ ottenendo $v_2$, ovvero $f(v_1)=v_2$,
    e analogamente quando si riflette $v_2$ rispetto a $r$, ottenendo $v_1$, ovvero $f(v_2)=v_1$.

    Quindi, riscrivendo $f(v_1)=v_2$ come $f(v_1)=0v_1+1v_2$ si vede che le coordinate di $f(v_2)$
    rispetto a $B$ sono $(1,0)$: disponendo tali coordinate in colonna, come previsto dalla
    definizione di matrice associata, si ottiene
    \begin{eqnarray*}
      M_B(f)=
      \begin{pmatrix}
        0 & 1 \\
        1 & 0
      \end{pmatrix}
    \end{eqnarray*}
    Se, dato sempre lo stesso endomorfismo, consideriando invece la base $B^\prime=\{v_1^\prime,
    v_2^\prime\}$ come da figura
    \begin{figure}[ht!]
      \centering
      \resizebox{5cm}{!}{\begin{tikzpicture}
	\begin{pgfonlayer}{nodelayer}
		\node [style=none] (0) at (-4, 4) {};
		\node [style=none] (1) at (4, 4) {};
		\node [style=none] (2) at (-4, -4) {};
		\node [style=none] (3) at (4, -4) {};
		\node [style=none] (4) at (0, 0) {};
		\node [style=none] (5) at (6, 6) {};
		\node [style=none] (6) at (0, -0.5) {$O$};
		\node [style=none] (7) at (6, 5.5) {$r$};
		\node [style=none] (8) at (4.75, 3.75) {$v_1^\prime=f(v_1^\prime)$};
		\node [style=none] (9) at (-5, 4) {$v^\prime_2$};
		\node [style=none] (10) at (5, -4) {$f(v_2^\prime)=-v_2^\prime$};
	\end{pgfonlayer}
	\begin{pgfonlayer}{edgelayer}
		\draw (1.center) to (5.center);
		\draw [style=Rightarrow] (4.center) to (0.center);
		\draw [style=Rightarrow] (4.center) to (1.center);
		\draw [style=Rightarrow] (4.center) to (3.center);
		\draw (4.center) to (2.center);
	\end{pgfonlayer}
\end{tikzpicture}
}
      \caption{Esempio di endomosfismo con $B^\prime=\{v_1^\prime,v_2^\prime\}$}
      \label{fig:mtxAsaplin9}
    \end{figure}

    allora si ha che $f(v_1^\prime)=v_1^\prime$ e $f(v_2^\prime)=-v_2^\prime$, cioè $f(v_1^\prime)
    =1v_1^\prime+0v_2^\prime$ e $f(v_2)=0v_1^\prime+(-1)v_2^\prime$ e quindi
    \begin{eqnarray*}
      M_{B^\prime}(f)=
      \begin{pmatrix}
        1 & 0 \\
        0 & -1
      \end{pmatrix}
    \end{eqnarray*}
    Questo esempio illustra il fatto oovvio che la matrice associata dipende dalla scelta delle
    basi.
  \end{enumerate}
\end{es}

\section{Iniettività e suriettività di applicazioni lineari}
\label{sec:inietesuriet}

Il primo problema che verrà affrontato sulle applicazionio lineari è determinare quanto
una tale funzione è iniettiva, suriettiva o biiettiva.

\subsection{Richiemi generali}
\label{sec:richgen}

Ricordando che una funzione $f:A\to B$ tra due insiemi $A$ e $B$ si dice \emph{suriettiva}
se ogni elemento del codominio $B$ risulta essere immagine di qualche elemento di $A$,
ovvero, se per ogni $b\in B$ esiste un $a\in A$ tale che $f(a)=b$. Ad esempio, delle
funzioni rappresentate nel seguente disegno, la prima non è suriettiva, la seconda invece sì.
\clearpage
\begin{figure}[ht!]
  \centering
  \resizebox{10cm}{!}{\begin{tikzpicture}
	\begin{pgfonlayer}{nodelayer}
		\node [style=none] (2) at (1.5, 5) {};
		\node [style=none] (3) at (1.5, 0) {};
		\node [style=none] (4) at (6, 5) {};
		\node [style=none] (5) at (6, 0) {};
		\node [style=none] (6) at (11, 5) {};
		\node [style=none] (7) at (11, 0) {};
		\node [style=none] (8) at (15.5, 5) {};
		\node [style=none] (9) at (15.5, 0) {};
		\node [style=dot] (10) at (1.5, 4) {};
		\node [style=dot] (11) at (1.5, 3) {};
		\node [style=dot] (12) at (1.5, 2) {};
		\node [style=dot] (13) at (1.5, 1) {};
		\node [style=dot] (14) at (6, 3.5) {};
		\node [style=dot] (15) at (6, 2.5) {};
		\node [style=dot] (16) at (6, 1.5) {};
		\node [style=dot] (17) at (11, 4) {};
		\node [style=dot] (18) at (11, 3) {};
		\node [style=dot] (19) at (11, 2) {};
		\node [style=dot] (20) at (11, 1) {};
		\node [style=dot] (21) at (15.5, 3.5) {};
		\node [style=dot] (22) at (15.5, 2.5) {};
		\node [style=dot] (23) at (15.5, 1.5) {};
		\node [style=none] (24) at (1, 4) {1};
		\node [style=none] (25) at (1, 3) {2};
		\node [style=none] (26) at (1, 2) {3};
		\node [style=none] (27) at (1, 1) {4};
		\node [style=none] (28) at (6.5, 3.5) {$a$};
		\node [style=none] (29) at (6.5, 2.5) {$b$};
		\node [style=none] (30) at (6.5, 1.5) {$c$};
		\node [style=none] (38) at (1.5, -0.5) {$A$};
		\node [style=none] (39) at (6, -0.5) {$B$};
		\node [style=none] (40) at (11, -0.5) {$A$};
		\node [style=none] (41) at (15.5, -0.5) {$B$};
		\node [style=none] (42) at (10.5, 4) {1};
		\node [style=none] (43) at (10.5, 3) {2};
		\node [style=none] (44) at (10.5, 2) {3};
		\node [style=none] (45) at (10.5, 1) {4};
		\node [style=none] (46) at (16, 3.5) {$a$};
		\node [style=none] (47) at (16, 2.5) {$b$};
		\node [style=none] (48) at (16, 1.5) {$c$};
		\node [style=none] (49) at (2.5, -0.5) {};
		\node [style=none] (50) at (5, -0.5) {};
		\node [style=none] (51) at (12, -0.5) {};
		\node [style=none] (52) at (14.5, -0.5) {};
	\end{pgfonlayer}
	\begin{pgfonlayer}{edgelayer}
		\draw [bend left=90] (2.center) to (3.center);
		\draw [bend right=90] (2.center) to (3.center);
		\draw [bend left=90] (4.center) to (5.center);
		\draw [bend right=90] (4.center) to (5.center);
		\draw [bend left=90] (6.center) to (7.center);
		\draw [bend right=90] (6.center) to (7.center);
		\draw [bend left=90] (8.center) to (9.center);
		\draw [bend right=90] (8.center) to (9.center);
		\draw [style=Rightarrow] (10) to (14);
		\draw [style=Rightarrow] (11) to (14);
		\draw [style=Rightarrow, bend left=15, looseness=1.25] (12) to (15);
		\draw [style=Rightarrow, bend left=15, looseness=0.75] (13) to (15);
		\draw [style=Rightarrow] (17) to (21);
		\draw [style=Rightarrow] (18) to (21);
		\draw [style=Rightarrow, bend left=15] (19) to (22);
		\draw [style=Rightarrow, bend left=15] (20) to (23);
		\draw [style=Rightarrow] (49.center) to (50.center);
		\draw [style=Rightarrow] (51.center) to (52.center);
	\end{pgfonlayer}
\end{tikzpicture}
}
  \caption{esempio di surietività}
  \label{fig:ricgendelfab}
\end{figure}
Un modo alternativo di dire che una funzione è suriettiva, è proprio quello di fare riferimento
alla cosiddetta \textit{immagine} $I_m(f)$ di $f$: per definizione, l'immagine di una funzione
$f:A\to B$ è il sotoinsieme di $B$ costituito da tutti gli elementi che sono immagine di qualche
elemento di $A$\footnote{In riferimento alle figure, quegli elemnti (raggiunti da una freccia che
  proviene da $A^{\prime\prime}$)}, ovvero
\begin{eqnarray*}
  I_m(f)=\{b\in B\text{ | }b=f(a) \text{ per qualche } a\in A\}
\end{eqnarray*}
Ad esempio, la funzione a sinistra nella figura precedente ha $I_m(f)=\{a,b\}$, mentre la funzione
a destra $I_m(f)=\{a,b,c\}$: una funzione è suriettiva esattamente quando $I_m(f)=B$, ovvero l'immagine
coincide con tutto il codominio\footnote{dire $I_m(f)=B$ significa effetti dire che ogni elemento di
  $B$ è immagine di elemento di $A$}.\\
Una funzione $f:A\to B$ si dice invece \textit{iniettiva} se non succede che due elementi diversi di $A$
abbiano la stessa immagine\footnote{in formula, $f$ è iniettiva se $a_1\neq a_2\Rightarrow
  f(a_1)\neq f(a_1)$ o equivalentemente $a_1= a_2\Rightarrow f(a_1)= f(a_1)$}. Ad esempio, delle funzioni
rappresentate nella seguente figura, la prima non è iniettiva\footnote{in quanto nonostante $1\neq 2$
  si ha $f(1)=f(2)=a$}, la seconda sì.
\begin{figure}[ht!]
  \centering
  \resizebox{10cm}{!}{\begin{tikzpicture}
	\begin{pgfonlayer}{nodelayer}
		\node [style=none] (0) at (2, 5) {};
		\node [style=none] (1) at (2, 0) {};
		\node [style=none] (2) at (7, 5) {};
		\node [style=none] (3) at (7, 0) {};
		\node [style=none] (4) at (13.5, 5) {};
		\node [style=none] (5) at (13.5, 0) {};
		\node [style=none] (6) at (18.5, 5) {};
		\node [style=none] (7) at (18.5, 0) {};
		\node [style=dot] (8) at (7, 4) {};
		\node [style=dot] (9) at (7, 3) {};
		\node [style=dot] (10) at (7, 2) {};
		\node [style=dot] (11) at (7, 1) {};
		\node [style=dot] (12) at (2, 3.5) {};
		\node [style=dot] (13) at (2, 2.5) {};
		\node [style=dot] (14) at (2, 1.5) {};
		\node [style=none] (15) at (7.75, 4) {$a$};
		\node [style=none] (16) at (7.75, 3) {$b$};
		\node [style=none] (17) at (7.75, 2) {$c$};
		\node [style=none] (18) at (7.75, 1) {$d$};
		\node [style=none] (19) at (1.5, 3.5) {$1$};
		\node [style=none] (20) at (1.5, 2.5) {2};
		\node [style=none] (21) at (1.5, 1.5) {3};
		\node [style=none] (22) at (3, -0.75) {};
		\node [style=none] (23) at (6, -0.75) {};
		\node [style=none] (24) at (2, -0.75) {A};
		\node [style=none] (25) at (7, -0.75) {B};
		\node [style=none] (26) at (4.5, -1.25) {$f$};
		\node [style=none] (27) at (14.5, -0.75) {};
		\node [style=none] (28) at (17.5, -0.75) {};
		\node [style=none] (29) at (13.5, -0.75) {A};
		\node [style=none] (30) at (18.5, -0.75) {B};
		\node [style=none] (31) at (16, -1.25) {$f$};
		\node [style=dot] (32) at (13.5, 3.5) {};
		\node [style=dot] (33) at (13.5, 2.5) {};
		\node [style=dot] (34) at (13.5, 1.5) {};
		\node [style=dot] (35) at (18.5, 4) {};
		\node [style=dot] (36) at (18.5, 3) {};
		\node [style=dot] (37) at (18.5, 2) {};
		\node [style=dot] (38) at (18.5, 1) {};
		\node [style=none] (39) at (19.25, 4) {$a$};
		\node [style=none] (40) at (19.25, 3) {$b$};
		\node [style=none] (41) at (19.25, 2) {$c$};
		\node [style=none] (42) at (19.25, 1) {$d$};
		\node [style=none] (43) at (13, 3.5) {$1$};
		\node [style=none] (44) at (13, 2.5) {2};
		\node [style=none] (45) at (13, 1.5) {3};
	\end{pgfonlayer}
	\begin{pgfonlayer}{edgelayer}
		\draw [bend right=90] (1.center) to (0.center);
		\draw [bend right=90] (0.center) to (1.center);
		\draw [bend right=90] (3.center) to (2.center);
		\draw [bend right=90] (2.center) to (3.center);
		\draw [bend right=90] (5.center) to (4.center);
		\draw [bend right=90] (4.center) to (5.center);
		\draw [bend right=90] (7.center) to (6.center);
		\draw [bend right=90] (6.center) to (7.center);
		\draw [style=Rightarrow] (12) to (8);
		\draw [style=Rightarrow] (13) to (8);
		\draw [style=Rightarrow] (14) to (10);
		\draw [style=Rightarrow] (22.center) to (23.center);
		\draw [style=Rightarrow] (27.center) to (28.center);
		\draw [style=Rightarrow] (32) to (35);
		\draw [style=Rightarrow] (33) to (36);
		\draw [style=Rightarrow] (34) to (37);
	\end{pgfonlayer}
\end{tikzpicture}
}
  \caption{esempio di iniettività}
  \label{fig:ricgendelfab2}
\end{figure}

La nozione di iniettività può essere riformulata tramite il concetto di \textit{controimmagine}: dato
un elemento $b$ del codominio $B$, la sua controimmagine, denotata $f^{-1}(b)$, è l'insieme di tutti gli
elementi di $A$ che hanno $b$ come immagne, ovvero
\begin{eqnarray*}
  f^{-1}(b)=\{a\in A \text{ | } f(a)=b\}
\end{eqnarray*}
Dal momento che una funzione è iniettiva quando non esistono due elementi diversi che la stessa immagine,
dire che tutte le controimmagine che non siano vuote\footnote{se la funzione non è suriettiva, ci
  saranno elementi $b\in B$ tali che non esiste nessun $a\in A$ con $f(a)=b$, e quindi la cui
  controimmagine $f^{-1}(b)$ non ha elementi.} hanno un solo elemento.\\
Ad esempio per la funzione nella figura precedente si ha $f^{-1}(a)=\{1,2\},f^{-1}(d)=\diameter,
f^{-1}(c)=\{3\}$: essa non è iniettiva in quanto la controimmagine di $a$ ha due elementi. Per la
funzione a destra, invece, si ha $f^{-1}(a)=\{1\},f^{-1}(b)=\{2\}, f^{-1}(c)=\{3\},f^{-1}(d)=\diameter$:
essa è iniettiva in quanto le controimmagini non vuote hanno tutte un solo elemento. Infine, una funzione
si dice invece \textit{biiettiva} se è sia iniettiva che suriettiva. Ad esempio, la funzione
rappresentata nella seguente figura è biiettiva.
\clearpage
\begin{figure}[ht!]
  \centering
  \resizebox{10cm}{!}{\begin{tikzpicture}
	\begin{pgfonlayer}{nodelayer}
		\node [style=none] (0) at (2, 4) {};
		\node [style=none] (1) at (2, 0) {};
		\node [style=none] (2) at (6.5, 4) {};
		\node [style=none] (3) at (6.5, 0) {};
		\node [style=dot] (4) at (2, 3) {};
		\node [style=dot] (5) at (2, 2) {};
		\node [style=dot] (6) at (2, 1) {};
		\node [style=dot] (7) at (6.5, 3) {};
		\node [style=dot] (8) at (6.5, 2) {};
		\node [style=dot] (9) at (6.5, 1) {};
		\node [style=none] (10) at (2, -0.5) {A};
		\node [style=none] (11) at (6.5, -0.5) {B};
		\node [style=none] (12) at (2.5, -0.5) {};
		\node [style=none] (13) at (6, -0.5) {};
		\node [style=none] (14) at (4.25, -1) {$f$};
		\node [style=none] (15) at (1.5, 3) {1};
		\node [style=none] (16) at (1.5, 2) {2};
		\node [style=none] (17) at (1.5, 1) {3};
		\node [style=none] (18) at (7, 3) {$a$};
		\node [style=none] (19) at (7, 2) {$b$};
		\node [style=none] (20) at (7, 1) {$c$};
	\end{pgfonlayer}
	\begin{pgfonlayer}{edgelayer}
		\draw [bend right=90, looseness=1.25] (0.center) to (1.center);
		\draw [bend left=90, looseness=1.25] (0.center) to (1.center);
		\draw [bend right=90, looseness=1.25] (2.center) to (3.center);
		\draw [bend left=90, looseness=1.25] (2.center) to (3.center);
		\draw [style=Rightarrow, bend left] (4) to (7);
		\draw [style=Rightarrow, bend left] (5) to (8);
		\draw [style=Rightarrow, bend left] (6) to (9);
		\draw [style=Rightarrow] (12.center) to (13.center);
	\end{pgfonlayer}
\end{tikzpicture}
}
  \caption{caso di biiettività}
  \label{fig:ricgendelfab3}
\end{figure}

\subsection{Suriettività di applicazioni lineari}
\label{sec:suriappllin}
avendo detto che una funzione $f:A\to B$ è suriettiva se e solo se per
ogni elemento $b\in B$ esiste un $a\in A$ tale che $f(a)=b$, ovvero
equivalentemente se e solo se la sua immagine $I_m(f)$ coincide con tutto
il codominio. Quindi, per capire se un'applicazione è suriettiva,
bisogna determinare l'insieme $I_m(f)$.\\
Nel csaso di un'applicazione lineare $f:V\to W$, vale la seguente
importante
\begin{prop}
  \label{prop:suriappllin1}
  Sia $f:V\to W$ un'applicazione lineare. Allora $I_m(f)$ è un
  sottospazio vettoriale di $W$.
\end{prop}
\begin{proof}
  Bisogna verificare che $I_m(f)$ è chiuso rispetto alla somma e al
  prodotto per scalari. Per la prima proprietà bisogni prevedere $w,
  w^\prime\in I_m(f)$ e vedere se $w+w^\prime\in I_m(f)$ , per definizione
  di $I_m(f)$ e vedere se $w,w^\prime\in I_m(f)$. Ora, se
  $w,w^\prime\in I_m(f)$, per definizione di $I_m(f)$ significa che
  esistono un vettore $v\in V$ tale che $w=f(v)$ e un vettore
  $v^\prime\in V$ tale $w^\prime=f(v^\prime)$. Ma allora, sfruttando il
  fatto che $f$ è lineare, si ha
  \begin{eqnarray*}
    w+w^\prime=f(v)+f(v^\prime)=f(v+v^\prime)
  \end{eqnarray*}
  che afferma, anche che $w+w^\prime$ è immagine un elemento del dominio
  (cioè $v+v^\prime$) e quindi $w+w^\prime\in I_m(f)$.
  Per la chiusura rispetto al prodotto per scalari, bisogna verificare
  che se $w\in I_m(f)$ $c\in \mathds{K}$, allora $cw\in I_m(f)$. Ma, come
  prima, se $w\in I_m(f)$ allora per definizione di $I_m(f)$ esiste un
  vettore $w=f(v)$, e quindi, usando sempre il fatto che $f$ e lineare,
  \begin{eqnarray*}
    cw=cf(v)=f(cv)
  \end{eqnarray*}
  che ci dice anche $cw$ è immagine di un elemento del dominio (cioè
  $cv$) e quindi $cw\in I_m(f)$.
\end{proof}
Il fatto che $I_m(f)$ sia un sottospazio vettoriale ci dice che per
dterminarla è possibile trovare un sistema di generatori o una base.
Questo si fa facilmente grazie alla seguente
\begin{prop}
  \label{prop:suriappllin2}
  Sia $f:V\to W$ un'applicazione lineare e siano $v_1,\dots, v_n$
  generatori di $V$. Allora le immagini $f(v_1),\dots,f(v_n)$ generano
  $I_m(f)$\footnote{in simboli, $I_m(f)=(f(v_1),\dots,f(v_n))$}
\end{prop}
\begin{proof}
  Per definizione di generatori, bisogna verificare che ogni vettore
  $w\in I_m(f)$ si può scrivere come combinazione lineare dei vettori
  $f(v_1),\dots,f(v_n)$. Si sa che un vettore $w\in I_m(f)$ è tale che
  $w=f(v)$ per qualche vettore $v\in V.$ Ma essendo per ipotesi $v_1,
  \dots,v_n$ generatori di $V$, il vettore $v$ potrà essere scritto come
  loro combinazione lineare $v=x_1v_2+\cdots+x_nv_n$. Quindi, sfruttando
  la linearità di $f$ si ha
  \begin{eqnarray*}
    w=f(v)=f(x_1v_1+\cdots+x_nv_n)=x_1f(v_1)+\cdot+x_nf(v_n)
  \end{eqnarray*}
  che dimostra proprio che $w$ si scrive come combinazione lineare di
  $f(v_1),\dots,f(v_n)$.
\end{proof}
A questo punto, per definire se un'applicazione lineare $f:V\to W$ è
suriettiva, basta scegliere dei generatori $v_1,\dots,v_n$ di $V$,
prendere le loro immagini $f(v_1),\dots,f(v_n)$ che, formano un insieme
di generatori di $I_m(f)$, e poi estrarre da quest'ultima insieme una
base eliminando gli eventuali vettori che sono dipendenti dai rimanenti:
contando i vettori della base ottenuta, si saprà la dimensione di
$I_m(f)$ e quindi $f$ sarà se e solo se\footnote{Questa affermazione è
  giustificata dal fatto, che non è stato possibile dimostrare, che se
  $S$ è un sottospazio vettoriale di uno spazio vettoriale $V$, allora
  $\dim (V)$ e le dimensioni ccoincidono se e solo se $S=V$.}
$\dim(I_m(f))=\dim(W)$.\\
È il caso di vedere come tale criterio di suriettività si rivela
particolarmente utile e di semplice applicazione nel caso
dell'applicazione nel caso dell'applicazione lineare $L_A:\mathds{K}^n
\to \mathds{K}^m$ determinata da un matrice $A$, ovvero della forma data
dalla (\ref{eq:mtxAsaplin7}). Ora, se come generatore del dominio
$V=\mathds{K}^n$ si scelgono i vettori della base canonica $v_1=
(1,0,\dots,0),v_2=(0,1,\dots,0),v_n=(0,0,\dots,1)$, dalla
(\ref{eq:mtxAsaplin7}) si ha
\begin{eqnarray*}
  f(v_1)=L_A
  \begin{pmatrix}
    1\\
    0\\
    \vdots\\
    0
  \end{pmatrix}=
  \begin{pmatrix}
    a_{11}\\
    a_{21}\\
    \vdots\\
    a_{m1}
  \end{pmatrix},f(v_2)=L_A
  \begin{pmatrix}
    0\\
    1\\
    \vdots\\
    a_{m2}
  \end{pmatrix},\dots,f(v_n)=L_A
  \begin{pmatrix}
    0\\
    0\\
    \vdots\\
    1
  \end{pmatrix}=
  \begin{pmatrix}
    a_{1n}\\
    a_{2n}\\
    \vdots\\
    a_{mn}
  \end{pmatrix}
\end{eqnarray*}
cioè $f(v_1),f(v_2),\dots,f(v_n)$ sono le colonne $C_1,C_2,\dots,C_n$
della matrice $A$ che determina l'applicazione. Quindi, in base alla
Proposizione \ref{prop:suriappllin2}, si ha $I_m(f)=(C_1,C_2,\dots,C_n)$
e la funzione è suriettiva se e solo se $\dim(C_1,C_2,\dots,C_n)=
\dim(\mathds{K}^m)=m$. Ma poiché la dimensione del sottospazio generato
dalle colonne di una matrice è per definizione il suo rango, è possibile
concludere che \textit{un'applicazione del tipo (\ref{eq:mtxAsaplin7})
  è suriettiva se e solo se il rango di $A$ uguale a $m$}.

\subsection{Iniettività di applicazioni lineari}
\label{sec:iniediappllin}

Mentre, nel caso della suriettività si è visto che per verificare se una
funzione $f$ è suriettiva basta controllare un solo sottoinsieme del
codominio\footnote{l'immagine $I_m(f)$}, in generale per verificare se
$f$ e iniettiva bisogna controllare tutte le controimmagini degli
elementi del codominio e verificare che queste, quando non sono vuote,
hanno un solo elemento.\\
Ora, per una generica funzione $f:A\to B$ le controimmagine degli
elementi di $B$ sono sottoinsimi del tutto indipendenti tra loro: come
nella seguente figura
\begin{figure}[ht!]
  \centering
  \resizebox{8cm}{!}{\begin{tikzpicture}
	\begin{pgfonlayer}{nodelayer}
		\node [style=none] (0) at (2, 7) {};
		\node [style=none] (1) at (2, 0) {};
		\node [style=dot] (2) at (2, 6) {};
		\node [style=none] (3) at (2, 6.5) {};
		\node [style=none] (4) at (2, 5.5) {};
		\node [style=dot] (5) at (2, 4) {};
		\node [style=dot] (6) at (2, 3.5) {};
		\node [style=dot] (7) at (2, 2) {};
		\node [style=dot] (8) at (1.25, 1.5) {};
		\node [style=dot] (9) at (2, 0.75) {};
		\node [style=none] (10) at (2, 4.5) {};
		\node [style=none] (11) at (2, 3) {};
		\node [style=none] (12) at (1.25, 2.25) {};
		\node [style=none] (13) at (2.5, 0.5) {};
		\node [style=none] (14) at (8, 7) {};
		\node [style=none] (15) at (8, 0) {};
		\node [style=dot] (16) at (8, 5.5) {};
		\node [style=dot] (17) at (8, 3.5) {};
		\node [style=dot] (18) at (8, 1.5) {};
		\node [style=none] (19) at (8.5, 5.5) {$a$};
		\node [style=none] (20) at (8.5, 3.5) {$b$};
		\node [style=none] (21) at (8.5, 1.5) {$c$};
		\node [style=none] (22) at (-1, 6) {$f^{-1}(a)$};
		\node [style=none] (23) at (-1, 4) {$f^{-1}(b)$};
		\node [style=none] (24) at (-1, 1.25) {$f^{-1}(c)$};
		\node [style=none] (25) at (2, -0.5) {A};
		\node [style=none] (26) at (8, -0.5) {B};
		\node [style=none] (27) at (2.75, -0.5) {};
		\node [style=none] (28) at (7.25, -0.5) {};
		\node [style=none] (29) at (5, -1) {$f$};
	\end{pgfonlayer}
	\begin{pgfonlayer}{edgelayer}
		\draw [bend left=90] (1.center) to (0.center);
		\draw [bend left=90] (0.center) to (1.center);
		\draw [bend right=75, looseness=1.50] (4.center) to (3.center);
		\draw [bend left=75, looseness=1.50] (4.center) to (3.center);
		\draw [bend left=90, looseness=1.50] (10.center) to (11.center);
		\draw [bend right=90, looseness=1.50] (10.center) to (11.center);
		\draw [bend left=90, looseness=1.75] (12.center) to (13.center);
		\draw [bend right=90, looseness=1.50] (12.center) to (13.center);
		\draw [bend left=90] (14.center) to (15.center);
		\draw [bend right=90] (14.center) to (15.center);
		\draw [style=Rightarrow] (2) to (16);
		\draw [style=Rightarrow] (5) to (17);
		\draw [style=Rightarrow] (6) to (17);
		\draw [style=Rightarrow] (7) to (18);
		\draw [style=Rightarrow] (9) to (18);
		\draw [style=Rightarrow] (8) to (18);
		\draw [style=Rightarrow] (27.center) to (28.center);
	\end{pgfonlayer}
\end{tikzpicture}
}
  \caption{Iniettiità di applicazioni lineari}
  \label{fig:iniediappllin}
\end{figure}

può accadere che un elemento abbia controimmagine costituita da un solo
elemento ma altri abbiano controimmagine costituita da più elementi.\\
Mentre, le applicazione linari hanno il particolare comportamento per cui
le controimmagini degli elementi di $B$, se non solo vuote, o sono
\textit{tutte} costituite da un solo elemento o hanno \textit{tutti} più
di un elemento: quindi basta controllare una sola controimmagine non vuota
come sono fatte tutte le altre. Più precisamente si ha la sequente
\begin{prop}
  \label{prop:iniediappllin1}
  Sia $f:V\to W$ un'applicazione lineare. Allora valgono i seguenti fatti:
  \begin{enumerate}
  \item la controimmagine $f^{-1}(\bar{0})=\{v\in V \text{ | }
    f(v)=\bar{0}\}$ del vettore nullo di $W$ è un sottospazio vettoriale
    di un sottospazio vettoriale $V$ (detto \textit{nucleo di} $f$ e
    denotato $N(f)$)
  \item per ogni $w_0\in W$, la controimmagine $f^{-1}(w_0)$, se non è
    vuota, è sottospazio affine di $V$, e più precisamente
    \begin{eqnarray*}
      f^{-1}(w_0)=v_0N(f)=\{v_0+n \text{ | } n\in N(f)\}
    \end{eqnarray*}
    dove $v_0$ è un qualunque elemento fissato di $f^{-1}(w_0)$.
  \end{enumerate}
\end{prop}
\begin{proof}
  Per dimostrare il primo punto, bisogna osservare che il nucleo di $f$
  non è mai vuoto, in quanto il vettore nullo di $V$ è sicuramente tale
  che $f(\bar{0})=\bar{0}$: infatti, è possibile concepire il vettore
  nullo $\bar{0}$ di $V$ come $0v$ e quindi, sfruttando la linearità di
  $f$, si ha $f(\bar{0})=f(0v)=0f(v)=\bar{0}$.
  È il momento di dimostrare $N(f)$ è chiuso rispetto alla somma e al
  prodotto per scalari. Siano $v,v^\prime$ due vettori di $N(f)$, cioè
  $f(v)=\bar{0}$ e $f(v^\prime)=\bar{0}$. Allora, essendo $f$ lineare,
  \begin{eqnarray*}
    f(v+v^\prime)=f(v)+f(v^\prime)=\bar{0}+\bar{0}=\bar{0}
  \end{eqnarray*}
  e quindi anche $v+v^\prime\in N(f)$: questo si dice che $N(f)$ è chiuso
  rispetto alla somma. Dati invece un vettore $v$ del
  nucleo\footnote{quindi $f(v)=\bar{0}$} e uno scalare $c\in \mathds{K}$,
  allora, sempre per la linearità di $f$,
  \begin{eqnarray*}
    f(cv)=cf(v)=c\bar{0}=\bar{0}
  \end{eqnarray*}
  ovvero $cv\in N(f)$: questo dice che $N(f)$ è chiuso rispetto al
  prodotto per scalari. Con questo il punto 1 è stato dimostrato.\\
  Per dimostrare il secondo punto, ovvero l'uguaglianza $v_0+N(f)=
  f^{-1}(w_0)$, bisogna dimostrare che ogni elemento di $v_0+N(f)$ sta
  nella controimmagine $f^{-1}(w_0)$ di $w_0$ (ovvero $v_0+N\subseteq f^{-1}(w_0)$), e viceversa
  che ogni elemento di $f^{-1}(w_0)$ appartiene a $v_0+N(f)$\footnote{ovvero l'inclusione opposta
    $f^{-1}(w_0) \subseteq v_0+N(f)$}.\\
  Per dimostrare la prima inclusione, considerando il generico elemento di $v_0+N(f)$, cioè, per
  definizione di sottospazio affine, un vettore $v$ del tipo $v=v_0+n$, con $n\in N(f)$. Allora
  \begin{eqnarray*}
    f(v)=f(v_0+n)=f(v_0)+f(n)=f(v_0)+\bar{0}=f(v_0)=w_0
  \end{eqnarray*}
  Avendo quindi dimostrato che $f(v)=w_0$, cioè $v$ appartiene alla controimmagine $f^{-1}(w_0)$ di
  $w_0$. \\
  Per dimostrare la seconda inclusione, considerando un qualunque elemento $v$ della controimmagine
  di $w_0$, cioè $f(v)=w_0$. Essendo $w_0=f(v_0)$, si ha quindi $f(v)=f(v_0)$, da cui, portando a prima
  membro, $f(v)-f(v_0)=\bar{0}$. Essendo $f$ lineare, quest'ultima uguaglianza può essere riscritta
  come $f(v-v_0)=\bar{0}$, il che dice che il vettore $v-v_0$ appartiene al nucleo $N(f)$ di $f$.
  Ma allora, osservando che chiaramente $v=v_0+(v-v_0)$, vedendo che $v$ si decompone proprio come somma
  di $v_0$ e di un elemento del nucleo $N(f)$, cioè $v\in v_0+N(f)$.
\end{proof}
La proposizione appena dimostrata afferma in pratica che tutte le controimmagini non vuote di
un'applicazione lineare $f$ sono ``copie'' o traslati del nucleo $N(f)$: per illustrare ciò,
si consideri ad esempio lo spazio $V_o^2$ dei vettori nel piano applicati in $O$ e l'applicazione
lineare $f:V_o^2\to V_o^2$ data dalla proiezione su una retta $r$ fissata.\\
Come si vede, un vettore viene proiettato sul vettore nullo $\vec{OO}$\footnote{cioè appartiene al
  nucleo $N(f)$ della funzione} se e solo se appartiene alla retta passante per $O$ e ortogonale a $r$,
come i vettori $\vec{OP}, \vec{OQ}$ della seguente figura
\begin{figure}[ht!]
  \centering
  \resizebox{7cm}{!}{\begin{tikzpicture}
	\begin{pgfonlayer}{nodelayer}
		\node [style=none] (0) at (0, 0) {};
		\node [style=none] (1) at (0, 4) {};
		\node [style=none] (2) at (0, 2) {};
		\node [style=none] (3) at (4, 3.5) {};
		\node [style=none] (4) at (7, 0) {};
		\node [style=none] (5) at (-3, 0) {};
		\node [style=none] (6) at (4, 0) {};
		\node [style=none] (7) at (3.5, 0) {};
		\node [style=none] (8) at (3.5, 3) {};
		\node [style=none] (9) at (2.5, 2.25) {};
		\node [style=none] (10) at (2.5, 0) {};
		\node [style=none] (11) at (1.5, 1.25) {};
		\node [style=none] (12) at (1.5, 0) {};
		\node [style=none] (13) at (4.25, 3.75) {$R$};
		\node [style=none] (14) at (7.5, 0) {$r$};
		\node [style=none] (15) at (4, -0.5) {$f(\vec{OR})$};
		\node [style=none] (16) at (0, -0.5) {$O$};
		\node [style=none] (17) at (2, -1) {$f(\vec{OP})=f(\vec{OQ})=\vec{\infty}$};
	\end{pgfonlayer}
	\begin{pgfonlayer}{edgelayer}
		\draw [style=Rightarrow] (5.center) to (6.center);
		\draw [style=Rightarrow] (0.center) to (2.center);
		\draw [style=Rightarrow] (2.center) to (1.center);
		\draw [style=Rightarrow] (0.center) to (3.center);
		\draw (6.center) to (4.center);
		\draw [style=Rightarrow] (3.center) to (6.center);
		\draw [style=Rightarrow] (8.center) to (7.center);
		\draw [style=Rightarrow] (9.center) to (10.center);
		\draw [style=Rightarrow] (11.center) to (12.center);
	\end{pgfonlayer}
\end{tikzpicture}
}
  \caption{Proiezione sul vettore nullo $\vec{OO}$ mediante retta $O$}
  \label{fig:iniediappllin2}
\end{figure}
Ma, i vettori che stanno su una retta per $O$ formano un sottospazo vettoriale: questo conferma che
il nucleo $N(f)$ è un sottospazio vettoriale.\\
Per verificare che le controimmagini non vuote sono copie o traslat del nucleo, considerando come nella
figura seguente
\clearpage
\begin{figure}[ht!]
  \centering
  \resizebox{5cm}{!}{\begin{tikzpicture}
	\begin{pgfonlayer}{nodelayer}
		\node [style=none] (0) at (-3, 0) {};
		\node [style=none] (1) at (6, 0) {};
		\node [style=none] (2) at (0, 7) {};
		\node [style=none] (3) at (0, -6) {};
		\node [style=none] (4) at (4, 7) {};
		\node [style=none] (5) at (4, -6) {};
		\node [style=none] (6) at (0, 0) {};
		\node [style=none] (7) at (4, 0) {};
		\node [style=none] (8) at (-3, 0) {};
		\node [style=none] (9) at (4, 2) {};
		\node [style=none] (10) at (4, 4) {};
		\node [style=none] (11) at (0, 2) {};
		\node [style=none] (12) at (-3.25, 0) {$r$};
		\node [style=none] (13) at (0, 7.5) {$N(g)$};
		\node [style=none] (14) at (4, 7.5) {$\vec{OP}_o+N(g)$};
		\node [style=none] (15) at (4.25, -0.25) {$Q$};
		\node [style=none] (16) at (-0.25, -0.25) {$O$};
		\node [style=none] (17) at (-0.25, 2) {$R$};
		\node [style=none] (18) at (4.25, 4) {$P$};
		\node [style=none] (19) at (4.25, 2) {$P_0$};
	\end{pgfonlayer}
	\begin{pgfonlayer}{edgelayer}
		\draw (4.center) to (5.center);
		\draw (8.center) to (6.center);
		\draw (7.center) to (1.center);
		\draw [style=Rightarrow] (6.center) to (7.center);
		\draw [style=Rightarrow] (6.center) to (11.center);
		\draw [style=Rightarrow] (6.center) to (9.center);
		\draw [style=Rightarrow] (11.center) to (10.center);
		\draw (6.center) to (10.center);
		\draw (3.center) to (6.center);
		\draw (11.center) to (2.center);
	\end{pgfonlayer}
\end{tikzpicture}
}
  \caption{Immagine $f$ con $\vec{OQ},\vec{OR},\vec{OP}$ e $\vec{OP}_0$}
  \label{fig:iniediappllin3}
\end{figure}
un qualunque vettore $\vec{OQ}$ che stia nell'immagine di $f$, definendo $\vec{OQ}=f(\vec{OP}_0)$:
la sua controimagine, oltre che da $\vec{OP}_0$, è data da tutti i vettori $\vec{OP}$ che vengono
proiettati su $\vec{OQ}$, cioè, come si vede nella figura, tutti i vettori che hanno secondo estremo
sulla retta ortogonale a $r$ e passante per $Q$.\\
Ognuno di tali vettori $\vec{OP}$ si decompone come somma di $\vec{OP}_0$ più un vettore $\vec{OR}$
appartenente al nucleo $N(f)$: quindi, come previsto dalla Proposizione \ref{prop:iniediappllin1},
la controimmagine $f^{-1}(\vec{OP})$ è data dal sottospazio affine $\vec{OP}_0+N(f)$, traslato della
retta ortognonale a $r$ e passante per $O$ che rappresenta il nucleo. La Proposizione
\ref{prop:iniediappllin1} ha come immediato corollario il seguente criterio necessario e sufficiente
di iniettività per un'applicazione lineare:
\begin{corol}
  \label{corol:iniediappllin1}
  Un'applicazione lineare $f:V\to W$ è iniettiva se e solo se $N(f)=\{\bar{0}\}$. 
\end{corol}
\begin{proof}
  Un'applicazione lineare è iniettiva se e solo se la controimmagine di ogni elemento $w_0\in W$, se
  non è vuota\footnote{cosa che succede se $w_0$ non sta nell'immagine dell'applicazione}, contiene
  un solo elemento. Ma poiché, come visto nella proposizione, la controimmagine di ogni elemento $w_0$
  dell'immagine è del tipo $v_0+N(f)$, allora questa conterrà un solo elemento $v_0$ esattamente quando
  il nucleo contiene il solo vettore nullo $\bar{0}$, cioè $N(f)=\{\bar{0}\}$.
\end{proof}
Si noti che il nucleo sia un sottospazio vettoriale dice che si può parlare di base e dimensione del
nucleo, e che si possono riformulare il criterio di iniettività per un'applicazione lineare $f$ come
segue: $f$ \emph{è iniettiva se e solo se} $\dim(N(f))=0$\footnote{infatti, un sottospazio ha dimensione
  0 se e solo se è $\{\bar{0}\}$}.\\
Riassumendo quanto visto finora, come dimostrato per veriricare l'iniettività di un'applicazione
lineare $f$ si deve guardare la dimostrazione $\dim(N(f))$ del nucleo $N(f)$ di $f$, mentre per
verificare la suiriettività di $f$ si deve guardare la dimensione $\dim(I_m(f))$ dell'immagine
$I_m(f)$ di $f$.\\
Una cosa che appare lampante è che in realtà sarà sufficiente calcolare una sola di queste dimensioni
per determinare automaticamente anche l'altra. Infatti, queste dimensioni sono collegate dalla formula
del sequente risultato, dove anche \emph{teorema della dimensione} o \emph{teorema nullità più rango}.
\begin{teo}
  \label{teo:iniediappllin1}
  Sia $f:V\to W$ un'applicazione lineare, con $\dim(V)$ finita. Allora
  \begin{eqnarray}
    \label{eq:iniediappllin1-1}
    \dim(N(f))+\dim(I_m(f))=\dim(V).
  \end{eqnarray}
\end{teo}
\begin{proof}
  Supponendo che $\dim(N(f))=s$ e sia $v_1,\dots,v_s$ una base $N(f)$. Se $\dim(V)=n$, allora si
  può\footnote{Il fatto che data una base di un sottospazio, questa possa sempre essere completata
    a una base di tutto lo spazio aggiungendo dei vettori è in effetti un teorema detto \emph{teorema
      del completamento}.} aggiungere alla base del nucleo $n-s$ vettore $v_{s+1},\dots,v_n$ in modo
  che\\ $\{v_1,\dots,v_s,v_{s+1},\dots,v_n\}$ sia una base di $V$.\\
  Ora, dimostrando che le immagini $f(v_{s+1}),\dots,f(v_n)$ di questi ultimi $n-s$ vettori formano una
  base di $I_m(f)$:
  \begin{quote}
    Questo implicherà che $\dim(I_m(f)=n-s$, che assieme a $\dim(N(f))=s$ e $\dim(V)=n$ dà $\dim(N(f))+
    \dim(I_m(f))=s+(n-s)+\dim(V)$, cioè la formula.
  \end{quote}
  Per dimostrare che $\{f(v_{s+1},\dots,f(v_n)\}$ è una base di $I_m(f)$, bisogna dimostrare che i
  vettori generano $I$ e sono linearmente indipendenti. In effetti, sapendo che essendo $v_1,\dots,v_s,
  v_{s+1}\dots, v_n$ una base e quindi un insieme di generatori di $V$, le immagini $f(v_1),\dots,f(v_s),
  f(v_s+1),\dots,f(v_n)$ generatori $I_m(f)$; ma in questi generatori $f(v_1),\dots,f(v_s)$ sono uguali
  al vettore nullo $\bar{0}$, in quanto $v_1,\dots,v_s$ sono i vettori della base del nucleo fissata
  inzialmente. Quindi possono essere eliminati della lista $f(v_1),\dots,f(v_s), f(v_s+1),\dots,f(v_n)$
  dei generatori, concludendo che per generare $I_m(f)$ bastano $f(v_s+1),\dots,f(v_n)$.\\
  Ora dimostrando che $f(v_s+1),\dots,f(v_n)$ sono linearmente indipendenti. Bisogna dimostrare che se
  $c_{s+1}f(v_{s+1})+\cdots+c_nf(v_n)=\bar{0}$ allora $c_{s+1}=0,\dots,c_n=0$.\\
  In effetti, sfruttando il fatto che $f$ è lineare è possibile riscrivere l'uguaglianza
  $c_{s+1}f(v_{s+1})+\cdots+c_nf(v_n)=\bar{0}$ come $f(c_{s+1}v_{s+1}+\cdots+c_nv_n)=\bar{0}$: ma questa
  uguaglianza dice che il vettore $c_{s+1}v_{s+1}+\cdots+c_nv_n$ appartiene il nucleo $N(f)$, e quindi
  esso può esso può essere scritto come combinazione lineare di $v_1,\dots,v_s$\footnote{che
    del nucleo costituiscono una base}:
  \begin{eqnarray*}
    c_{s+1}v_{s+1}+\cdots+c_nv_n=c_1v_1+\cdots+c_sv_s.
  \end{eqnarray*}
  Portando tutto a primo membro in questa uguaglianza si ottiene
  \begin{eqnarray*}
    c_{s+1}v_{s+1}+\cdots+c_nv_n-c_1v_1-\cdots-c_sv_s=\bar{0}
  \end{eqnarray*}
  ovvero una combinazione lineare uguale al vettore nullo dei vettori $v_1,\dots,v_s, v_{s+1}\dots, v_n$:
  essendo questi vettori indipendenti\footnote{sono i vettori che formano la base completa di $V$}
  necessariamente tutti i coefficianti $c_1,\dots,c_s,c_{s+1},\dots,c_n$ sono uguali a zero, e in
  particolare $c_{s+1}=0,\dots,c_n=0$, che è quello che restava da dimostrare.
\end{proof}
Vedendo ora alcune notevoli conseguenze della formula (\ref{eq:iniediappllin1-1}):
\begin{corol}
  \label{corol:iniediappllin2}
  Sia $f:V\to W$ un'applicazione lineare, con $\dim (V)$ finita.
  Allora valgono le tre seguenti
  \begin{enumerate}[label=(\roman*)]
  \item se $\dim(V)>\dim(W)$ allora $f$ non è iniettiva;
  \item se $\dim(V)<\dim(W)$ allora $f$ non è suriettiva;
  \item se $\dim(V)=\dim(W)$ allora $f$ è iniettiva se e solo se è suriettiva (\texttt{caso di
      biiettività})
  \end{enumerate}
\end{corol}
\begin{proof}
  Quei di seguito verrano dimostrati i tre punti definiti nel Corollario \ref{corol:iniediappllin2}:
  \begin{enumerate}[label=(\roman*)]
  \item Per assurdo, se la funzioni $f$ fosse iniettiva, il suo nucleo, come visto nel Corollario,
    sasrebbe nullo, ovvero $\dim(N(f))=0$. Sostituendo questo nella formula (\ref{eq:iniediappllin1-1}),
    si avrà $\dim(V)=\dim(I_m(f))$. Ma essendo $I_m(f)$ un sottospazio di $W$, la sua dimensione è
    sicuramente minore o uguale a $\dim(W)$, contro l'ipotesi che $\dim(V)>\dim(W)$. Quindi $f$ non può
    essere iniettiva.
  \item Supponendo per assurdo che la funzione $f$ sia suriettiva: allora, essendo $I_m(f)=W$, si avrà
    $\dim(I_m(f))=\dim(W)$ che, sostituita nella formula (\ref{eq:iniediappllin1-1}), dà $\dim (V)=
    \dim(W)+\dim(N(f))$. Poiché $\dim(N(f))$ è un numero reale positivo o nullo, questa uguaglianza
    implica che $\dim(V)\leq \dim(W)$, contro l'ipotesi che $\dim(V)<\dim(W)$. Quindi $f$ non può essere
    suiettiva.
  \item Vicerversa, se $f$ è suriettiva, $\dim(V)=\dim(W)$ e quindi la formula
    (\ref{eq:iniediappllin1-1}) si riduce a $\dim(V)=\dim(N(f))+\dim(W)$. Essendo per ipotesi la
    dimensione di $V$ uguagle a quella di $W$, il primo membro $\dim(V)$ si semplifica con l'addendo
    $\dim(W)$ del secondo membro, e quindi rimane $=\dim(N(f))$, che implica che $f$ è iniettiva.
  \end{enumerate}
\end{proof}
Ad esempio, un'applicazione lineare $f:\mathds{R}^3\to\mathds{R}^2$ non può mai essere iniettiva
(ma può essere suriettiva); un'applicazionelineare $f:\mathds{R}^3\to\mathds{R}^4$ non è sicuramente
suriettiva (ma può essere iniettiva); un'applicazione lineare $f:\mathds{R}^3\to\mathds{R}^3$ o è
contemporaneamente iniettiva e suriettiva o nessuna della due: basta mostrare che vale una delle due
proprietà e automaticamente varrà anche l'altra.\\
Ora, supponendo di aver determinato grazie ai risultati precedenti che un'applicazione lineare
$f$ data è biiettiva.

\section{Composizione di applicazioni, inversa e prodotto di matrici}
\label{sec:Compinveeproddimatrici}

Ricordiamo che, data una funzione $f:X\to Y$ tra due insiemi, questa si
dice \textit{invertibile} se esiste una funzione $g:Y\to X$ (detta
appunto l'inversa di $f$) tale che
\begin{equation}
  \label{eq:Compinveeproddimatrici1}
  f \circ g=id_{y}, \text{ } g\circ f=id_{x}
\end{equation}
e si denota che con \textit{id} la funzione che manda ogni elemento in se
stesso\footnote{Più precisamente, $id_{x}$ è la funzione $X\to X$ che
  manda ogni elemento di $x$ in se stesso e $id_{y}$ denota la funzione
  $Y\to Y$ che manda ogni elemento di $Y$ in se stesso.} e con il simbolo
$\circ$ la composizione di funzioni, ovvero l'operazione che consiste
nell'applicare prima una fuzione e poi l'altra: più precisamente,
ricordiando che ogni volta che si hanno due funzioni $f:X\to Y$ e
$g:Y\to Z$, tali che \textit{il codominio della prima coincida con il
  della seconda,} allora, per ogni elemento che $Y$ è anche, per ogni
elemento $x\in X$, si può applicare prima $f$ ottenendo $f(x)\in Y$,
e poi dal momento che $Y$ è anche il dominio della $g$ si può applicare
la $g$ a $f(x)$. In questo modo si ottiene una nuova funzione che
associa a ogni elemento di $X$ un elemento di $Z$:
\begin{equation*}
  \underset{x\to g(f(x))}{f:X\to Z}
\end{equation*}
Quindi, le (\ref{eq:Compinveeproddimatrici1}) significano che una
funzione $f:X\to Y$ è invertibile se esiste una funzione $g:Y\to X$
tale $g(f(x))=x$ $x\in X$ e $f(g(y))=y$ per ogni $y\in Y$.\\
Ora, si può vedere che le uniche funzioni di $f$ invertibili sono
quelle biiettive. Ad esempio, considerando $X=\{1,2,3\}$, $Y=\{a,b,c\}$
e la funzione $f:X\to Y$ biiettiva, che ha come inversa la funzione
$g:Y\in X$ rappresentata nella figura:
\begin{figure}[ht!]
  \centering
  \resizebox{15cm}{!}{\begin{tikzpicture}
	\begin{pgfonlayer}{nodelayer}
		\node [style=none] (0) at (2, 5) {};
		\node [style=none] (1) at (2, 0) {};
		\node [style=none] (2) at (2, 4) {};
		\node [style=none] (3) at (2, 2.5) {};
		\node [style=none] (4) at (2, 1) {};
		\node [style=none] (5) at (8, 5) {};
		\node [style=none] (6) at (8, 0) {};
		\node [style=none] (7) at (8, 4) {};
		\node [style=none] (8) at (8, 2.5) {};
		\node [style=none] (9) at (8, 1) {};
		\node [style=none] (10) at (1.5, 4) {$a$};
		\node [style=none] (11) at (1.5, 2.5) {$b$};
		\node [style=none] (12) at (1.5, 1) {$c$};
		\node [style=none] (13) at (8.5, 4) {$1$};
		\node [style=none] (14) at (8.5, 2.5) {$2$};
		\node [style=none] (15) at (8.5, 1) {$3$};
		\node [style=none] (16) at (11.75, 5) {};
		\node [style=none] (17) at (11.75, 0) {};
		\node [style=none] (18) at (11.75, 4) {};
		\node [style=none] (19) at (11.75, 2.5) {};
		\node [style=none] (20) at (11.75, 1) {};
		\node [style=none] (21) at (17.75, 5) {};
		\node [style=none] (22) at (17.75, 0) {};
		\node [style=none] (23) at (17.75, 4) {};
		\node [style=none] (24) at (17.75, 2.5) {};
		\node [style=none] (25) at (17.75, 1) {};
		\node [style=none] (26) at (18.25, 4) {$a$};
		\node [style=none] (27) at (18.25, 2.5) {$b$};
		\node [style=none] (28) at (18.25, 1) {$c$};
		\node [style=none] (29) at (11.25, 4) {$1$};
		\node [style=none] (30) at (11.25, 2.5) {$2$};
		\node [style=none] (31) at (11.25, 1) {$3$};
		\node [style=none] (32) at (2, -0.5) {$X$};
		\node [style=none] (33) at (8, -0.5) {$Y$};
		\node [style=none] (34) at (5, -0.5) {$f$};
		\node [style=none] (35) at (11.75, -0.5) {$X$};
		\node [style=none] (36) at (17.75, -0.5) {$Y$};
		\node [style=none] (37) at (14.75, -0.5) {$y$};
	\end{pgfonlayer}
	\begin{pgfonlayer}{edgelayer}
		\draw [bend right=90] (1.center) to (0.center);
		\draw [bend right=90] (0.center) to (1.center);
		\draw [bend left=90] (5.center) to (6.center);
		\draw [bend left=90] (6.center) to (5.center);
		\draw [style=Rightarrow, bend left] (4.center) to (9.center);
		\draw [style=Rightarrow, bend left] (3.center) to (8.center);
		\draw [style=Rightarrow, bend left] (2.center) to (7.center);
		\draw [bend right=90] (17.center) to (16.center);
		\draw [bend right=90] (16.center) to (17.center);
		\draw [bend left=90] (21.center) to (22.center);
		\draw [bend left=90] (22.center) to (21.center);
		\draw [style=Rightarrow, bend left] (20.center) to (25.center);
		\draw [style=Rightarrow, bend left] (19.center) to (24.center);
		\draw [style=Rightarrow, bend left] (18.center) to (23.center);
	\end{pgfonlayer}
\end{tikzpicture}
}
  \caption{Funzione $f$ inversa di $X=\{1,2,3\}$ e $Y=\{a,b,c\}$}
  \label{fig:Compinveeproddimatrici1}
\end{figure}

Infatti, come si vede fin da subito, si ha
\begin{equation*}
  \begin{matrix}
    (g\circ f)(1)=g(f(1))=g(a)=1\\
    (g\circ f)(2)=g(f(2))=g(b)=2\\
    (g\circ f(3)=g(f(3))=g(c)=3
  \end{matrix}
\end{equation*}
ovvero $g\circ f:\{1,2,3\} \to \{1,2,3\}$ è la funzione che manda ogni elemento dell'insieme
$X=\{1,2,3\}$ in se stesso (ovvero la funzione identica $id_{X}$ di X) e analogamente
\begin{equation*}
  \begin{matrix}
    (g\circ f)(a)=g(f(a))=g(1)=a\\
    (g\circ f)(b)=g(f(b))=g(2)=b\\
    (g\circ f(c)=g(f(c))=g(3)=c
  \end{matrix}
\end{equation*}
ovvero $g\circ f:\{a,b,c\} \to \{a,b,c\}$ è la funzione che manda ogni elemento dell'insieme
$Y=\{a,b,c\}$ in se stesso (cioè la funzione identica $id_Y$). Per giusticare l'affermazione che le
funzioni biiettiva sono \textit{le sole} a essere invertibili, considerando ad esempio la funzione
$f$ rappresentata nella seguente figura, che è iniettiva ma non suriettiva
\clearpage
\begin{figure}[ht!]
  \centering
  \resizebox{15cm}{!}{\begin{tikzpicture}
	\begin{pgfonlayer}{nodelayer}
		\node [style=none] (0) at (2, 5) {};
		\node [style=none] (1) at (2, 0) {};
		\node [style=none] (2) at (2, 4) {};
		\node [style=none] (3) at (2, 2.5) {};
		\node [style=none] (4) at (2, 1) {};
		\node [style=none] (5) at (8, 5) {};
		\node [style=none] (6) at (8, 0) {};
		\node [style=none] (7) at (8, 4) {};
		\node [style=none] (8) at (8, 2.5) {};
		\node [style=none] (9) at (8, 1) {};
		\node [style=none] (10) at (1.5, 4) {$a$};
		\node [style=none] (11) at (1.5, 2.5) {$b$};
		\node [style=none] (12) at (1.5, 1) {$c$};
		\node [style=none] (13) at (8.5, 4) {$1$};
		\node [style=none] (14) at (8.5, 2.5) {$2$};
		\node [style=none] (15) at (8.5, 1) {$3$};
		\node [style=none] (16) at (11.75, 5) {};
		\node [style=none] (17) at (11.75, 0) {};
		\node [style=none] (18) at (11.75, 4) {};
		\node [style=none] (19) at (11.75, 2.5) {};
		\node [style=none] (20) at (11.75, 1) {};
		\node [style=none] (21) at (17.75, 5) {};
		\node [style=none] (22) at (17.75, 0) {};
		\node [style=none] (23) at (17.75, 4) {};
		\node [style=none] (24) at (17.75, 2.5) {};
		\node [style=none] (25) at (17.75, 1) {};
		\node [style=none] (26) at (18.25, 4) {$a$};
		\node [style=none] (27) at (18.25, 2.5) {$b$};
		\node [style=none] (28) at (18.25, 1) {$c$};
		\node [style=none] (29) at (11.25, 4) {$1$};
		\node [style=none] (30) at (11.25, 2.5) {$2$};
		\node [style=none] (31) at (11.25, 1) {$3$};
		\node [style=none] (32) at (2, -0.5) {$X$};
		\node [style=none] (33) at (8, -0.5) {$Y$};
		\node [style=none] (34) at (5, -0.5) {$f$};
		\node [style=none] (35) at (11.75, -0.5) {$X$};
		\node [style=none] (36) at (17.75, -0.5) {$Y$};
		\node [style=none] (37) at (14.75, -0.5) {$y$};
	\end{pgfonlayer}
	\begin{pgfonlayer}{edgelayer}
		\draw [bend right=90] (1.center) to (0.center);
		\draw [bend right=90] (0.center) to (1.center);
		\draw [bend left=90] (5.center) to (6.center);
		\draw [bend left=90] (6.center) to (5.center);
		\draw [style=Rightarrow, bend left] (4.center) to (9.center);
		\draw [style=Rightarrow, bend left] (3.center) to (8.center);
		\draw [style=Rightarrow, bend left] (2.center) to (7.center);
		\draw [bend right=90] (17.center) to (16.center);
		\draw [bend right=90] (16.center) to (17.center);
		\draw [bend left=90] (21.center) to (22.center);
		\draw [bend left=90] (22.center) to (21.center);
		\draw [style=Rightarrow, bend left] (20.center) to (25.center);
		\draw [style=Rightarrow, bend left] (19.center) to (24.center);
		\draw [style=Rightarrow, bend left] (18.center) to (23.center);
	\end{pgfonlayer}
\end{tikzpicture}
}
  \caption{Funzione $f$ non invertibile di $X=\{1,2,3\}$ e $Y=\{a,b,c,d\}$}
  \label{fig:Compinveeproddimatrici2}
\end{figure}
Si vede subito che $g\circ f$ è una funzione identica di $\{1,2,3\}$, ma $f\circ g$ non è una funzione identica
di $\{a,b,c,d\}$ in quanto pur avendo
\begin{equation*}
  \begin{matrix}
    (g\circ f)(1)=g(f(1))=g(a)=1\\
    (g\circ f)(2)=g(f(2))=g(b)=2\\
    (g\circ f(3)=g(f(3))=g(c)=3
  \end{matrix}
\end{equation*}
non si ha, invece,
\begin{equation*}
  \begin{matrix}
    (g\circ f)(a)=g(f(a))=g(1)=a\\
    (g\circ f)(b)=g(f(b))=g(2)=b\\
    (g\circ f(c)=g(f(c))=g(3)=c
  \end{matrix}
\end{equation*}
Come si evince, dalla Figura \ref{fig:Compinveeproddimatrici2}, è presente un equivalenza in più che rimanda
ad un elemento già puntato del gruppo $X$, $f(g(d))=f(1)=a$ ovvero $f\circ g$ non manda $d$ in se stesso.

Si osservi che non c'è alcun modo di modificare $g$ in modo che $f\circ g$ mandi ogni elemento di $\{a,b,c,d\}$ in
se stesso: qualunque valore venga assegnato a $d$ non sarà mai $f(g(d))=d$, perché $g(d)$ dovrebbe essere un elemento
mandato da $f$ di $d$, ma non esiste nssun elemento di $\{1,2,3\}$ che viene mandato da $f$ in $d$.\\
Espresso in altri termini, il modo per cui l'uguaglianza $f\circ g=id$ non può mai essere verificata è la non
surietività d $f$. Si dice che $f$ ha un'\textit{inversa sinistra}\footnote{cioè $g\circ f=id$ è verificato} ma non
ammette un'\textit{inversa destra} (cioè la $f\circ g=id$ non può valere per nessuna $g$). Analogamente, si prendano
$f:\{1,2,3\}\to \{a,b\}$ e $g:\{a,b\}\to \{1,2,3\}$ definite come dalla figura seguente
\begin{figure}[ht!]
  \centering
  \resizebox{15cm}{!}{\begin{tikzpicture}
	\begin{pgfonlayer}{nodelayer}
		\node [style=none] (4) at (1.5, 5.75) {};
		\node [style=none] (5) at (1.5, 0.75) {};
		\node [style=none] (6) at (6.5, 5.75) {};
		\node [style=none] (7) at (6.5, 0.75) {};
		\node [style=none] (29) at (1.5, 0) {X};
		\node [style=none] (30) at (6.5, 0) {Y};
		\node [style=none] (31) at (4, 0) {$f$};
		\node [style=dot] (32) at (1.5, 4.25) {};
		\node [style=dot] (33) at (1.5, 3.25) {};
		\node [style=dot] (34) at (1.5, 2.25) {};
		\node [style=dot] (35) at (6.5, 4.25) {};
		\node [style=dot] (36) at (6.5, 2.5) {};
		\node [style=none] (39) at (7.25, 4.25) {$a$};
		\node [style=none] (40) at (7.25, 2.5) {$b$};
		\node [style=none] (43) at (1, 4.25) {$1$};
		\node [style=none] (44) at (1, 3.25) {2};
		\node [style=none] (45) at (1, 2.25) {3};
		\node [style=none] (46) at (15, 5.75) {};
		\node [style=none] (47) at (15, 0.75) {};
		\node [style=none] (48) at (10, 5.75) {};
		\node [style=none] (49) at (10, 0.75) {};
		\node [style=none] (50) at (15, 0) {Y};
		\node [style=none] (51) at (10, 0) {X};
		\node [style=none] (52) at (12.5, 0) {$g$};
		\node [style=dot] (53) at (15, 4.25) {};
		\node [style=dot] (54) at (15, 3.25) {};
		\node [style=dot] (55) at (15, 2.25) {};
		\node [style=dot] (56) at (10, 4.25) {};
		\node [style=dot] (57) at (10, 3.25) {};
		\node [style=none] (60) at (9.5, 4.25) {$a$};
		\node [style=none] (61) at (9.5, 3.25) {$b$};
		\node [style=none] (64) at (15.5, 4.25) {$1$};
		\node [style=none] (65) at (15.5, 3.25) {2};
		\node [style=none] (66) at (15.5, 2.25) {3};
	\end{pgfonlayer}
	\begin{pgfonlayer}{edgelayer}
		\draw [bend right=90] (5.center) to (4.center);
		\draw [bend right=90] (4.center) to (5.center);
		\draw [bend right=90] (7.center) to (6.center);
		\draw [bend right=90] (6.center) to (7.center);
		\draw [style=Rightarrow] (32) to (35);
		\draw [style=Rightarrow] (33) to (36);
		\draw [bend right=90] (47.center) to (46.center);
		\draw [bend right=90] (46.center) to (47.center);
		\draw [bend right=90] (49.center) to (48.center);
		\draw [bend right=90] (48.center) to (49.center);
		\draw [style=Leftarrow] (53) to (56);
		\draw [style=Leftarrow] (54) to (57);
		\draw [style=Rightarrow] (34) to (36);
	\end{pgfonlayer}
\end{tikzpicture}
}
  \caption{Funzione $f$ non invertibile di $X=\{1,2,3\}$ e $Y=\{a,b\}$}
\label{fig:Compinveeproddimatrici3}
\end{figure}

Anllora, si vede fin da subito che $f\circ g$ è una funzione identica di $\{a,b\}$, ma $g\circ f$ non è la funzione
identica di $\{1,2,3\}$ in quanto pur essendo $g(f(1))=g(a)=1$ e $g(f(2))=g(b)=2$, si ha oltre tutto un $g(f(b))=g(b)=2$,
ovvero $g\circ f$ non manda 3 in se stesso.\\
Si osservi che anche qui non c'è alcun modo di modificare $g$: se si avesse posto $g(b)=23$ si otterrebbe
$g(f(3))=g(b)=3$ ma stavolta sarebbe stato $g(f(2))=g(b)=3$ saranno uguali e non potranno mai essere il
primo 2 e il secondo 3. In altri termini, il motivo per cui non esiste un'inversa sinistra di $f$ è dovuto
alla non iniettività di $f$: la $f$ ha un'inversa sinistra.

\begin{es}
  \label{es:Compinveeproddimatrici1}
  La funzione $f:\mathds{R}\to \mathds{R}$ data da $f(x)=x^2$, non essendo nè iniettiva\footnote{due numeri uno
    l'opposto dell'altro hanno lo stesso quadrato} nè suiettiva\footnote{i numeri negativi non sono quadrati di nessun
    numero reale} non ha né inversa sinistra né inversa a destra. In efetti, la candidata ad essere inversa di $f$,
  la radice quadrata $g(x)=\sqrt{x}$, da una parte non soddisfatta $g(f(x))=x$ se $x$ è negativa perché in tal caso
  $g(x^2)=\sqrt{x^2}=\abs{x}$, e $\abs{x}\neq x$ se essa è negativa $(\abs{-2}=+2\neq -2)$; dall'stra parte $f(g(y))=y$
  non ha senso se $y$ è negativo in quanto in tal caso $g(y)=\sqrt{y}$ non è neanche un numero reale. Per eliminare i
  due problemi (e rendere $g$ l'inversa di $f$) bisogna togliere i numeri negativi sia dal dominio che dal codominio di
  $f$, ovvero restringerla a una funzione $f:\mathds{R}_{\geq 0}\to \mathds{R}_{\geq0}$, dove $\mathds{R}_{\geq 0}$ denota
  l'insieme dei numeri non negativi: ma così facendo in effetti si fa proprio in modo che diventi una funzione
  biiettiva\footnote{non ci sono due elementi del dominio con lo stesso quadrato, e ogni elemento del codominio è
    quadrato di qualcosa}.
\end{es}
Ora, come spiegato sopra, si pone il problema di calcolare, se esiste, l'inversa di un applicazione lineare data. Prima
di fare ciò, dal momento che l'inversa è definita tramite la composizione, bisogna vedere cosa succede quando si compongono
due applicazioni lineari. In particolare, poiché ogni applicazione lineare può essere sempre tradotta in una formuala della
tipologia (\ref{eq:mtxAsaplin7}), e adesso sarà possibile osservare cosa succede quando si compongono due applicazioni di
questo tipo. Più nello specifico:
\begin{eqnarray}
  \label{eq:Compinveeproddimatrici1}
  \begin{matrix}
    V\to W\\
    \dim(V)=n\\
    \dim(W)=m\\
    L_A:\mathds{K}^n\to\mathds{K}^m
  \end{matrix} & f
                 \begin{pmatrix}
                   x_1\\
                   x_2\\
                   \vdots\\
                   x_n
                 \end{pmatrix}=
                 \begin{pmatrix}
                   a_{11}x_1+a_{12}x_2+\cdots+a_{1n}x_n\\
                   a_{21}x_1+a_{22}x_2+\cdots+a_{2n}x_n\\
                   \vdots\\
                   a_{m1}x_1+a_{m2}x_2+\cdots+a_{mn}x_n
                 \end{pmatrix}
\end{eqnarray}
Quindi dopo aver adattato la formula (\ref{eq:mtxAsaplin7}) alla $f=L_A:\mathds{K}^n\to \mathds{K}^m$, bisogna adattarla
alla $g=L_A:\mathds{K}^p\to \mathds{K}^n$
\begin{eqnarray}
  \label{eq:Compinveeproddimatrici2}
  \begin{matrix}
    Z\to V\\
    \dim(Z)=P\\
    L_B=\mathds{K}^P\to \mathds{K}^n
  \end{matrix}
  & g
  \begin{pmatrix}
    y_1\\
    y_2\\
    \cdots\\
    x_n
  \end{pmatrix}=
  \begin{pmatrix}
    b_{11}y_1+a_{12}y_2+\cdots +a_{1n}x_n\\
    b_{21}y_2+a_{22}y_2+\cdots+ a_{2n}x_n\\
    \cdots\\
    a_{m1}x_1+a_{m2}x_2+\cdots+ a_{mn}x_n
  \end{pmatrix}
\end{eqnarray}
Quindi, in sostanza, è possibile dire che:
\begin{eqnarray*}
  f\circ g: & \mathds{K}^P\overset{\color{red}g}{\to} \mathds{K}^n\overset{\color{red}f}{\to}
              \mathds{K}^m & A \text{ matrice $m\times n$ associata ad } f\\
            & y\to g(y)\to f(g(y)) &  B \text{ matrice $n\times p$ associata ad } g\\
  &&  AB \text{ matrice $m\times p$ associata ad } f\circ g
\end{eqnarray*}
In base a quanto appena visto, la composizione $f\circ g$ può essere calcolata in quanto il codominio di
$g$, ovvero $\mathds{K}^n$, è anche il dominio di $f$, Si ha, quindi
\begin{equation}
  \label{eq:Compinveeproddimatrici3}
  \begin{matrix}
    (f\circ g)
    \begin{pmatrix}
      y_1\\
      y_2\\
      \vdots\\
      y_p
    \end{pmatrix}=f
    \begin{pmatrix}
      b_{11}y_1+b_{12}y_2+\cdots+b_{1p}y_p\\
      b_{21}y_1+b_{22}y_2+\cdots+b_{2p}y_p\\
      \cdots\\
      b_{n1}y_1+b_{n2}y_2+\cdots+b_{np}y_p
    \end{pmatrix}=\\
    =\begin{pmatrix}
      a_{11}(b_{11}y_1+b_{12}y_2+\cdots+b_{1p}y_p)+\cdots+a_{1n}(b_{11}y_1+b_{12}y_2+\cdots+b_{np}y_p)\\
      a_{21}(b_{11}y_1+b_{12}y_2+\cdots+b_{1p}y_p)+\cdots+a_{2n}(b_{11}y_1+b_{12}y_2+\cdots+b_{np}y_p)\\
      \vdots\\
      a_{m1}(b_{11}y_1+b_{12}y_2+\cdots+b_{1p}y_p)+\cdots+a_{mn}(b_{11}y_1+b_{12}y_2+\cdots+b_{np}y_p)\\
    \end{pmatrix}=
  \end{matrix}
\end{equation}
La prossima cosa da fare è raggruppare per $y_1,y_2,\dots,y_p$, ottenendo quindi:
\begin{eqnarray*}
  =\begin{pmatrix}
    (a_{11}b_{11}+\cdots+a_{1n}b_{n1})y_1+(a_{11}b_{12}+\cdots+a_{1n}b_{n2})y_2
    +\cdots+(a_{11}b_{1p}+\cdots+a_{1n}b_{np})y_p\\
    (a_{21}b_{11}+\cdots+a_{1n}b_{n1})y_1+(a_{21}b_{12}+\cdots+a_{1n}b_{n2})y_2
    +\cdots+(a_{21}b_{1p}+\cdots+a_{2n}b_{np})y_p\\
    \vdots\\
    (a_{m1}b_{11}+\cdots+a_{mn}b_{n1})y_1+(a_{m1}b_{12}+\cdots+a_{mn}b_{n2})y_2
    +\cdots+(a_{m1}b_{1p}+\cdots+a_{mn}b_{np})y_p
  \end{pmatrix}
\end{eqnarray*}
Da quest'ultima espressione, è possibile vedere che la composizione $f\circ g$ è ancora una funzione
del tipo (\ref{eq:mtxAsaplin7}), cioè determinata da una matrice $C$, e più precisamente
\begin{equation}
  \label{eq:Compinveeproddimatrici4}
  C=
  \left(
    \begin{array}{cccc}
      a_{11}b_{11}+\cdots+a_{1n}b_{n1}& a_{11}b_{12}+\cdots+a_{1n}b_{n2}& \cdots & a_{11}b_{1p}+\cdots+a_{1n}
                                                                                   b_{np}\\
      a_{21}b_{11}+\cdots+a_{2n}b_{n1} & a_{21}b_{12}+\cdots+a_{2n}b_{n2}& \cdots & a_{21}b_{1p}+\cdots+a_{2n}
                                                                                    b_{np}\\
                                      & \vdots\\
      a_{m1}b_{11}+\cdots+a_{mn}b_{n1} & a_{m1}b_{12}+\cdots+a_{mn}b_{n2} & \cdots & a_{m1}b_{1n}+\cdots+a_{mn}
                                                                                     b_{np}
    \end{array}
  \right)
\end{equation}
Ora, si può notare che le entrate $c_{ij}$ di tale matrice sono tutte espressioni del tipo
\begin{equation}
  \label{eq:Compinveeproddimatrici5}
  c_{ij}=a_{i1}b_{1j}+\cdots+a_{in} b_{nj}
\end{equation}
In parte, è stato definito il prodotto di due matrici $A$ e $B$ in modo che la matrice $C=AB$ ottenuta sia
la matrice che determina la composizione $L_A\circ L_B$ delle funzioni determinate da $A$ e $B$.\\
Ricordando che la composizione $L_A\circ L_B$ delle due funzioni $L_A$ e $L_B$ può essere fatta solo sotto
opportune condizioni\footnote{il codominio di $L_B:\mathds{K}^p\to\mathds{K}^n$ deve essere uguale al dominio
  di $L_A:\mathds{K}^n\to K^m$} si osserva di conseguenza che anche il prodotto di due matrice può essere
fatto solo sotto opportune condizioni: più precisamente, dal momento che la matrice di $L_A$ ha $m$ righe
e $n$ colonne, mentre la matrice $B$ di $L_B$ ha $n$ righe e $p$ colonne, vediamo che si possono moltiplicare
tra loro due matrici $A$ e $B$ (in quest'ordine) se e solo se \textit{il numero di colonne di $A$ è uguale
  al numero di righe di $B$}.
\begin{es}
  \label{es:Compinveeproddimatrici2}
  Come visto nella (\ref{eq:mtxAsaplin10}), la matrice associata alle rotazione $f$ di angolo $\theta$ in
  senso antiorario attorno a $O$ rispetto a una base ortonormale di $V_O^2$ è $A=
  \begin{pmatrix}
    \cos \theta & -\sin\theta\\
    \sin \theta & \cos\theta
  \end{pmatrix}
  $: la funzione $L_A:\mathds{R}^2\to \mathds{R}^2$ determinata da $A$, che manda $(x_1,x_2)$ in
  $(\cos\theta x_1-\sin\theta x_2, \sin\theta x_1+\cos\theta x_2)$, è la traduzione in coordinate di $f$.\\
  Analogamente, se $g$ denota la rotazione di angolo $\phi$, la traduzione in coordinate di $g$ sarà data
  della funzione $L_B$ determinata dalla matrice $B=
  \begin{pmatrix}
    \cos \phi & -\sin\phi\\
    \sin \phi & \cos \phi
  \end{pmatrix}
  $. Dal momento che il prodotto di matrici dà la composizione delle applicazioni corrispondenti, se
  si moltiplicano $A$ e $B$ è possibile ottenere la matrice che rappresenta in coordinate la composizione
  delle due rotazioni: infatti
  \begin{equation*}
    \begin{matrix}
      AB=\begin{pmatrix}
      \cos \theta & -\sin\theta\\
      \sin \theta & \cos\theta
    \end{pmatrix} \begin{pmatrix}
      \cos \phi & -\sin\phi\\
      \sin \phi & \cos \phi
    \end{pmatrix}=\begin{pmatrix}
      \cos \theta \cos \phi -\sin\theta\sin\phi & -\cos\theta\sin\phi-\sin\theta \cos \phi\\
      \sin \theta\cos \phi +\cos\theta\sin\phi & -\sin\theta\sin\phi +\cos\theta \cos \phi
    \end{pmatrix}\\
      =
      \begin{pmatrix}
        \cos(\theta+\phi) & -\sin(\theta+\phi)\\
        \sin(\theta+\phi) & \cos(\theta+\phi)
      \end{pmatrix}
    \end{matrix}
  \end{equation*}
  che è ancora una matrice di rotazione ma relativa al angolo $\theta+\phi$: questo era prevedibile
  in quanto la composizione non fa altro che applicare una rotazione di angolo $\phi$ seguita da una rotazione
  di angolo $\theta$.
\end{es}
\begin{oss}
  \label{oss:Compinveeproddimatrici1}
  Il motivo della non commutatività in generale del prodotto di due matrici $A$ e $B$ è che, tale prodotto
  rappresenta la composizione delle applicazioni $L_A$ e $L_B$ corrispondenti, e la composizione di funzioni
  non gode in generale della proprietà commutativa: ad esempio, date le due funzioni
  \begin{eqnarray*}
    f:\mathds{R}\to \mathds{R},\text{ } f(x)=x+1, & g: \mathds{R}\to\mathds{R}, \text{ } g(x)=x^2
  \end{eqnarray*}
  si ha $f(g(x))=f(x^2)=x^2+1$, mentre, $g(f(x))=g(x+1)=(x+1)^2$, e quindi $f\circ g\neq g\neq f$. Per un
  esempio geometrico, si considerino la funzione $f:V_O^2\to V_O^2$ che ruota ogni vettore del piano applicato
  in $O$ di 90 gradi in senso antiorario, e la funzione $g:V_O^2\to V_O^2$che rifletta ogni vettore attorno a
  una retta $r$ passante per $O$: allora, come si vede nella seguente figura, applicare prima la rotazione $f$
  e poi la riflessione $g$ oppure viceversa porta in generale a risultati diversi (ovvero $f\circ g\neq
  g\circ f$)
  \begin{figure}[ht!]
    \centering
    \resizebox{9cm}{!}{\begin{tikzpicture}
	\begin{pgfonlayer}{nodelayer}
		\node [style=none] (0) at (0, 0) {};
		\node [style=none] (1) at (6, 0) {};
		\node [style=dot] (2) at (2.75, 0) {};
		\node [style=none] (3) at (1, 3) {};
		\node [style=none] (4) at (1, -3) {};
		\node [style=none] (5) at (5.5, 2) {};
		\node [style=none] (6) at (3, -0.25) {$O$};
		\node [style=none] (7) at (5.75, 2) {$P$};
		\node [style=none] (8) at (1, 3.25) {$f(\vec{OP})$};
		\node [style=none] (9) at (-0.25, 0) {$r$};
		\node [style=none] (10) at (4.5, 1.25) {};
		\node [style=none] (11) at (2, 1.25) {};
		\node [style=none] (12) at (7, 0) {$r$};
		\node [style=none] (13) at (7.25, 0) {};
		\node [style=none] (14) at (13, 0) {};
		\node [style=dot] (15) at (10, 0) {};
		\node [style=none] (16) at (11, 3) {};
		\node [style=none] (17) at (12, 2) {};
		\node [style=none] (18) at (12, -2) {};
		\node [style=none] (19) at (11, -1) {};
		\node [style=none] (20) at (10.5, 1.5) {};
		\node [style=none] (21) at (11, 3.25) {$f(g(\vec{OP}))$};
		\node [style=none] (22) at (9.75, -0.25) {$O$};
		\node [style=none] (23) at (12.25, 2) {$P$};
		\node [style=none] (24) at (12, -2.25) {$g(\vec{OP})$};
	\end{pgfonlayer}
	\begin{pgfonlayer}{edgelayer}
		\draw [style=Rightarrow] (2) to (3.center);
		\draw (0.center) to (1.center);
		\draw [style=Rightarrow] (2) to (5.center);
		\draw [style=Rightarrow] (2) to (4.center);
		\draw [style=DashedLine] (3.center) to (4.center);
		\draw [style=Leftarrow, bend left=15] (11.center) to (10.center);
		\draw (13.center) to (14.center);
		\draw [style=DashedLine] (17.center) to (18.center);
		\draw [style=Rightarrow] (15) to (17.center);
		\draw [style=Rightarrow] (15) to (18.center);
		\draw [style=Rightarrow] (15) to (16.center);
		\draw [style=Rightarrow, bend right=60, looseness=1.25] (19.center) to (20.center);
	\end{pgfonlayer}
\end{tikzpicture}
}
    \caption{situazione di disegueglianza tra $f\circ g$ e $g\circ f$}
    \label{fig:Compinveeproddimatrici4}
  \end{figure}
\end{oss}
\begin{oss}
  \label{oss:Compinveeproddimatrici2}
  l'associatività del prodotto di matrici si spiega facilmente ricordando che tale prodotto rappresenta la
  composizione delle funzioni corrispondenti, ovvero $(AB)C$ rappresenta $L_A\circ (L_B\circ L_C)$. Ma è facile
  vedere che la composizione di funzioni gode della proprietà associativa: infatti, applicando $(L_A\circ
  L_B)\circ h$ a $x$ si ottiene $(f\circ g)(h(x))=f(g(h(x)))$, e analogamente applicando $f\circ(g\circ h)$ a
  $x$ ottenendo sempre $f((g\circ h)(x))=f(g(h(x)))$, ovvero $(f\circ g)\circ h=f\circ (g\circ h)$.
\end{oss}
Ora, se il prodotto di matrici ammetta un elemento neutro che svolga la stesso ruolo che svolge il numero $1$
per il prodotto tra numeri, per cui si ha $a\cdot 1=1\cdot a =a$ per ogni numero $a$. La risposta è
affermativa: più precisamente, per ogni $n$ considerando la matrice con $n$ righe e $n$ colonne seguente
\begin{equation}
  \label{eq:Compinveeproddimatrici6}
  \begin{pmatrix}
    1 & 0 & \dots & 0\\
    0 & 1 & \dots & 0\\
      && \vdots\\
    0 & 0 & \dots & 1
  \end{pmatrix}
\end{equation}
ovvero la matrice che ha $1$ nelle entrate con stesso indice di riga e di colonne ($a_{11},a_{22}$, etc.) e 0
in tutte le altre entrate. Tale matrice si chiama \textit{matrice identica di ordine $n$} e si denota $I_n$.
Ad esempio,
\begin{eqnarray*}
  I_2=
  \begin{pmatrix}
    1 & 0\\
    0 & 1
  \end{pmatrix}, & I_3=
                   \begin{pmatrix}
                     1 & 0 & 0\\
                     0 & 1 & 0\\
                     0 & 0 & 1
                   \end{pmatrix}.
\end{eqnarray*}
\begin{oss}
  \label{oss:Compinveeproddimatrici3}
  In genere, data una matrice quadrata di ordine $n$, le entrate $a_{11},a_{22},\dots,a_{nn}$ che hanno stesso
  indice di riga e di colonna formano la cosiddetta \textit{diagonale della matrice}. La matrice identica
  $I_n$ può essere quindi descritta come la matrice che ha $1$ sulla diagonale e $0$ nelle altre entrate.
  Le entrate di $I_n$ si denotano solitamente con il simbolo $\delta_{ij}$, detto \textit{delta di
    Kronecker}, che vale quindi $1$ se $i=j$ e 0 se $i\neq j$.\\
  Ora,si può verificare che, per ogni $A\in M_{m,n}(\mathds{K})$ si ha
  \begin{eqnarray*}
    AI_n=A, & I_mA=A
  \end{eqnarray*}
  l'ordine della matrice identica cambia perché deve essere tale che si possa svolgere il prodotto e quindi
  la quindi la matrice identica svolge esattamente il ruolo di elemento neutro per il prodotto righe per
  colonne. Ad esempio,
  \begin{eqnarray*}
    \begin{pmatrix}
      1 & 2 & 3\\
      4 & 5 & 6 
    \end{pmatrix}
    \begin{pmatrix}
      1 & 0 & 0\\
      0 & 1 & 0\\
      0 & 0 & 1
    \end{pmatrix}=
    \begin{pmatrix}
      1 & 2 & 3\\
      4 & 5 & 6
    \end{pmatrix}\\
    \begin{pmatrix}
      1 & 0 \\
      0 & 1
    \end{pmatrix}
    \begin{pmatrix}
      1 & 2 & 3\\
      4 & 5 & 6 
    \end{pmatrix}=
    \begin{pmatrix}
      1 & 2 & 3\\
      4 & 5 & 6
    \end{pmatrix}
  \end{eqnarray*}
\end{oss}
\begin{oss}
  \label{oss:Compinveeproddimatrici4}
  Si noti che la matrice identica $I_n$ non è nient'altro che la matrice che determina la funzione identica
  $id_{\mathds{K}^n}:\mathds{K}^n\to\mathds{K}^n$ che manda ogni elemento in se stesso: infatti,
  \begin{equation}
    \label{eq:Compinveeproddimatrici7}
    id_{\mathds{K}^n}
    \begin{pmatrix}
      x_1\\
      x_2\\
      \vdots\\
      x_n
    \end{pmatrix}=
    \begin{pmatrix}
      x_1\\
      x_2\\
      \vdots\\
      x_n 
    \end{pmatrix}=
    \begin{pmatrix}
      1x_1+0x_2+\cdots+0x_n\\
      0x_1+1x_2+\cdots+0x_n\\
      \vdots\\
      0x_1+0x_2+\cdots+1x_n
    \end{pmatrix}
  \end{equation}
  Questo spiega perché tale matrice sia l'elemento neutro per il prodotto, in quanto il prodotto tra matrici
  rappresenta la composizione delle applicazioni corrispondenti e la funzione identica è esattamente
  l'elemento neutro per la composizione.
\end{oss}
Continuando con l'analogia con il prodotto tra numeri, è il caso ora di esaminare la questione
\textit{dell'esistenza dell'inverso}: nell'insieme dei numeri reali $\mathds{R}$ per ogni numero $a$ diverso
d zero esiste un numero $b$ tale che $ab=ba=1$, detto appunto inverso di $a$ (e denotato $a^{-1}$).
Nel caso delle matrici, quindi per capire se data una matrice $A\in M_{m,n}(\mathds{K})$ esiste una
matrice\footnote{La scelta del numero di righe e colonne di $B$ è obbligata se poter eseguire sia il
  prodotto $AB$ che quello $BA$. Una matrice invertibile si può anche chiamare \textsc{Non Singolare}, mentre,
  una matrice non invertibile si può anche chiamare \textsc{Singolare}} $B\in M_{n,m}(\mathds{K})$ tale che
$AB=I_m$ e $BA=I_n$. Tuttavia, se una tale $B$ esiste, dal momento che il prodotto di $A$ e $B$ corrisponde alla
composizione delle applicazioni lineari $L_A:\mathds{K}^n\to\mathds{K}^m$ e $L_B:\mathds{K}^m\to\mathds{K}^n$
mentre la matrice identica corrisponde alla funzione identica, si ha $L_A\circ L_B=id$ e $L_B\circ L_A=id$,
ovvero la funzione $L_A$ deve essere invertibile. Riassumendo, come mostrato, il problema dell'inverso si
pone solo per matrici che hanno stesso numero di righe e colonne, ovvero solo se $A\in M_{n,n}(\mathds{K})$.
Tali matrici si dicono \emph{quadrate} e il numero $n$ comune di righe e colonne si dice \textit{ordine delle
  matrice}. Per semplicità, l'insieme $M_{n,n}(\mathds{K})$ si denota $M_n(\mathds{K})$. 
\begin{defi}
  \label{defi:Compinveeproddimatrici1}
  Una matrice $A\in M_n(\mathds{K})$ quadrata d ordine $n$ si dice \textit{invertibile} se esiste una matrice
  $B\in M_n(\mathds{K})$ tale che
  \begin{eqnarray}
    \label{eq:Compinveeproddimatrici8}
    AB=I_n, & BA=I_n
  \end{eqnarray}
  In tal caso, $B$ si chiama \textit{matrice inversa di} $A$ e si denota $A^{-1}$. Bisogna, dimostrare il
  seguente risultato, contrariamente a quello che accade nel campo dei campo dei numeri reali dove l'unico
  numero non invertibile è lo zero, nell'insieme delle matrici, anche limitandosi alle sole matrici quadrate,
  ci sono molte matrici non invertibili:
\end{defi}
\begin{teo}
  \label{teo:Compinveeproddimatrici1}
  Una matrice $A\in M_n(\mathds{K})$ è invertibile se e solo se il rango di $A$ è uguale a $n$.
\end{teo}
\begin{proof}
  Come ricordato poco fa nella Definizione \ref{defi:Compinveeproddimatrici1}, l'invertibilità di $A$, ovvero
  l'esistenza di una matrice $B$ tale che $AB=BA=I_n$, equivale a dire $L_A\circ L_B=L_B\circ L_A=id$, ovvero
  che $L_A$ è invertibile\footnote{Per essere rigorosi, $L_A\circ L_B=L_B\circ L_A=id$ implica chiaramente che
    $L_A$ sia invertibile, ma viceversa il fatto che $L_A$ sia invertibile implica solo che esista una
    funzione $g$ tale che $g\circ L_A=L_A\circ g=id$ che a priori non si sa se è della forma $g=L_B$ per
    qualche matrice $B$. In realtà, si può dimostrare che $g$ deve necessariamente essere implica che esista
    una matrice $B$ per cui $L_A\circ L_B=L_B\circ L_A=id$ (e quindi $AB=BA=I_n$).} e quindi biiettiva. Quindi
  ci basta dimostrare che $L_A$ è biiettiva se e solo se il rango di $A$ è $n$. Una un funzione lineare in cui
  dominio e codominio abbiano la stessa dimensione (e $L_A:\mathds{K}^n\to \mathds{K}^n$ verifica questa
  condizione) è iniettiva se e solo se è suriettiva, ovvero basta una delle due proprietà per avere anche
  l'altra: in altri termini, per avere entrambe le proprietà
  (e quindi la biiettività) è necessaria e sufficiente una delle due
  proprietà. Quindi, possiamo dire che $L_A$ è biiettiva se e solo
  se e solo se è suriettiva. 
\end{proof}
Vediamo ora come si calcola l'inversa di una matrice $A$.
supponendo che questa esista (cioè che $A$ sia invertibile).

A questo punto siamo pronti a descrivere il primo metodo per la
determinazione dell'inversa di una matrice: grazie all'osservazione
preliminare appera fatta, se si sa che l'inversa di $A$ equivale a
trovare una matrice $B$ tale che $AB=I_n$ (senza dover verificare
anche $BA=I_n$), ovvero
\begin{eqnarray*}
  \begin{pmatrix}
    a_{11} & a_{12} & \dots & a_{1n}\\
    a_{21} & a_{22} & \dots & a_{2n}\\
           && \dots \\
    a_{n1} & a_{n2} &\dots & a_{nn}
  \end{pmatrix}
  \begin{pmatrix}
    b_{11} & b_{12} & \dots & b_{1n}\\
    b_{21} & b_{22} & \dots & b_{2n}\\
           && \dots\\
    b_{n1} & b_{n2} & \dots & b_{nn}
  \end{pmatrix}=
  \begin{pmatrix}
    1 & 0 & \dots & 0\\
    0 & 1 & \dots & 0\\
      && \dots\\
    0 & 0 & \dots & 1
  \end{pmatrix}
\end{eqnarray*}
Tenuto conto della definizione di prodotto righe per colonne,
vediamo che moltiplcando le righe di $A$ per la prima colonna di $B$
deve essere
\begin{eqnarray*}
  \begin{cases}
    a_{11}b_{11} +a_{12}b_{21}+\cdots + a_{1n}b_{n1}=1\\
    a_{21}b_{11} +a_{22}b_{21}+\cdots + a_{2n}b_{n1}=0\\
    \dots\\
    a_{n1}b_{11}+a_{n2}b_{21}+\cdots + a_{nn}b_{n1}=0
  \end{cases}
\end{eqnarray*}
ovvero la prima colonna di $B$ soddisfa il sistema
\begin{eqnarray}
  \label{eq:Compinveeproddimatrici9}
  \begin{cases}
    a_{11}x_1+a_{12}x_2+\cdots+a_{1n}x_n=1\\
    a_{21}x_{22}x_2+\cdots +a_{2n}x_n=0\\
    \cdots\\
    a_{n1}x_1+a_{n2}x_2+\cdots+a_{nn}x_n=0
  \end{cases}
\end{eqnarray}
con matrice dei coefficienti uguale a $A$ e termini noti uguali alla
prima colonna della matrice identica. Analogamente, moltiplicando
le righe di $A$ per la seconda colonna di $B$ si vede che devono
essere soddisfatte le seguenti
\begin{eqnarray*}
  \begin{cases}
    a_{11}b_{11} +a_{12}b_{21}+\cdots + a_{1n}b_{n2}=0\\
    a_{21}b_{11} +a_{22}b_{21}+\cdots + a_{2n}b_{n2}=1\\
    \dots\\
    a_{n1}b_{11}+a_{n2}b_{21}+\cdots + a_{nn}b_{n2}=0
  \end{cases}
\end{eqnarray*}
ovvero la prima colonna di $B$ soddisfa il sistema
\begin{eqnarray}
  \label{eq:Compinveeproddimatrici10}
  \begin{cases}
    a_{11}x_1+a_{12}x_2+\cdots+a_{1n}x_n=0\\
    a_{21}x_{22}x_2+\cdots +a_{2n}x_n=1\\
    \cdots\\
    a_{n1}x_1+a_{n2}x_2+\cdots+a_{nn}x_n=0
  \end{cases}
\end{eqnarray}
sempre con matrice dei coefficienti uguale a $A$ ma stavolta con
termini noti uguali alla seconda colonna della matrice identica,
e così via, è possibile ragionare allo stesso modo fino all'ultima
colonna di $B$ che dovrà soddisfare
\begin{eqnarray*}
  \begin{cases}
    a_{11}b_{1n}+a_{12}b_{2n}+\cdots +a_{1n}b_{nn}=0\\
    a_{21}b_{1n}+a_{22}b_{2n}+\cdots+a_{2n}b_{nn}=0\\
    \dots\\
    a_{n1}b_{1n}+a_{n2}b_{2n}+\cdots+a_{nn}b_{nn}=1
  \end{cases}
\end{eqnarray*}
ovvero il sistema
\begin{eqnarray}
  \label{eq:Compinveeproddimatrici11}
  \begin{cases}
    a_{11}x_1+a_{12}x_2+\cdots+a_{1n}x_n=0\\
  a_{21}x_1+a_{22}x_2+\cdots+a_{2n}x_n=0\\
  \cdots\\
  a_{n1}x_1+a_{n2}x_2+\cdots+a_{nn}x_n=1
  \end{cases}
\end{eqnarray}
Riassumendo, si ha $AB=I_n$ se e solo se le colonne
\begin{eqnarray*}
  \begin{pmatrix}
    b_{11}\\
    b_{21}\\
    \cdots\\
    b_{nn}
  \end{pmatrix}, &
                   \begin{pmatrix}
                     b_{12}\\
                     b_{22}\\
                     \cdots\\
                     b_{n2}
                   \end{pmatrix}, \text{ } \dots, &
                                            \begin{pmatrix}
                                              b_{1n}\\
                                              b_{2n}\\
                                              \cdots\\

                                              b_{nn}
                                            \end{pmatrix}
\end{eqnarray*}
sono soluzioni rispettivamente degli $n$ sistemi della
(\ref{eq:Compinveeproddimatrici9}),
(\ref{eq:Compinveeproddimatrici10}),
\dots, (\ref{eq:Compinveeproddimatrici11}). Per risolvere un sistema
basta scriverne la matrice completa, composta da matrice dei
coefficienti delle incognite e colonna dei termini noti, e ridurla
a gradini madiante operazioni elementari sulle righe. Poiché i
sistemi (\ref{eq:Compinveeproddimatrici9}),
(\ref{eq:Compinveeproddimatrici10}),
\dots, (\ref{eq:Compinveeproddimatrici11}) hanno tutti la stessa
matrice dei coeffcienti, cioè $A$, e differiscono solo per i
termini noti, è possibile risolverli tutti contemporaneamente
eseguendo la stesse operazioni elementari: a questo scopo, basta
scrivere la matrice
\begin{eqnarray}
  \label{eq:Compinveeproddimatrici12}
  (A| I_n)=\left(
  \begin{array}{cccc|cccc}
    a_{11}& a_{12}& \dots& a_{1n}& 1& 0& \dots&0 \\
    a_{21} & a_{22} & \dots &a_{2n} & 0 & 1 &\dots &0\\
          &&\dots&&&&\dots\\
    a_{n1}& a_{n2} & \dots&a_{nn} & 0 & 0 & \dots & 1
  \end{array}
  \right)
\end{eqnarray}
ottenuta affiancando tutti i termini noti dei sistemi
(\ref{eq:Compinveeproddimatrici9}),
(\ref{eq:Compinveeproddimatrici10}),
\dots, (\ref{eq:Compinveeproddimatrici11}) e risolverli
contemporaneamente con una sola riduzione\footnote{chiaramente,
  in base al Teorema \ref{teo:Compinveeproddimatrici1} la matrice
  $A$ è invertibile se e solo se in seguito a tale riduzione non si
  annullerà nessuna delle sue righe}.

Ad esempio: data la matrice $A=
\begin{pmatrix}
  1 & -1\\
  1 & 2
\end{pmatrix}
$, verificando se essa è invertibile e, in caso affermativo,
calcolando l'inversa. Come detto sopra, affiancando a tale matrice
la matrice identità dello stesso ordine
\begin{eqnarray*}
  (A|I_n)=\left(
  \begin{array}{cc|cc}
    1 & -1 & 1 & 0\\
    1 &  2 & 0 & 1
  \end{array}
  \right)
\end{eqnarray*}
che rappresenta i due sistemi le cui soluzioni le colonne della
matrice inversa, e si inizia con l'applicare il procedimento di
risoluzione a gradini: a questo scopo basta il singolo passaggio
\begin{eqnarray}
  \label{eq:Compinveeproddimatrici13}
  (A|I_n)=\left(
  \begin{array}{cc|cc}
    1 & cc-1 & 1 & 0 \\
    1 & 2 & 0 & 1
  \end{array}\right) \underset{R_2\to R_2-R_1}{\to}\left(
  \begin{array}{cc|cc}
    1 & -1 & 1 & 0\\
    0 & 3 & -1 & 1
  \end{array}\right)
\end{eqnarray}
da cui si vede che, dopo la riduzione a gradini, ad $A$ non si
annulla nessuna riga e quindi, come detto sopra, $A$ è
invertibile.\\
La prima colonna dell'inversa $B$ in $A$ è data dalla soluzione del
sistema ridotto
\begin{eqnarray}
  \label{eq:Compinveeproddimatrici14}
  \begin{cases}
    x_1-x_2=1\\
    3x_2=-1
  \end{cases}
\end{eqnarray}
ovvero, come si vede risolvendo dal basso, la coppia
$\left(\frac{2}{3},-\frac{1}{3}\right)$, che è quindi la prima
colonna della matrice inversa. Analogamente, la seconda colonna
dell'inversa $B$ di $A$ è data dalla soluzione del sistema ridotto
\begin{eqnarray}
  \label{eq:Compinveeproddimatrici15}
  \begin{cases}
    x_1-x_2=0\\
    3x_2=1
  \end{cases}
\end{eqnarray}
ovvero, come si vede risolvendo dal basso, la coppia
$\left(\frac{1}{3},\frac{1}{3}\right)$, che è quindi la seconda
colonna della matrice inversa. In conclusione, l'inversa della
matrice $A$ è
\begin{equation*}
  A^{-1}=
  \begin{pmatrix}
    \frac{2}{3} & \frac{1}{3}\\
    -\frac{1}{3} & \frac{1}{3}
  \end{pmatrix}
\end{equation*}
Infatti, si verifica subito con un calcolo che
\begin{equation*}
  \begin{pmatrix}
    1 & -1\\
    1 & 2
  \end{pmatrix}
  \begin{pmatrix}
    \frac{2}{3} & \frac{1}{3}\\
    -\frac{1}{3} & \frac{1}{3}
  \end{pmatrix}=\begin{pmatrix}
    \frac{2}{3} & \frac{1}{3}\\
    -\frac{1}{3} & \frac{1}{3}
  \end{pmatrix}\begin{pmatrix}
    1 & -1\\
    1 & 2
  \end{pmatrix}=
  \begin{pmatrix}
    1 & 0\\
    0 & 1
  \end{pmatrix}
\end{equation*}
concordemente con la definizione di inversa.
Per evitare di scrivere e risolvere separatamente i sistemi
(\ref{eq:Compinveeproddimatrici14}) e 
(\ref{eq:Compinveeproddimatrici15}) e trovare invece, in modo
più diretto la matrice inversa, si può procedere come segue. Dopo
aver effettuato la riduzione a gradini in
(\ref{eq:Compinveeproddimatrici13}), si applicano ulteriori
operazioni elementari fino a trasformare la matrice $A$ del blocco
legge direttamente l'inversa. Per vederlo, bisogna riprendere da 
(\ref{eq:Compinveeproddimatrici13}) e facciamo comparire prima
uno zero in posizione 1 2 eseguendo
\begin{eqnarray*}  
  \left(
  \begin{array}{cc|cc}
    1 & -1 & 1 & 0 \\
    0 & 3 & -1 & 1
  \end{array}\right) \underset{R_1\to 3R_1+R_2}{\to}\left(
  \begin{array}{cc|cc}
    3 & 0 & 2 & 1\\
    0 & 3 & -1 & 1
  \end{array}\right)
\end{eqnarray*}
e poi bisogna applicare a ogni riga l'operazione elementare del
secondo tipo che consiste nel dividerla per l'elemento che si
trovare sulla diagonale:
\begin{eqnarray}
  \label{eq:Compinveeproddimatrici16}
  \left(\begin{array}{cc|cc}
    3 & 0 & 2 & 1\\
    0 & 3 & -1 & 1
  \end{array}\right) \underset{
  \begin{matrix}
    R_1=\left(\frac{1}{3}\right) R_1\\
    R_2=\left(\frac{1}{3}\right) R_2
  \end{matrix}
  }{\to}\left(
  \begin{array}{cc|cc}
    1 & 0 & \frac{2}{3} & \frac{1}{3}\\
    0 & 1 & -\frac{1}{3}& \frac{1}{3}
  \end{array}\right)
\end{eqnarray}
Come si vede dalla matrice identica che è stata affiancata ad $A$
si è trasformata nella matrice inversa di $A$ già trovata sopra.\\
Per capire il perché, bisogna ricorcare che le operazioni elementari
che vengono qui svolte servono a risolvere contemporaneamente i
sistemi che poi danno come soluzione le colonne della matrice
inversa: ma allora, riducendo la matrice $A$ alla matrice identica
come in (\ref{eq:Compinveeproddimatrici16}) non si sta facendo altro
che ridurre i due sistemi alla forma
\begin{eqnarray*}
  \begin{cases}
    x_1=\frac{2}{3}\\
    x_2=-\frac{1}{3}
  \end{cases}, &
                 \begin{cases}
                   x_1=\frac{1}{3}\\
                   x_2=\frac{1}{3}
                 \end{cases}
\end{eqnarray*}
e cioè far comparire direttamente le soluzioni cercate (\emph{che sono
proprio le colonne della matrice inversa}).

Un altro esempio:
\begin{equation*}
  A=
  \begin{pmatrix}
    1 & 1 & 2\\
    -1 & 1 & 0\\
    2 & 1 & 1
  \end{pmatrix}.
\end{equation*}
La prima cosa da fare è quella di trasformare $(A|I_n)$ in una matrice a gradini, come
prima andando ad affiancare la matrice per la sua identità:
\begin{eqnarray*}
  (A|I_n)=\left(
  \begin{array}{ccc|ccc}
    1 & 1 & 2 & 1 & 0 & 0\\
    -1 & 1 & 0 & 0 & 1 & 0\\
    2 & 1 & 1 & 0 & 0 & 1
  \end{array}
  \right)\underset{
  \begin{matrix}
    R_2\to R_2 +R_1\\
    R_3\to R_3-2R_1
  \end{matrix}
  }{\to}\left(
  \begin{array}{ccc|ccc}
    1 & 1 & 2 & 1 & 0 & 0\\
    0 & 2 & 2 & 1 & 1 & 0\\
    0 & -1 & -3 & -2 & 0 & 1
  \end{array}
  \right)\\
  \underset{R_3\to 2R_3+R_2}{\to}
  \left(
  \begin{array}{ccc|ccc}
    1 & 1 & 2 & 1& 0&0\\
    0 & 2 & 2 & 1 & 1 & 0\\
    0 & 0 & -4 & -3 & 1 & 2
  \end{array}
  \right)
\end{eqnarray*}
Il fatto che non si sia annullata nessuna riga nel blocco di sinistra dice che la matrice $A$ è
invertibile.

Ora, come spiagato sompra, bisogna far comparire zeri sopra la diagonale, effettuando una sorta di
riduzione a gradini ``all'incontrario'', dal basso verso l'alto e destra verso sinistra\footnote{
  in ogni passaggio, bisogna mettere in evidenza in grassetto i nuovi zeri che si fanno comparire}
\begin{eqnarray*}
  \left(
  \begin{array}{ccc|ccc}
    1 & 1 & 2 & 1 & 0 & 0\\
    0 & 2 & 2 & 1 & 1 & 0\\
    0 & 0 & -4 & -3 & 1 & 2
  \end{array}\right) \underset{
  \begin{matrix}
    R_2\to R_2+R_3\\
    R_1\to 3R_1+R_3
  \end{matrix}
  }{\to}\left(
  \begin{array}{ccc|ccc}
    2 &2 & 0 & -1 & 1 & 2 \\
      0 & 4 &0 &-1 &3 &2 \\
      0 & 0 & -4 & -3 & 1&2
  \end{array}
  \right)\\
  \underset{
  R_1\to R_1-R_2
  }{\to}\left(
  \begin{array}{ccc|ccc}
    4 & 0 & 0 & -1 & -1 & 2 \\
    0 & 4 & 0 & -1 & 3 & 2 \\
    0 & 0 & -4 & -3 & 1 & 2 
  \end{array}
  \right)
\end{eqnarray*}
Infine, si divide ogni riga per l'elemento sulla diagonale applicando operazioni elementari
del secondo tipo
\begin{eqnarray*}
  \left(
  \begin{array}{ccc|ccc}
    4 & 0 & 0 & -1 & -1 & 2 \\
    0 & 4 & 0 & -1 & 3 & 2 \\
    0 & 0 & -4 & -3 & 1 & 2 
  \end{array}
  \right)\underset{
  \begin{matrix}
    R_1\to \frac{1}{4}R_1\\
    R_1\to \frac{1}{4}R_2\\
    R_3\to -\frac{1}{4}R_3
  \end{matrix}
  }{\to} \left(
  \begin{array}{ccc|ccc}
    1 & 0 & 0 & -\frac{1}{4} &-\frac{1}{4}&\frac{1}{2}\\
    0 & 1 & 0 & -\frac{1}{4} & \frac{3}{4} & \frac{1}{2}\\
    0 & 0 & 1 & \frac{3}{4} & -\frac{1}{4} & -\frac{1}{2}
  \end{array}
  \right)
\end{eqnarray*}
Quindi
\begin{equation*}
  A^{-1}=
  \begin{pmatrix}
    -\frac{1}{4} & -\frac{1}{4} & \frac{1}{2}\\
    -\frac{1}{4} & \frac{3}{4} & \frac{1}{2}\\
    \frac{3}{4} & -\frac{1}{4} & -\frac{1}{2}
  \end{pmatrix}.
\end{equation*}
Un modo alternativo per calcolare l'inversa di una matrice invertibile, basato sul determinante e sulla
notizoa di cofattore. Come visto nel Teorema \ref{teo:Compinveeproddimatrici1} che una matrice $A$ di
ordine $n$ è invertibile se e solo se il suo rango è $n$. Ma questo equivale ad avere $\det (A)\neq 0$.
Quindi, è necessario dimostrare la seguente
\begin{prop}
  \label{prop:Compinveeproddimatrici3}
  Sia $A\in M_n(\mathds{K})$ una matrice invertibile (ovvero con $\det(A)\neq 0$). Allora la sua
  inversa $A^{-1}$ è data da
  \begin{equation}
    \label{eq:Compinveeproddimatrici17}
    A^{-1}=\frac{1}{\det(A)}
    \begin{pmatrix}
      C_{11} & C_{21} & \dots & C_{n1}\\
      C_{12} & C_{22} & \dots & C_{n2}\\
             & \vdots\\
      C_{1n} & C_{2n} &\dots & C_{nn}
    \end{pmatrix}
  \end{equation}
  dove $C_{ij}$ indica il cofattore di $a_{ij}$, e $\frac{1}{\det(A)}$ davanti alla matrice dei cofattori
  significa che ogni entrata di tale matrice deve essere moltiplicata per $\frac{1}{\det(A)}$.
\end{prop}
\begin{oss}
  \label{oss:Compinveeproddimatrici4}
  A proposito delle disposizione dei cofattori nella (\ref{eq:Compinveeproddimatrici17}), si notiche i
  cofattori delle entrate della prima \textit{riga} di $A$ sono nella prima \textit{colonna} della
  (\ref{eq:Compinveeproddimatrici17}), i cofattori delle entrate della seconda \textit{riga} di $A$
  sono nella seconda \textit{colonna} della (\ref{eq:Compinveeproddimatrici17}), e così via.
\end{oss}
\end{document}

\documentclass{article}

\usepackage[utf8]{inputenc}
\usepackage{titlesec}
\usepackage{easylist}
\usepackage{hanging}
\usepackage{hyperref}
\usepackage[a4paper,top=2.0cm,bottom=2.0cm,left=2.0cm,right=2.0cm]{geometry}
\usepackage{blindtext}
\usepackage{tipa}
\usepackage{epigraph}
\usepackage{enumerate}
\usepackage{longtable}
\usepackage{setspace}
\usepackage{verbatim}
\usepackage[T1]{fontenc}
\usepackage{graphicx}
\usepackage[italian]{babel}
\usepackage{amsmath}
\usepackage{pbox}
\usepackage{fancyhdr}
\usepackage{cancel}
\usepackage{tabularx}
\usepackage{booktabs}
\usepackage{multirow}
\usepackage{longtable}
\usepackage{tikz}
\usepackage{tikz-qtree}
\usepackage{subfig}
\usepackage{xcolor}
\usepackage{amssymb}
\usepackage{mathrsfs}
\usepackage{textcomp}
\usepackage{dsfont}
\usepackage{listings}
\usepackage{color}

\definecolor{mygreen}{rgb}{0,0.6,0}
\definecolor{mygray}{rgb}{0.5,0.5,0.5}
\definecolor{mymauve}{rgb}{0.58,0,0.82}

\lstset{ 
  backgroundcolor=\color{white},   % choose the background color; you must add \usepackage{color} or \usepackage{xcolor}; should come as last argument
  basicstyle=\footnotesize,        % the size of the fonts that are used for the code
  breakatwhitespace=false,         % sets if automatic breaks should only happen at whitespace
  breaklines=true,                 % sets automatic line breaking
  captionpos=b,                    % sets the caption-position to bottom
  commentstyle=\color{mygreen},    % comment style
  deletekeywords={...},            % if you want to delete keywords from the given language
  escapeinside={\%*}{*)},          % if you want to add LaTeX within your code
  extendedchars=true,              % lets you use non-ASCII characters; for 8-bits encodings only, does not work with UTF-8
  firstnumber=1000,                % start line enumeration with line 1000
  frame=single,	                   % adds a frame around the code
  keepspaces=true,                 % keeps spaces in text, useful for keeping indentation of code (possibly needs columns=flexible)
  keywordstyle=\color{blue},       % keyword style
  language=Octave,                 % the language of the code
  morekeywords={*,...},            % if you want to add more keywords to the set
  numbers=left,                    % where to put the line-numbers; possible values are (none, left, right)
  numbersep=5pt,                   % how far the line-numbers are from the code
  numberstyle=\tiny\color{mygray}, % the style that is used for the line-numbers
  rulecolor=\color{black},         % if not set, the frame-color may be changed on line-breaks within not-black text (e.g. comments (green here))
  showspaces=false,                % show spaces everywhere adding particular underscores; it overrides 'showstringspaces'
  showstringspaces=false,          % underline spaces within strings only
  showtabs=false,                  % show tabs within strings adding particular underscores
  stepnumber=2,                    % the step between two line-numbers. If it's 1, each line will be numbered
  stringstyle=\color{mymauve},     % string literal style
  tabsize=2,	                   % sets default tabsize to 2 spaces
  title=\lstname                   % show the filename of files included with \lstinputlisting; also try caption instead of title
}

\linespread{1.5} % l'interlinea

\frenchspacing

\newcommand{\abs}[1]{\lvert#1\rvert}

\usepackage{floatflt,epsfig}

\usepackage{multicol}
\newcommand\yellowbigsqcup[1][\displaystyle]{%
  \fboxrule0pt
  \ifx#1\textstyle\fboxsep-0.6pt\else\fboxsep-1.25pt\fi
  \mathrel{\fcolorbox{white}{yellow}{$#1\bigsqcup$}}}

% definizioni
\newtheorem{teorema}{Teorema}[section]
\newtheorem{nota}{Nota}[section]
\newtheorem{notab}{Nota bene}[section]
\newtheorem{attenzione}{Attenzione}[section]
\newtheorem{defi}{Definizione}[section]
\newtheorem{proposizione}{Proposizione}[section]
\newtheorem{esempio}{Esempio}[section]
\newtheorem{esercizio}{Esercizio}[section]
\newtheorem{osservazione}{Osservazione}[section]
\newtheorem{corollario}{Corollario}[section]

\title{Formulario Geometria}
\author{Nicola Ferru}
\begin{document}
\maketitle

\section{Prodotto scalare, norma di un vettore e prodotto vettoriale}
\label{sec:prodScalNormvetteprodvett}
\begin{defi}
  Siano $\vec{u}=[u_1,u_2,u_3]$ e $\vec{v}=[v_1,v_2,v_3]$ due vettori. il loro
  prodotto scalare, denotato $\vec{u}\cdot \vec{v}$, è definito da
  \begin{eqnarray}
    \label{eq:prodottoscalare}
    \vec{u}\cdot \vec{v}=u_1v_1+u_2v_2+u_3v_3 & \left(\sum\limits_{i=1}^3 u_i v_3\right)
  \end{eqnarray}
\end{defi}
La proprietà fondamentale del prodotto scalare è
\begin{eqnarray}
  \label{eq:prodottoscalare1}
  \vec{u}\cdot \vec{v}=\abs{\vec{u}}\cdot\abs{\vec{v}}\cos\theta
\end{eqnarray}
%\begin{esempio}
%  \begin{eqnarray*}
%    \vec{u} \left(\frac{7}{3};-6;2k\right)\\
%    \vec{v} \left( -3k; -\frac{1}{2};k\right)\\
%    \vec{u}\cdot \vec{v}=\frac{7}{\not{3}}(-\not{3}k)+(-\not{6})(-\frac{1}{2})+
%    (2k)(k)=0\\
%    -7k+3+2k^2 =0\\
%    2k^2-7k+3=0\\
%    \Delta k=-7^2-4(2)(3)=49-24=25\\
%    k_{\frac{1}{2}}=\frac{+7\pm \sqrt{23}}{4}=
%    \begin{cases}
%      \frac{7+5}{4}=\frac{12}{4}=3\\
%      \frac{7-5}{4}=\frac{2}{4}=\frac{1}{2}
%    \end{cases}
%  \end{eqnarray*}
%\end{esempio}

\subsection{Norme di un vettore}
\label{sec:normVett}
\begin{defi}
  La norma di un vettore $\vec{u}=(u_1,u_2,\dots,u_n)\in \mathds{R}^n$ è una
  applicazione che a un vettore associa un numero reale
  \begin{eqnarray*}
    ||\cdot||: \mathds{R}^n\to \mathds{R}\\
    \vec{u}\mapsto ||\vec{u}||
  \end{eqnarray*}
  così definito:
  \begin{eqnarray*}
    ||\vec{u}||=\sqrt{v_1^2+u_2^2+\dots+u^2_n}
  \end{eqnarray*}
  In modo equivalente possiamo esprimere la norma di un vettore in termini di
  prodotto scalare:
  \begin{eqnarray*}
    ||\vec{u}||=\sqrt{\vec{u}\cdot\vec{u}}
  \end{eqnarray*}
  infatti
  \begin{eqnarray*}
    ||\vec{u}||=\sqrt{\vec{u}\cdot\vec{u}}=\sqrt{u_1u_1+u_2u_2+\dots+u_nu_n}=
    \sqrt{u_1^2+u_2^2+\dots+u_n^2}
  \end{eqnarray*}
\end{defi}
\subsection{Prodotto Vettoriale}
\label{sec:prodVett}

\begin{defi}
  Siano $\vec{u}=[u_1,u_2,u_3]$ e $\vec{v}=[v_1,v_2,v_3]$. Il loro prodotto
  vettoriale (indicato $\vec{u} \wedge \vec{v}$, oppure $\vec{u}\times \vec{v}$)
  è il vettore definito da
  \begin{eqnarray*}
    \vec{u}\wedge \vec{v}=[u_2v_3-u_3v_2,u_3v_1-u_1v_3,u_1v_2-u_2v_1]
  \end{eqnarray*}
\end{defi}
%\begin{esempio}
%  Siano $\vec{u}=[1,2,1],$ $\vec{v}=[6,-4,1]$.
%  \begin{eqnarray*}
%    \vec{u}\wedge \vec{v}=[2\cdot 1 -1 \cdot (-4), 1 \cdot 6-1\cdot 1,1\cdot (-4) -2\cdot 6]= [6,5,-16]
%  \end{eqnarray*}
%\end{esempio}

\section{Metodo di Cramer per sistemi lineari}
\label{sec:Metododicramer}
\begin{defi}
  Il \textbf{metodo di Cramer per sistemi lineari} è un procedimento per la
  risoluzione dei sistemi di equazioni lineari, e prevede di determinare le
  soluzioni dei sistemi lineari quadrati mediante il calcolo del determinante
  assoluto. Nel caso delle matrici 2x2:
  \begin{equation}
    \label{eq:MotododiCrmer}
    \det(A)=
    \begin{vmatrix}
      a_1 & b_1\\
      a_2 & b_2
    \end{vmatrix} = (a_1\cdot b_2) - (a_2 \cdot b_1)
  \end{equation}
  mentre, nel caso di una matrice 3x3:
  \begin{align}&\det(A) = \det\begin{bmatrix}a_1 & b_1 & c_1 \\ a_2 & b_2 & c_2 \\ a_3 & b_3 &
c_3\end{bmatrix}={+a_1\cdot b_2\cdot c_3+ b_1 \cdot c_2 \cdot a_3 + c_1 \cdot a_2
                  \cdot b_3 - b_1 \cdot a_2 \cdot c_3 - a_1 \cdot c_2 \cdot b_3 - c_1\cdot b_2 \cdot a_3}
  \end{align}
  Con un caso di matrice estesa che viene fatto nel seguente modo $
  \det(A)=\begin{array}{|ccc|cc} 
    a_1&b_1&c_1&a_1&b_1\\ 
    a_2&b_2&c_2&a_2&b_2\\
    a_3&b_3&c_3&a_3&b_3\\
  \end{array}
  $
\end{defi}
%\begin{enumerate}
%\item Prendendo una matrice $
%\det(A)=\begin{vmatrix}
%    3 & 2 \\
%    5 & 1
%\end{vmatrix}= (3\cdot 1)-(5\cdot 2)= 3-10=-7
%$.
%\item Questo era un esempio di matrici 3x3
%  \begin{eqnarray*}
%    \det(A)=\det\begin{bmatrix}
%      2 & 4 & 1\\
%      3 & 9 & 2\\
%      5 & 8 & 6        
%            \end{bmatrix}= +2\cdot 9\cdot 6+3\cdot 2 \cdot 1 + 5\cdot 4\cdot 2-
%    4\cdot2\cdot 8-1\cdot9\cdot 5=23
 % \end{eqnarray*}
%\end{enumerate}
\section{Algoritmo di Gauss Jordan}
\label{sec:algoritmodigauss}

\section{Sviluppo di Laplace per determinanti}
\label{sec:laplace}

\section{Polinomio caratteristico di un'applicazione lineare}
\label{sec:polinomicaratteristici}


\end{document}

\chapter {Numeri Complessi}
\newtheorem{NumComp}{Numeri reali}
\begin{NumComp}
	Un numero complesso è definito come un numero della forma $x+iy$, con x e y numeri reali e i una
	soluzione dell'equazione $x^2=-1$ detta unità immaginaria. i numeri reali
	sono
\end{NumComp}
\section{Operazioni con Numeri complessi}
\begin{enumerate}
\item Modulo e distanza
	\begin{equation}
		\abs{z}=\sqrt{x^2+y^2}
	\end{equation}
	Il valore assoluto (modulo) ha proprietà queste proprietà:
	\begin{equation*}
		\abs{z+w}\geq \abs{z}+\abs{w}, \text{ } \abs{zw}=\abs{z}\abs{w}, \text{ } \left|\frac{z}{w}\right|=\frac{\abs z}{\abs w}
	\end{equation*}
	Valide per tutti i numeri complessi $z$ e $w$. La prima proprietà è una versione della disuguaglianza triangolare.

\end{enumerate}
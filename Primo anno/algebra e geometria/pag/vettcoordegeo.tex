
\chapter{Vettori, coordinate e geometria}
\label{chap:vettcoordegeo}
Uno degli argomenti su cui il corso si basa sono proprio i \textit{vettori}. All'interno di questo capitolo saranno presenti nozioni e definizioni legate alla natura stessa di queste entità matematiche dai rudimenti ad alcuni spetti più avanzati.

\section{Vettori Geometrici}
\label{sec:vettorigeo}
\begin{defi}
  \label{def:vettorigeo}
  Un vettore geometrico applicato nel piano è un segmento orientato che va da un punto fisso O ``Origine'' verso un secondo punto $P$ del piano, come mostrato nella figira \ref{fig:vettorigeo}: 
  \begin{figure}[ht!]
    \centering
    \resizebox{3cm}{!}{
      \begin{tikzpicture}
	\begin{pgfonlayer}{nodelayer}
		\node [style=dot] (0) at (2, 1) {};
		\node [style=none] (1) at (-2, 1) {};
		\node [style=none] (2) at (-1, 5) {};
		\node [style=none] (3) at (4, 5) {};
		\node [style=none] (4) at (5, -3) {};
		\node [style=none] (5) at (-0.75, 5.5) {$S$};
		\node [style=none] (6) at (3.75, 5.5) {$P$};
		\node [style=none] (7) at (-2.25, 1.5) {$R$};
		\node [style=none] (8) at (5.25, -2.75) {$Q$};
		\node [style=none] (9) at (2.5, 1) {$O$};
	\end{pgfonlayer}
	\begin{pgfonlayer}{edgelayer}
		\draw [style=Rightarrow] (0) to (2.center);
		\draw [style=Rightarrow] (0) to (1.center);
		\draw [style=Rightarrow] (0) to (3.center);
		\draw [style=Rightarrow] (0) to (4.center);
	\end{pgfonlayer}
\end{tikzpicture}

    }
    \caption{Esempio vettori geometrici}
    \label{fig:vettorigeo}
  \end{figure}\\
  Analogamente, se il punto $P$ (\textit{e quindi il segmento}) è libero di variare in tutto lo spazio tridimensionale. In ambo i casi il vettore sarà denotato $\vec{OP}$ (\textit{si denota che il punto finale $P$ può anche uguale a $O$, ovvero il vettore può essere molto ravvicinato al punto $O$}).
\end{defi}
\begin{nota}
  \label{nota:vettorigeo}
  La direzione è indicata dalla simbolo freccia, graficamente la lunghezza e direzione del vettore implicano il modo in cui agisce nello spazio, ad esempio, se due vettori hanno direzioni opposte uno si sottrarrà potenzialmente al altro.
\end{nota}

\paragraph{Denotare che}
con $V_O^2$ l'unsieme dei vettori geometrici applicati in $O$ nel piano, e con $V_O^3$ l'insieme dei vettori geometrici applicati in $O$ liberi di variare in tutto lo spazio tridimensionale. I vettori orientati sono utilizzati infisica, dove vengono usati per rappresentare le forze applicate sul punto $O$.
\begin{es}
  Si può immaginare che in $O$ si trovi un oggetto sul quale viene esercitata una forza che lo ``trascina'' nella direzione e nel verso dati dalla freccia come evidenziato nella nota (\ref{nota:vettorigeo}), mentre l'intensità della forza esercitata è rappresentata dlla lunghezza del segmento.
\begin{figure}[ht!]
  \centering
  \resizebox{7.5cm}{!}{
      \begin{tikzpicture}
	\begin{pgfonlayer}{nodelayer}
		\node [style=none] (0) at (-2, 4) {};
		\node [style=none] (1) at (-4, 1) {};
		\node [style=none] (2) at (4, 4) {};
		\node [style=none] (3) at (2, 1) {};
		\node [style=none] (4) at (-2, 1) {};
		\node [style=none] (5) at (0, 4) {};
		\node [style=none] (6) at (2, 4) {};
		\node [style=none] (7) at (0, 1) {};
		\node [style=none] (8) at (-3.25, 2) {};
		\node [style=none] (9) at (2.5, 2) {};
		\node [style=none] (10) at (4.25, 4.25) {$P_3$};
		\node [style=none] (11) at (2.25, 0.5) {$P_1$};
		\node [style=none] (12) at (-4.5, 0.75) {$O$};
		\node [style=none] (13) at (-2, 4.25) {$P_2$};
		\node [style=none] (14) at (7, 4) {};
		\node [style=none] (15) at (5, 1) {};
		\node [style=none] (16) at (11, 1) {};
		\node [style=none] (17) at (13, 4) {};
		\node [style=none] (18) at (13.25, 4.25) {$P_3$};
		\node [style=none] (19) at (11.25, 0.5) {$P_1$};
		\node [style=none] (20) at (4.5, 0.75) {$O$};
		\node [style=none] (21) at (7, 4.25) {$P_2$};
	\end{pgfonlayer}
	\begin{pgfonlayer}{edgelayer}
		\draw [style=Rightarrow] (1.center) to (0.center);
		\draw [style=Rightarrow] (1.center) to (3.center);
		\draw [style=Dashedrightarrow] (4.center) to (5.center);
		\draw [style=Dashedrightarrow] (7.center) to (6.center);
		\draw [style=Rightarrow] (3.center) to (2.center);
		\draw [style=Rightarrow, bend left] (8.center) to (9.center);
		\draw [style=Rightarrow] (15.center) to (14.center);
		\draw [style=Rightarrow] (15.center) to (16.center);
		\draw [style=Rightarrow] (15.center) to (17.center);
		\draw [style=campitura] (17.center) to (16.center);
		\draw [style=campitura] (14.center) to (17.center);
	\end{pgfonlayer}
\end{tikzpicture}

    }
  \caption{Somma vettoriale}
  \label{fig:sommavect}
\end{figure}
Dal momento che $\vec{OP}_3$ rappresenta la forza totale esercitata la forza totale esercitata su $O$ quando si applicano contemporaneamente $\vec{OP_1}$ e $\vec{OP}_2$, il meccanismo più immediato è associare l'operazione ad una addizione, infatti, essa viene scritta come:
\begin{equation}
  \label{eq:sommavect}
  \vec{OP}_3=\vec{OP}_1+\vec{OP}_2
\end{equation}
La rappresentazione grafica è presente in figura \ref{fig:sommavect} definisce in modo in cui un'operazione di somma sull'insieme di vettori geometrici (del piono o dello spazio) viene rappresentata.
\end{es}
Per i vettori che non hanno la stessa direzione, si denota che $OP_3$ è la direzionale del parallelogramma che ha $OP_1$ e $OP_2$ come lati (infatti, viene definita anche come \textit{regola del parallelogramma}). Il motodo descrittivo funziona comunque anche per sommare due o più vettori che hanno la stessa direzione:
\begin{figure}[ht!]
  \centering
  \resizebox{9cm}{!}{
      \begin{tikzpicture}
	\begin{pgfonlayer}{nodelayer}
		\node [style=none] (0) at (0, 2) {};
		\node [style=none] (1) at (3, 2) {};
		\node [style=none] (2) at (5, 2) {};
		\node [style=none] (3) at (8, 2) {};
		\node [style=none] (4) at (1.5, 2.5) {};
		\node [style=none] (5) at (4.75, 2.5) {};
		\node [style=none] (6) at (1, 4.5) {};
		\node [style=none] (7) at (3.25, 3.25) {};
		\node [style=none] (8) at (8, 3.25) {};
		\node [style=none] (9) at (6, 4.5) {};
		\node [style=none] (10) at (6, 4) {$P_2$};
		\node [style=none] (11) at (1, 4) {$O$};
		\node [style=none] (12) at (2.75, 3.25) {$O$};
		\node [style=none] (13) at (8.25, 3.25) {$P_2$};
		\node [style=none] (14) at (5, 1.5) {$P_2$};
		\node [style=none] (15) at (3, 1.5) {$P_1$};
		\node [style=none] (16) at (0, 1.5) {$O$};
		\node [style=none] (17) at (10, 2) {};
		\node [style=none] (18) at (13, 2) {};
		\node [style=none] (19) at (15, 2) {};
		\node [style=none] (20) at (18, 2) {};
		\node [style=none] (21) at (15, 1.5) {$P_2$};
		\node [style=none] (22) at (13, 1.5) {$P_1$};
		\node [style=none] (23) at (10, 1.5) {$O$};
		\node [style=none] (24) at (18, 1.5) {$P_3$};
	\end{pgfonlayer}
	\begin{pgfonlayer}{edgelayer}
		\draw [style=Rightarrow] (0.center) to (1.center);
		\draw [style=Rightarrow] (1.center) to (2.center);
		\draw [style=Dashedrightarrow] (2.center) to (3.center);
		\draw [style=Rightarrow, bend left=75, looseness=2.75] (4.center) to (5.center);
		\draw [style=Dashedrightarrow] (6.center) to (9.center);
		\draw [style=Dashedrightarrow] (7.center) to (8.center);
		\draw [style=Rightarrow] (17.center) to (18.center);
		\draw [style=Rightarrow] (18.center) to (19.center);
		\draw [style=Rightarrow] (19.center) to (20.center);
	\end{pgfonlayer}
\end{tikzpicture}

    }
  \caption{Regola del parallelogramma}
  \label{fig:metparallelogramma}
\end{figure}\\
Anche in questo caso vale la formula \ref{eq:sommavect}, infatti, graficamente la $OP_3$ è chiaramente frutto di una somma tra il segmento $OP_1$ e $OP_2$. Un'altra operazione è il prodotto del vettore per un numero reale: nel contesto delle forze, il concetto è quella di rappresentare una variazione dell'intensità e eventualmente del verso della forza rappresentata dal vettore.\\
Più precisamente, dati un vettore geometrico $\vec{OP}$ e un numero releale $c\in\mathds{R}$, si può definire $c\vec{OP}$ come il vettore che sta sulla stessa retta a cui appartiene $\vec{OP}$, ma avente:
\begin{enumerate}
\item Stesso verso e lunghezza $c$ volte la lunghezza di $\vec{OP}$, se $c$ è positivo;
\item Verso opposto e lunghezza $-c$ volte quella di $\vec{OP}$, se $c$ è negativo;
\item Lunghezza ulla se c=0, cioè $0\vec{OP}=\vec{OO}$.
\end{enumerate}
\begin{figure}[ht!]
  \centering
  \resizebox{6cm}{!}{
      \begin{tikzpicture}
	\begin{pgfonlayer}{nodelayer}
		\node [style=none] (0) at (0, 1) {};
		\node [style=none] (1) at (2, 3) {};
		\node [style=none] (2) at (4, 5) {};
		\node [style=none] (3) at (-1, 0) {};
		\node [style=none] (4) at (-0.25, 1.25) {O};
		\node [style=none] (5) at (1.75, 3.25) {P};
		\node [style=none] (6) at (-1.25, -0.5) {$\frac{1}{2}\vec{OP}$};
		\node [style=none] (7) at (4.25, 5.5) {$2\vec{OP}$};
	\end{pgfonlayer}
	\begin{pgfonlayer}{edgelayer}
		\draw [style=Rightarrow] (0.center) to (1.center);
		\draw [style=Rightarrow] (1.center) to (2.center);
		\draw [style=Dashedrightarrow] (0.center) to (3.center);
	\end{pgfonlayer}
\end{tikzpicture}

    }
  \caption{Prodotto vettoriale}
  \label{fig:prodottovect}
\end{figure}
Nel contesto dei vettori, i numeri reali si chiamano anche \textit{scalari}.\\
Come si vedra nel ultima parte del capitolo, la nozione di vettore geometrico e le operazioni di somma tra vettori e prodotto di un vettore per un numero che appena definito saranno fornamentali per impostare e risolvere problemi geometrici nel piano e nello spazio. Per questo motivo, è necessario conoscere e mettere in evidenza le proprietà di cui godono tali operazionim che permettono di manipolare le espressioni e formule che coinvolgono i vettori. Si può verificare che valgono le seguenti:
\begin{enumerate}
\item La somma è \textit{associativa}
  \begin{equation}
    \label{eq:sommaassociativa}
    (\vec{OP}_1+\vec{OP}_2)+\vec{OP}_3=\vec{OP}_1+(\vec{OP}_2+\vec{OP}_3)
  \end{equation}
\item La somma è \textit{commutativa}
  \begin{equation}
    \label{eq:commutativa}
    \vec{OP}_1+\vec{OP}_2=\vec{OP}_2+\vec{OP}_1
  \end{equation}
\item Il vettore $\vec{OO}$ funge da elemento neutro per la somma:
  \begin{equation}
    \label{eq:sommaelementoneutro}
    \vec{OP}+\vec{OO}=\vec{OO}+\vec{OP}=\vec{OP}
  \end{equation}
\item Per ogni vettore $\vec{OP}$, il vettore $(-1)\vec{OP}$ (ovvero il vettore che si ottiene da $\vec{OP}$ basterà invertire il verso, senza modificare direzione e lunghezza) è il suo inverso additivo o opposto rispetto alla somma:
  \begin{equation}
    \label{eq:sommainversa}
    \vec{OP}+(-1)\vec{OP}=(-1)\vec{OP}+\vec{OP}=\vec{OO}
  \end{equation}
\item Dati due numeri reali $c_1$, $c_2$ e un vettore $\vec{OP}$, si ha
  \begin{equation}
    \label{eq:prodottoconduenumerireali}
    c_1(c_2\vec{OP})=(c_1c_2)\vec{OP}
  \end{equation}
  (\textit{Una situazione molto similare alla proprietà associativa del prodotto}).
\item Per ogni vettore $\vec{OP}$, si ha
  \begin{equation}
    \label{eq:perognivecOP}
    1\vec{OP}=\vec{OP}
  \end{equation}
  (\textit{ovvero il numero 1 funge da elemento neutro rispetto al prodotto per scalari}).
\item Dati due numeri reali $c_1$, $c_2$ ed un vettore $\vec{OP}$, si ha
  \begin{equation}
    \label{eq:numrealeVectOP}
    (c_1+c_2)\vec{OP}=c_1\vec{OP}+c_2\vec{OP}
  \end{equation}
\item Dati un numero reale $c$ e due vettori $\vec{OP}$, $\vec{OP}$ si ha
  \begin{equation}
    \label{eq:prodottoconduenumerirealiperunnumeroreale}
    c(\vec{OP}_1+\vec{OP}_2)=c\vec{OP}_1+c\vec{OP}_2
  \end{equation}
\end{enumerate}
Lo sviluppo suggerisce che valga la proprietà distributiva rispetto alla somma di numeri reale o rispetto alla somma di vettori.
\begin{oss}
  \label{oss:vettgeo1}
  Come esempio di applicazione delle proprietà appena elencate, è il caso di motrare che in un'uguaglianza tra vettori, esattamente come si fa in un'uguagliana tra numeri, si possono ``spostare i vettori'' da un membro all'altro cambiandoli di segno:
  \begin{equation*}
    \vec{OP}_1+\vec{OP}_2=\vec{OP}_3 \to \vec{OP}_1=\vec{OP}_3-\vec{OP}_2
  \end{equation*}
  Dove, come nel caso dei numeri lo spostamento dall'altra parte dell'uguaglianza comporta il cambiamento di segno scritto come $\vec{OP}_3-\vec{OP}_2$ che risulta essere la forma semplificata di $\vec{OP}_3+(-1)\vec{OP}_2$.\\
  Per vederlo, basterà sommare ad antrambi i membri di $\vec{OP}_1+\vec{OP}_2=\vec{OP}_3$ il vettore $(-1)\vec{OP}_2$:
  \begin{equation*}
    (\vec{OP}_1+\vec{OP}_2)+(-1)\vec{OP}_2=\vec{OP}_3+(-1)\vec{OP}_2
  \end{equation*}
  Applicando la propriatà associativa (\ref{eq:sommaassociativa}) a primo membro:
  \begin{equation*}
    \vec{OP}_!+\left[\vec{OP}_2+(-1)\vec{OP}_2\right]=\vec{OP}_3+(-1)\vec{OP}_2
  \end{equation*}
  Dopo aver fatto questo passaggio, sarà necessario applicare la proprietà (\ref{eq:sommainversa}) che afferma che $(-1)\vec{OP}_2$ è l'opposto di $\vec{OP}_2$:
  \begin{equation*}
    \vec{OP}_2+\vec{OO}=\vec{OP}_3+(-1)\vec{OP}_2
  \end{equation*}
  e infine va applicato la proprietà (\ref{eq:sommaelementoneutro}) che afferma che il vettore nullo funge da elemento neutro:
  \begin{equation*}
    \vec{OP}_1=\vec{OP}_3+(-1)\vec{OP}_2
  \end{equation*}
  e con questo è stata confermata l'affermazione iniziale.
\end{oss}

\section{Coordinate}
\label{sec:coordinate}

Considerando due vettori geometrici $\vec{OP}_1$ e $\vec{OP}_2$ nel piano, e si può supporre che $\vec{OP}_1$ e $\vec{OP}_2$ non abbiano la stessa dimensione. \\
Affermando che ogni vettore $\vec{OP}\in V_O^2$ può essere ottenuto sommando multipli opportuni di $\vec{OP}_1$ e $\vec{OP}_2$, ovvero:
\begin{equation*}
  \vec{OP}=c_1\vec{OP}_1+c_2\vec{OP}_2
\end{equation*}
dove $c_1$, $c_2$ sono opportuni numeri reali.
\clearpage
Infatti, questo può essere facilmente visto graficamente: come nel disegno seguente, prolungando i vettori $\vec{OP}_1$ e $\vec{OP}_2$ disegnando le due rette $r_1$ e $r_2$; proiettando quindi i punti $P$ su $r_1$ seguendo la direzione parallela a $\vec{OP}_2$, e chiamando il punto proiettato $Q_1$: e chiamandolo punto proiettato $Q_2$.
\begin{figure}[ht!]
  \centering
  \resizebox{6cm}{!}{
      \begin{tikzpicture}
	\begin{pgfonlayer}{nodelayer}
		\node [style=none] (0) at (0, 0) {};
		\node [style=none] (1) at (1, 2) {};
		\node [style=none] (2) at (2, 4) {};
		\node [style=none] (3) at (3.5, 7) {};
		\node [style=none] (4) at (2, 0) {};
		\node [style=none] (5) at (4, 0) {};
		\node [style=none] (6) at (7, 0) {};
		\node [style=none] (7) at (6, 4) {};
		\node [style=none] (8) at (2.75, 6.25) {$r_2$};
		\node [style=none] (9) at (1.5, 4) {$Q_2$};
		\node [style=none] (10) at (0.5, 2) {$P_2$};
		\node [style=none] (11) at (-0.5, -0.25) {O};
		\node [style=none] (12) at (2, -0.5) {$P_1$};
		\node [style=none] (13) at (4, -0.5) {$Q_1$};
		\node [style=none] (14) at (7, -0.5) {$r_1$};
		\node [style=none] (15) at (6.5, 4.25) {$P$};
	\end{pgfonlayer}
	\begin{pgfonlayer}{edgelayer}
		\draw [style=Rightarrow] (0.center) to (1.center);
		\draw [style=Rightarrow] (1.center) to (2.center);
		\draw (2.center) to (3.center);
		\draw [style=Rightarrow] (0.center) to (4.center);
		\draw [style=Rightarrow] (4.center) to (5.center);
		\draw (5.center) to (6.center);
		\draw [style=campitura] (5.center) to (7.center);
		\draw [style=campitura] (2.center) to (7.center);
		\draw [style=Rightarrow] (0.center) to (7.center);
	\end{pgfonlayer}
\end{tikzpicture}

    }
  \caption{Costruzione grafica $\vec{OP}=c_1\vec{OP}_1+c_2\vec{OP}_2$}
  \label{fig:costruvectgraph}
\end{figure}\\
Avendo costruito le due proiezioni parallelamente a $\vec{OP}_1$ e $\vec{OP}_2$ come lati e $\vec{OP}$ come diagonale, quindi per definizione di somma tra vettori geometrici si ha $\vec{OP}=\vec{OQ}_1+\vec{OQ}_2$.\\ Ma dal momento che $\vec{OQ}_1$ si trova sulla stessa retta di $\vec{OP}_1$ per come è definito il prodotto dei vettori per i numeri realim esisterà un numero reale $c_1$ tale che $\vec{OQ}_1=c_1\vec{OP}_1$ (dove $c_1$ dipende semplicemente dal rappotro tra la lunghezza di $\vec{OQ}_1$ e quella di $\vec{OP}_1$).\\
Si conclude che $\vec{OP}=c_1\vec{OP}_1+c_2\vec{OP}_2$.
Si noti che nella situazione considerata nel disegno, $c_1,c_2 > 0$ in quanto $\vec{OQ}_1$ e $\vec{OQ}_2$ hanno lo stesso verso di $\vec{OP}_1$ e $\vec{OP}_2$ rispettivamente. In generale, la stessa costruzione può essere effettuata per qualunque vettore $\vec{OP}$ del piano e i coefficienti $c_1$ e $c_2$ potranno anche essere negativi\footnote{Può essrere anche $c_1=0$ o $c_2=0$: nel primo caso, si ha $\vec{OP}=c_2\vec{OP}_2$, nel secondo $\vec{OP}=c_1\vec{OP}_1$, cioè $\vec{OP}$ non sta all'interno di uno dei quadranti in cui le rette $r_1,r_2$ dividono il piano, ma sta sulla retta $r_2$ (se $\vec{OP}=c_2\vec{OP}_2$) o sulla retta $r_1$ (se $\vec{OP}=c_1\vec{OP}_1$).} a seconda del quadrante nel quale si trova $\vec{OP}$, ovvero a seconda che la proiezione di $P$ sulle rette $r_1$, $r_2$ cada dalla stessa parte o dalla parte opposta dei punti $P_1$ e $P_2$.
\begin{figure}[ht!]
  \centering
  \resizebox{6cm}{!}{
      \begin{tikzpicture}
	\begin{pgfonlayer}{nodelayer}
		\node [style=none] (0) at (5, 4) {};
		\node [style=none] (1) at (3, 4) {};
		\node [style=none] (2) at (7, 4) {};
		\node [style=none] (3) at (1, 4) {};
		\node [style=none] (4) at (10, 4) {};
		\node [style=none] (5) at (11, 4) {};
		\node [style=none] (6) at (7, 8) {};
		\node [style=none] (7) at (4, 2) {};
		\node [style=none] (8) at (-1, 4) {};
		\node [style=none] (9) at (8, 10) {};
		\node [style=none] (10) at (3, 0) {};
		\node [style=none] (11) at (5, 8) {};
		\node [style=none] (12) at (9, 2) {};
		\node [style=none] (13) at (3.5, 1) {};
		\node [style=none] (14) at (-0.5, 1) {};
		\node [style=none] (15) at (5, 3.5) {O};
		\node [style=none] (16) at (7, 3.5) {$P_1$};
		\node [style=none] (17) at (9.25, 1.75) {$P$};
		\node [style=none] (18) at (6.5, 7) {};
		\node [style=none] (19) at (7, 7) {$P_2$};
		\node [style=none] (20) at (4.75, 8.25) {P};
		\node [style=none] (21) at (-1, 0.75) {$P$};
		\node [style=none] (22) at (0, 0) {$c_1<0$};
		\node [style=none] (23) at (0, -0.5) {$c_2<0$};
		\node [style=none] (24) at (10.75, 2) {$c_1>0$};
		\node [style=none] (25) at (10.75, 1.5) {$c_2<0$};
		\node [style=none] (26) at (3.25, 8.75) {$c_1<0$};
		\node [style=none] (27) at (3.25, 8.25) {$c_2>0$};
	\end{pgfonlayer}
	\begin{pgfonlayer}{edgelayer}
		\draw [style=Rightarrow] (0.center) to (2.center);
		\draw [style=Rightarrow] (2.center) to (4.center);
		\draw [style=Rightarrow] (0.center) to (1.center);
		\draw [style=Rightarrow] (1.center) to (3.center);
		\draw [style=Rightarrow] (0.center) to (7.center);
		\draw [style=campitura] (3.center) to (8.center);
		\draw [style=campitura] (4.center) to (5.center);
		\draw [style=campitura] (6.center) to (9.center);
		\draw [style=campitura] (1.center) to (11.center);
		\draw [style=campitura] (11.center) to (6.center);
		\draw [style=campitura] (4.center) to (12.center);
		\draw [style=campitura] (12.center) to (7.center);
		\draw [style=campitura] (3.center) to (14.center);
		\draw [style=campitura] (14.center) to (13.center);
		\draw [style=Rightarrow] (7.center) to (13.center);
		\draw [style=campitura] (13.center) to (10.center);
		\draw [style=Rightarrow] (0.center) to (11.center);
		\draw [style=Rightarrow] (0.center) to (12.center);
		\draw [style=Rightarrow] (0.center) to (14.center);
		\draw [style=Rightarrow] (0.center) to (18.center);
		\draw [style=Rightarrow] (18.center) to (6.center);
	\end{pgfonlayer}
\end{tikzpicture}

    }
  \caption{Condizione della formula $\vec{OP}=c_1\vec{OP}_1+c_2\vec{OP}_2$ in base ai reali $c_1,c_2$}
  \label{fig:condizionic1c2}
\end{figure}
\begin{defi}
  \label{defi:coppiaC1eC2talcheOp1}
  La coppia ($c_1,c_2$) di numeri reali tale che $\vec{OP}=c_1\vec{OP}_1+c_2\vec{OP}_2$ si dice la \textit{coppia delle coordinate} del vettore $\vec{OP}$ rispetto ai vettori base $\vec{OP}_1, \vec{OP}_2$.\\
  Le coordinate $c_1$ e $c_2$ di un vettore dipendono chiaramente dalla scelta dei vettori base $\vec{OP}_1$, $\vec{OP}_2$, ma una volta che essi sono stati fissati seriveremo $\vec{OP}\equiv (c_1,c_2)$, identificando di fatto il vettore con la coppia delle sua coordinate, e quindi l'insieme $\vec{V}_O^2$ con l'insieme $\mathds{R}^2$ delle coppie di numeri reali.
\end{defi}
\begin{oss}
  Bisognerebbe porsi il problema dell'\textit{unicità} di $c_1$ e $c_2$: se esistessero due modi diversi, diciamo $\vec{OP}=c_1\vec{OP}_1+c_2\vec{OP}_2$ e $\vec{OP}=c_1^\prime\vec{OP}_1+c_2^\prime\vec{OP}_2$, di decomporre $\vec{OP}$, non avremmo una e una spola coppia di numeri con cui identificarlo: in realtà, la costruzione grafica già suggerisce che l'unicità è garantita, ma si tornerà su tel questione nel paragrafo \ref{}.
\end{oss}
Un risultato analogo a quello visto per i vettori nel piano può essere ottenuto anche nell'insieme $V_O^3$ dei vettori geometri nello spazio tridimensionale. In questo non si deve però partire da una coppia di vettori non allineati ma da una terna di vettori $\vec{OP}_1$, $\vec{OP}_2$ e $\vec{OP}_3$ \textit{che non siano tutti e tre sullo stesso piano}: alloram, è semplice vedere graficamente, utilizzando proiezioni come fatto nel caso di due vettori nel piano, che ogni vettore $\vec{OP}\in V_O^3$ può essere scritto come combinazione $c_1\vec{OP}_1+c_2\vec{OP}_2+c_3\vec{OP}_3$.
\begin{figure}[ht!]
  \centering
  \resizebox{6cm}{!}{
      \begin{tikzpicture}
	\begin{pgfonlayer}{nodelayer}
		\node [style=none] (0) at (-1, 6) {};
		\node [style=none] (1) at (-1.5, 5) {};
		\node [style=none] (2) at (-2.25, 3.5) {};
		\node [style=none] (3) at (-4.5, -1) {};
		\node [style=none] (4) at (-1, -1) {};
		\node [style=none] (5) at (1.5, -1) {};
		\node [style=none] (6) at (3, 1) {};
		\node [style=none] (7) at (-2.5, 1) {};
		\node [style=none] (8) at (3.75, 4) {};
		\node [style=none] (10) at (4, 4.5) {$P$};
		\node [style=none] (11) at (-1.5, 6) {$r_3$};
		\node [style=none] (12) at (-7, 3) {};
		\node [style=none] (13) at (-10, -4) {};
		\node [style=none] (14) at (7, 3) {};
		\node [style=none] (15) at (4, -4) {};
		\node [style=none] (16) at (-2.5, 4) {$P_3$};
		\node [style=none] (17) at (-4.5, 2) {$c_3\vec{OP}_3$};
		\node [style=none] (18) at (-3.25, 0.25) {};
		\node [style=none] (19) at (-2.25, 1.5) {$c_2\vec{OP}_2$};
		\node [style=none] (20) at (3.5, 1.25) {$Q$};
		\node [style=none] (21) at (-5, -1.25) {O};
		\node [style=none] (22) at (-1, -1.5) {$P_1$};
		\node [style=none] (23) at (1.5, -1.5) {$c_1\vec{OP}_1$};
		\node [style=none] (24) at (-3.25, 0.5) {$P_2$};
	\end{pgfonlayer}
	\begin{pgfonlayer}{edgelayer}
		\draw [style=Rightarrow] (3.center) to (2.center);
		\draw [style=Rightarrow] (2.center) to (1.center);
		\draw [style=campitura] (1.center) to (0.center);
		\draw [style=Rightarrow] (3.center) to (4.center);
		\draw [style=Rightarrow] (4.center) to (5.center);
		\draw [style=Rightarrow] (3.center) to (6.center);
		\draw [style=campitura] (5.center) to (6.center);
		\draw [style=campitura] (7.center) to (6.center);
		\draw [style=campitura] (6.center) to (8.center);
		\draw [style=Rightarrow] (3.center) to (8.center);
		\draw [style=campitura] (2.center) to (8.center);
		\draw (13.center) to (12.center);
		\draw (12.center) to (14.center);
		\draw (14.center) to (15.center);
		\draw (15.center) to (13.center);
		\draw [style=Rightarrow] (3.center) to (18.center);
		\draw [style=Rightarrow] (18.center) to (7.center);
	\end{pgfonlayer}
\end{tikzpicture}

    }
  \caption{Vettori su spazio tridimensionale}
  \label{fig:vectspaztridim}
\end{figure}\\
Come rappresentato in figura \ref{fig:vectspaztridim}, si proietta il punto su cui stanno $\vec{OP}_1$ e $\vec{OP}_2$ seguendo la direzione $\vec{OP}_3$ e si individua così un punto $Q$; proiettando poi $P$ sulla retta $r_3$ parallelamente al vettore $\vec{OQ}$, risulta individuato un parallelogramma, che ci dice che $\vec{OP}$ si scrive come somma $\vec{OP}=\vec{OQ}+c_3\vec{OP}_3$ di $\vec{OQ}$ e di un opportuno multiplo $c_3\vec{OP}_3$ di $\vec{OP}_3$. A questo punto si osserva che $\vec{OQ}$, stando sul piano di $\vec{OP}_1$ e $\vec{OP}_2$ si scriverà come loro combinazione lineare $\vec{OQ}= c_1\vec{OP}_1+c_2\vec{OP}_2+c_3\vec{OP}_3$. In modo analogo a quato già fatto per i vettori geometrico del piano, si può dire che:
\begin{defi}
  Ka terna ($c_1,c_2,c_3$) di numeri reali tale che $\vec{OQ}= c_1\vec{OP}_1+c_2\vec{OP}_2+c_3\vec{OP}_3$ si dice la \textit{terna delle coordinate} del vettore $\vec{OP}$ rispetto ai vettori di base $\vec{OP}_1,\vec{OP}_2, \vec{OP}_3$.
\end{defi}
Come osservato per i vettori del piano, le coordinate $c_1, \text{ } c_2, \text{ } c_3$ di un vettore dipendono chiaramente dalla scelta dei vettori base $\vec{OP}_1,\vec{OP}_2, \vec{OP}_3$, ma una volta che essi sono stati fissati si potrà scrivere $\vec{OP}\equiv (c_1,c_2,c_3)$, identificando di fatto il vettore con la terna delle sue coordinate, e quindi l'insieme $\vec{V}_O^2$ con l'insieme $\mathds{R}^3$ della terna di numeri reali.\\
L'importanza delle coordinate consiste nel fatto che esse, permattendoci di rappresentare i vettori mediamente coppie o terne di numeri, permettano di tradurre in calcolo tra vettori: questa è un'importante semplificazione, in quanto è più semplice lavorare con numeri che con costruzioni o dimostrazioni di geometria eoclidea che sarebbero altrimenti necessarie per lavorare con i vettori, che sono oggetti (entità) geometrici. Per dare un idea più chiara delle affermazioni esposte precedentemente è necessario stimare questo importante risultato:
\begin{prop}
  \label{prop:coordinate1}
  Sia $\vec{OP}_1$, $\vec{OP}_2$ una coppia di vettori base non allineati nell'insieme $V_O^2$. Le coordinate rispetto a $\vec{OP}_1$, $\vec{OP}_2$ hanno le seguenti proprietà:
  \begin{enumerate}
  \item Se $\vec{OP}$ e $\vec{OP}^\prime$ hanno coordinate rispettivamente $(x_1,x_2)$ e $(x^\prime_1,x^\prime_2)$, le coordinate di $\vec{OP}+\vec{OP}^\prime$ sono date dalla coppia $(x_1+x^\prime_1,x_2+x^\prime_2)$ ottenuta sommando componete per componente le coppie delle coordinate dei due vettori.
  \item Se $\vec{OP}$ ha coordinate $(x_1,x_2)$ e $c\in \mathds{R}$ è un numero reale, allora le coordinate di $c\vec{OP}$ sono date dalla coppia $(cx_1,cx_2)$ ottenuta moltiplicando per $c$ le coordinate di $\vec{OP}$.
  \end{enumerate}
\end{prop}
\begin{proof}
  Il fatto che $\vec{OP}$ abbia coordinate $(x_1,x_2)$ rispetto a $\vec{OP}_1$, $\vec{OP}_2$ significa per definizione che $\vec{OP}^\prime= x_1\vec{OP}_1+x_2\vec{OP}_2$, e analogamente il fatto che $\vec{OP}^\prime$ abbia coordinate $(x^\prime_1,x^\prime_2)$ significa che $\vec{OP}^\prime= x^\prime_1\vec{OP}_1+x^\prime_2\vec{OP}_2$. Ma allora
  \begin{equation*}
    \vec{OP}+\vec{OP}^\prime=(x_1\vec{OP}_1+x_2\vec{OP}_2)+(x^\prime_1\vec{OP}_1+x^\prime_2\vec{OP}_2)=
  \end{equation*}
  Riordinando gli addendi e raccogliendoli diversamente sfruttando le proprietà associativa e commutativa della somma tra vettori
  \begin{equation*}
    =(x_1\vec{OP}_1+x^\prime_1\vec{OP}_1)+(x_2\vec{OP}_2+x^\prime_2\vec{OP}_2)=
  \end{equation*}
  Sfruttando la proprietà \ref{eq:numrealeVectOP} sia nella prima parentesi che nella saconda, effettuato il raggruppamento mettendo in evvidenza nel caso della prima parentesi $\vec{OP}_1$, mentre, nel caso del secondo mettendo in evvidenza $\vec{OP}_2$, il risultato sarà
  \begin{equation*}
    =(x_1+x^\prime_1)\vec{OP}_1+(x_2+x^\prime_2)\vec{OP}_2
  \end{equation*}
  Ma questo, per definizione di coordinate, significa proprio che le coordinate di $\vec{OP}+\vec{OP}^\prime$ sono date dalla coppia $(x_1+x^\prime_1,x_2+x^\prime_2)$, come affermato nel punto 1 della Proposizione \ref{prop:coordinate1}.\\
  Per dimostrare la (2), bisogna partire sempre dal fatto che $\vec{OP}$ abbia coordinate $(x_1,x_2)$ significa per definizione che $\vec{OP}= x_1\vec{OP}_1+x_2\vec{OP}_2$. Allora
  \begin{equation*}
    c\vec{OP}=c(x_1\vec{OP}_1+x_2\vec{OP}_1)=
  \end{equation*}
  Applicando la proprietà (\ref{eq:prodottoconduenumerirealiperunnumeroreale}) otterremo la divisione in due gruppi di parentesi, con c messo in evidenza messi tra di loro in forma di addizione.
  \begin{equation*}
    =c(x_1\vec{OP}_1)+c(x_2\vec{OP}_2)=
  \end{equation*}
  Applicando la proprietà (\ref{eq:prodottoconduenumerireali}) a entrambi gli addendi si otterrà:
  \begin{equation*}
    =(cx_!)\vec{OP}_1+(cx_2)\vec{OP}_2
  \end{equation*}
  Ma questo, per definizione di coordinate, ci dice proprio che le coordinate di $c\vec{OP}$ sono date dalla coppia ($cx_1,cx_2$), come affermato nella (2) della  Proposizione \ref{prop:coordinate1}.
\end{proof}
\begin{es}
  \label{es:coordinate1}
  Per un esempio di quanto appena dimostrato, si prendano i vettori base $\vec{OP}_1$ e $\vec{OP}_2$ come nel disegno seguente, e si considerino i due $\vec{OQ}_1$ e $\vec{OQ}_2$
  \begin{figure}[ht!]
  \centering
  \resizebox{4cm}{!}{
      \begin{tikzpicture}
	\begin{pgfonlayer}{nodelayer}
		\node [style=none] (0) at (0, 3) {};
		\node [style=none] (1) at (0, 0) {};
		\node [style=none] (2) at (3, 0) {};
		\node [style=none] (3) at (8, 3) {};
		\node [style=none] (4) at (8, 0) {};
		\node [style=none] (5) at (0, 8) {};
		\node [style=none] (6) at (3, 8) {};
		\node [style=none] (7) at (-0.25, -0.25) {$O$};
		\node [style=none] (8) at (-0.75, 3) {$P_2$};
		\node [style=none] (9) at (3, -0.5) {$P_1$};
		\node [style=none] (10) at (8.25, 3.5) {$Q_1$};
		\node [style=none] (11) at (3, 8.5) {$Q_2$};
	\end{pgfonlayer}
	\begin{pgfonlayer}{edgelayer}
		\draw [style=Rightarrow] (1.center) to (0.center);
		\draw [style=Rightarrow] (1.center) to (2.center);
		\draw [style=Rightarrow] (1.center) to (6.center);
		\draw [style=Rightarrow] (1.center) to (3.center);
		\draw [style=DashedCampitura] (2.center) to (4.center);
		\draw [style=DashedCampitura] (4.center) to (3.center);
		\draw [style=DashedCampitura] (0.center) to (3.center);
		\draw [style=DashedCampitura] (2.center) to (6.center);
		\draw [style=DashedCampitura] (6.center) to (5.center);
		\draw [style=DashedCampitura] (5.center) to (0.center);
	\end{pgfonlayer}
\end{tikzpicture}

    }
  \caption{Rappresentazione grafica $OQ_1$ e $OQ_2$}
  \label{fig:coordinate1-1}
\end{figure}\\
Come si vede dalla figura (\ref{fig:coordinate1-1}), si ha $\vec{OQ}_1=2\vec{OP}_1+\vec{OP}_2$ e $\vec{OQ}_1=\vec{OP}_1+2\vec{OP}_2$, ovvero le coordinate $\vec{OP}_1$ sono date dalla coppia (2, 1).\\
Allora, in base alla (1) della Proposizione \ref{prop:coordinate1}, la somma $\vec{OQ}_1+\vec{OQ}_2$ ha coordinate (\textit{sempre rispetto a $\vec{OP}_1$ e $\vec{OP}_2$}) date da 
\begin{eqnarray*}
  \vec{OQ}_1=
  \begin{vmatrix}
    2\\
    1
  \end{vmatrix},\text{ } \vec{OQ}_2=
  \begin{vmatrix}
    1\\
    2
  \end{vmatrix} &\to& \vec{OQ}_1+\vec{OQ}_2=
                  \begin{vmatrix}
                    2{\color{red}+1}\\
                    1{\color{red}+2}
                  \end{vmatrix}=
                  \begin{vmatrix}
                    3\\
                    3
                  \end{vmatrix}= (3,3).
\end{eqnarray*}
ovvero si ha $\vec{OQ}_1+\vec{OQ}_2=3\vec{OP}_1+3\vec{OP}_2$. In effetti, questo può essere verificato graficamente costruendo con la regola del parallelogramma la somma $\vec{OQ}_1+\vec{OQ}_2$, come nella figura seguente
  \begin{figure}[ht!]
  \centering
  \resizebox{3.4cm}{!}{
      \begin{tikzpicture}
	\begin{pgfonlayer}{nodelayer}
		\node [style=none] (0) at (-3, 3) {};
		\node [style=none] (1) at (-3, 0) {};
		\node [style=none] (2) at (0, 0) {};
		\node [style=none] (3) at (-3, 6) {};
		\node [style=none] (4) at (-3, 9) {};
		\node [style=none] (5) at (0, 9) {};
		\node [style=none] (6) at (0, 6) {};
		\node [style=none] (7) at (0, 3) {};
		\node [style=none] (8) at (3, 9) {};
		\node [style=none] (9) at (3, 6) {};
		\node [style=none] (10) at (3, 3) {};
		\node [style=none] (11) at (3, 0) {};
		\node [style=none] (12) at (6, 9) {};
		\node [style=none] (13) at (6, 6) {};
		\node [style=none] (14) at (6, 3) {};
		\node [style=none] (15) at (6, 0) {};
		\node [style=none] (16) at (-0.75, 5.75) {$Q_2$};
		\node [style=none] (17) at (3.5, 2.5) {$Q_1$};
		\node [style=none] (18) at (6, 9.75) {$\vec{OQ}_1+\vec{OQ}_2$};
		\node [style=none] (19) at (-3.75, 3) {$P_2$};
		\node [style=none] (20) at (0, -0.5) {$P_1$};
		\node [style=none] (21) at (-3.25, -0.25) {O};
	\end{pgfonlayer}
	\begin{pgfonlayer}{edgelayer}
		\draw [style=DashedCampitura] (4.center) to (12.center);
		\draw [style=DashedCampitura] (12.center) to (15.center);
		\draw [style=DashedCampitura] (4.center) to (0.center);
		\draw [style=DashedCampitura] (2.center) to (15.center);
		\draw [style=DashedCampitura] (8.center) to (11.center);
		\draw [style=DashedCampitura] (5.center) to (2.center);
		\draw [style=DashedCampitura] (3.center) to (13.center);
		\draw [style=DashedCampitura] (14.center) to (0.center);
		\draw [style=Rightarrow] (1.center) to (0.center);
		\draw [style=Rightarrow] (1.center) to (2.center);
		\draw [style=Rightarrow] (1.center) to (12.center);
		\draw [style=Rightarrow] (1.center) to (6.center);
		\draw [style=Rightarrow] (1.center) to (10.center);
		\draw [style=Rightarrow] (6.center) to (12.center);
		\draw [style=Rightarrow] (10.center) to (12.center);
	\end{pgfonlayer}
\end{tikzpicture}

    }
  \caption{Rappresentazione grafica $\vec{OQ}_1+\vec{OQ}_2$}
  \label{fig:coordinate1-2}
\end{figure}\\
L'aspetto notevole è che si può dimostrare chi era il vettore $\vec{OQ}_1+\vec{OQ}_2$ (in coordinate) con un semplice conto aritmetico, anche prima di disegnarlo con la costruzione geometrica del parallelogramma.
\end{es}
\begin{oss}
  \label{oss:coordinate2}
  Affermazioni del tutto analoghe a quelle della Proposizione \ref{prop:coordinate1} valgono anche nel caso dei vettori nello spazio. Più precisamente, si ha che fissata una terna $\vec{OP}_1, \vec{OP}_2,\vec{OP}_3$ di vettori non complanari nell'insieme $V_O^3$ dei vettori dello spazio tridimensionale, allora le coordiante rispetto a tale terna di base hanno le seguenti proprietà:
  \begin{enumerate}
  \item Se $\vec{OP}$ e $\vec{OP}^\prime$ hanno coordinate rispettivamete $(x_1,x_2,x_3)$ e $(x_1^\prime,x_2^\prime,x_3^\prime)$, le coordinate di $\vec{OP}_1+\vec{OP}_1^\prime$ sono date dalla terna $(x_1+x_1^\prime,x_2+x_2^\prime,x_3+x_3^\prime)$ ottenuta sommando componente per componente le terne delle coordiante dei due vettori.
  \item Se $\vec{OP}$ ha coordinate $(x_1,x_2,x_3)$ e $c\in \mathds{R}$ è un numero reale, allora le coordinate, di $c\vec{OP}$ sono date dalla terna $(cx_1,cx_2,cx_3)$ ottenuta moltiplicando per $c$ le coordinate di $\vec{OP}$.
  \end{enumerate}
  La dimostrazione è perfettamente analoga a quella della Proposizione \ref{prop:coordinate1}.
\end{oss}

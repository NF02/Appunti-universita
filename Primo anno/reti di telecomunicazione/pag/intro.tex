\section{Sommario}
Qui di seguito sono riportati i concetti fondamentali trattati all'interno
del documento
\begin{itemize}
	\item Informazione e segnali;
		\begin{tasks}(2)
			\task L'informazione sussiste solo se il ricevente della
			trasmissione non conosce il contenuto della suddetta;
			\task Per esistere una trasmissione devono esserci:
				\begin{enumerate}
					\item Comunicazione;
					\item mezzo di trasmissione;
					\item informazione.
				\end{enumerate}
		\end{tasks}
	\item Informazioni analogiche e digitali.
		\begin{itemize}
			\item Informazioni analogiche: si dicono grandezze analogiche
				quelle che possono assumere tutti i valori intermedi
				all'interno di un dato intervallo; Si dicono grandezze digitali
				quelle che vengono espresse in modo numerico, senza possibilità
				di discriminare valori intermedi tra due cifre consecutive.
				Ulteriori approfondimenti presenti in (\ref{sanalog})
				\begin{center}
					By
					\underline{\href{https://it.wikipedia.org/wiki/Analogico}{Wikipedia}}
				\end{center}
			\item Informazioni Digitali: Con digitale o numerico, in
				informatica ed elettronica, ci si riferisce a tutto ciò che
				viene rappresentato con numeri o che opera manipolando numeri,
				contrapposto all'analogico. Ulteriori approfondimenti presenti
				in (\ref{sdigital})
				\begin{center}
					By \underline{\href{https://it.wikipedia.org/wiki/Digitale_(informatica)}{Wikipedia}}
				\end{center}
		\end{itemize}
\end{itemize}
\texttt{Oggi ormai utilizziamo il digitale perché effettivamente i calcolatori
elettronici gestiscono meglio una codifica rispetto a dei numeri reali. Per di
più costa meno produrre un dispositivo che gestisca segnali digitali rispetto
ad un dispositivo che gestisce mezzi analogici, per esempio la differenza tra
lo standard VHS e lo standard CD/DVD/Blue Ray.}\\
Bisogna anche dire che le trasmissione vengono comunque trasmessi tramite dei
canali fisici ({\bf Analogici}), semplicemente all'interno dei dispositivi che si
occupando i convertire da analogico a digitale e viceversa.
\subsection{Alcune osservazioni}
\begin{itemize}
	\item Non tutte le informazioni costituiscono informazione
		\begin{enumerate}
			\item La notizia comunicata deve per noi essere eclatante;
			\item una persona noiosa non apporta informazione perché ripete
				continuamente gli stessi argomenti.
		\end{enumerate}
	\item Problema di misurazione del contenuto informativo
		\begin{itemize}
			\item {\bf Claude E. Shannon} ({\tt 1916-2001}), fondatore della
				\textit{Teoria Matematica dell'Informazione}, è stato il primo
				ad introdurre la distinzione tra forma e significato nel
				processo comunicativo.
		\end{itemize}
\end{itemize}
\subsubsection{I risultati di Shannon}
\begin{itemize}
	\item Non è possibile definire la quantità di informazione associata ad un
		messaggio già ricevuto, ma piuttosto la quantità di informazione
		associata ad un papabile messaggio
		\begin{itemize}
			\item \textit{``information is that which reduces uncertainty''}
		\end{itemize}
	\item La quantità di informazione associata ad un massaggio è tanto più
		altra quanto più esso è inatteso
		\begin{itemize}
			\item il messaggio ``{\bf domani sorgerà il sole}'' ha un bassissimo
				contenuto informativo perché è assolutamente scontato e banale
			\item il messaggio ``{\bf Domani scoppierà la guerra}'' ha un alto
				contenuto informativo.
		\end{itemize}
\end{itemize}
\section{I segnali}
\begin{itemize}
	\item \textit{Grandezze fisiche variabili nel tempo a cui è associata
		un'informazione};
	\item \textit{L'informazione è associata ad una variazione ({\color{red}
		aleatorio} e non deterministica) della grandezza fisica};
	\item Aleatorio (dal latino ``alea'', gioco di dati) è sinonimo di non
		predicibile a priori (in contrapposizione con deterministico).
\end{itemize}
\subsection{Rappresentazione dell'informazione}
\begin{itemize}
	\item Associazione tra caratteristiche (di valore e temporali) dei segnali
		e le informazioni che essi rappresentano;
	\item Le caratteristiche sono impresse dal dispositivo generatore del
		segnale;
	\item Quali caratteristiche?
		\begin{itemize}
			\item valore, andamento temporale ed eventi del segnale (es.
				superare una soglia), etc.
		\end{itemize}
\end{itemize}
\subsection{Osservazione}
\begin{itemize}
	\item L'associazione informazione-segnale può essere arbitraria,
		tecnologie permettendo
	\item Esempi
		\begin{itemize}
			\item valore costante $\to$ segnale a frequenza costante evento
				$\to$ segnale ad ampiezza costante
		\end{itemize}
	\item Occorre quindi chiaramente distinguere tra stati, andamenti ed eventi
		del segnale e del fenomeno (\textit{cioè dell'informazione}) da esso
		descritto.
\end{itemize}
\subsection{Classificazione di segnali}
\begin{tasks}(3)
	\task In base alla loro natura fisica [grandezza fisica $\to$ trasduttore
	$\to$ segnale elettrico]
	\begin{itemize}
		\item elettrici;
		\item acustici;
		\item etc.
	\end{itemize}
	\task In base a come vengono rappresentati
	\begin{itemize}
		\item analogici;
		\item digitali.
	\end{itemize}
	\task Segnali elettrici
	\begin{itemize}
		\item Trasmissione di informazione tramite una variazione di corrente
			elettrica o di tensione all'interno di un conduttore oppure in un
			punto di un circuito elettrico o elettronico.
		\item grazie ai trasduttori qualsiasi segnale fisico può diventare
			elettrico
		\item Esempio: segnale acustico (\textit{vibrazione})
	\end{itemize}
\end{tasks}
\section{Segnali}
\subsection{Segnali analogici \label{sanalog}}
\begin{itemize}
	\item il valore dell'informazione rappresentata è una funzione continua
		della grandezza significativa;
	\item rappresentazione attraverso un numero reale (\textit{con precisione
		teoricamente infinita})
	\item generati da sensori o trasduttori che creano una corrispondenza tra
		la grandezza fisica che è oggetto di informazione (\textbf{esempio
		temperatura}) e il segnale (\textbf{esempio tensione elettrica})
\end{itemize}
\paragraph{Esempi}
\begin{itemize}
	\item \textbf{Temperatura:} altezza in \texttt{mm} del mercurio nel
		termometro;
	\item \textbf{Acustico:} variazione di pressione ad un microfono;
	\item \textbf{Elettrico:} tensione ai capi di un conduttore.
\end{itemize}
\subsection{Segnali digitali\label{sdigital}}
\begin{itemize}
	\item rappresentazione come sequenza di numeri presi da un insieme di
		valori discreti, ovvero appartenenti a uno stesso insieme ben definito
		e circoscritto;
	\item rappresentazione ``{\bf a fasce}''
\end{itemize}
\paragraph{Osservazione importante}
\begin{itemize}
	\item L'attributo ``analogico'' o ``digitale'' non si riferisce a
		caratteristiche intrinseche del segnale ma a caratteristiche
		dell'informazione da esso rappresentato:
	\item {\color{red} I segnali digitali nascono come analogici}
\end{itemize}
\subsection{Pregi e difetti}
\paragraph{Analogico}
\subparagraph{Pregi}
\begin{tasks}(2)
	\task Sono più ``naturali'', le leggi della fisica classica operano
	tipicamente nel ``continuo'';
	\task Il rumore deforma ma non stravolge il segnale (errori proporzionali
	all'entità del disturbo ``{\bf in onde media la radio analogica la senti,
	anche se con un forte rumore bianco di fondo.}'')
\end{tasks}
\subparagraph{Difetti}
\begin{tasks}(2)
	\task Dispositivi di elaborazione relativamente poco precisi, poco stabili
	nel tempo ``maggiormente predisposti ai guasti, alle intemperie e anche a
	potenziali variazioni atmosferiche'' e poco immuni alle perturbazioni;\\
	({\tt esempio}: il video registratore VHS ``M-matic'' o sony U-matic, sono
	apparecchi estremamente complessi, soprattutto gli ultimi per metà digitali
	con tante funzionalità e tasti programmabili per fasce orarie, perfetti per
	registrale le trasmissioni in modo autonomo.)
	\task Le elaborazioni su di essi sono poco flessibili e producono degrado.
\end{tasks}
\paragraph{Digitale}
\subparagraph{Pregi}
\begin{tasks}(3)
	\task Rappresentazione esatta di simboli di un alfabeto finito;
	\task Semplicità e robustezza dei circuiti di gestione ed elaborazione;
	\task Elevata immunità ai disturbi.
\end{tasks}
\subparagraph{Difetti}
\begin{tasks}(3)
	\task Gli errori possono stravolgere l'informazione;
	\task Servono molti bit per rappresentare informazioni ricche;
	\task Le informazioni intrinsecamente continue vanno convertite.
\end{tasks}
\subsection{Il binario}
\begin{itemize}
	\item Due stati previsti e possibili:
		\begin{itemize}
			\item 1 = Vero ``TRUE'' = ON = HIGH ``\textit{Livello alto}''
			\item 0 = Falso ``False'' = OFF = Low ``\textit{Livello basso}''
		\end{itemize}
	\item Logica positiva o negativa;
	\item Rappresentazione necessaria per un calcolatore.
	\item Rappresentazione senza segno
		\begin{itemize}
			\item base \textit{b} e lunghezza \textit{n};
			\item conversione in base 10;
			\item con \textit{n} bit rappresento qualsiasi decimale senza segno
				tra 0 e $2^n-1$
				\begin{equation}
					(a_{n-1},\dots, a_1,a_0)_2 \to
					\sum^{n-1}_{\substack{i=0}}a_ib^i
				\end{equation}
				Dove ($a_{n-1}$) è la cifra più significativa e $a_0$ è quella
				meno significativa.\\
				Un esempio:
				\begin{equation}
					(10101110)_2\Leftrightarrow (174)_{10} \Leftrightarrow
					(AE)_{16}
				\end{equation}
		\end{itemize}
	\item Rappresentazione in modulo e segno
		\begin{itemize}
			\item il bit più significativo rappresenta il segno (0 = positivo e
				1 = segno negativo), mentre i restanti rappresentano il modulo;
			\item scomoda per operazioni aritmetiche (due modi per scrivere 0);
			\item la somma tra due numeri A e B si svolge come da tabella
				\begin{table}[h!]
					\centering
					\begin{tabular}{l|ccc|}
						&&\multicolumn{2}{c|}{Segno di B} \\
						&&+&-\\
						\multirow{2}{*}{Segno di A}&+&$A+B$&$a-\abs B$\\
						&-&$B-\abs A$&$\abs A +\abs B$
					\end{tabular}
					\caption {Somma tra A e B}
				\end{table}
		\end{itemize}
	\item Rappresentazione in complemento a 1
		\begin{tasks}
			\task il bit più significativo rappresenta ({\tt come prima}) il
			segno
			\begin{itemize}
				\item stesso intervallo di valori rappresentabili con modulo e
					segno.
			\end{itemize}
			\task un numero negativo si ottiene dal suo positivo e cambiando
			tutti i bit
			\begin{itemize}
				\item esempio a 4 bit: 0110 corrisponde a 6, mentre, 1001
					corrisponde a -6
			\end{itemize}
			\task nella operazione aritmetiche (\textit{ad esempio la somma})
			si utilizza l'eventuale riporto in fase di somma.
			\task esempio: 22+3=25
			\begin{center}
				\begin{tabular}{lccc}
					riporto&&{\color{red}1100}\\
					&0001&0110&(22)\\
					+&0000&0011&(3)\\\hline
					&0001&1001&(25)
				\end{tabular}
			\end{center}
		\end{tasks}
	\item Rappresentazione in complemento a 2
	\begin{tasks}(2)
		\task vantaggio: unica rappresentazione per lo zero;
		\task si ignora l'overflow;
		\task esempio: 31-5=26
			\begin{center}
				\begin{tabular}{lccc}
					riporto&{\color{red}1111}&{\color{red}1110}\\
					&0001&1111&(31)\\
					+&0000&1011&(-5)\\\hline
					&0001&1010&(26)
				\end{tabular}
			\end{center}
	\end{tasks}
\end{itemize}
\subsection{Confronto}
\begin{tasks}(3)
	\task Rappresentazione in modulo e il segno
	\begin{itemize}
		\item il bit più significativo rappresenta il segno (\textit{1 =
			negativo}) e i restanti il modulo.
		\item scomoda per operazioni aritmetiche
	\end{itemize}
	\task Rappresentazione in complemento a 1
	\begin{itemize}
		\item si complementano tutti i bit
		\item utilizzo del riporto in fase di somma
	\end{itemize}
	\task Rappresentazione in complemento a 2
	\begin{itemize}
		\item si complementano tutti i bit e si somma 1
		\item unica rappresentazione per per lo zero
	\end{itemize}
\end{tasks}
\begin{table}[h!]
	\centering
	\begin{tabular}{|c|c|c|c|c|c|c|}
		\hline
		\multirow{2}{*}{Stringa}&\multicolumn{4}{c|}{rappresentazione}\\
		&senza segno&modulo e segno&complemento a 1&complemento a 2\\\hline
		000&0&0&0&0\\\hline
		001&1&1&1&1\\\hline
		010&2&2&2&2\\\hline
		011&3&3&3&3\\\hline
		100&4&0&-3&-4\\\hline
		101&5&-1&-2&-3\\\hline
		110&6&-2&-1&-2\\\hline
		111&7&-3&0&-1\\\hline
	\end{tabular}
	\caption {Confronto tra il modulo e segno, complemento 1 \& 2}
\end{table}

